\documentclass[border=10pt]{standalone}
\usepackage[UTF8]{ctex}
\usepackage{tikz}
% \usetikzlibrary{shapes}

\newcommand{\iwd}{1000}
\newcommand{\fs}[1]{\fontsize{#1 pt}{0pt}\selectfont}
\newcommand{\osilayer}[5]{
    \node [root] (#1) at ([yshift=30pt]#2.north east) {\kaishu \textbf{#4} \\[5pt] \color{red!70!black}{\fs{15} \textit{#5}}};
    \node [font=\fs{25}, anchor=west, draw=black, shape=circle, line width=2pt] at ([xshift=20pt, yshift=5pt]#1.east) {\textbf{#3}};
}

\begin{document}
\begin{tikzpicture}[
    x=1pt, y=1pt,
]

\tikzstyle{root}=[align=right, font={\fs{20}}, anchor=south east, text=cyan!80!black]

\coordinate (ori) at (0, 0);
\osilayer{lay1}{ori}{1}{物理层}{Physical Layer}
\osilayer{lay2}{lay1}{2}{数据链路层}{Data Link Layer}
\osilayer{lay3}{lay2}{3}{网络层}{Network Layer}
\osilayer{lay4}{lay3}{4}{传输层}{Transport Layer}
\osilayer{lay5}{lay4}{5}{会话层}{Session Layer}
\osilayer{lay6}{lay5}{6}{表示层}{Presentation Layer}
\osilayer{lay7}{lay6}{7}{应用层}{Application Layer}


\end{tikzpicture}
\end{document}