% \part{Ti\emph{k}Z ist \emph{kein} Zeichenprogramm}
\part{绘何物为}\footnote{\color{blue} 该部分标题原文为德文:Ti\emph{k}Z ist \emph{kein} Zeichenprogramm,翻译成英文就是 “Ti\emph{k}Z is not a drawing program”,中文意思是“Ti\emph{k}Z 不是一个绘图程序”。然而德文采用的是 “GNU's Not Unix!” 式的递归缩写,这里如果直接采用原文也并非不可以,但是中文博大精深,一定有对应的贴切的翻译。
\\ \indent \indent 我这里译成“绘何物为”,用了拼音的递归:Huì hé wù wéi。意即“‘绘’是什么呢”,当然也可以将“绘”直接作为动词,理解成“绘制什么呢”。这样中文含义就和原文含义形成一问一答,无论是形式上还是内容上,都有了合理的对应。当然,这里着实夹杂了译者的私货,正文中依旧使用 \tikzname\ 来指代这一绘图系统。}
\label{part-tikz}

{\Large \emph{by Till Tantau}}


\bigskip
\noindent
\vskip3cm
\begin{codeexample}[graphic=white]

\begin{tikzpicture}[angle radius=.75cm]

  \node (A) at (-2,0)    [red,left]   {$A$};
  \node (B) at ( 3,.5)    [red,right]  {$B$};
  \node (C) at (-2,2)  [blue,left]  {$C$};
  \node (D) at ( 3,2.5)  [blue,right] {$D$};
  \node (E) at (60:-5mm) [below]      {$E$};
  \node (F) at (60:3.5cm)  [above]      {$F$};

  \coordinate (X) at (intersection cs:first line={(A)--(B)}, second line={(E)--(F)});
  \coordinate (Y) at (intersection cs:first line={(C)--(D)}, second line={(E)--(F)});
  
  \path 
    (A) edge [red, thick]  (B)
    (C) edge [blue, thick] (D)
    (E) edge [thick]       (F)
      pic ["$\alpha$",  draw, fill=yellow]   {angle = F--X--A}
      pic ["$\beta$",   draw, fill=green!30] {angle = B--X--F}
      pic ["$\gamma$",  draw, fill=yellow]   {angle = E--Y--D}
      pic ["$\delta$",  draw, fill=green!30] {angle = C--Y--E};

  \node at ($ (D)!.5!(B) $) [right=1cm,text width=6cm,rounded corners,fill=red!20,inner sep=1ex]
    {
      When we assume that $\color{red}AB$ and $\color{blue}CD$ are
      parallel, i.\,e., ${\color{red}AB} \mathbin{\|} \color{blue}CD$,
      then $\alpha = \delta$ and $\beta = \gamma$.
    };
\end{tikzpicture}
\end{codeexample}

% Copyright 2006 by Till Tantau
%
% This file may be distributed and/or modified
%
% 1. under the LaTeX Project Public License and/or
% 2. under the GNU Free Documentation License.
%
% See the file doc/generic/pgf/licenses/LICENSE for more details.


\section{Design Principles}

This section describes the design principles behind the \tikzname\ frontend,
where \tikzname\ means ``\tikzname\ ist \emph{kein} Zeichenprogramm''. To use
\tikzname, as a \LaTeX\ user say |\usepackage{tikz}| somewhere in the preamble,
as a plain \TeX\ user say |\input tikz.tex|. \tikzname's job is to make your
life easier by providing an easy-to-learn and easy-to-use syntax for describing
graphics.

The commands and syntax of \tikzname\ were influenced by several sources. The
basic command names and the notion of path operations is taken from
\textsc{metafont}, the option mechanism comes from \textsc{pstricks}, the
notion of styles is reminiscent of \textsc{svg}, the graph syntax is taken from
\textsc{graphviz}. To make it all work together, some compromises were
necessary. I also added some ideas of my own, like coordinate transformations.

The following basic design principles underlie \tikzname:
%
\begin{enumerate}
    \item Special syntax for specifying points.
    \item Special syntax for path specifications.
    \item Actions on paths.
    \item Key--value syntax for graphic parameters.
    \item Special syntax for nodes.
    \item Special syntax for trees.
    \item Special syntax for graphs.
    \item Grouping of graphic parameters.
    \item Coordinate transformation system.
\end{enumerate}


\subsection{Special Syntax For Specifying Points}

\tikzname\ provides a special syntax for specifying points and coordinates. In
the simplest case, you provide two \TeX\ dimensions, separated by commas, in
round brackets as in |(1cm,2pt)|.

You can also specify a point in polar coordinates by using a colon instead of a
comma as in |(30:1cm)|, which means ``1cm in a 30 degrees direction''.

If you do not provide a unit, as in |(2,1)|, you specify a point in \pgfname's
$xy$-coordinate system. By default, the unit $x$-vector goes 1cm to the right
and the unit $y$-vector goes 1cm upward.

By specifying three numbers as in |(1,1,1)| you specify a point in \pgfname's
$xyz$-coordinate system.

It is also possible to use an anchor of a previously defined shape as in
|(first node.south)|.

You can add two plus signs before a coordinate as in |++(1cm,0pt)|. This means
``1cm to the right of the last point used''. This allows you to easily specify
relative movements. For example, |(1,0) ++(1,0) ++(0,1)| specifies the three
coordinates |(1,0)|, then |(2,0)|, and |(2,1)|.

Finally, instead of two plus signs, you can also add a single one. This also
specifies a point in a relative manner, but it does not ``change'' the current
point used in subsequent relative commands. For example, |(1,0) +(1,0) +(0,1)|
specifies the three coordinates |(1,0)|, then |(2,0)|, and |(1,1)|.


\subsection{Special Syntax For Path Specifications}

When creating a picture using \tikzname, your main job is the specification of
\emph{paths}. A path is a series of straight or curved lines, which need not be
connected. \tikzname\ makes it easy to specify paths, partly using the syntax
of \textsc{metapost}. For example, to specify a triangular path you use
%
\begin{codeexample}[code only]
(5pt,0pt) -- (0pt,0pt) -- (0pt,5pt) -- cycle
\end{codeexample}
%
and you get \tikz \draw (5pt,0pt) -- (0pt,0pt) -- (0pt,5pt) -- cycle; when you
draw this path.


\subsection{Actions on Paths}

A path is just a series of straight and curved lines, but it is not yet
specified what should happen with it. One can \emph{draw} a path, \emph{fill} a
path, \emph{shade} it, \emph{clip} it, or do any combination of these. Drawing
(also known as \emph{stroking}) can be thought of as taking a pen of a certain
thickness and moving it along the path, thereby drawing on the canvas. Filling
means that the interior of the path is filled with a uniform color. Obviously,
filling makes sense only for \emph{closed} paths and a path is automatically
closed prior to filling, if necessary.

Given a path as in |\path (0,0) rectangle (2ex,1ex);|, you can draw it by
adding the |draw| option as in |\path[draw] (0,0) rectangle (2ex,1ex);|, which
yields \tikz \path[draw] (0,0) rectangle (2ex,1ex);. The |\draw| command is
just an abbreviation for |\path[draw]|. To fill a path, use the |fill| option
or the |\fill| command, which is an abbreviation for |\path[fill]|. The
|\filldraw| command is an abbreviation for |\path[fill,draw]|. Shading is
caused by the |shade| option (there are |\shade| and |\shadedraw|
abbreviations) and clipping by the |clip| option. There is also a |\clip|
command, which does the same as |\path[clip]|, but not commands like
|\drawclip|. Use, say, |\draw[clip]| or |\path[draw,clip]| instead.

All of these commands can only be used inside |{tikzpicture}| environments.

\tikzname\ allows you to use different colors for filling and stroking.


\subsection{Key--Value Syntax for Graphic Parameters}

Whenever \tikzname\ draws or fills a path, a large number of graphic parameters
influences the rendering. Examples include the colors used, the dashing
pattern, the clipping area, the line width, and many others. In \tikzname, all
these options are specified as lists of so called key--value pairs, as in
|color=red|, that are passed as optional parameters to the path drawing and
filling commands. This usage is similar to \textsc{pstricks}. For example, the
following will draw a thick, red triangle;
%
\begin{codeexample}[]
\tikz \draw[line width=2pt,color=red] (1,0) -- (0,0) -- (0,1) -- cycle;
\end{codeexample}


\subsection{Special Syntax for Specifying Nodes}

\tikzname\ introduces a special syntax for adding text or, more generally,
nodes to a graphic. When you specify a path, add nodes as in the following
example:
%
\begin{codeexample}[]
\tikz \draw (1,1) node {text} -- (2,2);
\end{codeexample}
%
Nodes are inserted at the current position of the path, but either \emph{after}
(the default) or \emph{before} the complete path is rendered. When special
options are given, as in |\draw (1,1) node[circle,draw] {text};|, the text is
not just put at the current position. Rather, it is surrounded by a circle and
this circle is ``drawn''.

You can add a name to a node for later reference either by using the option
|name=|\meta{node name} or by stating the node name in parentheses outside the
text as in |node[circle](name){text}|.

Predefined shapes include |rectangle|, |circle|, and |ellipse|, but it is
possible (though a bit challenging) to define new shapes.


\subsection{Special Syntax for Specifying Trees}

The ``node syntax'' can also be used to draw tress: A |node| can be followed by
any number of children, each introduced by the keyword |child|. The children
are nodes themselves, each of which may have children in turn.
%
\begin{codeexample}[]
\begin{tikzpicture}
  \node {root}
    child {node {left}}
    child {node {right}
      child {node {child}}
      child {node {child}}
    };
\end{tikzpicture}
\end{codeexample}
%
Since trees are made up from nodes, it is possible to use options to modify the
way trees are drawn. Here are two examples of the above tree, redrawn with
different options:
%
\begin{codeexample}[preamble={\usetikzlibrary{arrows.meta,trees}}]
\begin{tikzpicture}
  [edge from parent fork down, sibling distance=15mm, level distance=15mm,
   every node/.style={fill=red!30,rounded corners},
   edge from parent/.style={red,-{Circle[open]},thick,draw}]
  \node {root}
      child {node {left}}
      child {node {right}
        child {node {child}}
        child {node {child}}
      };
\end{tikzpicture}
\end{codeexample}

\begin{codeexample}[]
\begin{tikzpicture}
  [parent anchor=east,child anchor=west,grow=east,
   sibling distance=15mm, level distance=15mm,
   every node/.style={ball color=red,circle,text=white},
   edge from parent/.style={draw,dashed,thick,red}]
  \node {root}
      child {node {left}}
      child {node {right}
        child {node {child}}
        child {node {child}}
      };
\end{tikzpicture}
\end{codeexample}


\subsection{Special Syntax for Graphs}

The |\node| command gives you fine control over where nodes should be placed,
what text they should use, and what they should look like. However, when you
draw a graph, you typically need to create numerous fairly similar nodes that
only differ with respect to the name they show. In these cases, the |graph|
syntax can be used, which is another syntax layer build ``on top'' of the node
syntax.
%
\begin{codeexample}[preamble={\usetikzlibrary{graphs}}]
\tikz \graph [grow down, branch right] {
  root -> { left, right -> {child, child} }
};
\end{codeexample}
%
The syntax of the |graph| command extends the so-called \textsc{dot}-notation
used in the popular \textsc{graphviz} program.

Depending on the version of \TeX\ you use (it must allow you to call Lua code,
which is the case for Lua\TeX), you can also ask \tikzname\ to do automatically
compute good positions for the nodes of a graph using one of several integrated
\emph{graph drawing algorithms}.


\subsection{Grouping of Graphic Parameters}

Graphic parameters should often apply to several path drawing or filling
commands. For example, we may wish to draw numerous lines all with the same
line width of 1pt. For this, we put these commands in a |{scope}| environment
that takes the desired graphic options as an optional parameter. Naturally, the
specified graphic parameters apply only to the drawing and filling commands
inside the environment. Furthermore, nested |{scope}| environments or
individual drawing commands can override the graphic parameters of outer
|{scope}| environments. In the following example, three red lines, two green
lines, and one blue line are drawn:
%
\begin{codeexample}[]
\begin{tikzpicture}
  \begin{scope}[color=red]
    \draw (0mm,10mm) -- (10mm,10mm);
    \draw (0mm, 8mm) -- (10mm, 8mm);
    \draw (0mm, 6mm) -- (10mm, 6mm);
  \end{scope}
  \begin{scope}[color=green]
    \draw             (0mm, 4mm) -- (10mm, 4mm);
    \draw             (0mm, 2mm) -- (10mm, 2mm);
    \draw[color=blue] (0mm, 0mm) -- (10mm, 0mm);
  \end{scope}
\end{tikzpicture}
\end{codeexample}

The |{tikzpicture}| environment itself also behaves like a |{scope}|
environment, that is, you can specify graphic parameters using an optional
argument. These optional apply to all commands in the picture.


\subsection{Coordinate Transformation System}

\tikzname\ supports both \pgfname's \emph{coordinate} transformation system to
perform transformations as well as \emph{canvas} transformations, a more
low-level transformation system. (For details on the difference between
coordinate transformations and canvas transformations see
Section~\ref{section-design-transformations}.)

The syntax is set up in such a way that it is harder to use canvas
transformations than coordinate transformations. There are two reasons for
this: First, the canvas transformation must be used with great care and often
results in ``bad'' graphics with changing line width and text in wrong sizes.
Second, \pgfname\ loses track of where nodes and shapes are positioned when
canvas transformations are used. So, in almost all circumstances, you should
use coordinate transformations rather than canvas transformations.

% % Copyright 2006 by Till Tantau
%
% This file may be distributed and/or modified
%
% 1. under the LaTeX Project Public License and/or
% 2. under the GNU Free Documentation License.
%
% See the file doc/generic/pgf/licenses/LICENSE for more details.

\section[Hierarchical Structures: Package, Environments, Scopes, and Styles]
{Hierarchical Structures:\\
  Package, Environments, Scopes, and Styles}

The present section explains how your files should be structured when
you use \tikzname. On the top level, you need to include the |tikz|
package. In the main text, each graphic needs to be put in a
|{tikzpicture}| environment. Inside these environments, you can use
|{scope}| environments to create internal groups. Inside the scopes
you use |\path| commands to actually draw something. On all levels
(except for the package level), graphic options can be given that
apply to everything within the environment.



\subsection{Loading the Package and the Libraries}

\begin{package}{tikz}
  This package does not have any options.

  This will automatically load the \pgfname\ and the |pgffor| package.

  \pgfname\ needs to know what \TeX\ driver you are intending to use. In
  most cases \pgfname\ is clever enough to determine the correct driver
  for you; this is true in particular if you use \LaTeX. One
  situation where \pgfname\ cannot know the driver ``by itself'' is when
  you use plain \TeX\ or Con\TeX t together with |dvipdfm|. In this case,
  you have to write |\def\pgfsysdriver{pgfsys-dvipdfm.def}|
  \emph{before} you input |tikz.tex|.
\end{package}


\begin{command}{\usetikzlibrary\marg{list of libraries}}
  Once \tikzname\ has been loaded, you can use this command to load
  further libraries. The list of libraries should contain the names of
  libraries separated by commas. Instead of curly braces, you can also
  use square brackets, which is something Con\TeX t users will
  like. If you try to load a library a second time, nothing will
  happen.

  \example |\usetikzlibrary{arrows.meta}|

  The above command will load a whole bunch of extra arrow tip
  definitions.

  What this command does is to load the file
  |tikzlibrary|\meta{library}|.code.tex| for each \meta{library} in
  the \meta{list of libraries}. If this file does not exist, the file
  |pgflibrary|\meta{library}|.code.tex| is loaded instead. If this
  file also does not exist, an error message is printed. Thus, to
  write your own library file, all you need to do is to place a file
  of the appropriate name somewhere where \TeX\ can find it. \LaTeX,
  plain \TeX, and Con\TeX t users can then use your library.
\end{command}



\subsection{Creating a Picture}

\subsubsection{Creating a Picture Using an Environment}

The ``outermost'' scope of \tikzname\ is the |{tikzpicture}|
environment. You may give drawing commands only inside this
environment, giving them outside (as is possible in many other
packages) will result in chaos.

In \tikzname, the way graphics are rendered is strongly influenced by
graphic options. For example, there is an option for setting the color used
for drawing, another for setting the color used for filling, and also
more obscure ones like the option  for setting the prefix used in the
filenames of temporary files written while plotting functions using an
external program. The graphic options are specified in \emph{key
  lists}, see Section~\ref{section-graphic-options} below for
details. All graphic options are local to the |{tikzpicture}| to which
they apply.

\begin{environment}{{tikzpicture}\opt{\oarg{options}}}
  All \tikzname\ commands should be given inside this
  environment, except for the |\tikzset| command. Unlike other
  packages, it is not possible to use, say, |\pgfpathmoveto| outside
  this environment and doing so will result in chaos. For \tikzname,
  commands like |\path| are only defined inside this environment, so
  there is little chance that you will do something wrong here.

  When this environment is encountered, the \meta{options} are
  parsed, see Section~\ref{section-graphic-options}. All options given
  here will apply to the whole picture.

  Next, the contents of the environment is processed and the graphic
  commands therein are put into a box. Non-graphic text is suppressed
  as well as possible, but non-\pgfname\ commands inside a
  |{tikzpicture}| environment should not produce any ``output'' since
  this may totally scramble the positioning system of the backend
  drivers. The suppressing of normal text, by the way, is done by
  temporarily switching the font to |\nullfont|. You can, however,
  ``escape back'' to normal \TeX\ typesetting. This happens, for
  example, when you specify a node.

  At the end of the environment, \pgfname\ tries to make a good guess
  at the size of a bounding box of the graphic and
  then resizes the picture box such that the box has this size. To ``make its
  guess,'' everytime \pgfname\ encounters a coordinate, it updates the
  bounding box's size such that it encompasses all these
  coordinates. This will usually give a good
  approximation of the bounding box, but will not always be
  accurate. First, the line thickness of diagonal lines is not taken
  into account correctly. Second, control points of a curve often lie far
  ``outside'' the curve and make the bounding box too large. In this
  case, you should use the |[use as bounding box]| option.

  The following key influences the baseline of the resulting
  picture:
  \begin{key}{/tikz/baseline=\meta{dimension or coordinate or \texttt{default}} (default 0pt)}
    Normally, the lower end of the picture is put on the baseline of
    the surrounding text. For example, when you give the code
    |\tikz\draw(0,0)circle(.5ex);|, \pgfname\ will find out that the
    lower end of the picture is at $-.5\mathrm{ex} - 0.2\mathrm{pt}$
    (the 0.2pt are half the line width, which is 0.4pt) and that the
    upper end is at $.5\mathrm{ex}+.5\mathrm{pt}$. Then, the lower end
    will be put on the baseline, resulting in the following:
    \tikz\draw(0,0)circle(.5ex);. 

    Using this option, you can specify that the picture should be
    raised or lowered such that the height \meta{dimension} is on the
    baseline. For example, |\tikz[baseline=0pt]\draw(0,0)circle(.5ex);|
    yields \tikz[baseline=0pt]\draw(0,0)circle(.5ex); since, now, the
    baseline is on the height of the $x$-axis.

    This options is often useful for ``inlined'' graphics as in
\begin{codeexample}[]
$A \mathbin{\tikz[baseline] \draw[->>] (0pt,.5ex) -- (3ex,.5ex);} B$
\end{codeexample}

    Instead of a \meta{dimension} you can also provide a coordinate in
    parentheses. Then the effect is to put the baseline on the
    $y$-coordinate that the given \meta{coordinate} has \emph{at the
      end of the picture}. This means that, at the end of the picture,
    the \meta{coordinate} is evaluated and then the baseline is set
    to the $y$-coordinate of the resulting point. This makes it easy
    to reference the $y$-coordinate of, say, the baseline of nodes.
\begin{codeexample}[]
Hello
\tikz[baseline=(X.base)]
  \node [cross out,draw] (X) {world.};
\end{codeexample}

\begin{codeexample}[]
Top align:
\tikz[baseline=(current bounding box.north)]
  \draw (0,0) rectangle (1cm,1ex);
\end{codeexample}
	
	Use |baseline=default| to reset the |baseline| option to its initial configuration.
  \end{key}

  \begin{key}{/tikz/execute at begin picture=\meta{code}}
    This option causes \meta{code} to be executed
    at the beginning of the picture. This option must be
    given in the argument of the |{tikzpicture}| environment itself
    since this option will not have an effect otherwise. After all,
    the picture has already ``started'' later on. The effect of
    multiply setting this option accumulates.

    This option is mainly used in styles like the |every picture|
    style to execute certain code at the start  of a picture.
  \end{key}

  \begin{key}{/tikz/execute at end picture=\meta{code}}
    This option installs \meta{code} that will be executed
    at the end of the picture. Using this option multiple times will
    cause the code to accumulate. This option must also be given in
    the optional argument of the |{tikzpicture}| environment.

\begin{codeexample}[]
\begin{tikzpicture}[execute at end picture=%
  {
    \begin{pgfonlayer}{background}
      \path[fill=yellow,rounded corners]
        (current bounding box.south west) rectangle
        (current bounding box.north east);
    \end{pgfonlayer}
  }]
  \node at (0,0) {X};
  \node at (2,1) {Y};
\end{tikzpicture}
\end{codeexample}
  \end{key}

  All options ``end'' at the end of the picture. To set an option
  ``globally'' change the following style:
  \begin{stylekey}{/tikz/every picture (initially \normalfont empty)}
    This style is installed at the beginning of each picture.
\begin{codeexample}[code only]
\tikzset{every picture/.style=semithick}
\end{codeexample}
  \end{stylekey}

  Note that you should not use |\tikzset| to set options directly. For
  instance, if you want to use a line width of |1pt| by default, do
  not try to say |\tikzset{line width=1pt}| at the beginning of your
  document. This will not work since the line width is changed in many
  places. Instead, say
\begin{codeexample}[code only]
\tikzset{every picture/.style={line width=1pt}}
\end{codeexample}
  This will have the desired effect.
\end{environment}

In other \TeX\ formats, you should use the following commands instead:

\begin{plainenvironment}{{tikzpicture}\opt{\oarg{options}}}
  This is the plain \TeX\ version of the environment.
\end{plainenvironment}

\begin{contextenvironment}{{tikzpicture}\opt{\oarg{options}}}
  This is the Con\TeX t version of the environment.
\end{contextenvironment}


\subsubsection{Creating a Picture Using a Command}

The following command is an alternative to |{tikzpicture}| that is
particular useful for graphics consisting of a single or few
commands.

\begin{command}{\tikz\opt{\oarg{options}}\marg{path commands}}
  This command places the \meta{path commands} inside a
  |{tikzpicture}| environment. The \meta{path commands} may
  contain paragraphs and fragile material (like verbatim text).

  If there is only one path command, it need not be surrounded by
  curly braces, if there are several, you need to add them (this is
  similar to the |\foreach| statement and also to the rules in
  programming languages like Java or C concerning the placement of
  curly braces).

  \example |\tikz{\draw (0,0) rectangle (2ex,1ex);}| yields
  \tikz{\draw (0,0) rectangle (2ex,1ex);}

  \example |\tikz \draw (0,0) rectangle (2ex,1ex);| yields
  \tikz \draw (0,0) rectangle (2ex,1ex);
\end{command}


\subsubsection{Handling Catcodes and the Babel Package}

Inside a \tikzname\ picture, most symbols need to have the category 
code 12 (normal text) in order to ensure that the parser works
properly. This is typically not the case when packages like |babel|
are used, which change catcodes aggressively.

To solve this problem, \tikzname\ provides a small library also called
|babel| (which can, however, also be used together with any other
package that globally changes category codes). What it does is to
reset the category codes at the beginning of every |{tikzpicture}| and
to restore them at the beginning of every node. In almost all cases,
this is exactly would you would expect and need, so I recommend to
always load this library by saying |\usetikzlibrary{babel}|. For
details on what, exactly, happens with the category codes, see
Section~\ref{section-library-babel}.


\subsubsection{Adding a Background}

By default, pictures do not have any background, that is, they are
``transparent'' on all parts on which you do not draw
anything. You may instead wish to have a colored background behind
your picture or a black frame around it or lines above and below it or
some other kind of decoration.

Since backgrounds are often not needed at all, the definition of
styles for adding backgrounds has been put in the library package
|backgrounds|. This package is documented in
Section~\ref{section-tikz-backgrounds}.


\subsection{Using Scopes to Structure a Picture}

Inside a |{tikzpicture}| environment you can create scopes
using the |{scope}| environment. This environment is available only
inside the |{tikzpicture}| environment, so once more, there is little
chance of doing anything wrong.

\subsubsection{The Scope Environment}

\begin{environment}{{scope}\opt{\oarg{options}}}
  All \meta{options} are local to the \meta{environment
  contents}. Furthermore, the clipping path is also local to the
  environment, that is, any clipping done inside the environment
  ``ends'' at its end.

\begin{codeexample}[]
\begin{tikzpicture}[ultra thick]
  \begin{scope}[red]
    \draw (0mm,10mm) -- (10mm,10mm);
    \draw (0mm,8mm) -- (10mm,8mm);
  \end{scope}
  \draw (0mm,6mm) -- (10mm,6mm);
  \begin{scope}[green]
    \draw (0mm,4mm) -- (10mm,4mm);
    \draw (0mm,2mm) -- (10mm,2mm);
    \draw[blue] (0mm,0mm) -- (10mm,0mm);
  \end{scope}
\end{tikzpicture}
\end{codeexample}

  The following style influences scopes:
  \begin{stylekey}{/tikz/every scope (initially \normalfont empty)}
    This style is installed at the beginning of every scope.
  \end{stylekey}

  The following options are useful for scopes:
  \begin{key}{/tikz/execute at begin scope=\meta{code}}
    This option install some code that will be executed
    at the beginning of the scope. This option must be
    given in the argument of the |{scope}| environment.

    The effect applies only to the current scope, not to subscopes.
  \end{key}
  \begin{key}{/tikz/execute at end scope=\meta{code}}
    This option installs some code that will be executed
    at the end of the  current scope. Using this option multiple times
    will  cause the code to accumulate. This option must also be given
    in the optional argument of the |{scope}| environment.

    Again, the effect applies only to the current scope, not to subscopes.
  \end{key}
\end{environment}

\begin{plainenvironment}{{scope}\opt{\oarg{options}}}
  Plain \TeX\ version of the environment.
\end{plainenvironment}

\begin{contextenvironment}{{scope}\opt{\oarg{options}}}
  Con\TeX t version of the environment.
\end{contextenvironment}


\subsubsection{Shorthand for Scope Environments}

There is a small library that makes using scopes a bit easier:
\begin{tikzlibrary}{scopes}
  This library defines a shorthand for starting and ending |{scope}|
  environments.
\end{tikzlibrary}
When this library is loaded, the following happens: At certain places
inside a \tikzname\ picture, it is allowed to start a scope just using
a single brace, provided the single brace is followed by options in
square brackets:

\begin{codeexample}[]
\begin{tikzpicture}
  { [ultra thick]
    { [red]
      \draw (0mm,10mm) -- (10mm,10mm);
      \draw (0mm,8mm)  -- (10mm,8mm);
    }
    \draw (0mm,6mm) -- (10mm,6mm);
  }
  { [green]
    \draw (0mm,4mm) -- (10mm,4mm);
    \draw (0mm,2mm) -- (10mm,2mm);
    \draw[blue] (0mm,0mm) -- (10mm,0mm);
  }
\end{tikzpicture}
\end{codeexample}

In the above example, |{ [thick]| actually causes a
  |\begin{scope}[thick]| to be inserted, and the corresponding closing
    |}| causes an |\end{scope}| to be inserted.

The ``certain places'' where an opening brace has this special meaning
are the following: First, right after the semicolon that ends a path. Second,
right after the end of a scope. Third, right at the beginning of a
scope, which includes the beginning of a picture. Also note that some
square bracket must follow, otherwise the brace is treated as a normal
\TeX\ scope.


\subsubsection{Single Command Scopes}

In some situations it is useful to create a scope for a single
command. For instance, when you wish to use algorithm graph drawing
in order to layout a tree, the path of the tree needs to be surrounded
by a scope whose only purpose is to take a key that selects a layout
for the scope. Similarly, in order to put something on a background
layer, a scope needs to be created. In such cases, where it will 
cumbersome to create a |\begin{scope}| and |\end{scope}| pair just for
a single command, the |\scoped| command may be useful:

\begin{command}{\scoped\opt{\oarg{options}}\meta{path command}}
  This command works like |\tikz|, only you can use it inside a
  |{tikzpicture}|. It will take the following \meta{path command} and
  put it inside a |{scope}| with the \meta{options} set. The
  \meta{path command} may either be a single command ended by a
  semicolon or it may contain multiple commands, but then they must be
  surrounded by curly braces.
  \begin{codeexample}[]
\begin{tikzpicture}
  \node [fill=white] at (1,1) {Hello world};
  \scoped [on background layer]
    \draw (0,0) grid (3,2);
\end{tikzpicture}
  \end{codeexample}
\end{command}


\subsubsection{Using Scopes Inside Paths}

The |\path| command, which is described in much more detail in later
sections, also takes graphic options. These options are local to the
path. Furthermore, it is possible to create local scopes within a
path simply by using curly braces as in
\begin{codeexample}[]
\tikz \draw (0,0) -- (1,1)
           {[rounded corners] -- (2,0) -- (3,1)}
           -- (3,0) -- (2,1);
\end{codeexample}

Note that many options apply only to the path as a whole and cannot be
scoped in this way. For example, it is not possible to scope the
|color| of the path. See the explanations in the section on paths for
more details.

Finally, certain elements that you specify in the argument to the
|\path| command also take local options. For example, a node
specification takes options. In this case, the options apply only to
the node, not to the surrounding path.



\subsection{Using Graphic Options}
\label{section-graphic-options}

\subsubsection{How Graphic Options Are Processed}

Many commands and environments of \tikzname\ accept
\emph{options}. These options are so-called \emph{key lists}. To
process the options, the following command is used, which you can also
call yourself. Note that it is usually better not to call this command
directly, since this will ensure that the effect of options are local
to a well-defined scope.

\begin{command}{\tikzset\marg{options}}
  This command will process the \meta{options} using the |\pgfkeys|
  command, documented in detail in Section~\ref{section-keys}, with
  the default path set to |/tikz|. Under normal circumstances, the
  \meta{options} will be lists of comma-separated pairs of the form
  \meta{key}|=|\meta{value}, but more fancy things can happen when you
  use the power of the |pgfkeys| mechanism, see
  Section~\ref{section-keys} once more.

  When a pair \meta{key}|=|\meta{value} is processed, the following
  happens:
  \begin{enumerate}
  \item If the \meta{key} is a full key (starts with a slash) it is
    handled directly as described in Section~\ref{section-keys}.
  \item Otherwise (which is usually the case), it is checked whether
    |/tikz/|\meta{key} is a key and, if so, it is executed.
  \item Otherwise, it is checked whether |/pgf/|\meta{key} is a key
    and, if so, it is executed.
  \item Otherwise, it is checked whether \meta{key} is a color and, if
    so, |color=|\meta{key} is executed.
  \item Otherwise, it is checked whether \meta{key} contains a dash
    and, if so, |arrows=|\meta{key} is executed.
  \item Otherwise, it is checked whether \meta{key} is the name of a
    shape and, if so, |shape=|\meta{key} is executed.
  \item Otherwise, an error message is printed.
  \end{enumerate}

  Note that by the above description, all keys starting with |/tikz|
  and also all keys starting with |/pgf| can be used as \meta{key}s in
  an \meta{options} list.
\end{command}


\subsubsection{Using Styles to Manage How Pictures Look}

There is a way of organizing sets of graphic options ``orthogonally''
to the normal scoping mechanism. For example, you might wish all your
``help lines'' to be drawn in a certain way like, say, gray and thin
(do \emph{not} dash them, that distracts). For this, you can use
\emph{styles}.

A style is a key that, when used, causes a set of graphic options to
be processed. Once a style has been defined, it can be used like any
other key. For example, the predefined |help lines| style, which you
should use for lines in the background like grid lines or construction
lines.
\begin{codeexample}[]
\begin{tikzpicture}
  \draw             (0,0) grid +(2,2);
  \draw[help lines] (2,0) grid +(2,2);
\end{tikzpicture}
\end{codeexample}

Defining styles is also done using options. Suppose we wish to define
a style called |my style| and when this style is used, we want the
draw color to be set to |red| and the fill color be set to
|red!20|. To achieve this, we use the following option:
\begin{codeexample}[code only]
my style/.style={draw=red,fill=red!20}
\end{codeexample}

The meaning of the curious |/.style| is the following: ``The key
|my style| should not be used here but, rather, be defined. So, set up
things such that using the key |my style| will, in the following, have
the same effect as if we had written |draw=red,fill=red!20| instead.''

Returning to the help lines example, suppose we prefer blue help
lines. This could be achieved as follows:
\begin{codeexample}[]
\begin{tikzpicture}[help lines/.style={blue!50,very thin}]
  \draw             (0,0) grid +(2,2);
  \draw[help lines] (2,0) grid +(2,2);
\end{tikzpicture}
\end{codeexample}

Naturally, one of the main ideas behind styles is that they can be
used in different pictures. In this case, we have to use the
|\tikzset| command somewhere at the beginning.
\begin{codeexample}[]
\tikzset{help lines/.style={blue!50,very thin}}
% ...
\begin{tikzpicture}
  \draw             (0,0) grid +(2,2);
  \draw[help lines] (2,0) grid +(2,2);
\end{tikzpicture}
\end{codeexample}

Since styles are just special cases of |pgfkeys|'s general style
facility, you can actually do quite a bit more. Let us start with
adding options to an already existing style. This is done using
|/.append style| instead of |/.style|:
\begin{codeexample}[]
\begin{tikzpicture}[help lines/.append style=blue!50]
  \draw             (0,0) grid +(2,2);
  \draw[help lines] (2,0) grid +(2,2);
\end{tikzpicture}
\end{codeexample}
In the above example, the option |blue!50| is appended to the style
|help lines|, which now has the same effect as
|black!50,very thin,blue!50|. Note that two colors are set, so the
last one will ``win.'' There also exists a handler called
|/.prefix style| that adds something at the beginning of the style.

Just as normal keys, styles can be parameterized. This means that you
write \meta{style}|=|\meta{value} when you use the style instead of
just \meta{style}. In this case, all occurrences of |#1| in
\meta{style} are replaced by \meta{value}. Here is an example that
shows how this can be used.

\begin{codeexample}[]
\begin{tikzpicture}[outline/.style={draw=#1,thick,fill=#1!50}]
  \node [outline=red]  at (0,1) {red};
  \node [outline=blue] at (0,0) {blue};
\end{tikzpicture}
\end{codeexample}

For parameterized styles you can also set a \emph{default} value using
the |/.default| handler:

\begin{codeexample}[]
\begin{tikzpicture}[outline/.style={draw=#1,thick,fill=#1!50},
                    outline/.default=black]
  \node [outline]      at (0,1) {default};
  \node [outline=blue] at (0,0) {blue};
\end{tikzpicture}
\end{codeexample}

For more details on using and setting styles, see also
Section~\ref{section-keys}.

% % Copyright 2006 by Till Tantau
%
% This file may be distributed and/or modified
%
% 1. under the LaTeX Project Public License and/or
% 2. under the GNU Free Documentation License.
%
% See the file doc/generic/pgf/licenses/LICENSE for more details.


\section{Specifying Coordinates}

\subsection{Overview}

A \emph{coordinate} is a position on the canvas on which your picture is drawn.
\tikzname\ uses a special syntax for specifying coordinates. Coordinates are
always put in round brackets. The general syntax is
\declare{|(|\opt{|[|\meta{options}|]|}\meta{coordinate  specification}|)|}.

The \meta{coordinate specification} specifies coordinates using one of many
different possible \emph{coordinate systems}. Examples are the Cartesian
coordinate system or polar coordinates or spherical coordinates. No matter
which coordinate system is used, in the end, a specific point on the canvas is
represented by the coordinate.

There are two ways of specifying which coordinate system should be used:
%
\begin{description}
    \item[Explicitly] You can specify the coordinate system explicitly. To do
        so, you give the name of the coordinate system at the beginning,
        followed by |cs:|, which stands for ``coordinate system'', followed by
        a specification of the coordinate using the key--value syntax. Thus,
        the general syntax for \meta{coordinate specification} in the explicit
        case is |(|\meta{coordinate system}| cs:|\meta{list of key--value pairs
        specific to the coordinate system}|)|.
    \item[Implicitly] The explicit specification is often too verbose when
        numerous coordinates should be given. Because of this, for the
        coordinate systems that you are likely to use often a special syntax
        is provided. \tikzname\ will notice when you use a coordinate
        specified in a special syntax and will choose the correct coordinate
        system automatically.
\end{description}

Here is an example in which explicit the coordinate systems are specified
explicitly:
%
\begin{codeexample}[]
\begin{tikzpicture}
  \draw[help lines] (0,0) grid (3,2);
  \draw (canvas cs:x=0cm,y=2mm)
     -- (canvas polar cs:radius=2cm,angle=30);
\end{tikzpicture}
\end{codeexample}
%
In the next example, the coordinate systems are implicit:
%
\begin{codeexample}[]
\begin{tikzpicture}
  \draw[help lines] (0,0) grid (3,2);
  \draw (0cm,2mm) -- (30:2cm);
\end{tikzpicture}
\end{codeexample}

It is possible to give options that apply only to a single coordinate, although
this makes sense for transformation options only. To give transformation
options for a single coordinate, give these options at the beginning in
brackets:
%
\begin{codeexample}[]
\begin{tikzpicture}
  \draw[help lines] (0,0) grid (3,2);
  \draw      (0,0) -- (1,1);
  \draw[red] (0,0) -- ([xshift=3pt] 1,1);
  \draw      (1,0) -- +(30:2cm);
  \draw[red] (1,0) -- +([shift=(135:5pt)] 30:2cm);
\end{tikzpicture}
\end{codeexample}


\subsection{Coordinate Systems}

\subsubsection{Canvas, XYZ, and Polar Coordinate Systems}

Let us start with the basic coordinate systems.

\begin{coordinatesystem}{canvas}
    The simplest way of specifying a coordinate is to use the |canvas|
    coordinate system. You provide a dimension $d_x$ using the |x=| option and
    another dimension $d_y$ using the |y=| option. The position on the canvas
    is located at the position that is $d_x$ to the right and $d_y$ above the
    origin.

    \begin{key}{/tikz/cs/x=\meta{dimension} (initially 0pt)}
        Distance by which the coordinate is to the right of the origin. You can
        also write things like |1cm+2pt| since the mathematical engine is used
        to evaluate the \meta{dimension}.
    \end{key}

    \begin{key}{/tikz/cs/y=\meta{dimension} (initially 0pt)}
        Distance by which the coordinate is above the origin.
    \end{key}

\begin{codeexample}[]
\begin{tikzpicture}
  \draw[help lines] (0,0) grid (3,2);

  \fill (canvas cs:x=1cm,y=1.5cm)    circle (2pt);
  \fill (canvas cs:x=2cm,y=-5mm+2pt) circle (2pt);
\end{tikzpicture}
\end{codeexample}

    To specify a coordinate in the coordinate system implicitly, you use two
    dimensions that are separated by a comma as in |(0cm,3pt)| or
    |(2cm,\textheight)|.
    %
\begin{codeexample}[]
\begin{tikzpicture}
  \draw[help lines] (0,0) grid (3,2);

  \fill (1cm,1.5cm)    circle (2pt);
  \fill (2cm,-5mm+2pt) circle (2pt);
\end{tikzpicture}
\end{codeexample}
    %
\end{coordinatesystem}

\begin{coordinatesystem}{xyz}
    The |xyz| coordinate system allows you to specify a point as a multiple of
    three vectors called the $x$-, $y$-, and $z$-vectors. By default, the
    $x$-vector points 1cm to the right, the $y$-vector points 1cm upwards, but
    this can be changed arbitrarily as explained in Section~\ref{section-xyz}.
    The default $z$-vector points to
    $\bigl(-3.85\textrm{mm},-3.85\textrm{mm}\bigr)$.

    To specify the factors by which the vectors should be multiplied before
    being added, you use the following three options:
    %
    \begin{key}{/tikz/cs/x=\meta{factor} (initially 0)}
        Factor by which the $x$-vector is multiplied.
    \end{key}
    %
    \begin{key}{/tikz/cs/y=\meta{factor} (initially 0)}
        Works like |x|.
    \end{key}
    %
    \begin{key}{/tikz/cs/z=\meta{factor} (initially 0)}
        Works like |x|.
    \end{key}

\begin{codeexample}[]
\begin{tikzpicture}[->]
  \draw (0,0) -- (xyz cs:x=1);
  \draw (0,0) -- (xyz cs:y=1);
  \draw (0,0) -- (xyz cs:z=1);
\end{tikzpicture}
\end{codeexample}

    This coordinate system can also be selected implicitly. To do so, you just
    provide two or three comma-separated factors (not dimensions).
    %
\begin{codeexample}[]
\begin{tikzpicture}[->]
  \draw (0,0) -- (1,0);
  \draw (0,0) -- (0,1,0);
  \draw (0,0) -- (0,0,1);
\end{tikzpicture}
\end{codeexample}
    %
\end{coordinatesystem}

\emph{Note:} It is possible to use coordinates like |(1,2cm)|, which are
neither |canvas| coordinates nor |xyz| coordinates. The rule is the following:
If a coordinate is of the implicit form |(|\meta{x}|,|\meta{y}|)|, then
\meta{x} and \meta{y} are checked, independently, whether they have a dimension
or whether they are dimensionless. If both have a dimension, the |canvas|
coordinate system is used. If both lack a dimension, the |xyz| coordinate
system is used. If \meta{x} has a dimension and \meta{y} has not, then the sum
of two coordinate |(|\meta{x}|,0pt)| and |(0,|\meta{y}|)| is used. If \meta{y}
has a dimension and \meta{x} has not, then the sum of two coordinate
|(|\meta{x}|,0)| and |(0pt,|\meta{y}|)| is used.

\emph{Note furthermore:} An expression like |(2+3cm,0)| does \emph{not} mean
the same as |(2cm+3cm,0)|. Instead, if \meta{x} or \meta{y} internally uses a
mixture of dimensions and dimensionless values, then all dimensionless values
are ``upgraded'' to dimensions by interpreting them as |pt|. So, |2+3cm| is the
same dimension as |2pt+3cm|.

\begin{coordinatesystem}{canvas polar}
    The |canvas polar| coordinate system allows you to specify polar
    coordinates. You provide an angle using the |angle=| option and a radius
    using the |radius=| option. This yields the point on the canvas that is at
    the given radius distance from the origin at the given degree. An angle of
    zero degrees to the right, a degree of 90 upward.
    %
    \begin{key}{/tikz/cs/angle=\meta{degrees}}
        The angle of the coordinate. The angle must always be given in degrees
        and should be between $-360$ and $720$.
    \end{key}
    %
    \begin{key}{/tikz/cs/radius=\meta{dimension}}
        The distance from the origin.
    \end{key}
    %
    \begin{key}{/tikz/cs/x radius=\meta{dimension}}
        A polar coordinate is, after all, just a point on a circle of the given
        \meta{radius}. When you provide an $x$-radius and also a $y$-radius,
        you specify an ellipse instead of a circle. The |radius| option has the
        same effect as specifying identical |x radius| and |y radius| options.
    \end{key}
    %
    \begin{key}{/tikz/cs/y radius=\meta{dimension}}
        Works like |x radius|.
    \end{key}
    %
\begin{codeexample}[]
\tikz \draw (0,0) -- (canvas polar cs:angle=30,radius=1cm);
\end{codeexample}

    The implicit form for canvas polar coordinates is the following: you
    specify the angle and the distance, separated by a colon as in |(30:1cm)|.
    %
\begin{codeexample}[]
\tikz \draw    (0cm,0cm) -- (30:1cm) -- (60:1cm) -- (90:1cm)
            -- (120:1cm) -- (150:1cm) -- (180:1cm);
\end{codeexample}

    Two different radii are specified by writing |(30:1cm and 2cm)|.

    For the implicit form, instead of an angle given as a number you can also
    use certain words. For example, |up| is the same as |90|, so that you can
    write |\tikz \draw (0,0) -- (2ex,0pt) -- +(up:1ex);| and get
    \tikz \draw (0,0) -- (2ex,0pt) -- +(up:1ex);. Apart from |up| you can use
    |down|, |left|, |right|, |north|, |south|, |west|, |east|, |north east|,
    |north west|, |south east|, |south west|, all of which have their natural
    meaning.
\end{coordinatesystem}

\begin{coordinatesystem}{xyz polar}
    This coordinate system work similarly to the |canvas polar| system.
    However, the radius and the angle are interpreted in the $xy$-coordinate
    system, not in the canvas system. More detailed, consider the circle or
    ellipse whose half axes are given by the current $x$-vector and the current
    $y$-vector. Then, consider the point that lies at a given angle on this
    ellipse, where an angle of zero is the same as the $x$-vector and an angle
    of 90 is the $y$-vector. Finally, multiply the resulting vector by the
    given radius factor. Voil\`a.
    %
    \begin{key}{/tikz/cs/angle=\meta{degrees}}
        The angle of the coordinate interpreted in the ellipse whose axes are
        the $x$-vector and the $y$-vector.
    \end{key}
    %
    \begin{key}{/tikz/cs/radius=\meta{factor}}
        A factor by which the $x$-vector and $y$-vector are multiplied prior to
        forming the ellipse.
    \end{key}
    %
    \begin{key}{/tikz/cs/x radius=\meta{dimension}}
        A specific factor by which only the $x$-vector is multiplied.
    \end{key}
    %
    \begin{key}{/tikz/cs/y radius=\meta{dimension}}
        Works like |x radius|.
    \end{key}
    %
\begin{codeexample}[]
\begin{tikzpicture}[x=1.5cm,y=1cm]
  \draw[help lines] (0cm,0cm) grid (3cm,2cm);

  \draw (0,0) -- (xyz polar cs:angle=0,radius=1);
  \draw (0,0) -- (xyz polar cs:angle=30,radius=1);
  \draw (0,0) -- (xyz polar cs:angle=60,radius=1);
  \draw (0,0) -- (xyz polar cs:angle=90,radius=1);

  \draw (xyz polar cs:angle=0,radius=2)
     -- (xyz polar cs:angle=30,radius=2)
     -- (xyz polar cs:angle=60,radius=2)
     -- (xyz polar cs:angle=90,radius=2);
 \end{tikzpicture}
\end{codeexample}

    The implicit version of this option is the same as the implicit version of
    |canvas polar|, only you do not provide a unit.

\begin{codeexample}[]
\tikz[x={(0cm,1cm)},y={(-1cm,0cm)}]
  \draw  (0,0) -- (30:1) -- (60:1) -- (90:1)
             -- (120:1) -- (150:1) -- (180:1);
\end{codeexample}
    %
\end{coordinatesystem}

\begin{coordinatesystem}{xy polar}
    This is just an alias for |xyz polar|, which some people might prefer as
    there is no z-coordinate involved in the |xyz polar| coordinates.
\end{coordinatesystem}


\subsubsection{Barycentric Systems}
\label{section-barycentric-coordinates}

In the barycentric coordinate system a point is expressed as the linear
combination of multiple vectors. The idea is that you specify vectors $v_1$,
$v_2$, \dots, $v_n$ and numbers $\alpha_1$, $\alpha_2$, \dots, $\alpha_n$. Then
the barycentric coordinate specified by these vectors and numbers is
%
\begin{align*}
    \frac{\alpha_1 v_1 + \alpha_2 v_2 + \cdots + \alpha_n v_n}{\alpha_1
        + \alpha_2 + \cdots + \alpha_n}
\end{align*}

The |barycentric cs| allows you to specify such coordinates easily.

\begin{coordinatesystem}{barycentric}
    For this coordinate system, the \meta{coordinate specification} should be a
    comma-separated list of expressions of the form \meta{node
    name}|=|\meta{number}. Note that (currently) the list should not contain
    any spaces before or after the \meta{node name} (unlike normal key--value
    pairs).

    The specified coordinate is now computed as follows: Each pair provides one
    vector and a number. The vector is the |center| anchor of the \meta{node
    name}. The number is the \meta{number}. Note that (currently) you cannot
    specify a different anchor, so that in order to use, say, the |north|
    anchor of a node you first have to create a new coordinate at this north
    anchor. (Using for instance \texttt{\string\coordinate (mynorth) at
    (mynode.north);}.)
    %
\begin{codeexample}[]
\begin{tikzpicture}
  \coordinate (content)   at (90:3cm);
  \coordinate (structure) at (210:3cm);
  \coordinate (form)      at (-30:3cm);

  \node [above]       at (content)   {content oriented};
  \node [below left]  at (structure) {structure oriented};
  \node [below right] at (form)      {form oriented};

  \draw [thick,gray] (content.south) -- (structure.north east) -- (form.north west) -- cycle;

  \small
  \node at (barycentric cs:content=0.5,structure=0.1 ,form=1)    {PostScript};
  \node at (barycentric cs:content=1  ,structure=0   ,form=0.4)  {DVI};
  \node at (barycentric cs:content=0.5,structure=0.5 ,form=1)    {PDF};
  \node at (barycentric cs:content=0  ,structure=0.25,form=1)    {CSS};
  \node at (barycentric cs:content=0.5,structure=1   ,form=0)    {XML};
  \node at (barycentric cs:content=0.5,structure=1   ,form=0.4)  {HTML};
  \node at (barycentric cs:content=1  ,structure=0.2 ,form=0.8)  {\TeX};
  \node at (barycentric cs:content=1  ,structure=0.6 ,form=0.8)  {\LaTeX};
  \node at (barycentric cs:content=0.8,structure=0.8 ,form=1)    {Word};
  \node at (barycentric cs:content=1  ,structure=0.05,form=0.05) {ASCII};
\end{tikzpicture}
\end{codeexample}
    %
\end{coordinatesystem}


\subsubsection{Node Coordinate System}
\label{section-node-coordinates}

In \pgfname\ and in \tikzname\ it is quite easy to define a node that you wish
to reference at a later point. Once you have defined a node, there are
different ways of referencing points of the node. To do so, you use the
following coordinate system:

\begin{coordinatesystem}{node}
    This coordinate system is used to reference a specific point inside or on
    the border of a previously defined node. It can be used in different ways,
    so let us go over them one by one.

    You can use three options to specify which coordinate you mean:
    %
    \begin{key}{/tikz/cs/name=\meta{node name}}
        Specifies the node that you wish to use to specify a coordinate. The
        \meta{node name} is the name that was previously used to name the node
        using the |name=|\meta{node name} option or the special node name
        syntax.
    \end{key}
    %
    \begin{key}{/tikz/anchor=\meta{anchor}}
        Specifies an anchor of the node. Here is an example:
        %
\begin{codeexample}[preamble={\usetikzlibrary{arrows.meta}}]
\begin{tikzpicture}
  \node (shape)   at (0,2)  [draw] {|class Shape|};
  \node (rect)    at (-2,0) [draw] {|class Rectangle|};
  \node (circle)  at (2,0)  [draw] {|class Circle|};
  \node (ellipse) at (6,0)  [draw] {|class Ellipse|};

  \draw (node cs:name=circle,anchor=north) |- (0,1);
  \draw (node cs:name=ellipse,anchor=north) |- (0,1);
  \draw [arrows = -{Triangle[open, angle=60:3mm]}]
           (node cs:name=rect,anchor=north)
        |- (0,1) -| (node cs:name=shape,anchor=south);
\end{tikzpicture}
\end{codeexample}
    \end{key}
    %
    \begin{key}{/tikz/cs/angle=\meta{degrees}}
        It is also possible to provide an angle \emph{instead} of an anchor.
        This coordinate refers to a point of the node's border where a ray shot
        from the center in the given angle hits the border. Here is an example:
        %
\begin{codeexample}[preamble={\usetikzlibrary{shapes.geometric}}]
\begin{tikzpicture}
  \node (start) [draw,shape=ellipse] {start};
  \foreach \angle in {-90, -80, ..., 90}
    \draw (node cs:name=start,angle=\angle)
      .. controls +(\angle:1cm) and +(-1,0) .. (2.5,0);
  \end{tikzpicture}
\end{codeexample}
    \end{key}

    It is possible to provide \emph{neither} the |anchor=| option nor the
    |angle=| option. In this case, \tikzname\ will calculate an appropriate
    border position for you. Here is an example:
    %
\begin{codeexample}[preamble={\usetikzlibrary{shapes.geometric}}]
\begin{tikzpicture}
  \path (0,0)  node(a) [ellipse,rotate=10,draw] {An ellipse}
        (3,-1) node(b) [circle,draw]            {A circle};
  \draw[thick] (node cs:name=a) -- (node cs:name=b);
\end{tikzpicture}
\end{codeexample}

    \tikzname\ will be reasonably clever at determining the border points that
    you ``mean'', but, naturally, this may fail in some situations. If
    \tikzname\ fails to determine an appropriate border point, the center will
    be used instead.

    Automatic computation of anchors works only with the line-to operations
    |--|, the vertical/horizontal versions \verb!|-! and \verb!-|!, and with
    the curve-to operation |..|. For other path commands, such as |parabola| or
    |plot|, the center will be used. If this is not desired, you should give a
    named anchor or an angle anchor.

    Note that if you use an automatic coordinate for both the start and the end
    of a line-to, as in |--(node cs:name=b)--|, then \emph{two} border
    coordinates are computed with a move-to between them. This is usually
    exactly what you want.

    If you use relative coordinates together with automatic anchor coordinates,
    the relative coordinates are computed relative to the node's center, not
    relative to the border point. Here is an example:
    %
\begin{codeexample}[]
\tikz \draw (0,0) node(x) [draw] {Text}
            rectangle (1,1)
            (node cs:name=x) -- +(1,1);
\end{codeexample}

    Similarly, in the following examples both control points are $(1,1)$:
    %
\begin{codeexample}[]
\tikz \draw (0,0) node(x) [draw] {X}
            (2,0) node(y) {Y}
            (node cs:name=x) .. controls +(1,1) and +(-1,1) ..
            (node cs:name=y);
\end{codeexample}

    The implicit way of specifying the node coordinate system is to simply use
    the name of the node in parentheses as in |(a)| or to specify a name
    together with an anchor or an angle separated by a dot as in |(a.north)| or
    |(a.10)|.

    Here is a more complete example:
    %
\begin{codeexample}[preamble={\usetikzlibrary{shapes.geometric}}]
\begin{tikzpicture}[fill=blue!20]
  \draw[help lines] (-1,-2) grid (6,3);
  \path (0,0)  node(a) [ellipse,rotate=10,draw,fill]    {An ellipse}
        (3,-1) node(b) [circle,draw,fill]               {A circle}
        (2,2)  node(c) [rectangle,rotate=20,draw,fill]  {A rectangle}
        (5,2)  node(d) [rectangle,rotate=-30,draw,fill] {Another rectangle};
  \draw[thick] (a.south) -- (b) -- (c) -- (d);
  \draw[thick,red,->] (a) |- +(1,3) -| (c) |- (b);
  \draw[thick,blue,<->] (b) .. controls +(right:2cm) and +(down:1cm) .. (d);
\end{tikzpicture}
\end{codeexample}
    %
\end{coordinatesystem}

% -----------------------------------------------------------------------------
% Deprecated:
% -----------------------------------------------------------------------------
%
% \subsubsection{Intersection Coordinate Systems}
%
% Often you wish to specify a point that is on the
% intersection of two lines or shapes. For this, the following
% coordinate system is useful:
%
% \begin{coordinatesystem}{intersection}
%   First, you must specify two objects that should be
%   intersected. These ``objects'' can either be lines or the shapes of
%   nodes. There are two option to specify the first object:
%   \begin{key}{/tikz/cs/first line={\ttfamily\char`\{}|(|\meta{first
%           coordinate}|)--(|\meta{second coordinate}|)|{\ttfamily\char`\}}}
%     Specifies that the first object is a line that goes from
%     \meta{first coordinate} to meta{second coordinate}.
%   \end{key}
%   Note that you have to write |--| between the coordinate, but this
%   does not mean that anything is added to the path. This is simply a
%   special syntax.
%   \begin{key}{/tikz/cs/first node=\meta{node}}
%     Specifies that the first object is a previously defined node named
%     \meta{node}.
%   \end{key}
%
%   To specify the second object, you use one of the following keys:
%   \begin{key}{/tikz/cs/second line={\ttfamily\char`\{}|(|\meta{first
%           coordinate}|)--(|\meta{second coordinate}|)|{\ttfamily\char`\}}}
%     As above.
%   \end{key}
%   \begin{key}{/tikz/cs/second node=\meta{node}}
%     Specifies that the second object is a previously defined node
%     named \meta{node}.
%   \end{key}
%
%   Since it is possible that two objects have multiple intersections,
%   you may need to specify which solution you want:
%   \begin{key}{/tikz/cs/solution=\meta{number} (initially 1)}
%     Specifies which solution should be used. Numbering starts with 1.
%   \end{key}
%   The coordinate specified in this way is the \meta{number}th
%   intersection of the two objects.  If the objects do not intersect,
%   an error may occur.
%
% \begin{codeexample}[]
% \begin{tikzpicture}
%   \draw[help lines] (0,0) grid (3,2);
%   \draw (0,0) coordinate (A) -- (3,2) coordinate (B)
%         (1,2)                -- (3,0);
%
%   \fill[red] (intersection cs:
%     first line={(A)--(B)},
%     second line={(1,2)--(3,0)}) circle (2pt);
% \end{tikzpicture}
% \end{codeexample}
%
%   The implicit way of specifying this coordinate system is to write
%   \declare{|(intersection |\opt{\meta{number}}| of |\meta{first
%       object}%
%     | and |\meta{second object}|)|}. Here, \meta{first object} either
%   has the form \meta{$p_1$}|--|\meta{$p_2$} or it is just a node
%   name. Likewise for \meta{second object}. Note that there are \emph{no}
%   parentheses around the $p_i$. Thus, you would write
%   |(intersection of A--B and 1,2--3,0)|  for the intersection of the
%   line through the coordinates |A| and |B| and the line through the
%   points $(1,2)$ and $(3,0)$. You would write
%   |(intersection 2 of c_1 and c_2)| for the second
%   intersection of the node named |c_1| and the node named
%   |c_2|.
%
%   \tikzname\ needs an explicit algorithm for computing the
%   intersection of two shapes and such an algorithm is available only
%   for few shapes. Currently, the following intersection will be
%   computed correctly:
%   \begin{itemize}
%   \item a line and a line
%   \item a |circle| node and a line (in any order)
%   \item a |circle| and a |circle|
%   \end{itemize}
% \begin{codeexample}[]
% \begin{tikzpicture}[scale=.25]
%   \coordinate [label=-135:$a$] (a) at ($ (0,0)   + (rand,rand) $);
%   \coordinate [label=45:$b$]   (b) at ($ (3,2) + (rand,rand) $);
%
%   \coordinate [label=-135:$u$] (u) at (-1,1);
%   \coordinate [label=45:$v$]   (v) at (6,0);
%
%   \draw (a) -- (b)
%         (u) -- (v);
%
%   \node (c1) at (a) [draw,circle through=(b)] {};
%   \node (c2) at (b) [draw,circle through=(a)] {};
%
%   \coordinate [label=135:$c$] (c) at (intersection 2 of c1 and c2);
%   \coordinate [label=-45:$d$] (d) at (intersection of u--v and c2);
%   \coordinate [label=135:$e$] (e) at (intersection of u--v and a--b);
%
%   \foreach \p in {a,b,c,d,e,u,v}
%     \fill [opacity=.5] (\p) circle (8pt);
% \end{tikzpicture}
% \end{codeexample}
% \end{coordinatesystem}
% -----------------------------------------------------------------------------


\subsubsection{Tangent Coordinate Systems}

\begin{coordinatesystem}{tangent}
    This coordinate system, which is available only when the \tikzname\ library
    |calc| is loaded, allows you to compute the point that lies tangent to a
    shape. In detail, consider a \meta{node} and a \meta{point}. Now, draw a
    straight line from the \meta{point} so that it ``touches'' the \meta{node}
    (more formally, so that it is \emph{tangent} to this \meta{node}). The
    point where the line touches the shape is the point referred to by the
    |tangent| coordinate system.

    The following options may be given:
    %
    \begin{key}{/tikz/cs/node=\meta{node}}
        This key specifies the node on whose border the tangent should lie.
    \end{key}
    %
    \begin{key}{/tikz/cs/point=\meta{point}}
        This key specifies the point through which the tangent should go.
    \end{key}
    %
    \begin{key}{/tikz/cs/solution=\meta{number}}
        Specifies which solution should be used if there are more than one.
    \end{key}

    A special algorithm is needed in order to compute the tangent for a given
    shape. Currently, tangents can be computed for nodes whose shape is one of
    the following:
    %
    \begin{itemize}
        \item |coordinate|
        \item |circle|
    \end{itemize}
    %
\begin{codeexample}[preamble={\usetikzlibrary{calc}}]
\begin{tikzpicture}
  \draw[help lines] (0,0) grid (3,2);

  \coordinate (a) at (3,2);

  \node [circle,draw] (c) at (1,1) [minimum size=40pt] {$c$};

  \draw[red] (a)  -- (tangent cs:node=c,point={(a)},solution=1) --
       (c.center) -- (tangent cs:node=c,point={(a)},solution=2) -- cycle;
\end{tikzpicture}
\end{codeexample}

    There is no implicit syntax for this coordinate system.
\end{coordinatesystem}


\subsubsection{Defining New Coordinate Systems}

While the set of coordinate systems that \tikzname\ can parse via their special
syntax is fixed, it is possible and quite easy to define new explicitly named
coordinate systems. For this, the following commands are used:

\begin{command}{\tikzdeclarecoordinatesystem\marg{name}\marg{code}}
    This command declares a new coordinate system named \meta{name} that can
    later on be used by writing |(|\meta{name}| cs:|\meta{arguments}|)|. When
    \tikzname\ encounters a coordinate specified in this way, the
    \meta{arguments} are passed to \meta{code} as argument |#1|.

    It is now the job of \meta{code} to make sense of the \meta{arguments}. At
    the end of \meta{code}, the two \TeX\ dimensions |\pgf@x| and |\pgf@y|
    should be have the $x$- and $y$-canvas coordinate of the coordinate.

    It is not necessary, but customary, to parse \meta{arguments} using the
    key--value syntax. However, you can also parse it in any way you like.

    In the following example, a coordinate system |cylindrical| is defined.
    %
\begin{codeexample}[]
\makeatletter
\define@key{cylindricalkeys}{angle}{\def\myangle{#1}}
\define@key{cylindricalkeys}{radius}{\def\myradius{#1}}
\define@key{cylindricalkeys}{z}{\def\myz{#1}}
\tikzdeclarecoordinatesystem{cylindrical}%
{%
  \setkeys{cylindricalkeys}{#1}%
  \pgfpointadd{\pgfpointxyz{0}{0}{\myz}}{\pgfpointpolarxy{\myangle}{\myradius}}
}
\begin{tikzpicture}[z=0.2pt]
  \draw [->] (0,0,0) -- (0,0,350);
  \foreach \num in {0,10,...,350}
    \fill (cylindrical cs:angle=\num,radius=1,z=\num) circle (1pt);
\end{tikzpicture}
\end{codeexample}
    %
\end{command}

\begin{command}{\tikzaliascoordinatesystem\marg{new name}\marg{old name}}
    Creates an alias of \meta{old name}.
\end{command}


\subsection{Coordinates at Intersections}
\label{section-intersection-coordinates}

You will wish to compute the intersection of two paths. For the special and
frequent case of two perpendicular lines, a special coordinate system called
|perpendicular| is available. For more general cases, the |intersection|
library can be used.


\subsubsection{Intersections of Perpendicular Lines}

A frequent special case of path intersections is the intersection of a vertical
line going through a point $p$ and a horizontal line going through some other
point $q$. For this situation there is a useful coordinate system.

\begin{coordinatesystem}{perpendicular}
    You can specify the two lines using the following keys:

    \begin{key}{/tikz/cs/horizontal line through={\ttfamily\char`\{}|(|\meta{coordinate}|)|{\ttfamily\char`\}}}
        Specifies that one line is a horizontal line that goes through the
        given coordinate.
    \end{key}
    %
    \begin{key}{/tikz/cs/vertical line through={\ttfamily\char`\{}|(|\meta{coordinate}|)|{\ttfamily\char`\}}}
        Specifies that the other line is vertical and goes through the given
        coordinate.
    \end{key}

    However, in almost all cases you should, instead, use the implicit syntax.
    Here, you write \declare{|(|\meta{p}\verb! |- !\meta{q}|)|} or
    \declare{|(|\meta{q}\verb! -| !\meta{p}|)|}.

    For example, \verb!(2,1 |- 3,4)! and  \verb!(3,4 -| 2,1)! both yield the
    same as \verb!(2,4)! (provided the $xy$-co\-or\-di\-nate system has not
    been modified).

    The most useful application of the syntax is to draw a line up to some
    point on a vertical or horizontal line. Here is an example:
    %
\begin{codeexample}[]
\begin{tikzpicture}
  \path (30:1cm) node(p1) {$p_1$}   (75:1cm) node(p2) {$p_2$};

  \draw (-0.2,0) -- (1.2,0) node(xline)[right] {$q_1$};
  \draw (2,-0.2) -- (2,1.2) node(yline)[above] {$q_2$};

  \draw[->] (p1) -- (p1 |- xline);
  \draw[->] (p2) -- (p2 |- xline);
  \draw[->] (p1) -- (p1 -| yline);
  \draw[->] (p2) -- (p2 -| yline);
\end{tikzpicture}
\end{codeexample}

    Note that in \declare{|(|\meta{c}\verb! |- !\meta{d}|)|} the coordinates
    \meta{c} and \meta{d} are \emph{not} surrounded by parentheses. If they
    need to be complicated expressions (like a computation using the
    |$|-syntax), you must surround them with braces; parentheses will then be   %$
    added around them.

    As an example, let us specify a point that lies horizontally at the middle
    of the line from $A$ to~$B$ and vertically at the middle of the line from
    $C$ to~$D$:
    %
\begin{codeexample}[preamble={\usetikzlibrary{calc}}]
\begin{tikzpicture}
  \node (A) at (0,1)    {A};
  \node (B) at (1,1.5)  {B};
  \node (C) at (2,0)    {C};
  \node (D) at (2.5,-2) {D};

  \draw (A) -- (B) node [midway] {x};
  \draw (C) -- (D) node [midway] {x};

  \node at ({$(A)!.5!(B)$} -| {$(C)!.5!(D)$}) {X};
\end{tikzpicture}
\end{codeexample}
    %
\end{coordinatesystem}


\subsubsection{Intersections of Arbitrary Paths}

\begin{tikzlibrary}{intersections}
    This library enables the calculation of intersections of two arbitrary
    paths. However, due to the low accuracy of \TeX, the paths should not be
    ``too complicated''. In particular, you should not try to intersect paths
    consisting of lots of very small segments such as plots or decorated paths.
\end{tikzlibrary}

To find the intersections of two paths in \tikzname, they must be ``named''. A
``named path'' is, quite simply, a path that has been named using the following
key (note that this is a \emph{different} key from the |name| key, which only
attaches a hyperlink target to a path, but does not store the path in a way the
is useful for the intersection computation):

\begin{keylist}{%
    /tikz/name path=\meta{name},
    /tikz/name path global=\meta{name}%
}
    The effect of this key is that, after the path has been constructed, just
    before it is used, it is associated with \meta{name}. For |name path|, this
    association survives beyond the final semi-colon of the path but not the
    end of the surrounding scope. For |name path global|, the association will
    survive beyond any scope as well. Handle with care.

    Any paths created by nodes on the (main) path are ignored, unless this key
    is explicitly used. If the same \meta{name} is used for the main path and
    the node path(s), then the paths will be added together and then associated
    with \meta{name}.
\end{keylist}

To find the intersection of named paths, the following key is used:

\begin{key}{/tikz/name intersections=\marg{options}}
    This key changes the key path to |/tikz/intersection| and processes
    \meta{options}. These options determine, among other things, which paths to
    use for the intersection. Having processed the options, any intersections
    are then found. A coordinate is created at each intersection, which by
    default, will be named |intersection-1|, |intersection-2|, and so on.
    Optionally, the prefix |intersection| can be changed, and the total number
    of intersections stored in a \TeX-macro.
    %
\begin{codeexample}[preamble={\usetikzlibrary{intersections}}]
\begin{tikzpicture}[every node/.style={opacity=1, black, above left}]
  \draw [help lines] grid (3,2);
  \draw [name path=ellipse] (2,0.5) ellipse (0.75cm and 1cm);
  \draw [name path=rectangle, rotate=10] (0.5,0.5) rectangle +(2,1);
  \fill [red, opacity=0.5, name intersections={of=ellipse and rectangle}]
    (intersection-1) circle (2pt) node {1}
    (intersection-2) circle (2pt) node {2};
\end{tikzpicture}
\end{codeexample}

    The following keys can be used in \meta{options}:

    \begin{key}{/tikz/intersection/of=\meta{name path 1}| and |\meta{name path 2}}
        This key is used to specify the names of the paths to use for the
        intersection.
    \end{key}

    \begin{key}{/tikz/intersection/name=\meta{prefix} (initially intersection)}
        This key specifies the prefix name for the coordinate nodes placed at
        each intersection.
    \end{key}

    \begin{key}{/tikz/intersection/total=\meta{macro}}
        This key means that the total number of intersections found will be
        stored in \meta{macro}.
    \end{key}

\begin{codeexample}[preamble={\usetikzlibrary{intersections}}]
\begin{tikzpicture}
  \clip (-2,-2) rectangle (2,2);
  \draw [name path=curve 1] (-2,-1) .. controls (8,-1) and (-8,1) .. (2,1);
  \draw [name path=curve 2] (-1,-2) .. controls (-1,8) and (1,-8) .. (1,2);

  \fill [name intersections={of=curve 1 and curve 2, name=i, total=\t}]
        [red, opacity=0.5, every node/.style={above left, black, opacity=1}]
        \foreach \s in {1,...,\t}{(i-\s) circle (2pt) node {\footnotesize\s}};
\end{tikzpicture}
\end{codeexample}

    \begin{key}{/tikz/intersection/by=\meta{comma-separated list}}
        This key allows you to specify a list of names for the intersection
        coordinates. The intersection coordinates will still be named
        \meta{prefix}|-|\meta{number}, but additionally the first coordinate
        will also be named by the first element of the \meta{comma-separated
        list}. What happens is that the \meta{comma-separated list} is passed
        to the |\foreach| statement and for \meta{list member} a coordinate is
        created at the already-named intersection.
        %
\begin{codeexample}[preamble={\usetikzlibrary{intersections}}]
\begin{tikzpicture}
  \clip (-2,-2) rectangle (2,2);
  \draw [name path=curve 1] (-2,-1) .. controls (8,-1) and (-8,1) .. (2,1);
  \draw [name path=curve 2] (-1,-2) .. controls (-1,8) and (1,-8) .. (1,2);

  \fill [name intersections={of=curve 1 and curve 2, by={a,b}}]
        (a) circle (2pt)
        (b) circle (2pt);
\end{tikzpicture}
\end{codeexample}

        You can also use the |...| notation of the |\foreach| statement inside
        the \meta{comma-separated list}.

        In case an element of the \meta{comma-separated list} starts with
        options in square brackets, these options are used when the coordinate
        is created. A coordinate name can still, but need not, follow the
        options. This makes it easy to add labels to intersections:
        %
\begin{codeexample}[preamble={\usetikzlibrary{intersections}}]
\begin{tikzpicture}
  \clip (-2,-2) rectangle (2,2);
  \draw [name path=curve 1] (-2,-1) .. controls (8,-1) and (-8,1) .. (2,1);
  \draw [name path=curve 2] (-1,-2) .. controls (-1,8) and (1,-8) .. (1,2);

  \fill [name intersections={
          of=curve 1 and curve 2,
          by={[label=center:a],[label=center:...],[label=center:i]}}];
\end{tikzpicture}
\end{codeexample}
    \end{key}

    \begin{key}{/tikz/intersection/sort by=\meta{path name}}
        By default, the intersections are simply returned in the order that the
        intersection algorithm finds them. Unfortunately, this is not
        necessarily a ``helpful'' ordering. This key can be used to sort the
        intersections along the path specified by \meta{path name}, which
        should be one of the paths mentioned in the |/tikz/intersection/of|
        key.
        %
\begin{codeexample}[preamble={\usetikzlibrary{intersections}}]
\begin{tikzpicture}
\clip (-0.5,-0.75) rectangle (3.25,2.25);
\foreach \pathname/\shift in {line/0cm, curve/2cm}{
  \tikzset{xshift=\shift}
  \draw [->, name path=curve] (1,1.5) .. controls (-1,1) and (2,0.5) .. (0,0);
  \draw [->, name path=line]  (0,-.5) -- (1,2) ;
  \fill [name intersections={of=line and curve,sort by=\pathname, name=i}]
    [red, opacity=0.5, every node/.style={left=.25cm, black, opacity=1}]
    \foreach \s in {1,2,3}{(i-\s) circle (2pt) node {\footnotesize\s}};
}
\end{tikzpicture}
\end{codeexample}
    \end{key}
\end{key}


\subsection{Relative and Incremental Coordinates}

\subsubsection{Specifying Relative Coordinates}

You can prefix coordinates by |++| to make them ``relative''. A coordinate such
as |++(1cm,0pt)| means ``1cm to the right of the previous position, making this
the new current position''. Relative coordinates are often useful in ``local''
contexts:
%
\begin{codeexample}[]
\begin{tikzpicture}
  \draw (0,0)     -- ++(1,0) -- ++(0,1) -- ++(-1,0) -- cycle;
  \draw (2,0)     -- ++(1,0) -- ++(0,1) -- ++(-1,0) -- cycle;
  \draw (1.5,1.5) -- ++(1,0) -- ++(0,1) -- ++(-1,0) -- cycle;
\end{tikzpicture}
\end{codeexample}

Instead of |++| you can also use a single |+|. This also specifies a relative
coordinate, but it does not ``update'' the current point for subsequent usages
of relative coordinates. Thus, you can use this notation to specify numerous
points, all relative to the same ``initial'' point:

\begin{codeexample}[]
\begin{tikzpicture}
  \draw (0,0)     -- +(1,0) -- +(1,1) -- +(0,1) -- cycle;
  \draw (2,0)     -- +(1,0) -- +(1,1) -- +(0,1) -- cycle;
  \draw (1.5,1.5) -- +(1,0) -- +(1,1) -- +(0,1) -- cycle;
\end{tikzpicture}
\end{codeexample}

There is a special situation, where relative coordinates are interpreted
differently. If you use a relative coordinate as a control point of a Bézier
curve, the following rule applies: First, a relative first control point is
taken relative to the beginning of the curve. Second, a relative second control
point is taken relative to the end of the curve. Third, a relative end point of
a curve is taken relative to the start of the curve.

This special behavior makes it easy to specify that a curve should ``leave or
arrive from a certain direction'' at the start or end. In the following
example, the curve ``leaves'' at $30^\circ$ and ``arrives'' at $60^\circ$:
%
\begin{codeexample}[]
\begin{tikzpicture}
  \draw (1,0) .. controls +(30:1cm) and +(60:1cm) .. (3,-1);
  \draw[gray,->] (1,0) -- +(30:1cm);
  \draw[gray,<-] (3,-1) -- +(60:1cm);
\end{tikzpicture}
\end{codeexample}


\subsubsection{Rotational Relative Coordinates}

You may sometimes wish to specify points relative not only to the previous
point, but additionally relative to the tangent entering the previous point.
For this, the following key is useful:

\begin{key}{/tikz/turn}
    This key can be given as an option to a \meta{coordinate} as in the
    following example:
    %
\begin{codeexample}[]
\tikz \draw (0,0) -- (1,1) -- ([turn]-45:1cm) -- ([turn]-30:1cm);
\end{codeexample}
    %
    The effect of this key is to locally shift the coordinate system so that
    the last point reached is at the origin and the coordinate system is
    ``turned'' so that the $x$-axis points in the direction of a tangent
    entering the last point. This means, in effect, that when you use polar
    coordinates of the form \meta{relative angle}|:|\meta{distance} together
    with the |turn| option, you specify a point that lies at \meta{distance}
    from the last point in the direction of the last tangent entering the last
    point, but with a rotation of \meta{relative angle}.

    This key also works with curves \dots
    %
\begin{codeexample}[]
\tikz [delta angle=30, radius=1cm]
  \draw (0,0) arc [start angle=0]  -- ([turn]0:1cm)
              arc [start angle=30] -- ([turn]0:1cm)
              arc [start angle=60] -- ([turn]30:1cm);
\end{codeexample}
\begin{codeexample}[]
\tikz \draw (0,0) to [bend left] (2,1) -- ([turn]0:1cm);
\end{codeexample}
    %
    \dots and with plots \dots
    %
\begin{codeexample}[]
\tikz \draw plot coordinates {(0,0) (1,1) (2,0) (3,0) } -- ([turn]30:1cm);
\end{codeexample}

    Although the above examples use polar coordinates with |turn|, you can also
    use any normal coordinate. For instance, |([turn]1,1)| will append a line
    of length $\sqrt 2$ that is turns by $45^\circ$ relative to the tangent to
    the last point.
    %
\begin{codeexample}[]
\tikz \draw (0.5,0.5) -| (2,1) -- ([turn]1,1)
         .. controls ([turn]0:1cm) .. ([turn]-90:1cm);
\end{codeexample}
    %
\end{key}


\subsubsection{Relative Coordinates and Scopes}
\label{section-scopes-relative}

An interesting question is, how do relative coordinates behave in the presence
of scopes? That is, suppose we use curly braces in a path to make part of it
``local'', how does that affect the current position? On the one hand, the
current position certainly changes since the scope only affects options, not
the path itself. On the other hand, it may be useful to ``temporarily escape''
from the updating of the current point.

Since both interpretations of how the current point and scopes should
``interact'' are useful, there is a (local!) option that allows you to decide
which you need.

\begin{key}{/tikz/current point is local=\opt{\meta{boolean}} (initially false)}
    Normally, the scope path operation has no effect on the current point. That
    is, curly braces on a path have no effect on the current position:
    %
\begin{codeexample}[]
\begin{tikzpicture}
  \draw      (0,0) -- ++(1,0)   -- ++(0,1)   -- ++(-1,0);
  \draw[red] (2,0) -- ++(1,0) { -- ++(0,1) } -- ++(-1,0);
\end{tikzpicture}
\end{codeexample}
    %
    If you set this key to |true|, this behaviour changes. In this case, at the
    end of a group created on a path, the last current position reverts to
    whatever value it had at the beginning of the scope. More precisely, when
    \tikzname\ encounters |}| on a path, it checks whether at this particular
    moment the key is set to |true|. If so, the current position reverts to the
    value it had when the matching |{| was read.
    %
\begin{codeexample}[]
\begin{tikzpicture}
  \draw      (0,0) -- ++(1,0)   -- ++(0,1)   -- ++(-1,0);
  \draw[red] (2,0) -- ++(1,0)
     { [current point is local] -- ++(0,1) } -- ++(-1,0);
\end{tikzpicture}
\end{codeexample}
    %
    In the above example, we could also have given the option outside the
    scope, for instance as a parameter to the whole scope.
\end{key}


\subsection{Coordinate Calculations}
\label{tikz-lib-calc}

\begin{tikzlibrary}{calc}
    You need to load this library in order to use the coordinate calculation
    functions described in the present section.
\end{tikzlibrary}

It is possible to do some basic calculations that involve coordinates. In
essence, you can add and subtract coordinates, scale them, compute midpoints,
and do projections. For instance, |($(a) + 1/3*(1cm,0)$)| is the coordinate
that is $1/3 \text{cm}$ to the right of the point |a|:
%
\begin{codeexample}[preamble={\usetikzlibrary{calc}}]
\begin{tikzpicture}
  \draw [help lines] (0,0) grid (3,2);

  \node (a) at (1,1) {A};
  \fill [red] ($(a) + 1/3*(1cm,0)$) circle (2pt);
\end{tikzpicture}
\end{codeexample}


\subsubsection{The General Syntax}

The general syntax is the following:
%
\begin{quote}
    \declare{|(|\opt{|[|\meta{options}|]|}|$|\meta{coordinate computation}|$)|}.
\end{quote}

As you can see, the syntax uses the \TeX\ math symbol |$| to %$
indicate that a ``mathematical computation'' is involved. However, the |$| %$
has no other effect, in particular, no mathematical text is typeset.

The \meta{coordinate computation} has the following structure:
%
\begin{enumerate}
    \item It starts with
        %
        \begin{quote}
            \opt{\meta{factor}|*|}\meta{coordinate}\opt{\meta{modifiers}}
        \end{quote}
    \item This is optionally followed by |+| or |-| and then another
        %
        \begin{quote}
            \opt{\meta{factor}|*|}\meta{coordinate}\opt{\meta{modifiers}}
        \end{quote}
    \item This is once more followed by |+| or |-| and another of the above
        modified coordinate; and so on.
\end{enumerate}

In the following, the syntax of factors and of the different modifiers
is explained in detail.


\subsubsection{The Syntax of Factors}

The \meta{factor}s are optional and detected by checking whether the
\meta{coordinate computation} starts with a |(|. Also, after each $\pm$ a
\meta{factor} is present if, and only if, the |+| or |-| sign is not directly
followed by~|(|.

If a \meta{factor} is present, it is evaluated using the |\pgfmathparse| macro.
This means that you can use pretty complicated computations inside a factor. A
\meta{factor} may even contain opening parentheses, which creates a
complication: How does \tikzname\ know where a \meta{factor} ends and where a
coordinate starts? For instance, if the beginning of a \meta{coordinate
computation} is |2*(3+4|\dots, it is not clear whether |3+4| is part of a
\meta{coordinate} or part of a \meta{factor}. Because of this, the following
rule is used: Once it has been determined, that a \meta{factor} is present, in
principle, the \meta{factor} contains everything up to the next occurrence of
|*(|. Note that there is no space between the asterisk and the parenthesis.

It is permissible to put the \meta{factor} in curly braces. This can be used
whenever it is unclear where the \meta{factor} would end.

Here are some examples of coordinate specifications that consist of exactly one
\meta{factor} and one \meta{coordinate}:
%
\begin{codeexample}[preamble={\usetikzlibrary{calc}}]
\begin{tikzpicture}
  \draw [help lines] (0,0) grid (3,2);

  \fill [red] ($2*(1,1)$) circle (2pt);
  \fill [green] (${1+1}*(1,.5)$) circle (2pt);
  \fill [blue] ($cos(0)*sin(90)*(1,1)$) circle (2pt);
  \fill [black] (${3*(4-3)}*(1,0.5)$) circle (2pt);
\end{tikzpicture}
\end{codeexample}


\subsubsection{The Syntax of Partway Modifiers}

A \meta{coordinate} can be followed by different \meta{modifiers}. The first
kind of modifier is the \emph{partway modifier}. The syntax (which is loosely
inspired by Uwe Kern's |xcolor| package) is the following:
%
\begin{quote}
    \meta{coordinate}\declare{|!|\meta{number}|!|\opt{\meta{angle}|:|}\meta{second coordinate}}
\end{quote}
%
One could write for instance
%
\begin{codeexample}[code only]
(1,2)!.75!(3,4)
\end{codeexample}
%
The meaning of this is: ``Use the coordinate that is three quarters on the way
from |(1,2)| to |(3,4)|.'' In general, \meta{coordinate
x}|!|\meta{number}|!|\meta{coordinate y} yields the coordinate $(1 -
\meta{number})\meta{coordinate x} + \meta{number} \meta{coordinate y}$. Note
that this is a bit different from the way the \meta{number} is interpreted in
the |xcolor| package: First, you use a factor between $0$ and $1$, not a
percentage, and, second, as the \meta{number} approaches $1$, we approach the
second coordinate, not the first. It is permissible to use a \meta{number} that
is smaller than $0$ or larger than $1$. The \meta{number} is evaluated using
the |\pgfmathparse| command and, thus, it can involve complicated computations.
%
\begin{codeexample}[preamble={\usetikzlibrary{calc}}]
\begin{tikzpicture}
  \draw [help lines] (0,0) grid (3,2);

  \draw (1,0) -- (3,2);

  \foreach \i in {0,0.2,0.5,0.9,1}
    \node at ($(1,0)!\i!(3,2)$) {\i};
\end{tikzpicture}
\end{codeexample}

The \meta{second coordinate} may be prefixed by an \meta{angle}, separated with
a colon, as in |(1,1)!.5!60:(2,2)|. The general meaning of
\meta{a}|!|\meta{factor}|!|\meta{angle}|:|\meta{b} is: ``First, consider the
line from \meta{a} to \meta{b}. Then rotate this line by \meta{angle}
\emph{around the point \meta{a}}. Then the two endpoints of this line will be
\meta{a} and some point \meta{c}. Use this point \meta{c} for the subsequent
computation, namely the partway computation.''

Here are two examples:
%
\begin{codeexample}[preamble={\usetikzlibrary{calc}}]
\begin{tikzpicture}
  \draw [help lines] (0,0) grid (3,3);

  \coordinate (a) at (1,0);
  \coordinate (b) at (3,2);

  \draw[->] (a) -- (b);

  \coordinate (c) at ($ (a)!1! 10:(b) $);

  \draw[->,red] (a) -- (c);

  \fill ($ (a)!.5! 10:(b) $) circle (2pt);
\end{tikzpicture}
\end{codeexample}

\begin{codeexample}[preamble={\usetikzlibrary{calc}}]
\begin{tikzpicture}
  \draw [help lines] (0,0) grid (4,4);

  \foreach \i in {0,0.1,...,2}
    \fill ($(2,2) !\i! \i*180:(3,2)$) circle (2pt);
\end{tikzpicture}
\end{codeexample}

You can repeatedly apply modifiers. That is, after any modifier you can add
another (possibly different) modifier.
%
\begin{codeexample}[preamble={\usetikzlibrary{calc}}]
\begin{tikzpicture}
  \draw [help lines] (0,0) grid (3,2);

  \draw (0,0) -- (3,2);
  \draw[red] ($(0,0)!.3!(3,2)$) -- (3,0);
  \fill[red] ($(0,0)!.3!(3,2)!.7!(3,0)$) circle (2pt);
\end{tikzpicture}
\end{codeexample}


\subsubsection{The Syntax of Distance Modifiers}

A \emph{distance modifier} has nearly the same syntax as a partway modifier,
only you use a \meta{dimension} (something like |1cm|) instead of a
\meta{factor} (something like |0.5|):
%
\begin{quote}
    \meta{coordinate}\declare{|!|\meta{dimension}|!|\opt{\meta{angle}|:|}\meta{second coordinate}}
\end{quote}

When you write \meta{a}|!|\meta{dimension}|!|\meta{b}, this means the
following: Use the point that is distanced \meta{dimension} from \meta{a} on
the straight line from \meta{a} to \meta{b}. Here is an example:
%
\begin{codeexample}[preamble={\usetikzlibrary{calc}}]
\begin{tikzpicture}
  \draw [help lines] (0,0) grid (3,2);

  \draw (1,0) -- (3,2);

  \foreach \i in {0cm,1cm,15mm}
    \node at ($(1,0)!\i!(3,2)$) {\i};
\end{tikzpicture}
\end{codeexample}

As before, if you use a \meta{angle}, the \meta{second coordinate} is rotated
by this much around the \meta{coordinate} before it is used.

The combination of an \meta{angle} of |90| degrees with a distance can be used
to ``offset'' a point relative to a line. Suppose, for instance, that you have
computed a point |(c)| that lies somewhere on a line from |(a)| to~|(b)| and
you now wish to offset this point by |1cm| so that the distance from this
offset point to the line is |1cm|. This can be achieved as follows:
%
\begin{codeexample}[preamble={\usetikzlibrary{calc}}]
\begin{tikzpicture}
  \draw [help lines] (0,0) grid (3,2);

  \coordinate (a) at (1,0);
  \coordinate (b) at (3,1);

  \draw (a) -- (b);

  \coordinate (c) at ($ (a)!.25!(b) $);
  \coordinate (d) at ($ (c)!1cm!90:(b) $);

  \draw [<->] (c) -- (d) node [sloped,midway,above] {1cm};
\end{tikzpicture}
\end{codeexample}


\subsubsection{The Syntax of Projection Modifiers}

The projection modifier is also similar to the above modifiers: It also gives a
point on a line from the \meta{coordinate} to the \meta{second coordinate}.
However, the \meta{number} or \meta{dimension} is replaced by a
\meta{projection coordinate}:
%
\begin{quote}
    \meta{coordinate}\declare{|!|\meta{projection coordinate}|!|\opt{\meta{angle}|:|}\meta{second coordinate}}
\end{quote}

Here is an example:
%
\begin{codeexample}[code only]
(1,2)!(0,5)!(3,4)
\end{codeexample}

The effect is the following: We project the \meta{projection coordinate}
orthogonally onto the line from \meta{coordinate} to \meta{second coordinate}.
This makes it easy to compute projected points:
%
\begin{codeexample}[preamble={\usetikzlibrary{calc}}]
\begin{tikzpicture}
  \draw [help lines] (0,0) grid (3,2);

  \coordinate (a) at (0,1);
  \coordinate (b) at (3,2);
  \coordinate (c) at (2.5,0);

  \draw (a) -- (b) -- (c) -- cycle;

  \draw[red]    (a) -- ($(b)!(a)!(c)$);
  \draw[orange] (b) -- ($(a)!(b)!(c)$);
  \draw[blue]   (c) -- ($(a)!(c)!(b)$);
\end{tikzpicture}
\end{codeexample}

% % Copyright 2006 by Till Tantau
%
% This file may be distributed and/or modified
%
% 1. under the LaTeX Project Public License and/or
% 2. under the GNU Free Documentation License.
%
% See the file doc/generic/pgf/licenses/LICENSE for more details.


\section{Syntax for Path Specifications}
\label{section-paths}

A \emph{path} is a series of straight and curved line segments. It is specified
following a |\path| command and the specification must follow a special syntax,
which is described in the subsections of the present section.

\begin{command}{\path\meta{specification}|;|}
    This command is available only inside a |{tikzpicture}| environment.

    The \meta{specification} is a long stream of \emph{path operations}. Most
    of these path operations tell \tikzname\ how the path is built. For
    example, when you write |--(0,0)|, you use a \emph{line-to operation} and
    it means ``continue the path from wherever you are to the origin''.

    At any point where \tikzname\ expects a path operation, you can also give
    some graphic options, which is a list of options in brackets, such as
    |[rounded corners]|. These options can have different effects:
    %
    \begin{enumerate}
        \item Some options take ``immediate'' effect and apply to all
            subsequent path operations on the path. For example, the
            |rounded corners| option will round all following corners, but not
            the corners ``before'' and if the |sharp corners| is given later on
            the path (in a new set of brackets), the rounding effect will end.
            %
\begin{codeexample}[]
\tikz \draw (0,0) -- (1,1)
           [rounded corners] -- (2,0) -- (3,1)
           [sharp corners] -- (3,0) -- (2,1);
\end{codeexample}
            %
            Another example are the transformation options, which also apply
            only to subsequent coordinates.
        \item The options that have immediate effect can be ``scoped'' by
            putting part of a path in curly braces. For example, the above
            example could also be written as follows:
            %
\begin{codeexample}[]
\tikz \draw (0,0) -- (1,1)
           {[rounded corners] -- (2,0) -- (3,1)}
           -- (3,0) -- (2,1);
\end{codeexample}
            %
        \item Some options only apply to the path as a whole. For example,
            the |color=| option for determining the color used for, say,
            drawing the path always applies to all parts of the path. If
            several different colors are given for different parts of the
            path, only the last one (on the outermost scope) ``wins'':
            %
\begin{codeexample}[]
\tikz \draw (0,0) -- (1,1)
           [color=red] -- (2,0) -- (3,1)
           [color=blue] -- (3,0) -- (2,1);
\end{codeexample}

            Most options are of this type. In the above example, we would
            have had to ``split up'' the path into several |\path| commands:
            %
\begin{codeexample}[]
\tikz{\draw (0,0) -- (1,1);
      \draw [color=red] (2,0) -- (3,1);
      \draw [color=blue] (3,0) -- (2,1);}
\end{codeexample}
    \end{enumerate}

    By default, the |\path| command does ``nothing'' with the path, it just
    ``throws it away''. Thus, if you write |\path(0,0)--(1,1);|, nothing is
    drawn in your picture. The only effect is that the area occupied by the
    picture is (possibly) enlarged so that the path fits inside the area. To
    actually ``do'' something with the path, an option like |draw| or |fill|
    must be given somewhere on the path. Commands like |\draw| do this
    implicitly.

    Finally, it is also possible to give \emph{node specifications} on a path.
    Such specifications can come at different locations, but they are always
    allowed when a normal path operation could follow. A node specification
    starts with |node|. Basically, the effect is to typeset the node's text as
    normal \TeX\ text and to place it at the ``current location'' on the path.
    The details are explained in Section~\ref{section-nodes}.

    Note, however, that the nodes are \emph{not} part of the path in any way.
    Rather, after everything has been done with the path what is specified by
    the path options (like filling and drawing the path due to a |fill| and a
    |draw| option somewhere in the \meta{specification}), the nodes are added
    in a post-processing step.
\end{command}

\begin{key}{/tikz/name=\meta{path name}}
    Assigns a name to the path for reference (specifically, for reference
    in animations; for reference in intersections, use the |name path|
    command, which has a different purpose, see the |intersections| library
    for details). Since the name is a ``high-level'' name (drivers never
    know of it), you can use spaces, number, letters, or whatever you like
    when naming a path, but the name may \emph{not} contain any punctuation
    like a dot, a comma, or a colon.
\end{key}

The following style influences scopes:
%
\begin{stylekey}{/tikz/every path (initially \normalfont empty)}
    This style is installed at the beginning of every path. This can be
    useful for (temporarily) adding, say, the |draw| option to everything
    in a scope.
    %
\begin{codeexample}[]
\begin{tikzpicture}
  [fill=yellow!80!black,      % only sets the color
   every path/.style={draw}]  % all paths are drawn
  \fill  (0,0) rectangle +(1,1);
  \shade (2,0) rectangle +(1,1);
\end{tikzpicture}
\end{codeexample}
    %
\end{stylekey}

\begin{key}{/tikz/insert path=\meta{path}}
    This key can be used inside an option to add something to the current path.
    This is mostly useful for defining styles that create graphic contents.
    This option should be used with care, for instance it should not be used as
    an argument of, say, a |node|. In the following example, we use a style to
    add little circles to a path.
    %
\begin{codeexample}[]
\tikz [c/.style={insert path={circle[radius=2pt]}}]
  \draw (0,0) -- (1,1) [c] -- (3,2) [c];
\end{codeexample}
    %
     The effect is the same as of
    |(0,0) -- (1,1) circle[radius=2pt] -- (3,2) circle[radius=2pt]|.
\end{key}

The following options are for experts only:

\begin{key}{/tikz/append after command=\meta{path}}
    Some of the path commands described in the following sections take optional
    arguments. For these commands, when you use this key inside these options,
    the \meta{path} will be inserted \emph{after} the path command is done. For
    instance, when you give this command in the option list of a node, the
    \meta{path} will be added after the node. This is used by, for instance,
    the |label| option to allow you to specify a label in the option list of a
    node, but have this |label| cause a node to be added after another node.
    %
\begin{codeexample}[]
\tikz \draw node [append after command={(foo)--(1,1)},draw] (foo){foo};
\end{codeexample}
    %
    If this key is called multiple times, the effects accumulate, that is, all
    of the paths are added in the order to keys were found.
\end{key}

\begin{key}{/tikz/prefix after command=\meta{path}}
    Works like |append after command|, only the accumulation order is inverse:
    The \meta{path} is added before any earlier paths added using either
    |append after command| or |prefix after command|.
\end{key}


\subsection{The Move-To Operation}

The perhaps simplest operation is the move-to operation, which is specified by
just giving a coordinate where a path operation is expected.

\begin{pathoperation}[noindex]{}{\meta{coordinate}}
        \index{empty@\protect\meta{empty} path operation}%
        \index{Path operations!empty@\protect\texttt{\meta{empty}}}%
    The move-to operation normally starts a path at a certain point. This does
    not cause a line segment to be created, but it specifies the starting point
    of the next segment. If a path is already under construction, that is, if
    several segments have already been created, a move-to operation will start
    a new part of the path that is not connected to any of the previous
    segments.
    %
\begin{codeexample}[]
\begin{tikzpicture}
  \draw (0,0) --(2,0) (0,1) --(2,1);
\end{tikzpicture}
\end{codeexample}

    In the specification |(0,0) --(2,0) (0,1) --(2,1)| two move-to operations
    are specified: |(0,0)| and |(0,1)|. The other two operations, namely
    |--(2,0)| and |--(2,1)| are line-to operations, described next.
\end{pathoperation}

There is special coordinate called |current subpath start| that is always at
the position of the last move-to operation on the current path.
%
\begin{codeexample}[]
\tikz [line width=2mm]
  \draw (0,0) -- (1,0) -- (1,1)
        -- (0,1) -- (current subpath start);
\end{codeexample}

Note how in the above example the path is not closed (as |--cycle| would do).
Rather, the line just starts and ends at the origin without being a closed
path.


\subsection{The Line-To Operation}

\subsubsection{Straight Lines}

\begin{pathoperation}{--}{\meta{coordinate or cycle}}
    The line-to operation extends the current path from the current point in a
    straight line to the given \meta{coordinate} (the ``or cycle'' part is
    explained in a moment). The ``current point'' is the endpoint of the
    previous drawing operation or the point specified by a prior move-to
    operation.

    When a line-to operation is used and some path segment has just been
    constructed, for example by another line-to operation, the two line
    segments become joined. This means that if they are drawn, the point where
    they meet is ``joined'' smoothly. To appreciate the difference, consider
    the following two examples: In the left example, the path consists of two
    path segments that are not joined, but they happen to share a point, while
    in the right example a smooth join is shown.
    %
\begin{codeexample}[]
\begin{tikzpicture}[line width=10pt]
  \draw (0,0) --(1,1)  (1,1) --(2,0);
  \draw (3,0) -- (4,1) -- (5,0);
  \useasboundingbox (0,1.5); % make bounding box higher
\end{tikzpicture}
\end{codeexample}

    Instead of a coordinate following the two minus signs, you can also use the
    text |cycle|. This causes the straight line from the current point to go to
    the last point specified by a move-to operation. Note that this need not be
    the beginning of the path. Furthermore, a smooth join is created between
    the first segment created after the last move-to operation and the straight
    line appended by the cycle operation.

    Consider the following example. In the left example, two triangles are
    created using three straight lines, but they are not joined at the ends. In
    the second example cycle operations are used.
    %
\begin{codeexample}[]
\begin{tikzpicture}[line width=10pt]
  \draw (0,0) -- (1,1) -- (1,0) -- (0,0) (2,0) -- (3,1) -- (3,0) -- (2,0);
  \draw (5,0) -- (6,1) -- (6,0) -- cycle (7,0) -- (8,1) -- (8,0) -- cycle;
  \useasboundingbox (0,1.5); % make bounding box higher
\end{tikzpicture}
\end{codeexample}
    %
\end{pathoperation}

Writing |cycle| instead of a coordinate at the end of a path operation is
possible with all path operations that end with a coordinate (such as |--| or
|..| or |sin| or |grid|, but not |graph| or |plot|). In all cases, the effect
is that the coordinate of the last moveto is used as the coordinate expected by
the path operation and that a smooth join is added. (What actually happens that
the text |cycle| used with any path operation other than |--| gets replaced by
|(current subpath start)--cycle|.)


\subsubsection{Horizontal and Vertical Lines}

Sometimes you want to connect two points via straight lines that are only
horizontal and vertical. For this, you can use two path construction
operations.

{\catcode`\|=12
\begin{pathoperation}[noindex]{-|}{\meta{coordinate or cycle}}
    \index{--1@\protect\texttt{-\protect\pgfmanualbar} path operation}%
    \index{Path operations!--1@\protect\texttt{-\protect\pgfmanualbar}}%
    \pgfmanualpdflabel[\catcode`\|=12 ]{-|}{}%
    This operation means ``first horizontal, then vertical''.
    %
\begin{codeexample}[]
\begin{tikzpicture}
  \draw (0,0) node(a) [draw] {A}  (1,1) node(b) [draw] {B};
  \draw (a.north) |- (b.west);
  \draw[color=red] (a.east) -| (2,1.5) -| (b.north);
\end{tikzpicture}
\end{codeexample}
    %
    Instead of a coordinate you can also write \verb!cycle! to close the path:
    %
\begin{codeexample}[]
\begin{tikzpicture}[ultra thick]
  \draw (0,0) -- (1,1) -| cycle;
\end{tikzpicture}
\end{codeexample}
\end{pathoperation}

\begin{pathoperation}[noindex]{|-}{\meta{coordinate or cycle}}
    \index{--2@\protect\texttt{\protect\pgfmanualbar-} path operation}%
    \index{Path operations!--2@\protect\texttt{\protect\pgfmanualbar-}}%
    \pgfmanualpdflabel[\catcode`\|=12 ]{|-}{}%
    This operations means ``first vertical, then horizontal''.
\end{pathoperation}
}


\subsection{The Curve-To Operation}

The curve-to operation allows you to extend a path using a Bézier curve.

\begin{pathoperation}{..}{\declare{|controls|}\meta{c}\opt{|and|\meta{d}}\declare{|..|\meta{y or cycle}}}
    This operation extends the current path from the current point, let us call
    it $x$, via a curve to a point~$y$ (if, instead of a coordinate you say
    |cycle| at the end, $y$ will be the coordinate of the last move-to
    operation). The curve is a cubic Bézier curve. For such a curve, apart
    from $y$, you also specify two control points $c$ and $d$. The idea is that
    the curve starts at $x$, ``heading'' in the direction of~$c$.
    Mathematically spoken, the tangent of the curve at $x$ goes through $c$.
    Similarly, the curve ends at $y$, ``coming from'' the other control
    point,~$d$. The larger the distance between $x$ and~$c$ and between $d$
    and~$y$, the larger the curve will be.

    If the ``|and|\meta{d}'' part is not given, $d$ is assumed to be equal to
    $c$.
    %
\begin{codeexample}[]
\begin{tikzpicture}
  \draw[line width=10pt] (0,0) .. controls (1,1) .. (4,0)
                               .. controls (5,0) and (5,1) .. (4,1);
  \draw[color=gray] (0,0) -- (1,1) -- (4,0) -- (5,0) -- (5,1) -- (4,1);
\end{tikzpicture}
\end{codeexample}

\begin{codeexample}[]
\begin{tikzpicture}
  \draw[line width=10pt] (0,0) -- (2,0) .. controls (1,1) .. cycle;
\end{tikzpicture}
\end{codeexample}

    As with the line-to operation, it makes a difference whether two curves are
    joined because they resulted from consecutive curve-to or line-to
    operations, or whether they just happen to have a common (end) point:
    %
\begin{codeexample}[]
\begin{tikzpicture}[line width=10pt]
  \draw (0,0) -- (1,1) (1,1) .. controls (1,0) and (2,0) .. (2,0);
  \draw [yshift=-1.5cm]
        (0,0) -- (1,1)       .. controls (1,0) and (2,0) .. (2,0);
\end{tikzpicture}
\end{codeexample}
    %
\end{pathoperation}


\subsection{The Rectangle Operation}

A rectangle can obviously be created using four straight lines and a cycle
operation. However, since rectangles are needed so often, a special syntax is
available for them.

\begin{pathoperation}{rectangle}{\meta{corner or cycle}}
    When this operation is used, one corner will be the current point, another
    corner is given by \meta{corner}, which becomes the new current point.
    %
\begin{codeexample}[]
\begin{tikzpicture}
  \draw (0,0) rectangle (1,1);
  \draw (.5,1) rectangle (2,0.5) (3,0) rectangle (3.5,1.5) -- (2,0);
\end{tikzpicture}
\end{codeexample}

    Just for consistency, you can also use |cycle| instead of a coordinate, but
    it is a bit unclear what use this might have.
\end{pathoperation}


\subsection{Rounding Corners}

All of the path construction operations mentioned up to now are influenced by
the following option:

\begin{key}{/tikz/rounded corners=\meta{inset} (default 4pt)}
    When this option is in force, all corners (places where a line is continued
    either via line-to or a curve-to operation) are replaced by little arcs so
    that the corner becomes smooth.
    %
\begin{codeexample}[]
\tikz \draw [rounded corners] (0,0) -- (1,1)
           -- (2,0) .. controls (3,1) .. (4,0);
\end{codeexample}

    The \meta{inset} describes how big the corner is. Note that the
    \meta{inset} is \emph{not} scaled along if you use a scaling option like
    |scale=2|.
    %
\begin{codeexample}[]
\begin{tikzpicture}
  \draw[color=gray,very thin] (10pt,15pt) circle[radius=10pt];
  \draw[rounded corners=10pt] (0,0) -- (0pt,25pt) -- (40pt,25pt);
\end{tikzpicture}
\end{codeexample}

    You can switch the rounded corners on and off ``in the middle of path'' and
    different corners in the same path can have different corner radii:
    %
\begin{codeexample}[]
\begin{tikzpicture}
  \draw (0,0) [rounded corners=10pt] -- (1,1) -- (2,1)
                     [sharp corners] -- (2,0)
               [rounded corners=5pt] -- cycle;
\end{tikzpicture}
\end{codeexample}

    Here is a rectangle with rounded corners:
    %
\begin{codeexample}[]
\tikz \draw[rounded corners=1ex] (0,0) rectangle (20pt,2ex);
\end{codeexample}

    You should be aware, that there are several pitfalls when using this
    option. First, the rounded corner will only be an arc (part of a circle) if
    the angle is $90^\circ$. In other cases, the rounded corner will still be
    round, but ``not as nice''.

    Second, if there are very short line segments in a path, the ``rounding''
    may cause inadvertent effects. In such case it may be necessary to
    temporarily switch off the rounding using |sharp corners|.
\end{key}

\begin{key}{/tikz/sharp corners}
    This options switches off any rounding on subsequent corners of the path.
\end{key}


\subsection{The Circle and Ellipse Operations}

Circles and ellipses are common path elements for which there is a special path
operation.

\begin{pathoperation}{circle}{\opt{|[|\meta{options}|]|}}
    This command adds a circle to the current path where the center of the
    circle is the current point by default, but you can use the |at| option to
    change this. The new current point of the path will be (typically just
    remain) the center of the circle.

    The radius of the circle is specified using the following options:
    %
    \begin{key}{/tikz/x radius=\meta{value}}
        Sets the horizontal radius of the circle (which, when this value is
        different form the vertical radius, is actually an ellipse). The
        \meta{value} may either be a dimension or a dimensionless number. In
        the latter case, the number is interpreted in the $xy$-coordinate
        system (if the $x$-unit is set to, say, |2cm|, then |x radius=3| will
        have the same effect as |x radius=6cm|).
    \end{key}
    %
    \begin{key}{/tikz/y radius=\meta{value}}
        Works like the |x radius|.
    \end{key}
    %
    \begin{key}{/tikz/radius=\meta{value}}
        Sets the |x radius| and |y radius| simultaneously.
    \end{key}
    %
    \begin{key}{/tikz/at=\meta{coordinate}}
        If this option is explicitly set inside the \meta{options} (or
        indirectly via the |every circle| style), the \meta{coordinate} is used
        as the center of the circle instead of the current point. Setting |at|
        to some value in an enclosing scope has no effect.
    \end{key}
    The \meta{options} may also contain additional options like, say, a
    |rotate| or |scale|, that will only have an effect on the circle.
    %
\begin{codeexample}[]
\begin{tikzpicture}
  \draw (1,0) circle [radius=1.5];
  \fill (1,0) circle [x radius=1cm, y radius=5mm, rotate=30];
\end{tikzpicture}
\end{codeexample}

    It is possible to set the |radius| also in some enclosing scope, in this
    case the options can be left out (but see the note below on what may
    follow):
    %
\begin{codeexample}[]
\begin{tikzpicture}[radius=2pt]
  \draw (0,0) circle -- (1,1) circle -- ++(0,1) circle;
\end{tikzpicture}
\end{codeexample}

    The following style is used with every circle:
    %
    \begin{stylekey}{/tikz/every circle}
        You can use this key to set up, say, a default radius for every circle.
        The key will also be used with the |ellipse| operation.
    \end{stylekey}

    In case you feel that the names |radius| and |x radius| are too long for
    your taste, you can easily created shorter aliases:
    %
\begin{codeexample}[code only]
\tikzset{r/.style={radius=#1},rx/.style={x radius=#1},ry/.style={y radius=#1}}
\end{codeexample}
    %
    You can then say |circle [r=1cm]| or |circle [rx=1,ry=1.5]|. The reason
    \tikzname\ uses the longer names by default is that it encourages people to
    write more readable code.

    \emph{Note:} There also exists an older syntax for circles, where the
    radius of the circle is given in parentheses right after the |circle|
    command as in |circle (1pt)|. Although this syntax is a bit more succinct,
    it is harder to understand for readers of the code and the use of
    parentheses for something other than a coordinate is ill-chosen.

    \tikzname\ will use the following rule to determine whether the old or the
    normal syntax is used: If |circle| is directly followed by something that
    (expands to) an opening parenthesis, then the old syntax is used and inside
    these following parentheses there must be a single number or dimension
    representing a radius. In all other cases the new syntax is used.
\end{pathoperation}

\begin{pathoperation}{ellipse}{|[|\meta{options}|]|}
    This command has exactly the same effect as |circle|. The older syntax for
    this command is |ellipse (|\meta{x radius} |and| \meta{y radius}|)|. As for
    the |circle| command, this syntax is not as good as the standard syntax.
    %
\begin{codeexample}[]
\begin{tikzpicture}
  \draw [help lines] (0,0) grid (3,2);
  \draw (1,1) ellipse [x radius=1cm,y radius=.5cm];
\end{tikzpicture}
\end{codeexample}
    %
\end{pathoperation}


\subsection{The Arc Operation}

The \emph{arc operation} allows you to add an arc to the current path.
%
\begin{pathoperation}{arc}{\oarg{options}}
    The |arc| operation adds a part of an ellipse to the current path. The
    radii of the ellipse are given by the values of |x radius| and |y radius|,
    which should be set in the \meta{options}. The arc will start at   the
    current point and will end at the end of the arc. The arc  will start and
    end at angles computed from the three keys |start angle|, |end angle|, and
    |delta angle|. Normally, the first two keys specify the start and end
    angle. However, in case one of them is empty, it is computed from the other
    key plus or minus the |delta angle|. In detail, if |end angle| is empty, it
    is set to the start angle plus the delta angle. If the start angle is
    missing, it is set to the end angle minus the delta angle. If all three
    keys are set, the delta angle is ignored.
    %
    \begin{key}{/tikz/start angle=\meta{degrees}}
        Sets the start angle.
    \end{key}
    %
    \begin{key}{/tikz/end angle=\meta{degrees}}
        Sets the end angle.
    \end{key}
    %
    \begin{key}{/tikz/delta angle=\meta{degrees}}
        Sets the delta angle.
    \end{key}

\begin{codeexample}[]
\begin{tikzpicture}[radius=1cm]
  \draw (0,0)  arc[start angle=180, end angle=90]
     -- (2,.5) arc[start angle=90,  delta angle=-90];
  \draw (4,0) -- +(30:1cm)
              arc [start angle=30,  delta angle=30] -- cycle;
  \draw (8,0) arc [start angle=0,   end angle=270,
                   x radius=1cm, y radius=5mm] -- cycle;
\end{tikzpicture}
\end{codeexample}

\begin{codeexample}[]
\begin{tikzpicture}[radius=1cm,delta angle=30]
  \draw (-1,0) -- +(3.5,0);
  \draw (1,0) ++(210:2cm) -- +(30:4cm);
  \draw (1,0) +(0:1cm) arc [start angle=0];
  \draw (1,0) +(180:1cm) arc [start angle=180];
  \path (1,0) ++(15:.75cm) node{$\alpha$};
  \path (1,0) ++(15:-.75cm) node{$\beta$};
\end{tikzpicture}
\end{codeexample}

    There also exists a shorter syntax for the arc operation, namely |arc|
    begin directly followed by
    |(|\meta{start angle}|:|\meta{end angle}|:|\meta{radius}). However, this
    syntax is harder to read, so the normal syntax should be preferred in
    general.
\end{pathoperation}


\subsection{The Grid Operation}

You can add a grid to the current path using the |grid| path operation.

\begin{pathoperation}{grid}{\opt{\oarg{options}}\meta{corner or cycle}}
    This operations adds a grid filling a rectangle whose two corners are given
    by \meta{corner} and by the previous coordinate. (Instead of a coordinate
    you can also say |cycle| to use the position of the last move-to as the
    corner coordinate, but it not very natural to do so.) corner Thus, the
    typical way in which a grid is drawn is |\draw (1,1) grid (3,3);|, which
    yields a grid filling the rectangle whose corners are at $(1,1)$ and
    $(3,3)$. All coordinate transformations apply to the grid.
    %
\begin{codeexample}[]
\tikz[rotate=30] \draw[step=1mm] (0,0) grid (2,2);
\end{codeexample}

    The \meta{options}, which are local to the |grid| operation, can be used to
    influence the appearance of the grid. The stepping of the grid is governed
    by the following options:
    %
    \begin{key}{/tikz/step=\meta{number or dimension or coordinate} (initially 1cm)}
        Sets the stepping in both the $x$ and $y$-direction. If a dimension is
        provided, this is used directly. If a number is provided, this number
        is interpreted in the $xy$-coordinate system. For example, if you
        provide the number |2|, then the $x$-step is twice the $x$-vector and
        the $y$-step is twice the $y$-vector set by the |x=| and |y=| options.
        Finally, if you provide a coordinate, then the $x$-part of this
        coordinate will be used as the $x$-step and the $y$-part will be used
        as the $y$-coordinate.
        %
\begin{codeexample}[]
\begin{tikzpicture}[x=.5cm]
  \draw[thick] (0,0) grid [step=1]     (3,2);
  \draw[red]   (0,0) grid [step=.75cm] (3,2);
\end{tikzpicture}
\begin{tikzpicture}
  \draw        (0,0) circle [radius=1];
  \draw[blue]  (0,0) grid [step=(45:1)] (3,2);
\end{tikzpicture}
\end{codeexample}

        A complication arises when the $x$- and/or $y$-vector do not point
        along the axes. Because of this, the actual rule for computing the
        $x$-step and the $y$-step is the following: As the $x$- and $y$-steps
        we use the $x$- and $y$-components or the following two vectors: The
        first vector is either $(\meta{x-grid-step-number},0)$ or
        $(\meta{x-grid-step-dimension},0\mathrm{pt})$, the second vector is
        $(0,\meta{y-grid-step-number})$ or
        $(0\mathrm{pt},\meta{y-grid-step-dimension})$.

        If the $x$-step or $y$-step is $0$ or negative the corresponding lines
        are not drawn.
    \end{key}

    \begin{key}{/tikz/xstep=\meta{dimension or number} (initially 1cm)}
        Sets the stepping in the $x$-direction.
        %
\begin{codeexample}[]
\begin{tikzpicture}
  \draw (0,0) grid [xstep=.5,ystep=.75] (3,2);
  \draw[ultra thick] (0,0) grid [ystep=0] (3,2);
\end{tikzpicture}
\end{codeexample}
    \end{key}

    \begin{key}{/tikz/ystep=\meta{dimension or number} (initially 1cm)}
        Sets the stepping in the $y$-direction.
    \end{key}

    It is important to note that the grid is always ``phased'' such that it
    contains the point $(0,0)$ if that point happens to be inside the
    rectangle. Thus, the grid does \emph{not} always have an intersection at
    the corner points; this occurs only if the corner points are multiples of
    the stepping. Note that due to rounding errors, the ``last'' lines of a
    grid may be omitted. In this case, you have to add an epsilon to the corner
    points.

    The following style is useful for drawing grids:
    %
    \begin{stylekey}{/tikz/help lines (initially {line width=0.2pt,gray})}
        This style makes lines ``subdued'' by using thin gray lines for them.
        However, this style is not installed automatically and you have to say
        for example:
        %
\begin{codeexample}[]
\tikz \draw[help lines] (0,0) grid (3,3);
\end{codeexample}
    \end{stylekey}
\end{pathoperation}


\subsection{The Parabola Operation}

The |parabola| path operation continues the current path with a parabola. A
parabola is a (shifted and scaled) curve defined by the equation $f(x) = x^2$
and looks like this: \tikz \draw (-1ex,1.5ex) parabola[parabola height=-1.5ex]
+(2ex,0ex);.

\begin{pathoperation}{parabola}{\opt{\oarg{options}|bend|\meta{bend
        coordinate}}\meta{coordinate or cycle}}
    This operation adds a parabola through the current point and the given
    \meta{coordinate} or, if |cycle| is used instead of coordinate at the end,
    the \meta{coordinate} is set to the position of the last move-to and the
    path gets closed after the parabola. If the |bend| is given, it specifies
    where the bend should go; the \meta{options} can also be used to specify
    where the bend is. By default, the bend is at the old current point.
    %
\begin{codeexample}[]
\begin{tikzpicture}
  \draw                (0,0) rectangle                (1,1.5)
                       (0,0) parabola                 (1,1.5);
  \draw[xshift=1.5cm]  (0,0) rectangle                (1,1.5)
                       (0,0) parabola[bend at end]    (1,1.5);
  \draw[xshift=3cm]    (0,0) rectangle                (1,1.5)
                       (0,0) parabola bend (.75,1.75) (1,1.5);

  \draw[yshift=-2cm]   (1,1.5) --
                       (0,0) parabola                 cycle;
\end{tikzpicture}
\end{codeexample}

    The following options influence parabolas:
    %
    \begin{key}{/tikz/bend=\meta{coordinate}}
        Has the same effect as saying |bend|\meta{coordinate} outside the
        \meta{options}. The option specifies that the bend of the parabola
        should be at the given \meta{coordinate}. You have to take care
        yourself that the bend position is a ``valid'' position; which means
        that if there is no parabola of the form $f(x) = a x^2 + b x + c$ that
        goes through the old current point, the given bend, and the new current
        point, the result will not be a parabola.

        There is one special property of the \meta{coordinate}: When a relative
        coordinate is given like |+(0,0)|, the position relative to this
        coordinate is ``flexible''. More precisely, this position lies
        somewhere on a line from the old current point to the new current
        point. The exact position depends on the next option.
    \end{key}

    \begin{key}{/tikz/bend pos=\meta{fraction}}
        Specifies where the ``previous'' point is relative to which the bend is
        calculated. The previous point will be at the \meta{fraction}th part of
        the line from the old current point to the new current point.

        The idea is the following: If you say |bend pos=0| and |bend +(0,0)|,
        the bend will be at the old current point. If you say |bend pos=1| and
        |bend +(0,0)|, the bend will be at the new current point. If you say
        |bend pos=0.5| and |bend +(0,2cm)| the bend will be 2cm above the
        middle of the line between the start and end point. This is most useful
        in situations such as the following:
        %
\begin{codeexample}[]
\begin{tikzpicture}
  \draw[help lines] (0,0) grid (3,2);
  \draw (-1,0) parabola[bend pos=0.5] bend +(0,2) +(3,0);
\end{tikzpicture}
\end{codeexample}

        In the above example, the |bend +(0,2)| essentially means ``a parabola
        that is 2cm high'' and |+(3,0)| means ``and 3cm wide''. Since this
        situation arises often, there is a special shortcut option:
        %
        \begin{key}{/tikz/parabola height=\meta{dimension}}
            This option has the same effect as
            |[bend pos=0.5,bend={+(0pt,|\meta{dimension}|)}]|.
            %
\begin{codeexample}[]
\begin{tikzpicture}
  \draw[help lines] (0,0) grid (3,2);
  \draw (-1,0) parabola[parabola height=2cm] +(3,0);
\end{tikzpicture}
\end{codeexample}
        \end{key}
    \end{key}

    The following styles are useful shortcuts:
    %
    \begin{stylekey}{/tikz/bend at start}
        This places the bend at the start of a parabola. It is a shortcut for
        the following options: |bend pos=0,bend={+(0,0)}|.
    \end{stylekey}

    \begin{stylekey}{/tikz/bend at end}
        This places the bend at the end of a parabola.
    \end{stylekey}
\end{pathoperation}


\subsection{The Sine and Cosine Operation}

The |sin| and |cos| operations are similar to the |parabola| operation. They,
too, can be used to draw (parts of) a sine or cosine curve.

\begin{pathoperation}{sin}{\meta{coordinate or cycle}}
    The effect of |sin| is to draw a scaled and shifted version of a sine curve
    in the interval $[0,\pi/2]$. The scaling and shifting is done in such a way
    that the start of the sine curve in the interval is at the old current
    point and that the end of the curve in the interval is at
    \meta{coordinate}. Here is an example that should clarify this:
    %
\begin{codeexample}[]
\tikz \draw (0,0) rectangle (1,1)     (0,0) sin (1,1)
            (2,0) rectangle +(1.57,1) (2,0) sin +(1.57,1);
\end{codeexample}
    %
\end{pathoperation}

\begin{pathoperation}{cos}{\meta{coordinate or cycle}}
    This operation works similarly, only a cosine in the interval $[0,\pi/2]$
    is drawn. By correctly alternating |sin| and |cos| operations, you can
    create a complete sine or cosine curve:
    %
\begin{codeexample}[]
\begin{tikzpicture}[xscale=1.57]
  \draw (0,0) sin (1,1) cos (2,0) sin (3,-1) cos (4,0) sin (5,1);
  \draw[color=red] (0,1.5) cos (1,0) sin (2,-1.5) cos (3,0) sin (4,1.5) cos (5,0);
\end{tikzpicture}
\end{codeexample}
    %
\end{pathoperation}

Note that there is no way to (conveniently) draw an interval on a sine or
cosine curve whose end points are not multiples of $\pi/2$.


\subsection{The SVG Operation}

The |svg| operation can be used to extend the current path by a path given in
the \textsc{svg} path data syntax. This syntax is described in detail in
Section~8.3 of the \textsc{svg 1.1} specification, please consult this
specification for details.

\begin{pathoperation}{svg}{\opt{\oarg{options}}\marg{path data}}
    This operation adds the path specified in the \meta{path data} in
    \textsc{svg 1.1 path data} syntax to the current path. Unlike the
    \textsc{svg}-specification, it \emph{is} permissible that the path data
    does not start with a move-to command (|m| or |M|), in which case the last
    point of the current path is used as start point. The optional
    \meta{options} apply locally to this path operation, typically you will use
    them to set up, say, some transformations.
    %
\begin{codeexample}[preamble={\usetikzlibrary{svg.path}}]
\begin{tikzpicture}
  \filldraw [fill=red!20] (0,1) svg[scale=2] {h 10 v 10 h -10}
    node [above left] {upper left} -- cycle;

  \draw svg {M 0 0 L 20 20 h 10 a 10 10 0 0 0 -20 0};
\end{tikzpicture}
\end{codeexample}

    An \textsc{svg} coordinate like |10 20| is always interpreted as
    |(10pt,20pt)|, so the basic unit is always points (|pt|). The
    $xy$-coordinate system is not used. However, you can use scaling to
    (locally) change the basic unit. For instance, |svg[scale=1cm]| (yes, this
    works, although some rather evil magic is involved) will cause 1cm to be
    the basic unit.

    Instead of curly braces, you can also use quotation marks to indicate the
    start and end of the \textsc{svg} path.

    \emph{Warning:} The arc operations (|a| and |A|) are  numerically instable.
    This means that they will be quite imprecise, except when the angle is a
    multiple of $90^\circ$ (as is, fortunately, most often the case).
\end{pathoperation}


\subsection{The Plot Operation}

The |plot| operation can be used to append a line or curve to the path that
goes through a large number of coordinates. These coordinates are either given
in a simple list of coordinates, read from some file, or they are computed on
the fly.

Since the syntax and the behaviour of this command are a bit complex, they are
described in the separated Section~\ref{section-tikz-plots}.


\subsection{The To Path Operation}

The |to| operation is used to add a user-defined path from the previous
coordinate to the following coordinate. When you write |(a) to (b)|, a straight
line is added from |a| to |b|, exactly as if you had written |(a) -- (b)|.
However, if you write |(a) to [out=135,in=45] (b)| a curve is added to the
path, which leaves at an angle of 135$^\circ$ at |a| and arrives at an angle of
45$^\circ$ at |b|. This is because the options |in| and |out| trigger a special
path to be used instead of the straight line.

\begin{pathoperation}{to}{\opt{|[|\meta{options}|]|}
        \opt{\meta{nodes}} \meta{coordinate or cycle}}
    This path operation inserts the path currently set via the |to path| option
    at the current position. The \meta{options} can be used to modify (perhaps
    implicitly) the |to path| and to set up how the path will be rendered.

    Before the |to path| is inserted, a number of macros are set up that can
    ``help'' the |to path|. These are |\tikztostart|, |\tikztotarget|, and
    |\tikztonodes|; they are explained in the following.

    \medskip
    \textbf{Start and Target Coordinates.}\ \
    The |to| operation is always followed by a \meta{coordinate}, called the
    target coordinate, or the text |cycle|, in which case the last move-to is
    used as a coordinate and the path gets closed. The macro |\tikztotarget| is
    set to this coordinate (without its parentheses). There is also a
    \emph{start coordinate}, which is the coordinate preceding the |to|
    operation. This coordinate can be accessed via the macro |\tikztostart|. In
    the following example, for the first |to|, the macro |\tikztostart| is
    |0pt,0pt| and the |\tikztotarget| is |0,2|. For the second |to|, the macro
    |\tikztostart| is |10pt,10pt| and |\tikztotarget| is |a|. For the third,
    they are set to |a| and |current subpath start|.
    %
\begin{codeexample}[]
\begin{tikzpicture}
  \draw[help lines] (0,0) grid (3,2);
  \node       (a)         at (2,2) {a};

  \draw       (0,0)       to (0,2);
  \draw[red]  (10pt,10pt) to (a);
  \draw[blue] (3,0) -- (3,2) -- (a) to cycle;
\end{tikzpicture}
\end{codeexample}

    \medskip
    \textbf{Nodes on to--paths.}\ \
    It is possible to add nodes to the paths constructed by a |to| operation.
    To do so, you specify the nodes between the |to| keyword and the coordinate
    (if there are options to the |to| operation, these come first). The effect
    of |(a) to node {x} (b)| (typically) is the same as if you had written
    |(a) -- node {x} (b)|, namely that the node is placed on the |to|. This can
    be used to add labels to |to|s:
    %
\begin{codeexample}[]
\begin{tikzpicture}
  \draw (0,0) to node [sloped,above] {x} (3,2);

  \draw (0,0) to[out=90,in=180] node [sloped,above] {x} (3,2);
\end{tikzpicture}
\end{codeexample}

    Instead of writing the node between the |to| keyword and the target
    coordinate, you may also use the following keys to create such nodes:
    %
    \begin{key}{/tikz/edge node=\meta{node specification}}
        This key can be used inside the \meta{options} of a |to| path command.
        It will add the \meta{node specification} to the list of nodes to be
        placed on the connecting line, just as if you had written the
        \meta{node specification} directly after the |to| keyword:
        %
\begin{codeexample}[]
\begin{tikzpicture}
  \draw (0,0) to [edge node={node [sloped,above] {x}}] (3,2);

  \draw (0,0) to [out=90,in=180,
                  edge node={node [sloped,above] {x}}] (3,2);
\end{tikzpicture}
\end{codeexample}
        %
        This key is mostly useful to create labels automatically using other
        keys.
    \end{key}
    %
    \begin{key}{/tikz/edge label=\meta{text}}
        A shorthand for |edge node={node[auto]{|\meta{text}|}}|.
        %
\begin{codeexample}[]
\tikz \draw (0,0) to [edge label=x] (3,2);
\end{codeexample}
    \end{key}
    %
    \begin{key}{/tikz/edge label'=\meta{text}}
        A shorthand for |edge node={node[auto,swap]{|\meta{text}|}}|.
        %
\begin{codeexample}[]
\tikz \draw (0,0) to [edge label=x, edge label'=y] (3,2);
\end{codeexample}
    \end{key}

    When the |quotes| library is loaded, additional ways of specifying nodes on
    to--paths become available, see Section~\ref{section-edge-quotes}.

    \medskip
    \textbf{Styles for to-paths.}\ \
    In addition to the \meta{options} given after the |to| operation, the
    following style is also set at the beginning of the to path:
    %
    \begin{stylekey}{/tikz/every to (initially \normalfont empty)}
        This style is installed at the beginning of every to.
        %
\begin{codeexample}[]
\tikz[every to/.style={bend left}]
  \draw (0,0) to (3,2);
\end{codeexample}
        %
        Note that, as explained below, every to path is implicitly surrounded
        by curly braces. This means that options like |draw| given in an
        |every to| do not actually influence the path. You can fix this by
        using the |append after command| option:
        %
\begin{codeexample}[]
\tikz[every to/.style={append after command={[draw,dashed]}}]
  \draw (0,0) to (3,2);
\end{codeexample}
    \end{stylekey}

    \medskip
    \textbf{Options.}\ \
    The \meta{options} given with the |to| allow you to influence the
    appearance of the |to path|. Mostly, these options are used to change the
    |to path|. This can be used to change the path from a straight line to,
    say, a curve.

    The path used is set using the following option:
    %
    \begin{key}{/tikz/to path=\meta{path}}
        Whenever a |to| operation is used, the \meta{path} is inserted. More
        precisely, the following path is added:
        %
        \begin{quote}
            |{[every to,|\meta{options}|] |\meta{path} |}|
        \end{quote}

        The \meta{options} are the options given to the |to| operation, the
        \meta{path} is the path set by this option |to path|.

        Inside the \meta{path}, different macros are used to reference the
        from- and to-coordinates. In detail, these are:
        %
        \begin{itemize}
            \item \declareandlabel{\tikztostart} will expand to the
                from-coordinate (without the parentheses).
            \item \declareandlabel{\tikztotarget} will expand to the
                to-coordinate.
            \item \declareandlabel{\tikztonodes} will expand to the nodes
                between the |to| operation and the coordinate. Furthermore,
                these nodes will have the |pos| option set implicitly.
        \end{itemize}

        Let us have a look at a simple example. The standard straight line for
        a |to| is achieved by the following \meta{path}:
        %
        \begin{quote}
            |-- (\tikztotarget) \tikztonodes|
        \end{quote}

        Indeed, this is the default setting for the path. When we write
        |(a) to (b)|, the \meta{path} will expand to |(a) -- (b)|, when we
        write
        %
        \begin{quote}
            |(a) to[red] node {x} (b)|
        \end{quote}
        %
        the \meta{path} will expand to
        %
        \begin{quote}
            |(a) -- (b) node[red] {x}|
        \end{quote}

        It is not possible to specify the path
        %
        \begin{quote}
            |-- \tikztonodes (\tikztotarget)|
        \end{quote}
        %
        since \tikzname\ does not allow one to have a macro after |--| that
        expands to a node.

        Now let us have a look at how we can modify the \meta{path} sensibly.
        The simplest way is to use a curve.
        %
\begin{codeexample}[]
\begin{tikzpicture}[to path={
    .. controls +(1,0) and +(1,0) .. (\tikztotarget) \tikztonodes}]

  \node (a) at (0,0) {a};
  \node (b) at (2,1) {b};
  \node (c) at (1,2) {c};

  \draw (a) to node {x} (b)
        (a) to          (c);
\end{tikzpicture}
\end{codeexample}

        Here is another example:
        %
\begin{codeexample}[]
\tikzset{
  my loop/.style={to path={
    .. controls +(80:1) and +(100:1) .. (\tikztotarget) \tikztonodes}},
  my state/.style={circle,draw}}

\begin{tikzpicture}[shorten >=2pt]
  \node [my state] (a) at (210:1) {$q_a$};
  \node [my state] (b) at (330:1) {$q_b$};

  \draw[->] (a) to           node[below]       {1} (b)
                to [my loop] node[above right] {0} (b);
\end{tikzpicture}
\end{codeexample}

        \begin{key}{/tikz/execute at begin to=\meta{code}}
            The \meta{code} is executed prior to the |to|. This can be used to
            draw one or more additional paths or to do additional computations.
        \end{key}

        \begin{key}{/tikz/execute at end to=\meta{code}}
            Works like the previous option, only this code is executed after
            the to path has been added.
            % FIXME : provide examples...
        \end{key}

        \begin{stylekey}{/tikz/every to (initially \normalfont empty)}
            This style is installed at the beginning of every to.
        \end{stylekey}
    \end{key}
\end{pathoperation}

There are a number of predefined |to path|s, see Section~\ref{library-to-paths}
for a reference.


\subsection{The Foreach Operation}

\begin{pathoperation}{foreach}{\meta{variables}\opt{\oarg{options}} |in|
        \marg{path commands}}
    The |foreach| operation can be used to repeatedly insert the \meta{path
    commands} into the current path. Naturally, the \meta{path commands} should
    internally reference some of the \meta{variables} so that you do not insert
    exactly the same path repeatedly, but rather variations. For historical
    reasons, you can also write |\foreach| instead of |foreach|.
    %
\begin{codeexample}[]
\tikz \draw (0,0) foreach \x in {1,...,3} { -- (\x,1) -- (\x,0) };
\end{codeexample}
    %
    See Section~\ref{section-foreach} for more details on the for-each-command.
\end{pathoperation}


\subsection{The Let Operation}

The \emph{let operation} is the first of a number of path operations that do
not actually extend that path, but have different, mostly local, effects.

\begin{pathoperation}{let}{\meta{assignment}
        \opt{|,|\meta{assignment}}%
        \opt{|,|\meta{assignment}\dots}\declare{| in |}}
    When this path operation is encountered, the \meta{assignment}s are
    evaluated, one by one. This will store coordinate and number in
    special \emph{registers} (which are local to \tikzname, they have
    nothing to do with \TeX\ registers). Subsequently, one can access the
    contents of these registers using the macros |\p|, |\x|, |\y|, and
    |\n|.

    The first kind of permissible \meta{assignment}s have the following form:
    %
    \begin{quote}
        |\n|\meta{number register}|={|\meta{formula}|}|
    \end{quote}
    %
    When an assignment has this form, the \meta{formula} is evaluated using the
    |\pgfmathparse| operation. The result is stored in the \meta{number
    register}. If the \meta{formula} involves a dimension anywhere (as in
    |2*3cm/2|), then the \meta{number register} stores the resulting dimension
    with a trailing |pt|.  A \meta{number register} can be named arbitrarily
    and is a normal \TeX\ parameter to the |\n| macro. Possible names are
    |{left corner}|, but also just a single digit like~|5|.

    Let us call the path that follows a let operation its \emph{body}. Inside
    the body, the |\n| macro can be used to access the register.
    %
    \begin{command}{\n\marg{number register}}
        When this macro is used on the left-hand side of an |=|-sign in a let
        operation, it has no effect and is just there for readability. When the
        macro is used on the right-hand side of an |=|-sign or in the body of
        the let operation, then it expands to the value stored in the
        \meta{number register}. This will either be a dimensionless number like
        |2.0| or a dimension like |5.6pt|.

        For instance, if we say |let \n1={1pt+2pt}, \n2={1+2} in ...|, then
        inside the |...| part the macro |\n1| will expand to |3pt| and |\n2|
        expands to |3|.
    \end{command}

    The second kind of \meta{assignments} have the following form:
    %
    \begin{quote}
        |\p|\meta{point register}|={|\meta{formula}|}|
    \end{quote}
    %
    Point position registers store a single point, consisting of an $x$-part
    and a $y$-part measured in \TeX\ points (|pt|). In particular, point
    registers do not store nodes or node names. Here is an example:
    %
\begin{codeexample}[preamble={\usetikzlibrary{calc}}]
\begin{tikzpicture}
  \draw [help lines] (0,0) grid (3,2);

  \draw let \p{foo} = (1,1), \p2 = (2,0) in
          (0,0) -- (\p2) -- (\p{foo});
\end{tikzpicture}
\end{codeexample}

    \begin{command}{\p\marg{point register}}
        When this macro is used on the left-hand side of an |=|-sign in a let
        operation, it has no effect and is just there for readability. When the
        macro is used on the right-hand side of an |=|-sign or in the body of
        the let operation, then it expands to the $x$-part (measured in \TeX\
        points) of the coordinate stored in the \meta{register}, followed, by a
        comma, followed by the $y$-part.

        For instance, if we say |let \p1=(1pt,1pt+2pt) in ...|, then inside the
        |...| part the macro |\p1| will expand to exactly the seven characters
        ``1pt,3pt''. This means that you when you write |(\p1)|, this expands
        to |(1pt,3pt)|, which is presumably exactly what you intended.
    \end{command}
    %
    \begin{command}{\x\marg{point register}}
        This macro expands just to the $x$-part of the point register. If we
        say as above, as we did above, |let \p1=(1pt,1pt+2pt) in ...|, then
        inside the |...| part the macro |\x1| expands to |1pt|.
    \end{command}
    %
    \begin{command}{\y\marg{point register}}
        Works like |\x|, only for the $y$-part.
    \end{command}
    %
    Note that the above macros are available only inside a let operation.

    Here is an example where let clauses are used to assemble a coordinate from
    the $x$-coordinate of a first point and the $y$-coordinate of a second
    point. Naturally, using the \verb!|-! notation, this could be written much
    more compactly.
    %
\begin{codeexample}[preamble={\usetikzlibrary{calc}}]
\begin{tikzpicture}
  \draw [help lines] (0,0) grid (3,2);

  \draw    (1,0) coordinate (first point)
        -- (3,2) coordinate (second point);

  \fill[red] let \p1 = (first point),
                 \p2 = (second point) in
               (\x1,\y2) circle [radius=2pt];
\end{tikzpicture}
\end{codeexample}

    Note that the effect of a let operation is local to the body of the let
    operation. If you wish to access a computed coordinate outside the body,
    you must use a |coordinate| path operation:
    %
\begin{codeexample}[preamble={\usetikzlibrary{calc}}]
\begin{tikzpicture}
  \draw [help lines] (0,0) grid (3,2);

  \path % let's define some points:
    let
      \p1        = (1,0),
      \p2        = (3,2),
      \p{center} = ($ (\p1) !.5! (\p2) $) % center
    in
      coordinate (p1) at (\p1)
      coordinate (p2) at (\p2)
      coordinate (center) at (\p{center});

  \draw (p1) -- (p2);
  \fill[red] (center) circle [radius=2pt];
\end{tikzpicture}
\end{codeexample}

    For a more useful application of the let operation, let us draw a circle
    that touches a given line:
    %
\begin{codeexample}[pre={\pgfmathsetseed{1}},preamble={\usetikzlibrary{calc}}]
\begin{tikzpicture}
  \draw [help lines] (0,0) grid (3,3);

  \coordinate (a) at (rnd,rnd);
  \coordinate (b) at (3-rnd,3-rnd);
  \draw (a) -- (b);

  \node (c) at (1,2) {x};

  \draw let \p1 = ($ (a)!(c)!(b) - (c) $),
            \n1 = {veclen(\x1,\y1)}
        in circle [at=(c), radius=\n1];
\end{tikzpicture}
\end{codeexample}
    %
\end{pathoperation}


\subsection{The Scoping Operation}

When \tikzname\ encounters and opening or a closing brace (|{| or~|}|) at some
point where a path operation should come, it will open or close a scope. All
options that can be applied ``locally'' will be scoped inside the scope. For
example, if you apply a transformation like |[xshift=1cm]| inside the scoped
area, the shifting only applies to the scope. On the other hand, an option like
|color=red| does not have any effect inside a scope since it can only be
applied to the path as a whole.

Concerning the effect of scopes on relative coordinates, please see
Section~\ref{section-scopes-relative}.


\subsection{The Node and Edge Operations}

The |node| operation adds a so-called node to a path. This operation is special
in the following sense: It does not change the current path in any way. In
other words, this operation is not really a path operation, but has an effect
that is ``external'' to the path. The |edge| operation has similar effect in
that it adds something \emph{after} the main path has been drawn. However, it
works like the |to| operation, that is, it adds a |to| path to the picture
after the main path has been drawn.

Since these operations are quite complex, they are described in the separate
Section~\ref{section-nodes}.


\subsection{The Graph Operation}

The |graph| operation can be used to specify easily how a large number of nodes
are connected. This operation is documented in a separate section, see
Section~\ref{section-library-graphs}.


\subsection{The Pic Operation}

The |pic| operation is used to insert a ``short picture'' (hence the ``short''
name) at the current position of the path. This operation is somewhat similar
to the |node| operation and discussed in detail in Section~\ref{section-pics}.


\subsection{The Attribute Animation Operation}

\begin{pathoperation}{:}{\meta{animation attribute}|=|\marg{options}}
    This path operation has the same effect as if you had said:
    %
    \begin{quote}
        |[animate = { myself:|\meta{animate attribute}|=|\marg{options}|} ]|
    \end{quote}
    %
    This causes an animation of \meta{animate attribute} to be added to the
    current path, see Section~\ref{section-tikz-animations} for details.
    %
\begin{codeexample}[width=2cm,preamble={\usetikzlibrary{animations}}]
\tikz \draw :xshift = {0s = "0cm", 30s = "-3cm", repeats} (0,0) circle (5mm);
\end{codeexample}
    %
\end{pathoperation}


\subsection{The PGF-Extra Operation}

In some cases you may need to ``do some calculations or some other stuff''
while a path is constructed. For this, you would like to suspend the
construction of the path and suspend \tikzname's parsing of the path, you would
then like to have some \TeX\ code executed, and would then like to resume the
parsing of the path. This effect can be achieved using the following path
operation |\pgfextra|. Note that this operation should only be used by real
experts and should only be used deep inside clever macros, not on normal paths.

\begin{command}{\pgfextra\marg{code}}
    This command may only be used inside a \tikzname\ path. There it is used
    like a normal path operation. The construction of the path is temporarily
    suspended and the \meta{code} is executed. Then, the path construction is
    resumed.
    %
\begin{codeexample}[]
\newdimen\mydim
\begin{tikzpicture}
  \mydim=1cm
  \draw (0pt,\mydim) \pgfextra{\mydim=2cm} -- (0pt,\mydim);
\end{tikzpicture}
\end{codeexample}
    %
\end{command}

\begin{command}{\pgfextra \meta{code} \texttt{\char`\\endpgfextra}}
    This is an alternative syntax for the |\pgfextra| command. If the code
    following |\pgfextra| does not start with a brace, the \meta{code} is
    executed until |\endpgfextra| is encountered. What actually happens is that
    when |\pgfextra| is not followed by a brace, this completely shuts down the
    \tikzname\ parser and |\endpgfextra| is a normal macro that restarts the
    parser.
    %
\begin{codeexample}[]
\newdimen\mydim
\begin{tikzpicture}
  \mydim=1cm
  \draw (0pt,\mydim)
    \pgfextra \mydim=2cm \endpgfextra -- (0pt,\mydim);
\end{tikzpicture}
\end{codeexample}
    %
\end{command}


\subsection{Interacting with the Soft Path subsystem}

During construction \tikzname\ stores the path internally as a \emph{soft
path}. Sometimes it is desirable to save a path during the stage of
construction, restore it elsewhere and continue using it. There are two keys to
facilitate this operation, which are explained below. To learn more about the
soft path subsystem, refer to section~\ref{section-soft-paths}.

\begin{key}{/tikz/save path=\meta{macro}}
    Save the current soft path into \meta{macro}.
\end{key}

\begin{key}{/tikz/use path=\meta{macro}}
    Set the current path to the soft path stored in \meta{macro}.
\end{key}

\begin{codeexample}[preamble={\usetikzlibrary{intersections}}]
\begin{tikzpicture}
  \path[save path=\pathA,name path=A] (0,1) to [bend left] (1,0);
  \path[save path=\pathB,name path=B]
    (0,0) .. controls (.33,.1) and (.66,.9) .. (1,1);

  \fill[name intersections={of=A and B}] (intersection-1) circle (1pt);

  \draw[blue][use path=\pathA];
  \draw[red] [use path=\pathB];
\end{tikzpicture}
\end{codeexample}

% % Copyright 2006 by Till Tantau
%
% This file may be distributed and/or modified
%
% 1. under the LaTeX Project Public License and/or
% 2. under the GNU Free Documentation License.
%
% See the file doc/generic/pgf/licenses/LICENSE for more details.

\section{Actions on Paths}

\subsection{Overview}

Once a path has been constructed, different things can be done with
it. It can be drawn (or stroked) with a ``pen,'' it can be filled with
a color or shading, it can be used for clipping subsequent drawing, it
can be used to specify the extend of the picture---or  any
combination of these actions at the same time.

To decide what is to be done with a path, two methods can be
used. First, you can use a special-purpose command like |\draw| to
indicate that the path should be drawn. However, commands like |\draw|
and |\fill| are just abbreviations for special cases of the more
general method: Here, the |\path| command is used to specify the
path. Then, options encountered on the path indicate what should be
done with the path.

For example, |\path (0,0) circle (1cm);| means ``This is a path
consisting of a circle around the origin. Do not do anything with it
(throw it away).'' However, if the option |draw| is encountered
anywhere on the path, the circle will be drawn. ``Anywhere'' is any
point on the path where an option can be given, which is everywhere
where a path command like |circle (1cm)| or |rectangle (1,1)| or even
just |(0,0)| would also be allowed. Thus, the following commands all
draw the same circle:
\begin{codeexample}[code only]
\path [draw] (0,0) circle (1cm);
\path (0,0) [draw] circle (1cm);
\path (0,0) circle (1cm) [draw];
\end{codeexample}
Finally, |\draw (0,0) circle (1cm);| also draws a path, because
|\draw| is an abbreviation for |\path [draw]| and thus the command
expands to the first line of the above example.

Similarly, |\fill| is an abbreviation for |\path[fill]| and
|\filldraw| is an abbreviation for the command
|\path[fill,draw]|. Since options accumulate, the following commands
all have the same effect:
\begin{codeexample}[code only]
\path [draw,fill]   (0,0) circle (1cm);
\path [draw] [fill] (0,0) circle (1cm);
\path [fill] (0,0) circle (1cm) [draw];
\draw [fill] (0,0) circle (1cm);
\fill (0,0) [draw] circle (1cm);
\filldraw (0,0) circle (1cm);
\end{codeexample}

In the following subsection the different actions that
can be performed on a path are explained. The following commands are abbreviations for
certain sets of actions, but for many useful combinations there are no
abbreviations:

\begin{command}{\draw}
  Inside |{tikzpicture}| this is an abbreviation for |\path[draw]|.
\end{command}

\begin{command}{\fill}
  Inside |{tikzpicture}| this is an abbreviation for |\path[fill]|.
\end{command}

\begin{command}{\filldraw}
  Inside |{tikzpicture}| this is an abbreviation for |\path[fill,draw]|.
\end{command}

\begin{command}{\pattern}
  Inside |{tikzpicture}| this is an abbreviation for |\path[pattern]|.
\end{command}

\begin{command}{\shade}
  Inside |{tikzpicture}| this is an abbreviation for |\path[shade]|.
\end{command}

\begin{command}{\shadedraw}
  Inside |{tikzpicture}| this is an abbreviation for |\path[shade,draw]|.
\end{command}

\begin{command}{\clip}
  Inside |{tikzpicture}| this is an abbreviation for |\path[clip]|.
\end{command}

\begin{command}{\useasboundingbox}
  Inside |{tikzpicture}| this is an abbreviation for |\path[use as bounding box]|.
\end{command}



\subsection{Specifying a Color}

The most unspecific option for setting colors is the following:

\begin{key}{/tikz/color=\meta{color name}}
  \indexoption{color option}%
  This option sets the color that is used for fill, drawing, and text
  inside the current scope. Any special settings for filling colors or
  drawing colors are immediately ``overruled'' by this option.

  The \meta{color name} is the name of a previously defined color. For
  \LaTeX\ users, this is just a normal ``\LaTeX-color'' and the
  |xcolor| extensions are allowed. Here is an example:

\begin{codeexample}[]
\tikz \fill[color=red!20] (0,0) circle (1ex);
\end{codeexample}

  It is possible to ``leave out'' the |color=| part and you can also
  write:
\begin{codeexample}[]
\tikz \fill[red!20] (0,0) circle (1ex);
\end{codeexample}
  What happens is that every option that \tikzname\ does not know, like
  |red!20|, gets a ``second chance'' as a color name.

  For plain \TeX\ users, it is not so easy to specify colors since
  plain \TeX\ has no ``standardized'' color naming
  mechanism. Because of this, \pgfname\ emulates the |xcolor| package,
  though the emulation is \emph{extremely basic} (more precisely, what
  I could hack together in two hours or so). The emulation allows you
  to do the following:
  \begin{itemize}
  \item Specify a new color using |\definecolor|. Only the two color
    models |gray| and |rgb| are supported\footnote{Con\TeX t users should be aware that \texttt{\textbackslash definecolor} has a different meaning in Con\TeX t. There is a low-level equivalent named \texttt{\textbackslash pgfutil@definecolor} which can be used instead.}.%
    \example |\definecolor{orange}{rgb}{1,0.5,0}|
  \item Use |\colorlet| to define a new color based on an old
    one. Here, the |!| mechanism is supported, though only ``once''
    (use multiple |\colorlet| for more fancy colors).
    \example |\colorlet{lightgray}{black!25}|
  \item Use |\color|\marg{color name} to set the color in the current
    \TeX\ group. |\aftergroup|-hackery is used to restore the color
    after the group.
  \end{itemize}
\end{key}

As pointed out above, the |color=| option applies to ``everything''
(except to shadings), which is not always what you want. Because of
this, there are several more specialized color options. For example,
the |draw=| option sets the color used for drawing, but does not
modify the color used for filling. These color options are documented
where the path action they influence is described.


\subsection{Drawing a Path}

You can draw a path using the following option:
\begin{key}{/tikz/draw=\meta{color} (default \normalfont is scope's color setting)}
  Causes the path to be drawn. ``Drawing'' (also known as
  ``stroking'') can be thought of as picking up a pen and moving it
  along the path, thereby leaving ``ink'' on the canvas.

  There are numerous parameters that influence how a line is drawn,
  like the thickness or the dash pattern. These options are explained
  below.

  If the optional \meta{color} argument is given, drawing is done
  using the given \meta{color}. This color can be different from the
  current filling color, which allows you to draw and fill a path with
  different colors. If no \meta{color} argument is given, the last
  usage of the |color=| option is used.

  If the special color name |none| is given, this option causes
  drawing to be ``switched off.'' This is useful if a style has
  previously switched on drawing and you locally wish to undo this
  effect.

  Although this option is normally used on paths to indicate that the
  path should be drawn, it also makes sense to use the option with a
  |{scope}| or |{tikzpicture}| environment. However, this will
  \emph{not} cause all paths to be drawn. Instead, this just sets the
  \meta{color} to be used for drawing paths inside the environment.

\begin{codeexample}[]
\begin{tikzpicture}
  \path[draw=red] (0,0) -- (1,1) -- (2,1) circle (10pt);
\end{tikzpicture}
\end{codeexample}
\end{key}


The following subsections list the different options that influence
how a path is drawn. All of these options only have an effect if the
|draw| option is given (directly or indirectly).

\subsubsection{Graphic Parameters: Line Width, Line Cap, and Line Join}

\label{section-cap-joins}

\begin{key}{/tikz/line width=\meta{dimension} (initially 0.4pt)}
  Specifies the line width. Note the space.

\begin{codeexample}[]
  \tikz \draw[line width=5pt] (0,0) -- (1cm,1.5ex);
\end{codeexample}
\end{key}

There are a number of predefined styles that provide more ``natural''
ways of setting the line width. You can also redefine these
styles.

\begin{stylekey}{/tikz/ultra thin}
  Sets the line width to 0.1pt.
\begin{codeexample}[]
  \tikz \draw[ultra thin] (0,0) -- (1cm,1.5ex);
\end{codeexample}
\end{stylekey}

\begin{stylekey}{/tikz/very thin}
  Sets the line width to 0.2pt.
\begin{codeexample}[]
  \tikz \draw[very thin] (0,0) -- (1cm,1.5ex);
\end{codeexample}
\end{stylekey}

\begin{stylekey}{/tikz/thin}
  Sets the line width to 0.4pt.
\begin{codeexample}[]
  \tikz \draw[thin] (0,0) -- (1cm,1.5ex);
\end{codeexample}
\end{stylekey}

\begin{stylekey}{/tikz/semithick}
  Sets the line width to 0.6pt.
\begin{codeexample}[]
  \tikz \draw[semithick] (0,0) -- (1cm,1.5ex);
\end{codeexample}
\end{stylekey}

\begin{stylekey}{/tikz/thick}
  Sets the line width to 0.8pt.
\begin{codeexample}[]
  \tikz \draw[thick] (0,0) -- (1cm,1.5ex);
\end{codeexample}
\end{stylekey}

\begin{stylekey}{/tikz/very thick}
  Sets the line width to 1.2pt.
\begin{codeexample}[]
  \tikz \draw[very thick] (0,0) -- (1cm,1.5ex);
\end{codeexample}
\end{stylekey}

\begin{stylekey}{/tikz/ultra thick}
  Sets the line width to 1.6pt.
\begin{codeexample}[]
  \tikz \draw[ultra thick] (0,0) -- (1cm,1.5ex);
\end{codeexample}
\end{stylekey}


\label{section-line-cap}

\begin{key}{/tikz/line cap=\meta{type} (initially butt)}
  Specifies how lines ``end.'' Permissible \meta{type} are |round|,
  |rect|, and |butt|. They have the following effects:

\begin{codeexample}[]
\begin{tikzpicture}
  \begin{scope}[line width=10pt]
    \draw[line cap=rect]  (0,0 ) -- (1,0);
    \draw[line cap=butt]  (0,.5) -- (1,.5);
    \draw[line cap=round] (0,1 ) -- (1,1);
  \end{scope}
  \draw[white,line width=1pt]
    (0,0 ) -- (1,0) (0,.5) -- (1,.5) (0,1 ) -- (1,1);
\end{tikzpicture}
\end{codeexample}
\end{key}

\begin{key}{/tikz/line join=\meta{type} (initially miter)}
  Specifies how lines ``join.'' Permissible \meta{type} are |round|,
  |bevel|, and |miter|. They have the following effects:

\begin{codeexample}[]
\begin{tikzpicture}[line width=10pt]
  \draw[line join=round] (0,0) -- ++(.5,1) -- ++(.5,-1);
  \draw[line join=bevel] (1.25,0) -- ++(.5,1) -- ++(.5,-1);
  \draw[line join=miter] (2.5,0) -- ++(.5,1) -- ++(.5,-1);
  \useasboundingbox (0,1.5); % enlarge bounding box
\end{tikzpicture}
\end{codeexample}

  \begin{key}{/tikz/miter limit=\meta{factor} (initially 10)}
    When you use the miter join and there is a very sharp corner (a
    small angle), the miter join may protrude very far over the actual
    joining point. In this case, if it were to protrude by
    more than \meta{factor} times the line width, the miter join is
    replaced by a bevel join.

\begin{codeexample}[]
\begin{tikzpicture}[line width=5pt]
  \draw                 (0,0) -- ++(5,.5) -- ++(-5,.5);
  \draw[miter limit=25] (6,0) -- ++(5,.5) -- ++(-5,.5);
  \useasboundingbox (14,0); % make bounding box bigger
\end{tikzpicture}
\end{codeexample}
  \end{key}
\end{key}

\subsubsection{Graphic Parameters: Dash Pattern}

\begin{key}{/tikz/dash pattern=\meta{dash pattern}}
  Sets the dashing pattern. The syntax is the same as in
  \textsc{metafont}. For example following pattern
  |on 2pt off 3pt on 4pt off 4pt| means ``draw
  2pt, then leave out 3pt, then draw 4pt once more, then leave out 4pt
  again, repeat''.

\begin{codeexample}[]
\begin{tikzpicture}[dash pattern=on 2pt off 3pt on 4pt off 4pt]
  \draw (0pt,0pt) -- (3.5cm,0pt);
\end{tikzpicture}
\end{codeexample}
\end{key}

\begin{key}{/tikz/dash phase=\meta{dash phase} (initially 0pt)}
  Shifts the start of the dash pattern by \meta{phase}.

\begin{codeexample}[]
\begin{tikzpicture}[dash pattern=on 20pt off 10pt]
  \draw[dash phase=0pt] (0pt,3pt) -- (3.5cm,3pt);
  \draw[dash phase=10pt] (0pt,0pt) -- (3.5cm,0pt);
\end{tikzpicture}
\end{codeexample}
\end{key}

As for the line thickness, some predefined styles allow you to set the
dashing conveniently.

\begin{stylekey}{/tikz/solid}
  Shorthand for setting a solid line as ``dash pattern.'' This is the default.

\begin{codeexample}[]
\tikz \draw[solid] (0pt,0pt) -- (50pt,0pt);
\end{codeexample}
\end{stylekey}

\begin{stylekey}{/tikz/dotted}
  Shorthand for setting a dotted dash pattern.

\begin{codeexample}[]
\tikz \draw[dotted] (0pt,0pt) -- (50pt,0pt);
\end{codeexample}
\end{stylekey}

\begin{stylekey}{/tikz/densely dotted}
  Shorthand for setting a densely dotted dash pattern.

\begin{codeexample}[]
\tikz \draw[densely dotted] (0pt,0pt) -- (50pt,0pt);
\end{codeexample}
\end{stylekey}

\begin{stylekey}{/tikz/loosely dotted}
  Shorthand for setting a loosely dotted dash pattern.

\begin{codeexample}[]
\tikz \draw[loosely dotted] (0pt,0pt) -- (50pt,0pt);
\end{codeexample}
\end{stylekey}

\begin{stylekey}{/tikz/dashed}
  Shorthand for setting a dashed dash pattern.

\begin{codeexample}[]
\tikz \draw[dashed] (0pt,0pt) -- (50pt,0pt);
\end{codeexample}
\end{stylekey}

\begin{stylekey}{/tikz/densely dashed}
  Shorthand for setting a densely dashed dash pattern.

\begin{codeexample}[]
\tikz \draw[densely dashed] (0pt,0pt) -- (50pt,0pt);
\end{codeexample}
\end{stylekey}

\begin{stylekey}{/tikz/loosely dashed}
  Shorthand for setting a loosely dashed dash pattern.

\begin{codeexample}[]
\tikz \draw[loosely dashed] (0pt,0pt) -- (50pt,0pt);
\end{codeexample}
\end{stylekey}


\begin{stylekey}{/tikz/dash dot}
  Shorthand for setting a dashed and dotted dash pattern.

\begin{codeexample}[]
\tikz \draw[dash dot] (0pt,0pt) -- (50pt,0pt);
\end{codeexample}
\end{stylekey}

\begin{stylekey}{/tikz/densely dash dot}
  Shorthand for setting a densely dashed and dotted dash pattern.

\begin{codeexample}[]
\tikz \draw[densely dash dot] (0pt,0pt) -- (50pt,0pt);
\end{codeexample}
\end{stylekey}

\begin{stylekey}{/tikz/loosely dash dot}
  Shorthand for setting a loosely dashed and dotted dash pattern.

\begin{codeexample}[]
\tikz \draw[loosely dash dot] (0pt,0pt) -- (50pt,0pt);
\end{codeexample}
\end{stylekey}


\begin{stylekey}{/tikz/dash dot dot}
  Shorthand for setting a dashed and dotted dash pattern with more dots.

\begin{codeexample}[]
\tikz \draw[dash dot dot] (0pt,0pt) -- (50pt,0pt);
\end{codeexample}
\end{stylekey}

\begin{stylekey}{/tikz/densely dash dot dot}
  Shorthand for setting a densely dashed and dotted dash pattern with more dots.

\begin{codeexample}[]
\tikz \draw[densely dash dot dot] (0pt,0pt) -- (50pt,0pt);
\end{codeexample}
\end{stylekey}

\begin{stylekey}{/tikz/loosely dash dot dot}
  Shorthand for setting a loosely dashed and dotted dash pattern with more dots.

\begin{codeexample}[]
\tikz \draw[loosely dash dot dot] (0pt,0pt) -- (50pt,0pt);
\end{codeexample}
\end{stylekey}


\subsubsection{Graphic Parameters: Draw Opacity}

When a line is drawn, it will normally ``obscure'' everything behind
it as if you had used perfectly opaque ink. It is also possible to ask
\tikzname\ to use an ink that is a little bit (or a big bit)
transparent using the |draw opacity| option. This is explained in
Section~\ref{section-tikz-transparency} on transparency in more detail.



\subsubsection{Graphic Parameters: Double Lines and Bordered Lines}

\begin{key}{/tikz/double=\meta{core color} (default white)}
  This option causes ``two'' lines to be drawn instead of a single
  one. However, this is not what really happens. In reality, the path
  is drawn twice. First, with the normal drawing color, secondly with
  the \meta{core color}, which is normally |white|. Upon the second
  drawing, the line width is reduced. The net effect is that it
  appears as if two lines had been drawn and this works well even with
  complicated, curved paths:

\begin{codeexample}[]
\tikz \draw[double]
  plot[smooth cycle] coordinates{(0,0) (1,1) (1,0) (0,1)};
\end{codeexample}

  You can also use the doubling option to create an effect in which a
  line seems to have a certain ``border'':

\begin{codeexample}[]
\begin{tikzpicture}
  \draw (0,0) -- (1,1);
  \draw[draw=white,double=red,very thick] (0,1) -- (1,0);
\end{tikzpicture}
\end{codeexample}
\end{key}

\begin{key}{/tikz/double distance=\meta{dimension} (initially 0.6pt)}
  Sets the distance the ``two'' lines are spaced apart. In reality,
  this is the thickness of the line that is used
  to draw the path for the second time. The thickness of the
  \emph{first} time the path is drawn is twice the normal line width
  plus the given \meta{dimension}. As a side-effect, this option
  ``selects'' the |double| option.

\begin{codeexample}[]
\begin{tikzpicture}
  \draw[very thick,double]              (0,0) arc (180:90:1cm);
  \draw[very thick,double distance=2pt] (1,0) arc (180:90:1cm);
  \draw[thin,double distance=2pt]       (2,0) arc (180:90:1cm);
\end{tikzpicture}
\end{codeexample}
\end{key}

\begin{key}{/tikz/double distance between line centers=\meta{dimension}}
  This option works like |double distance|, only the distance is not
  the distance between (inner) borders of the two main lines, but
  between their centers. Thus, the thickness the
  \emph{first} time the path is drawn is the normal line width
  plus the given \meta{dimension}, while the line width of the
  \emph{second} line that is drawn is \meta{dimension} minus the
  normal line width. As a side-effect, this option ``selects'' the
  |double| option.

\begin{codeexample}[]
\begin{tikzpicture}[double distance between line centers=3pt]
  \foreach \lw in {0.5,1,1.5,2,2.5}
    \draw[line width=\lw pt,double] (\lw,0) -- ++(4mm,0);
\end{tikzpicture}
\end{codeexample}
\begin{codeexample}[]
\begin{tikzpicture}[double distance=3pt]
  \foreach \lw in {0.5,1,1.5,2,2.5}
    \draw[line width=\lw pt,double] (\lw,0) -- ++(4mm,0);
\end{tikzpicture}
\end{codeexample}
\end{key}

\begin{stylekey}{/tikz/double equal sign distance}
  This style selects a double line distance such that it corresponds
  to the distance of the two lines in an equal sign.
\begin{codeexample}[]
\Huge $=\implies$\tikz[baseline,double equal sign distance]
                    \draw[double,thick,-implies](0,0.55ex) --++(3ex,0);
\end{codeexample}
\begin{codeexample}[]
\normalsize $=\implies$\tikz[baseline,double equal sign distance]
                          \draw[double,-implies](0,0.6ex) --++(3ex,0);
\end{codeexample}
\begin{codeexample}[]
\tiny $=\implies$\tikz[baseline,double equal sign distance]
                   \draw[double,very thin,-implies](0,0.5ex) -- ++(3ex,0);
\end{codeexample}
\end{stylekey}



\subsection{Adding Arrow Tips to a Path}
\label{section-arrow-tip-action}

In different situations, \tikzname\ will add arrow tips to the end of
a path. For this to happen, a number of different things need to be
specified:

\begin{enumerate}
\item You must have used the |arrows| key, explained in detail in
  Section~\ref{section-tikz-arrows}, to setup which kinds of arrow
  tips you would like.
\item The path may not be closed (like a circle or a rectangle) and,
  if it consists of several subpath, further restrictions apply as
  explained in Section~\ref{section-tikz-arrows}.
\item The |tips| key must be set to an appropriate value, see
  Section~\ref{section-tikz-arrows} once more.
\end{enumerate}

For the current section on paths, it is only important that when you
add the |tips| option to a path that is not drawn, arrow tips will
still be added at the beginning and at the end of the current
path. This is true even when ``only'' arrow tips get drawn for a path
without drawing the path itself. Here is an example:
\begin{codeexample}[width=2cm]
\tikz \path[tips, -{Latex[open,length=10pt,bend]}] (0,0) to[bend left] (1,0);
\end{codeexample}
\begin{codeexample}[width=2cm]
\tikz \draw[tips, -{Latex[open,length=10pt,bend]}] (0,0) to[bend left] (1,0);
\end{codeexample}


\subsection{Filling a Path}
\label{section-rules}
To fill a path, use the following option:
\begin{key}{/tikz/fill=\meta{color} (default \normalfont is scope's color setting)}
  This option causes the path to be filled. All unclosed parts of the
  path are first closed, if necessary. Then, the area enclosed by the
  path is filled with the current filling color, which is either the
  last color set using the general |color=| option or the optional
  color \meta{color}. For self-intersection paths and for paths
  consisting of several closed areas, the ``enclosed area'' is
  somewhat complicated to define and two different definitions exist,
  namely the nonzero winding number rule and the even odd rule, see
  the explanation of these options, below.

  Just as for the |draw| option, setting \meta{color} to |none|
  disables filling locally.

\begin{codeexample}[]
\begin{tikzpicture}
  \fill (0,0) -- (1,1) -- (2,1);
  \fill (4,0) circle (.5cm)  (4.5,0) circle (.5cm);
  \fill[even odd rule] (6,0) circle (.5cm)  (6.5,0) circle (.5cm);
  \fill (8,0) -- (9,1) -- (10,0) circle (.5cm);
\end{tikzpicture}
\end{codeexample}

  If the |fill| option is used together with the |draw| option (either
  because both are given as options or because a |\filldraw| command
  is used), the path is filled \emph{first}, then the path is drawn
  \emph{second}. This is especially useful if different colors are
  selected for drawing and for filling. Even if the same color is
  used, there is a difference between this command and a plain
  |fill|: A ``filldrawn'' area will be slightly larger than a filled
  area because of the thickness of the ``pen.''

\begin{codeexample}[]
\begin{tikzpicture}[fill=yellow!80!black,line width=5pt]
  \filldraw (0,0) -- (1,1) -- (2,1);
  \filldraw (4,0) circle (.5cm)  (4.5,0) circle (.5cm);
  \filldraw[even odd rule] (6,0) circle (.5cm)  (6.5,0) circle (.5cm);
  \filldraw (8,0) -- (9,1) -- (10,0) circle (.5cm);
\end{tikzpicture}
\end{codeexample}
\end{key}



\subsubsection{Graphic Parameters: Fill Pattern}

\label{section-fill-pattern}
Instead of filling a path with a single solid color, it is also
possible to fill it with a \emph{tiling pattern}. Imagine a small tile
that contains a simple picture like a star. Then these tiles are
(conceptually) repeated infinitely in all directions, but clipped
against the path.

Tiling patterns come in two variants: \emph{inherently
  colored patterns} and \emph{form-only patterns}. An inherently colored
pattern is, say, a red star with a black border and will always look
like this. A form-only pattern may have a different color each time
it is used, only the form of the pattern will stay the same. As such,
form-only patterns do not have any colors of their own, but when it
is used the current \emph{pattern color} is used as its color.

Patterns are not overly flexible. In particular, it is not possible to
change the size or orientation of a pattern without declaring a new
pattern. For complicated cases, it may be easier to use two nested
|\foreach| statements to simulate a pattern, but patterns are rendered
\emph{much} more quickly than simulated ones.

\begin{key}{/tikz/pattern=\meta{name} (default \normalfont is scope's pattern)}
  This option causes the path to be filled with a pattern. If the
  \meta{name} is given, this pattern is used, otherwise the pattern
  set in the enclosing scope is used. As for the |draw| and |fill|
  options, setting \meta{name} to |none| disables filling locally.

  The pattern works like a fill color. In particular, setting a new
  fill color will fill the path with a solid color once more.

  Strangely, no \meta{name}s are permissible by default. You need to
  load for instance the |patterns| library, see
  Section~\ref{section-library-patterns}, to install predefined
  patterns.

\begin{codeexample}[]
\begin{tikzpicture}
  \draw[pattern=dots] (0,0) circle (1cm);
  \draw[pattern=fivepointed stars] (0,0) rectangle (3,1);
\end{tikzpicture}
\end{codeexample}
\end{key}

\begin{key}{/tikz/pattern color=\meta{color}}
  This option is used to set the color to be used for form-only
  patterns. This option has no effect on inherently colored patterns.

\begin{codeexample}[]
\begin{tikzpicture}
  \draw[pattern color=red,pattern=fivepointed stars]  (0,0) circle (1cm);
  \draw[pattern color=blue,pattern=fivepointed stars] (0,0) rectangle (3,1);
\end{tikzpicture}
\end{codeexample}

\begin{codeexample}[]
\begin{tikzpicture}
  \def\mypath{(0,0) -- +(0,1) arc (180:0:1.5cm) -- +(0,-1)}
  \fill   [red]                                \mypath;
  \pattern[pattern color=white,pattern=bricks] \mypath;
\end{tikzpicture}
\end{codeexample}
\end{key}


\subsubsection{Graphic Parameters: Interior Rules}

The following two options can be used to decide how interior points
should be determined:
\begin{key}{/tikz/nonzero rule}
  If this rule is used (which is the default), the following method is
  used to determine whether a given point is ``inside'' the path: From
  the point, shoot a ray in some direction towards infinity (the
  direction is chosen such that no strange borderline cases
  occur). Then the ray may hit the path. Whenever it hits the path, we
  increase or decrease a counter, which is initially zero. If the ray
  hits the path as the path goes ``from left to right'' (relative to
  the ray), the counter is increased, otherwise it is decreased. Then,
  at the end, we check whether the counter is nonzero (hence the
  name). If so, the point is deemed to lie ``inside,'' otherwise it is
  ``outside.'' Sounds complicated? It is.

\begin{codeexample}[]
\begin{tikzpicture}
  \filldraw[fill=yellow!80!black]
  % Clockwise rectangle
  (0,0) -- (0,1) -- (1,1) -- (1,0) -- cycle
  % Counter-clockwise rectangle
  (0.25,0.25) -- (0.75,0.25) -- (0.75,0.75) -- (0.25,0.75) -- cycle;

  \draw[->] (0,1) -- (.4,1);
  \draw[->] (0.75,0.75) -- (0.3,.75);

  \draw[->] (0.5,0.5) -- +(0,1) node[above] {crossings: $-1+1 = 0$};

  \begin{scope}[yshift=-3cm]
    \filldraw[fill=yellow!80!black]
    % Clockwise rectangle
    (0,0) -- (0,1) -- (1,1) -- (1,0) -- cycle
    % Clockwise rectangle
    (0.25,0.25) -- (0.25,0.75) -- (0.75,0.75) -- (0.75,0.25) -- cycle;

    \draw[->] (0,1) -- (.4,1);
    \draw[->] (0.25,0.75) -- (0.4,.75);

    \draw[->] (0.5,0.5) -- +(0,1) node[above] {crossings: $1+1 = 2$};
  \end{scope}
\end{tikzpicture}
\end{codeexample}
\end{key}

\begin{key}{/tikz/even odd rule}
  This option causes a different method to be used for determining the
  inside and outside of paths. While it is less flexible, it turns out
  to be more intuitive.

  With this method, we also shoot rays from the point for which we
  wish to determine whether it is inside or outside the filling
  area. However, this time we only count how often we ``hit'' the path
  and declare the point to be ``inside'' if the number of hits is odd.

  Using the even-odd rule, it is easy to ``drill holes'' into a path.

\begin{codeexample}[]
\begin{tikzpicture}
  \filldraw[fill=yellow!80!black,even odd rule]
    (0,0) rectangle (1,1) (0.5,0.5) circle (0.4cm);
  \draw[->] (0.5,0.5) -- +(0,1) [above] node{crossings: $1+1 = 2$};
\end{tikzpicture}
\end{codeexample}
\end{key}



\subsubsection{Graphic Parameters: Fill Opacity}

\label{section-fill-opacity}
Analogously to the |draw opacity|, you can also set the fill opacity. Please see Section~\ref{section-tikz-transparency} for more
details.


\subsection{Generalized Filling: Using Arbitrary Pictures to Fill a Path}

Sometimes you wish to ``fill'' a path with something even more
complicated than a pattern, let alone a single color. For instance,
you might wish to use an image to fill the path or some other,
complicated drawing. In principle, this effect can be achieved
by first using the path for clipping and then, subsequently, drawing
the desired image or picture. However, there is an option that makes
this process much easier:

\begin{key}{/tikz/path picture=\meta{code}}
  When this option is given on a path and when the \meta{code} is not
  empty, the following happens: After all other ``filling'' operations
  are done with the path, which are caused by the options |fill|,
  |pattern| and  |shade|, a local scope is opened and the path is
  temporarily installed as a clipping path. Then, the \meta{code} is
  executed, which can now draw something. Then, the local scope ends
  and, possibly, the path is stroked, provided the |draw| option has
  been given.

  As with other keys like |fill| or |draw| this option needs to be given on a path, setting the |path picture| outside a path has no effect (the path picture is cleared at the beginning of each path).

  The \meta{code} can be any normal \tikzname\ code like |\draw ...|
  or |\node ...|. As always, when you include an external graphic, you need to put it inside a |\node|.

  Note that no special actions are taken to transform the origin in
  any way. This means that the coordinate |(0,0)| is still where is
  was when the path was being constructed and not -- as one might
  expect -- at the lower left corner of the path. However, you can use
  the following special node to access the size of the path:
  \begin{predefinednode}{path picture bounding box}
    This node is of shape |rectangle|. Its size and position are those
    of |current path bounding box| just before the \meta{code}
    of the path picture started to be executed. The \meta{code} can
    construct its own paths, so accessing the
    |current path bounding box| inside the \meta{code} yields the
    bounding box of any path that is currently being constructed
    inside the \meta{code}.
  \end{predefinednode}

\begin{codeexample}[]
\begin{tikzpicture}
  \draw [help lines] (0,0) grid (3,2);
  \filldraw [fill=blue!10,draw=blue,thick] (1.5,1) circle (1)
    [path picture={
      \node at (path picture bounding box.center) {
        This is a long text.
      };}
    ];
\end{tikzpicture}
\end{codeexample}

\begin{codeexample}[]
\begin{tikzpicture}[cross/.style={path picture={
      \draw[black]
            (path picture bounding box.south east) --
            (path picture bounding box.north west)
            (path picture bounding box.south west) --
            (path picture bounding box.north east);
    }}]
  \draw [help lines] (0,0) grid (3,2);
  \filldraw [cross,fill=blue!10,draw=blue,thick] (1,1) circle (1);
  \path     [cross,top color=red,draw=red,thick] (2,0) -- (3,2) -- (3,0);
\end{tikzpicture}
\end{codeexample}

\begin{codeexample}[]
  \begin{tikzpicture}[path image/.style={
      path picture={
        \node at (path picture bounding box.center) {
          \includegraphics[height=3cm]{#1}
        };}}]
  \draw     [help lines] (0,0) grid (3,2);

  \draw [path image=brave-gnu-world-logo,draw=blue,thick]
          (0,1) circle (1);
  \draw [path image=brave-gnu-world-logo,draw=red,very thick,->]
          (1,0) parabola[parabola height=2cm] (3,0);

\end{tikzpicture}
\end{codeexample}
\end{key}


\subsection{Shading a Path}

You can shade a path using the |shade| option. A shading is like a
filling, only the shading changes its color smoothly from one color to
another.

\begin{key}{/tikz/shade}
  Causes the path to be shaded using the currently selected shading
  (more on this later). If this option is used together with the
  |draw| option, then the path is first shaded, then drawn.

  It is not an error to use this option together with the |fill|
  option, but it makes no sense.

\begin{codeexample}[]
\tikz \shade (0,0) circle (1ex);
\end{codeexample}

\begin{codeexample}[]
\tikz \shadedraw (0,0) circle (1ex);
\end{codeexample}
\end{key}

For some shadings it is not really clear how they can ``fill'' the
path. For example, the |ball| shading normally looks like this: \tikz
\shade[shading=ball] (0,0) circle (0.75ex);. How is this supposed to
shade a rectangle? Or a triangle?

To solve this problem, the predefined shadings like |ball| or |axis|
fill a large rectangle completely in a sensible way. Then, when the
shading is used to ``shade'' a path, what actually happens is that the
path is temporarily used for clipping and then the rectangular shading
is drawn, scaled and shifted such that all parts of the path are
filled.

The default shading is a smooth transition from gray
to white and from top to bottom. However, other shadings are also
possible, for example a shading that will sweep a color from the
center to the corners outward. To choose the shading, you can use the
|shading=| option, which will also automatically invoke the |shade|
option. Note that this does \emph{not} change the shading color, only
the way the colors sweep. For changing the colors, other options are
needed, which are explained below.

\begin{key}{/tikz/shading=\meta{name}}
  This selects a shading named \meta{name}. The following shadings are
  predefined: |axis|, |radial|, and |ball|.
\begin{codeexample}[]
\tikz \shadedraw [shading=axis] (0,0) rectangle (1,1);
\tikz \shadedraw [shading=radial] (0,0) rectangle (1,1);
\tikz \shadedraw [shading=ball] (0,0) circle (.5cm);
\end{codeexample}

  The shadings as well as additional shadings are described in more
  detail in Section~\ref{section-library-shadings}.

  To change the color of a shading, special options are needed like
  |left color|, which sets the color of an axis shading from left to
  right. These options implicitly also select the correct shading type,
  see the following example
\begin{codeexample}[]
\tikz \shadedraw [left color=red,right color=blue]
    (0,0) rectangle (1,1);
\end{codeexample}

  For a complete list of the possible options see
  Section~\ref{section-library-shadings} once more.

  \begin{key}{/tikz/shading angle=\meta{degrees} (initially 0)}
    This option rotates the shading (not the path!) by the given
    angle. For example, we can turn a top-to-bottom axis shading into a
    left-to-right shading by rotating it by $90^\circ$.

\begin{codeexample}[]
\tikz \shadedraw [shading=axis,shading angle=90] (0,0) rectangle (1,1);
\end{codeexample}
  \end{key}
\end{key}

You can also define new shading types yourself. However, for this, you
need to use the basic layer directly, which is, well, more basic and
harder to use. Details on how to create a shading appropriate for
filling paths are given in Section~\ref{section-shading-a-path}.



\subsection{Establishing a Bounding Box}

\pgfname\ is reasonably good at keeping track of the size of your picture
and reserving just the right amount of space for it in the main
document. However, in some cases you may want to say things like
``do not count this for the picture size'' or ``the picture is
actually a little large.'' For this you can use the option
|use as bounding box| or the command |\useasboundingbox|, which is just
a shorthand for |\path[use as bounding box]|.

\begin{key}{/tikz/use as bounding box}
  Normally, when this option is given on a path, the bounding box of
  the present path is used to determine the size of the picture and
  the size of all \emph{subsequent} paths are
  ignored. However, if there were previous path operations that have
  already established a larger bounding box, it will not be made
  smaller by this operation (consider the |\pgfresetboundingbox| command
  to reset the previous bounding box).

  In a sense, |use as bounding box| has the same effect as clipping
  all subsequent drawing against the current path---without actually
  doing the clipping, only making \pgfname\ treat everything as if it
  were clipped.

  The first application of this option is to have a |{tikzpicture}|
  overlap with the main text:

\begin{codeexample}[]
Left of picture\begin{tikzpicture}
  \draw[use as bounding box] (2,0) rectangle (3,1);
  \draw (1,0) -- (4,.75);
\end{tikzpicture}right of picture.
\end{codeexample}

  In a second application this option can be used to get better
  control over the white space around the picture:

\begin{codeexample}[]
Left of picture
\begin{tikzpicture}
  \useasboundingbox (0,0) rectangle (3,1);
  \fill (.75,.25) circle (.5cm);
\end{tikzpicture}
right of picture.
\end{codeexample}

  Note: If this option is used on a path inside a \TeX\ group (scope),
  the effect ``lasts'' only until the end of the scope. Again, this
  behavior is the same as for clipping.


  Consider using |\useasboundingbox| together with |\pgfresetboundingbox| in order to replace the bounding box with a new one.
\end{key}

There is a node that allows you to get the size of the current
bounding box. The |current bounding box| node has the |rectangle|
shape and its size is always the size of the current
bounding box.

Similarly, the |current path bounding box| node has the |rectangle|
shape and the size of the bounding box of the current path.


\begin{codeexample}[]
\begin{tikzpicture}
  \draw[red] (0,0) circle (2pt);
  \draw[red] (2,1) circle (3pt);

  \draw (current bounding box.south west) rectangle
        (current bounding box.north east);

  \draw[red] (3,-1) circle (4pt);

  \draw[thick] (current bounding box.south west) rectangle
               (current bounding box.north east);
\end{tikzpicture}
\end{codeexample}


Occasionally, you may want to align multiple |tikzpicture| environments horizontally and/or vertically at some prescribed position. The vertical alignment can be realized by means of the |baseline| option since \TeX\ supports the concept of box depth natively. For horizontal alignment, things are slightly more involved. The following approach is realized by means of negative |\hspace|s before and/or after the picture, thereby removing parts of the picture. However, the actual amount of negative horizontal space is provided by means of image coordinates using the |trim left| and |trim right| keys:

\begin{key}{/tikz/trim left=\meta{dimension or coordinate or \texttt{default}} (default 0pt)}
	The |trim left| key tells \pgfname\space to discard everything which is left of the provided \meta{dimension or coordinate}. Here, \meta{dimension} is a single $x$ coordinate of the picture and \meta{coordinate} is a point with $x$ and $y$ coordinates (but only its $x$ coordinate will be used). The effect is the same as if you issue |\hspace{-s}| where |s| is the difference of the picture's bounding box lower left $x$ coordinate and the $x$ coordinate specified as \meta{dimension or coordinate}:
\begin{codeexample}[]
Text before image.%
	\begin{tikzpicture}[trim left]
		\draw (-1,-1) grid (3,2);
		\fill (0,0) circle (5pt);
	\end{tikzpicture}%
Text after image.
\end{codeexample}
	Since |trim left| uses the default |trim left=0pt|, everything left of $x=0$ is removed from the bounding box.

	The following example has once the relative long label $-1$ and once the shorter label $1$. Horizontal alignment is established with |trim left|:
\begin{codeexample}[pre={\vbox\bgroup\hsize=5cm},post=\egroup,width=8cm]
\begin{tikzpicture}
	\draw (0,1) -- (0,0) -- (1,1) -- cycle;
	\fill (0,0) circle (2pt);
	\node[left] at (0,0) {$-1$};
\end{tikzpicture}
\par
\begin{tikzpicture}
	\draw (0,1) -- (0,0) -- (1,1) -- cycle;
	\fill (0,0) circle (2pt);
	\node[left] at (0,0) {$1$};
\end{tikzpicture}
\par
\begin{tikzpicture}[trim left]
	\draw (0,1) -- (0,0) -- (1,1) -- cycle;
	\fill (0,0) circle (2pt);
	\node[left] at (0,0) {$-1$};
\end{tikzpicture}
\par
\begin{tikzpicture}[trim left]
	\draw (0,1) -- (0,0) -- (1,1) -- cycle;
	\fill (0,0) circle (2pt);
	\node[left] at (0,0) {$1$};
\end{tikzpicture}
\end{codeexample}
	
	Use |trim left=default| to reset the value.
\end{key}

\begin{key}{/tikz/trim right=\meta{dimension or coordinate or \texttt{default}}}
	This key is similar to |trim left|: it discards everything which is right of the provided \meta{dimension or coordinate}. As for |trim left|, \meta{dimension} denotes a single $x$ coordinate of the picture and \meta{coordinate} a coordinate with $x$ and $y$ value (although only its $x$ component will be used).

	We use the same example from above and add |trim right|:
\begin{codeexample}[]
Text before image.%
	\begin{tikzpicture}[trim left, trim right=2cm, baseline]
		\draw (-1,-1) grid (3,2);
		\fill (0,0) circle (5pt);
	\end{tikzpicture}%
Text after image.
\end{codeexample}
	In addition to |trim left=0pt|, we also discard everything which is right of $x$|=2cm|. Furthermore, the |baseline| key supports vertical alignment as well (using the $y$|=0cm| baseline).

	Use |trim right=default| to reset the value.
\end{key}

Note that |baseline|, |trim left| and |trim right| are currently the \emph{only} supported way of truncated bounding boxes which are compatible with image externalization (see the |external| library for details).

\begin{key}{/pgf/trim lowlevel=\mchoice{true,false} (initially false)}
	This affects only the basic level image externalization: the initial configuration |trim lowlevel=false| stores the normal image, without trimming, and the trimming into a separate file. This allows reduced bounding boxes without clipping the rest away. The |trim lowlevel=true| information causes the image externalization to store the trimmed image, possibly resulting in clipping.
\end{key}

\subsection{Clipping and Fading (Soft Clipping)}

\emph{Clipping path} means that all painting on the page is restricted
to a certain area. This area need not be rectangular, rather an
arbitrary path can be used to specify this area. The |clip| option,
explained below, is used to specify the region that is to be used for
clipping.

A \emph{fading} (a term that I propose, fadings are commonly known
as soft masks, transparency masks, opacity masks or soft clips) is
similar to clipping, but a fading allows parts of the picture to be
only ``half clipped.'' This means that a fading can specify that newly
painted pixels should be partly transparent. The specification
and handling of fadings is a bit complex and it is detailed in
Section~\ref{section-tikz-transparency}, which is devoted to
transparency in general.

\begin{key}{/tikz/clip}
  This option causes all subsequent drawings to be clipped against the
  current path and the size of subsequent paths will not be important
  for the picture size.  If you clip against a self-intersecting path,
  the even-odd rule or  the nonzero winding number rule is used to
  determine whether a point is inside or outside the clipping region.

  The clipping path is a graphic state parameter, so it will be reset
  at the end of the current scope. Multiple clippings accumulate, that
  is, clipping is always done against the intersection of all clipping
  areas that have been specified inside the current scopes. The only
  way of enlarging the clipping area is to end a |{scope}|.

\begin{codeexample}[]
\begin{tikzpicture}
  \draw[clip] (0,0) circle (1cm);
  \fill[red] (1,0) circle (1cm);
\end{tikzpicture}
\end{codeexample}

  It  is usually a \emph{very} good idea to apply the |clip| option only
  to the first path command in a scope.

  If you ``only wish to clip'' and do not wish to draw anything, you can
  use the |\clip| command, which is a shorthand for |\path[clip]|.

\begin{codeexample}[]
\begin{tikzpicture}
  \clip (0,0) circle (1cm);
  \fill[red] (1,0) circle (1cm);
\end{tikzpicture}
\end{codeexample}

  To keep clipping local, use |{scope}| environments as in the
  following example:

\begin{codeexample}[]
\begin{tikzpicture}
  \draw (0,0) -- ( 0:1cm);
  \draw (0,0) -- (10:1cm);
  \draw (0,0) -- (20:1cm);
  \draw (0,0) -- (30:1cm);
  \begin{scope}[fill=red]
    \fill[clip] (0.2,0.2) rectangle (0.5,0.5);

    \draw (0,0) -- (40:1cm);
    \draw (0,0) -- (50:1cm);
    \draw (0,0) -- (60:1cm);
  \end{scope}
  \draw (0,0) -- (70:1cm);
  \draw (0,0) -- (80:1cm);
  \draw (0,0) -- (90:1cm);
\end{tikzpicture}
\end{codeexample}

  There is a slightly annoying catch: You cannot specify certain graphic
  options for the command used for clipping. For example, in the above
  code we could not have moved the |fill=red| to the |\fill|
  command. The reasons for this have to do with the internals of the
  \pdf\ specification. You do not want to know the details. It is best
  simply not to specify any options for these
  commands.
\end{key}



\subsection{Doing Multiple Actions on a Path}

If more than one of the basic actions like drawing, clipping and
filling are requested, they are automatically applied in a sensible
order: First, a path is filled, then drawn, and then clipped (although
it took Apple two mayor revisions of their operating system to get
this right\dots). Sometimes, however, you need finer control over what
is done with a path. For instance, you might wish to first fill a path
with a color, then repaint the path with a pattern and then repaint it
with yet another pattern. In such cases you can use the following two
options:

\begin{key}{/tikz/preaction=\meta{options}}
  This option can be given to a |\path| command (or to derived
  commands like |\draw| which internally call |\path|). Similarly to
  options like |draw|, this option only has an effect when given to a
  |\path| or as part of the options of a |node|; as an option to a
  |{scope}| it has no effect.

  When this option is used on a |\path|, the effect is the following:
  When the path has been completely constructed and is about to be
  used, a scope is created. Inside this scope, the path is used but
  not with the original path options, but with \meta{options}
  instead. Then, the path is used in the usual manner. In other words,
  the path is used twice: Once with \meta{options} in force and then
  again with the normal path options in force.

  Here is an example in which the path consists of a rectangle. The
  main action is to draw this path in red (which is why we see a red
  rectangle). However, the preaction is to draw the path in blue,
  which is why we see a blue rectangle behind the red rectangle.
\begin{codeexample}[]
\begin{tikzpicture}
  \draw[help lines] (0,0) grid (3,2);

  \draw
    [preaction={draw,line width=4mm,blue}]
    [line width=2mm,red] (0,0) rectangle (2,2);
\end{tikzpicture}
\end{codeexample}

  Note that when the preactions are preformed, then the path is
  already ``finished.'' In particular, applying a coordinate
  transformation to the path has no effect. By comparison, applying a
  canvas transformation does have an effect. Let us use this to add a
  ``shadow'' to a path. For this, we use the preaction to fill the
  path in gray, shifted a bit to the right and down:

\begin{codeexample}[]
\begin{tikzpicture}
  \draw[help lines] (0,0) grid (3,2);
  \draw
    [preaction={fill=black,opacity=.5,
                transform canvas={xshift=1mm,yshift=-1mm}}]
    [fill=red] (0,0) rectangle (1,2)
               (1,2) circle (5mm);
\end{tikzpicture}
\end{codeexample}

  Naturally, you would normally create a style |shadow| that contains
  the above code. The shadow library, see
  Section~\ref{section-libs-shadows}, contains predefined shadows of
  this kind.

  It is possible to use the |preaction| option multiple times. In this
  case, for each use of the |preaction| option, the path is used again
  (thus, the \meta{options} do not accumulate in a single usage of the
  path). The path is used in the order of |preaction| options given.

  In the following example, we use one |preaction| to add a shadow and
  another to provide a shading, while the main action is to use a
  pattern.
\begin{codeexample}[]
\begin{tikzpicture}
  \draw[help lines] (0,0) grid (3,2);
  \draw [pattern=fivepointed stars]
    [preaction={fill=black,opacity=.5,
                transform canvas={xshift=1mm,yshift=-1mm}}]
    [preaction={top color=blue,bottom color=white}]
               (0,0) rectangle (1,2)
               (1,2) circle (5mm);
\end{tikzpicture}
\end{codeexample}

  A complicated application is shown in the following example, where
  the path is used several times with different fadings and shadings
  to create a special visual effect:
\begin{codeexample}[]
\begin{tikzpicture}
  [
    % Define an interesting style
    button/.style={
      % First preaction: Fuzzy shadow
      preaction={fill=black,path fading=circle with fuzzy edge 20 percent,
                 opacity=.5,transform canvas={xshift=1mm,yshift=-1mm}},
      % Second preaction: Background pattern
      preaction={pattern=#1,
                 path fading=circle with fuzzy edge 15 percent},
      % Third preaction: Make background shiny
      preaction={top color=white,
                 bottom color=black!50,
                 shading angle=45,
                 path fading=circle with fuzzy edge 15 percent,
                 opacity=0.2},
      % Fourth preaction: Make edge especially shiny
      preaction={path fading=fuzzy ring 15 percent,
                 top color=black!5,
                 bottom color=black!80,
                 shading angle=45},
      inner sep=2ex
    },
    button/.default=horizontal lines light blue,
    circle
  ]

  \draw [help lines] (0,0) grid (4,3);

  \node [button] at (2.2,1) {\Huge Big};
  \node [button=crosshatch dots light steel blue,
         text=white] at (1,1.5) {Small};
\end{tikzpicture}
\end{codeexample}
\end{key}

\begin{key}{/tikz/postaction=\meta{options}}
  The postactions work in the same way as the preactions, only they
  are applied \emph{after} the main action has been taken. Like
  preactions, multiple |postaction| options may be given to a |\path|
  command, in which case the path is reused several times, each time
  with a different set of options in force.

  If both pre- and postactions are specified, then the preactions are
  taken first, then the main action, and then the post actions.

  In the first example, we use a postaction to draw the path, after it
  has already been drawn:
\begin{codeexample}[]
\begin{tikzpicture}
  \draw[help lines] (0,0) grid (3,2);

  \draw
    [postaction={draw,line width=2mm,blue}]
    [line width=4mm,red,fill=white] (0,0) rectangle (2,2);
\end{tikzpicture}
\end{codeexample}

  In another example, we use a postaction to ``colorize'' a path:

\begin{codeexample}[]
\begin{tikzpicture}
  \draw[help lines] (0,0) grid (3,2);
  \draw
    [postaction={path fading=south,fill=white}]
    [postaction={path fading=south,fading angle=45,fill=blue,opacity=.5}]
    [left color=black,right color=red,draw=white,line width=2mm]
               (0,0) rectangle (1,2)
               (1,2) circle (5mm);
\end{tikzpicture}
\end{codeexample}
\end{key}



\subsection{Decorating and Morphing a Path}

Before a path is used, it is possible to first ``decorate'' and/or
``morph'' it. Morphing means that the path is replaced by another path
that is slightly varied. Such morphings are a special case
of the more general ``decorations'' described in detail in
Section~\ref{section-tikz-decorations}. For instance, in the following
example the path is drawn twice: Once normally and then in a morphed
(=decorated) manner.
\begin{codeexample}[]
\begin{tikzpicture}
  \draw (0,0) rectangle (3,2);
  \draw [red, decorate, decoration=zigzag]
        (0,0) rectangle (3,2);
\end{tikzpicture}
\end{codeexample}

Naturally, we could have combined this into a single command using
pre- or postaction. It is also possible to deform shapes:
\begin{codeexample}[]
\begin{tikzpicture}
  \node [circular drop shadow={shadow scale=1.05},minimum size=3.13cm,
         decorate, decoration=zigzag,
         fill=blue!20,draw,thick,circle] {Hello!};
\end{tikzpicture}
\end{codeexample}

% % Copyright 2006 by Till Tantau
%
% This file may be distributed and/or modified
%
% 1. under the LaTeX Project Public License and/or
% 2. under the GNU Free Documentation License.
%
% See the file doc/generic/pgf/licenses/LICENSE for more details.

\section{Arrows}
\label{section-tikz-arrows}

\subsection{Overview}

\tikzname\ allows you to add (multiple) arrow tips to the end of lines as in
\tikz [baseline] \draw [->>] (0,.5ex) -- (3ex,.5ex); or in \tikz [baseline]
\draw [-{Latex[]}] (0,.5ex) -- (3ex,.5ex);. It is possible to change which
arrow tips are used ``on-the-fly'', you can have several arrow tips in a row,
and you can change the appearance of each of them individually using a special
syntax. The following example is a perhaps slightly ``excessive'' demonstration
of what you can do (you need to load the |arrows.meta| library for it to work):
%
\begin{codeexample}[preamble={\usetikzlibrary{arrows.meta,bending,positioning}}]
\tikz {
  \node [circle,draw] (A)              {A};
  \node [circle,draw] (B) [right=of A] {B};

  \draw [draw = blue, thick,
         arrows={
           Computer Modern Rightarrow [sep]
         - Latex[blue!50,length=8pt,bend,line width=0pt]
           Stealth[length=8pt,open,bend,sep]}]
    (A) edge [bend left=45] (B)
    (B) edge [in=-110, out=-70,looseness=8] (B);
}
\end{codeexample}

There are a number of predefined generic arrow tip kinds whose appearance you
can modify in many ways using various options. It is also possible to define
completely new arrow tip kinds, see Section~\ref{section-arrows}, but doing
this is somewhat harder than configuring an existing kind (it is like the
difference between using a font at different sizes or faces like italics,
compared to designing a new font yourself).

In the present section, we go over the various ways in which you can configure
which particular arrow tips are \emph{used}. The glorious details of how new
arrow tips can be defined are explained in Section~\ref{section-arrows}.

At the end of the present section, Section~\ref{section-arrows-meta}, you will
find a description of the different predefined arrow tips from the
|arrows.meta| library.

\emph{Remark:} Almost all of the features described in the following were
introduced in version 3.0 of \tikzname. For compatibility reasons, the old
arrow tips are still available. To differentiate between the old and new arrow
tips, the following rule is used: The new, more powerful arrow tips start with
an uppercase letter as in |Latex|, compared to the old arrow tip |latex|.

\emph{Remark:} The libraries |arrows| and |arrows.spaced| are deprecated. Use
|arrows.meta| instead/additionally, which allows you to do all that the old
libraries offered, plus much more. However, the old libraries still work and
you can even mix old and new arrow tips (only, the old arrow tips cannot be
configured in the ways described in the rest of this section; saying |scale=2|
for a |latex| arrow has no effect for instance, while for |Latex| arrows it
doubles their size as one would expect.)


\subsection{Where and When Arrow Tips Are Placed}
\label{section-arrow-tips-where}

In order to add arrow tips to the lines you draw, the following conditions must
be met:
%
\begin{enumerate}
    \item You have specified that arrow tips should be added to lines, using
        the |arrows| key or its short form.
    \item You set the |tips| key to some value that causes tips to be drawn
        (to be explained later).
    \item You do not use the |clip| key (directly or indirectly) with the
        current path.
    \item The path actually has two ``end points'' (it is not ``closed'').
\end{enumerate}

Let us start with an introduction to the basics of the |arrows| key:

\begin{key}{/tikz/arrows=\meta{start arrow specification}|-|\meta{end arrow specification}}
    This option sets the arrow tip(s) to be used at the start and end of lines.
    An empty value as in |->| for the start indicates that no arrow tip should
    be drawn at the start.%
    \indexoption{arrows}

    \emph{Note: Since the arrow option is so often used, you can leave out the
    text |arrows=|.} What happens is that every (otherwise unknown) option that
    contains a |-| is interpreted as an arrow specification.
    %
\begin{codeexample}[preamble={\usetikzlibrary{arrows.meta}}]
\begin{tikzpicture}
  \draw[->]        (0,0)   -- (1,0);
  \draw[>-Stealth] (0,0.3) -- (1,0.3);
\end{tikzpicture}
\end{codeexample}

    In the above example, the first start specification is empty and the second
    is |>|. The end specifications are |>| for the first line and |Stealth| for
    the second line. Note that it makes a difference whether |>| is used in a
    start specification or in an end specification: In an end specification it
    creates, as one would expect, a pointed tip  at the end of the line. In the
    start specification, however, it creates a ``reversed'' version if this
    arrow -- which happens to be what one would expect here.

    The above specifications are very simple and only select a single arrow tip
    without any special configuration options, resulting in the ``natural''
    versions of these arrow tips. It is also possible to ``configure'' arrow
    tips in many different ways, as explained in detail in
    Section~\ref{section-arrow-config} below by adding options in square
    brackets following the arrow tip kind:
    %
\begin{codeexample}[preamble={\usetikzlibrary{arrows.meta}}]
\begin{tikzpicture}
  \draw[-{Stealth[red]}] (0,0)   -- (1,0);
\end{tikzpicture}
\end{codeexample}

    Note that in the example I had to surround the end specification by braces.
    This is necessary so that \tikzname\ does not mistake the closing square
    bracket of the |Stealth| arrow tip's options for the end of the options of
    the |\draw| command. In general, you often need to add braces when
    specifying arrow tips except for simple case like |->| or |<<->|, which are
    pretty frequent, though. When in doubt, say
    |arrows={|\meta{start spec}|-|\meta{end spec}|}|, that will always work.

    It is also possible to specify multiple (different) arrow tips in a row
    inside a specification, see Section~\ref{section-arrow-spec} below for
    details.
\end{key}

As was pointed out earlier, to add arrow tips to a path, the path must have
``end points'' and not be ``closed'' -- otherwise adding arrow tips makes
little sense, after all. However, a path can actually consist of several
subpath, which may be open or not and may even consist of only a single point
(a single move-to). In this case, it is not immediately obvious, where arrow
heads should be placed. The actual rules that \tikzname\ uses are governed by
the setting of the key |tips|:

\begin{key}{/pgf/tips=\meta{value} (default true, initially on draw)}
        \keyalias{tikz}
    This key governs in what situations arrow tips are added to a path. The
    following \meta{values} are permissible:
    %
    \begin{itemize}
        \item |true| (the value used when no \meta{value} is specified)
        \item |proper|
        \item |on draw| (the initial value, if the key has not yet been used
            at all)
        \item |on proper draw|
        \item |never| or |false| (same effect)
    \end{itemize}

    Firstly, there are a whole bunch of situations where the setting of
    these (or other) options causes no arrow tips to be shown:
    %
    \begin{itemize}
        \item If no arrow tips have been specified (for instance, by having
            said |arrows=-|), no arrow tips are drawn.
        \item If the |clip| option is set, no arrow tips are drawn.
        \item If |tips| has been set to |never| or |false|, no arrow tips are
            drawn.
        \item If |tips| has been set to |on draw| or |on proper draw|, but
            the |draw| option is not set, no arrow tips are drawn.
        \item If the path is empty (as in |\path ;|), no arrow tips are
            drawn.
        \item If at least one of the subpaths of a path is closed (|cycle| is
            used somewhere or something like |circle| or |rectangle|), arrow
            tips are never drawn anywhere -- even if there are open subpaths.
    \end{itemize}

    Now, if we pass all of the above tests, we must have a closer look at the
    path. All its subpaths must now be open and there must be at least one
    subpath. We consider the last one. Arrow tips will only be added to this
    last subpath.

    \begin{enumerate}
        \item If this last subpath not degenerate (all coordinates on the
            subpath are the same as in a single ``move-to'' |\path (0,0);| or
            in a ``move-to'' followed by a ``line-to'' to the same position
            as in |\path (1,2) -- (1,2)|), arrow tips are added to this last
            subpath now.
        \item If the last subpath is degenerate, we add arrow tips pointing
            upward at the single coordinate mentioned in the path, but only
            for |tips| begin set to |true| or to |on draw| -- and not for
            |proper| nor for |on proper draw|. In other words, ``proper''
            suppresses arrow tips on degenerate paths.
    \end{enumerate}

\begin{codeexample}[]
% No path, no arrow tips:
\tikz [<->] \draw;
\end{codeexample}
\begin{codeexample}[]
% Degenerate path, draw arrow tips (but no path, it is degenerate...)
\tikz [<->] \draw (0,0);
\end{codeexample}
\begin{codeexample}[]
% Degenerate path, tips=proper suppresses arrows
\tikz [<->] \draw [tips=proper] (0,0);
\end{codeexample}
\begin{codeexample}[]
% Normal case:
\tikz [<->] \draw (0,0) -- (1,0);
\end{codeexample}
\begin{codeexample}[]
% Two subpaths, only second gets tips
\tikz [<->] \draw (0,0) -- (1,0) (2,0) -- (3,0);
\end{codeexample}
\begin{codeexample}[]
% Two subpaths, second degenerate, but still gets tips
\tikz [<->] \draw (0,0) -- (1,0) (2,0);
\end{codeexample}
\begin{codeexample}[]
% Two subpaths, second degenerate, proper suppresses them
\tikz [<->] \draw [tips=on proper draw] (0,0) -- (1,0) (2,0);
\end{codeexample}
\begin{codeexample}[]
% Two subpaths, but one is closed: No tips, even though last subpath is open
\tikz [<->] \draw (0,0) circle[radius=2pt] (2,0) -- (3,0);
\end{codeexample}
    %
\end{key}

One common pitfall when arrow tips are added to a path should be addressed
right here at the beginning: When \tikzname\ positions an arrow tip at the
start, for all its computations it only takes into account the first segment of
the subpath to which the arrow tip is added. This ``first segment'' is the
first line-to or curve-to operation (or arc or parabola or a similar operation)
of the path; but note that decorations like |snake| will add many small line
segments to paths. The important point is that if this first segment is very
small, namely smaller that the arrow tip itself, strange things may result. As
will be explained in Section~\ref{section-arrow-flex}, \tikzname\ will modify
the path by shortening the first segment and shortening a segment below its
length may result in strange effects. Similarly, for tips at the end of a
subpath, only the last segment is considered.

The bottom line is that wherever an arrow tip is added to a path, the line
segment where it is added should be ``long enough''.


\subsection{Arrow Keys: Configuring the Appearance of a Single Arrow Tip}
\label{section-arrow-config}

For standard arrow tip kinds, like |Stealth| or |Latex| or |Bar|, you can
easily change their size, aspect ratio, color, and other parameters. This is
similar to selecting a font face from a font family: \emph{``This text''} is
not just typeset in the font ``Computer Modern'', but rather in ``Computer
Modern, italic face, 11pt size, medium weight, black color, no underline,
\dots'' Similarly, an arrow tip is not just a ``Stealth'' arrow tip, but rather
a ``Stealth arrow tip at its natural size, flexing, but not bending along the
path, miter line caps, draw and fill colors identical to the path draw color,
\dots''

Just as most programs make it easy to ``configure'' which font should be used
at a certain point in a text, \tikzname\ tries to make it easy to specify which
configuration of an arrow tip should be used. You use \emph{arrow keys}, where
a certain parameter like the |length| of an arrow is set to a given value using
the standard key--value syntax. You can provide several arrow keys following an
arrow tip kind in  an arrow tip specification as in
|Stealth[length=4pt,width=2pt]|.

While selecting a font may be easy, \emph{designing} a new font is a highly
creative and difficult process and more often than not, not all faces of a font
are available on any given system. The difficulties involved in designing a new
arrow tip are somewhat similar to designing a new letter for a font and, thus,
it may also happen that not all configuration options are actually implemented
for a given arrow tip. Naturally, for the standard arrow tips, all
configuration options are available -- but for special-purpose arrow tips it
may well happen that an arrow tip kind simply ``ignores'' some of the
configurations given by you.

Some of the keys explained in the following are defined in the library
|arrows.meta|, others are always available. This has to do with the question of
whether the arrow key needs to be supported directly in the \pgfname\ core or
not. In general, the following explanations assume that |arrows.meta| has been
loaded.


\subsubsection{Size}

The most important configuration parameter of an arrow tip is undoubtedly its
size. The following two keys are the main keys that are important in this
context:

\begin{key}{/pgf/arrow keys/length=\meta{dimension}| |\opt{\meta{line width factor}}%
        | |\opt{\meta{outer factor}}}
        \label{length-arrow-key}%
    This parameter is usually the most important parameter that governs the
    size of an arrow tip: The \meta{dimension} that you provide dictates the
    distance from the ``very tip'' of the arrow to its ``back end'' along the
    line:
    %
\begin{codeexample}[preamble={\usetikzlibrary{arrows.meta}}]
\tikz{
  \draw [-{Stealth[length=5mm]}] (0,0) -- (2,0);
  \draw [|<->|] (1.5,.4) -- node[above=1mm] {5mm} (2,.4);
}
\end{codeexample}
\begin{codeexample}[preamble={\usetikzlibrary{arrows.meta}}]
\tikz{
  \draw [-{Latex[length=5mm]}] (0,0) -- (2,0);
  \draw [|<->|] (1.5,.4) -- node[above=1mm] {5mm} (2,.4);
}
\end{codeexample}
\begin{codeexample}[preamble={\usetikzlibrary{arrows.meta}}]
\tikz{
  \draw [-{Classical TikZ Rightarrow[length=5mm]}] (0,0) -- (2,0);
  \draw [|<->|] (1.5,.6) -- node[above=1mm] {5mm} (2,.6);
}
\end{codeexample}

    \medskip
    \noindent \textbf{The Line Width Factors.}
    Following the \meta{dimension}, you may put a space followed by a
    \meta{line width factor}, which must be a plain number (no |pt| or |cm|
    following). When you provide such a number, the size of the arrow tip is
    not just \meta{dimension}, but rather $\meta{dimension} + \meta{line width
    factor}\cdot w$ where $w$ is the width of the to-be-drawn path. This makes
    it easy to vary the size of an arrow tip in accordance with the line width
    -- usually a very good idea since thicker lines will need thicker arrow
    tips.

    As an example, when you write |length=0pt 5|, the length of the arrow will
    be exactly five times the current line width. As another example, the
    default length of a |Latex| arrow is |length=3pt 4.5 0.8|. Let us ignore
    the 0.8 for a moment; the |4pt 4.5| then means that for the standard line
    width of |0.4pt|, the length of a |Latex| arrow will be exactly 4.8pt (3pt
    plus 4.5 times |0.4pt|).

    Following the line width factor, you can additionally provide an
    \meta{outer factor}, again preceded by a space (the |0.8| in the above
    example). This factor is taken into consideration only when the |double|
    option is used, that is, when a so-called ``inner line width''. For a
    double line, we can identify three different ``line widths'', namely the
    inner line width $w_i$, the line width  $w_o$ of the two outer lines, and
    the ``total line width'' $w_t = w_i + 2w_o$. In the below examples, we have
    $w_i = 3\mathrm{pt}$, $w_o=1\mathrm{pt}$, and $w_t = 5\mathrm{pt}$. It is
    not immediately clear which of these line widths should be considered as
    $w$ in the above formula $\meta{dimension} + \meta{line width factor}\cdot
    w$ for the computation of the length. One can argue both for $w_t$ and also
    for $w_o$. Because of this, you use the \meta{outer factor} to decide on
    one of them or even mix them: \tikzname\ sets $w = \meta{outer factor} w_o
    + (1-\meta{outer factor})w_t$. Thus, when the outer factor is $0$, as in
    the first of the following examples and as is the default when it is not
    specified, the computed $w$ will be the total line width $w_t =
    5\mathrm{pt}$. Since $w=5\mathrm{pt}$, we get a total length of $15pt$ in
    the first example (because of the factor |3|). In contrast, in the last
    example, the outer factor is 1 and, thus, $w = w_o = \mathrm{1pt}$ and the
    resulting length is 3pt. Finally, for the middle case, the ``middle''
    between 5pt and 1pt is 3pt, so the length is 9pt.
    %
\begin{codeexample}[preamble={\usetikzlibrary{arrows.meta}}]
\tikz \draw [line width=1pt, double distance=3pt,
             arrows = {-Latex[length=0pt 3 0]}] (0,0) -- (1,0);
\end{codeexample}
\begin{codeexample}[preamble={\usetikzlibrary{arrows.meta}}]
\tikz \draw [line width=1pt, double distance=3pt,
             arrows = {-Latex[length=0pt 3 .5]}] (0,0) -- (1,0);
\end{codeexample}
\begin{codeexample}[preamble={\usetikzlibrary{arrows.meta}}]
\tikz \draw [line width=1pt, double distance=3pt,
             arrows = {-Latex[length=0pt 3 1]} ] (0,0) -- (1,0);
\end{codeexample}

    \medskip
    \noindent \textbf{The Exact Length.}
    For an arrow tip kind that is just an outline that is filled with a color,
    the specified length should \emph{exactly} equal the distance from the tip
    to the back end. However, when the arrow tip is drawn by stroking a line,
    it is no longer obvious whether the |length| should refer to the extend of
    the stroked lines' path or of the resulting pixels (which will be wider
    because of the thickness of the stroking pen). The rules are as follows:
    %
    \begin{enumerate}
        \item If the arrow tip consists of a closed path (like |Stealth| or
            |Latex|), imagine the arrow tip drawn from left to right using a
            miter line cap. Then the |length| should be the horizontal
            distance from the first drawn ``pixel'' to the last drawn
            ``pixel''. Thus, the thickness of the stroked line and also the
            miter ends should be taken into account:
            %
\begin{codeexample}[preamble={\usetikzlibrary{arrows.meta}}]
\tikz{
  \draw [line width=1mm, -{Stealth[length=10mm, open]}]
          (0,0) -- (2,0);
  \draw [|<->|] (2,.6) -- node[above=1mm] {10mm} ++(-10mm,0);
}
\end{codeexample}
            %
        \item If, in the above case, the arrow is drawn using a round line
            join (see Section~\ref{section-arrow-key-caps} for details on how
            to select this), the size of the arrow should still be the same
            as in the first case (that is, as if a miter join were used).
            This creates some ``visual consistency'' if the two modes are
            mixed or if you later one change the mode.
            %
\begin{codeexample}[preamble={\usetikzlibrary{arrows.meta}}]
\tikz{
  \draw [line width=1mm, -{Stealth[length=10mm, open, round]}]
          (0,0) -- (2,0);
  \draw [|<->|] (2,.6) -- node[above=1mm] {10mm} ++(-10mm,0);
}
\end{codeexample}
            %
            As the above example shows, however, a rounded arrow will still
            exactly ``tip'' the point where the line should end (the point
            |(2,0)| in the above case). It is only the scaling of the arrow
            that is not affected.
  \end{enumerate}
\end{key}

\begin{key}{/pgf/arrow keys/width=\meta{dimension}| |\opt{\meta{line width factor}}%
        | |\opt{\meta{outer factor}}}
    This key works line the |length| key, only it specifies the ``width'' of
    the arrow tip; so if width and length are identical, the arrow will just
    touch the borders of a square. (An exception to this rule are ``halved''
    arrow tips, see Section~\ref{section-arrow-key-harpoon}.) The meaning of
    the two optional factor numbers following the \meta{dimension} is the same
    as for the |length| key.
    %
\begin{codeexample}[preamble={\usetikzlibrary{arrows.meta}}]
\tikz \draw [arrows = {-Latex[width=10pt, length=10pt]}] (0,0) -- (1,0);
\end{codeexample}
\begin{codeexample}[preamble={\usetikzlibrary{arrows.meta}}]
\tikz \draw [arrows = {-Latex[width=0pt 10, length=10pt]}] (0,0) -- (1,0);
\end{codeexample}
\end{key}

\begin{key}{/pgf/arrow keys/width'=\meta{dimension}| |\opt{\meta{length
        factor}| |\opt{\meta{line width factor}}}}
    The key (note the prime) has a similar effect as the |width| key. The
    difference is that the second, still optional parameter \meta{length
    factor} specifies the width of the key not as a multiple of the line width,
    but as a multiple of the arrow length.

    The idea is that if you write, say, |width'=0pt 0.5|, the width of the
    arrow will be half its length. Indeed, for standard arrow tips like
    |Stealth| the default width is specified in this way so that if you change
    the length of an arrow tip, you also change the width in such a way that
    the aspect ratio of the arrow tip is kept. The other way round, if you
    modify the factor in |width'| without changing the length, you change the
    aspect ratio of the arrow tip.

    Note that later changes of the length are taken into account for the
    computation. For instance, if you write
    %
\begin{codeexample}[code only]
length = 10pt, width'=5pt 2, length=7pt
\end{codeexample}
    %
    the resulting width will be $19\mathrm{pt} = 5\mathrm{pt} + 2\cdot
    7\mathrm{pt}$.
    %
\begin{codeexample}[preamble={\usetikzlibrary{arrows.meta}}]
\tikz \draw [arrows = {-Latex[width'=0pt .5, length=10pt]}] (0,0) -- (1,0);
\end{codeexample}
\begin{codeexample}[preamble={\usetikzlibrary{arrows.meta}}]
\tikz \draw [arrows = {-Latex[width'=0pt .5, length=15pt]}] (0,0) -- (1,0);
\end{codeexample}
    %
    The third, also optional, parameter allows you to add a multiple of the
    line width to the value computed in terms of the length.
\end{key}


\begin{key}{/pgf/arrow keys/inset=\meta{dimension}| |\opt{\meta{line width factor}}%
        | |\opt{\meta{outer factor}}}
    The key is relevant only for some arrow tips such as the |Stealth| arrow
    tip. It specifies a distance by which something inside the arrow tip is set
    inwards; for the |Stealth| arrow tip it is the distance by which the back
    angle is moved inwards.

    The computation of the distance works in the same way as for |length| and
    |width|: To the \meta{dimension} we add \meta{line width factor} times that
    line width, where the line width is computed based on the \meta{outer
    factor} as described for the |length| key.
    %
\begin{codeexample}[preamble={\usetikzlibrary{arrows.meta}}]
\tikz \draw [arrows = {-Stealth[length=10pt, inset=5pt]}] (0,0) -- (1,0);
\end{codeexample}
\begin{codeexample}[preamble={\usetikzlibrary{arrows.meta}}]
\tikz \draw [arrows = {-Stealth[length=10pt, inset=2pt]}] (0,0) -- (1,0);
\end{codeexample}

    For most arrows for which there is no ``natural inset'' like, say, |Latex|,
    this key has no effect.
\end{key}

\begin{key}{/pgf/arrow keys/inset'=\meta{dimension}| |\opt{\meta{length factor}}| |\opt{\meta{line width factor}}}
    This key works like |inset|, only like |width'| the second parameter is a
    factor of the arrow length rather than of the line width. For instance, the
    |Stealth| arrow sets |inset'| to |0pt 0.325| to ensure that the inset is
    always at $13/40$th of the arrow length if nothing else is specified.
\end{key}

\begin{key}{/pgf/arrow keys/angle=\meta{angle}|:|\meta{dimension}%
        | |\opt{\meta{line width factor}}%
        | |\opt{\meta{outer factor}}}
    This key sets the |length| and the |width| of an arrow tip at the same
    time. The length will be the cosine of \meta{angle}, while the width will
    be twice the sine of half the \meta{angle} (this slightly awkward rule
    ensures that a |Stealth| arrow will have an opening angle of \meta{angle}
    at its tip if this option is used). As for the |length| key, if the
    optional factors are given, they add a certain multiple of the line width
    to the \meta{dimension} before the sine and cosines are computed.
    %
\begin{codeexample}[preamble={\usetikzlibrary{arrows.meta}}]
\tikz \draw [arrows = {-Stealth[inset=0pt, angle=90:10pt]}] (0,0) -- (1,0);
\end{codeexample}
\begin{codeexample}[preamble={\usetikzlibrary{arrows.meta}}]
\tikz \draw [arrows = {-Stealth[inset=0pt, angle=30:10pt]}] (0,0) -- (1,0);
\end{codeexample}
    %
\end{key}

\begin{key}{/pgf/arrow keys/angle'=\meta{angle}}
    Sets the width of the arrow to twice the tangent of $\meta{angle}/2$ times
    the arrow length. This results in an arrow tip with an opening angle of
    \meta{angle} at its tip and with the specified |length| unchanged.
    %
\begin{codeexample}[preamble={\usetikzlibrary{arrows.meta}}]
\tikz \draw [arrows = {-Stealth[inset=0pt, length=10pt, angle'=90]}]
            (0,0) -- (1,0);
\end{codeexample}
\begin{codeexample}[preamble={\usetikzlibrary{arrows.meta}}]
\tikz \draw [arrows = {-Stealth[inset=0pt, length=10pt, angle'=30]}]
            (0,0) -- (1,0);
\end{codeexample}
    %
\end{key}


\subsubsection{Scaling}

In the previous section we saw that there are many options for getting ``fine
control'' overt the length and width of arrow tips. However, in some cases, you
do not really care whether the arrow tip is 4pt long or 4.2pt long, you ``just
want it to be a little bit larger than usual''. In such cases, the following
keys are useful:

\begin{key}{/pgf/arrows keys/scale=\meta{factor} (initially 1)}
    After all the other options listed in the previous (and also the following
    sections) have been processed, \tikzname\ applies a \emph{scaling} to the
    computed length, inset, and width of the arrow tip (and, possibly, to other
    size parameters defined by special-purpose arrow tip kinds). Everything is
    simply scaled by the given \meta{factor}.
    %
\begin{codeexample}[preamble={\usetikzlibrary{arrows.meta}}]
\tikz {
  \draw [arrows = {-Stealth[]}]          (0,1)   -- (1,1);
  \draw [arrows = {-Stealth[scale=1.5]}] (0,0.5) -- (1,0.5);
  \draw [arrows = {-Stealth[scale=2]}]   (0,0)   -- (1,0);
}
\end{codeexample}
    %
    Note that scaling has \emph{no} effect on the line width (as usual) and
    also not on the arrow padding (the |sep|).
\end{key}

You can get even more fine-grained control over scaling using the following
keys (the |scale| key is just a shorthand for setting both of the following
keys simultaneously):

\begin{key}{/pgf/arrows keys/scale length=\meta{factor} (initially 1)}
    This factor works like |scale|, only it is applied only to dimensions
    ``along the axis of the arrow'', that is, to the length and to the inset,
    but not to the width.
    %
\begin{codeexample}[preamble={\usetikzlibrary{arrows.meta}}]
\tikz {
  \draw [arrows = {-Stealth[]}]                 (0,1)   -- (1,1);
  \draw [arrows = {-Stealth[scale length=1.5]}] (0,0.5) -- (1,0.5);
  \draw [arrows = {-Stealth[scale length=2]}]   (0,0)   -- (1,0);
}
\end{codeexample}
    %
\end{key}

\begin{key}{/pgf/arrows keys/scale width=\meta{factor} (initially 1)}
    Like |scale length|, but for dimensions related to the width.
    %
\begin{codeexample}[preamble={\usetikzlibrary{arrows.meta}}]
\tikz {
  \draw [arrows = {-Stealth[]}]                 (0,1)   -- (1,1);
  \draw [arrows = {-Stealth[scale width=1.5]}] (0,0.5) -- (1,0.5);
  \draw [arrows = {-Stealth[scale width=2]}]   (0,0)   -- (1,0);
}
\end{codeexample}
    %
\end{key}


\subsubsection{Arc Angles}

A few arrow tips consist mainly of arcs, whose length can be specified. For
these arrow tips, you use the following key:

\begin{key}{/pgf/arrow keys/arc=\meta{degrees} (initially 180)}
    Sets the angle of arcs in arrows to \meta{degrees}. Note that this key is
    quite different from the |angle| key, which is ``just a fancy way of
    setting the length and width''. In contrast, the |arc| key is used to set
    the degrees of arcs that are part of an arrow tip:
    %
\begin{codeexample}[preamble={\usetikzlibrary{arrows.meta}}]
\tikz [ultra thick] {
  \draw [arrows = {-Hooks[]}]         (0,1)   -- (1,1);
  \draw [arrows = {-Hooks[arc=90]}]   (0,0.5) -- (1,0.5);
  \draw [arrows = {-Hooks[arc=270]}]  (0,0)   -- (1,0);
}
\end{codeexample}
    %
\end{key}


\subsubsection{Slanting}

You can ``slant'' arrow tips using the following key:

\begin{key}{/pgf/arrow keys/slant=\meta{factor} (initially 0)}
    Slanting is used to create an ``italics'' effect for arrow tips: All arrow
    tips get ``slanted'' a little bit relative to the axis of the arrow:
    %
\begin{codeexample}[preamble={\usetikzlibrary{arrows.meta}}]
\tikz {
  \draw [arrows = {->[]}]         (0,1)   -- (1,1);
  \draw [arrows = {->[slant=.5]}] (0,0.5) -- (1,0.5);
  \draw [arrows = {->[slant=1]}]  (0,0)   -- (1,0);
}
\end{codeexample}
    %
    There is one thing to note about slanting: Slanting is done using a
    so-called ``canvas transformation'' and has no effect on positioning of
    the arrow tip. In particular, if an arrow tip gets slanted so strongly that
    it starts to protrude over the arrow tip end, this does not change the
    positioning of the arrow tip.

    Here is another example where slanting is used to match italic text:
    %
\begin{codeexample}[preamble={\usetikzlibrary{arrows.meta,graphs}}]
\tikz [>={[slant=.3] To[] To[]}]
  \graph [math nodes] { A -> B <-> C };
\end{codeexample}
    %
\end{key}


\subsubsection{Reversing, Halving, Swapping}
\label{section-arrow-key-harpoon}

\begin{key}{/pgf/arrow keys/reversed}
    Adding this key to an arrow tip will ``reverse its direction'' so that is
    points in the opposite direction (but is still at that end of the line
    where the non-reversed arrow tip would have been drawn; so only the tip is
    reversed). For most arrow tips, this just results in an internal flip of a
    coordinate system, but some arrow tips actually use a slightly different
    version of the tip for reversed arrow tips (namely when the joining of the
    tip with the line would look strange). All of this happens automatically,
    so you do not need to worry about this.

    If you apply this key twice, the effect cancels, which is useful for the
    definition of shorthands (which will be discussed later).
    %
\begin{codeexample}[width=3cm,preamble={\usetikzlibrary{arrows.meta}}]
\tikz [ultra thick] \draw [arrows = {-Stealth[reversed]}] (0,0) -- (1,0);
\end{codeexample}
\begin{codeexample}[width=3cm,preamble={\usetikzlibrary{arrows.meta}}]
\tikz [ultra thick] \draw [arrows = {-Stealth[reversed, reversed]}] (0,0) -- (1,0);
\end{codeexample}
\end{key}

\begin{key}{/pgf/arrow keys/harpoon}
    The key requests that only the ``left half'' of the arrow tip should drawn:
    %
\begin{codeexample}[width=3cm,preamble={\usetikzlibrary{arrows.meta}}]
\tikz [ultra thick] \draw [arrows = {-Stealth[harpoon]}] (0,0) -- (1,0);
\end{codeexample}
\begin{codeexample}[width=3cm,preamble={\usetikzlibrary{arrows.meta}}]
\tikz [ultra thick] \draw [arrows = {->[harpoon]}] (0,0) -- (1,0);
\end{codeexample}
    %
    Unlike the |reversed| key, which all arrows tip kinds support at least in a
    basic way, designers of arrow tips really need to take this key into
    account in their arrow tip code and often a lot of special attention needs
    to do be paid to this key in the implementation. For this reason, only some
    arrow tips will support it.
\end{key}

\begin{key}{/pgf/arrow keys/swap}
    This key flips that arrow tip along the axis of the line. It makes sense
    only for asymmetric arrow tips like the harpoons created using the
    |harpoon| option.
    %
\begin{codeexample}[width=3cm,preamble={\usetikzlibrary{arrows.meta}}]
\tikz [ultra thick] \draw [arrows = {-Stealth[harpoon]}] (0,0) -- (1,0);
\end{codeexample}
\begin{codeexample}[width=3cm,preamble={\usetikzlibrary{arrows.meta}}]
\tikz [ultra thick] \draw [arrows = {-Stealth[harpoon,swap]}] (0,0) -- (1,0);
\end{codeexample}
    %
    Swapping is always possible, no special code is needed on behalf of an
    arrow tip implementer.
\end{key}

\begin{key}{/pgf/arrow keys/left}
    A shorthand for |harpoon|.
\end{key}

\begin{key}{/pgf/arrow keys/right}
    A shorthand for |harpoon, swap|.
    %
\begin{codeexample}[width=3cm,preamble={\usetikzlibrary{arrows.meta}}]
\tikz [ultra thick] \draw [arrows = {-Stealth[left]}] (0,0) -- (1,0);
\end{codeexample}
\begin{codeexample}[width=3cm,preamble={\usetikzlibrary{arrows.meta}}]
\tikz [ultra thick] \draw [arrows = {-Stealth[right]}] (0,0) -- (1,0);
\end{codeexample}
    %
\end{key}


\subsubsection{Coloring}

Arrow tips are drawn using the same basic mechanisms as normal paths, so arrow
tips can be stroked (drawn) and/or filled. However, we usually want the color
of arrow tips to be identical to the color used to draw the path, even if a
different color is used for filling the path. On the other hand, we may also
sometimes wish to use a special color for the arrow tips that is different from
both the line and fill colors of the main path.

The following options allow you to configure how arrow tips are colored:

\begin{key}{/pgf/arrow keys/color=\meta{color or empty} (initially \normalfont empty)}
    Normally, an arrow tip gets the same color as the path to which it is
    attached. More precisely, it will get the current ``draw color'', also
    known as ``stroke color'', which you can set using |draw=|\meta{some
    color}. By adding the option |color=| to an arrow tip (note that an
    ``empty'' color is specified in this way), you ask that the arrow tip gets
    this default draw color of the path. Since this is the default behaviour,
    you usually do not need to specify anything:
    %
\begin{codeexample}[width=3cm,preamble={\usetikzlibrary{arrows.meta}}]
\tikz [ultra thick] \draw [red, arrows = {-Stealth}] (0,0) -- (1,0);
\end{codeexample}
\begin{codeexample}[width=3cm,preamble={\usetikzlibrary{arrows.meta}}]
\tikz [ultra thick] \draw [blue, arrows = {-Stealth}] (0,0) -- (1,0);
\end{codeexample}

    Now, when you provide a \meta{color} with this option, you request that the
    arrow tip should get this color \emph{instead} of the color of the main
    path:
    %
\begin{codeexample}[width=3cm,preamble={\usetikzlibrary{arrows.meta}}]
\tikz [ultra thick] \draw [red, arrows = {-Stealth[color=blue]}] (0,0) -- (1,0);
\end{codeexample}
\begin{codeexample}[width=3cm,preamble={\usetikzlibrary{arrows.meta}}]
\tikz [ultra thick] \draw [red, arrows = {-Stealth[color=black]}] (0,0) -- (1,0);
\end{codeexample}

    Similar to the |color| option used in normal \tikzname\ options, you may
    omit the |color=| part of the option. Whenever an \meta{arrow key} is
    encountered that \tikzname\ does not recognize, it will test whether the
    key is the name of a color and, if so, execute |color=|\meta{arrow key}.
    So, the first of the above examples can be rewritten as follows:
    %
\begin{codeexample}[width=3cm,preamble={\usetikzlibrary{arrows.meta}}]
\tikz [ultra thick] \draw [red, arrows = {-Stealth[blue]}] (0,0) -- (1,0);
\end{codeexample}

    The \meta{color} will apply both to any drawing and filling operations used
    to construct the path. For instance, even though the |Stealth| arrow tips
    looks like a filled quadrilateral, it is actually constructed by drawing a
    quadrilateral and then filling it in the same color as the drawing (see the
    |fill| option below to see the difference).

    When |color| is set to an empty text, the drawing color is always used to
    fill the arrow tips, even if a different color is specified for filling the
    path:
    %
\begin{codeexample}[width=3cm,preamble={\usetikzlibrary{arrows.meta}}]
\tikz [ultra thick] \draw [draw=red, fill=red!50, arrows = {-Stealth[length=10pt]}]
                          (0,0) -- (1,1) -- (2,0);
\end{codeexample}
    %
    As you can see in the above example, the filled area is not quite what you
    might have expected. The reason is that the path was actually internally
    shortened a bit so that the end of the ``fat line'' as inside the arrow tip
    and we get a ``clear'' arrow tip.

    In general, it is a good idea not to add arrow tips to paths that are
    filled.
\end{key}

\begin{key}{/pgf/arrow keys/fill=\meta{color or |none|}}
    Use this key to explicitly set the color used for filling the arrow tips.
    This color can be different from the color used to draw (stroke) the arrow
    tip:
    %
\begin{codeexample}[width=3cm,preamble={\usetikzlibrary{arrows.meta}}]
\tikz {
  \draw [help lines] (0,-.5) grid [step=1mm] (1,.5);
  \draw [thick, red, arrows = {-Stealth[fill=white,length=15pt]}] (0,0) -- (1,0);
}
\end{codeexample}
    %
    You can also specify the special ``color'' |none|. In this case, the arrow
    tip is not filled at all (not even with white):
    %
\begin{codeexample}[width=3cm,preamble={\usetikzlibrary{arrows.meta}}]
\tikz {
  \draw [help lines] (0,-.5) grid [step=1mm] (1,.5);
  \draw [thick, red, arrows = {-Stealth[fill=none,length=15pt]}] (0,0) -- (1,0);
}
\end{codeexample}
    %
    Note that such ``open'' arrow tips are a bit difficult to draw in some
    case: The problem is that the line must be shortened by just the right
    amount so that it ends exactly on the back end of the arrow tip. In some
    cases, especially when double lines are used, this will not be possible.

    \begin{key}{/pgf/arrow keys/open}
        A shorthand for |fill=none|.
    \end{key}

    When you use both the |color| and |fill| option, the |color| option must
    come first since it will reset the filling to the color specified for
    drawing.
    %
\begin{codeexample}[width=3cm,preamble={\usetikzlibrary{arrows.meta}}]
\tikz {
  \draw [help lines] (0,-.5) grid [step=1mm] (1,.5);
  \draw [thick, red, arrows = {-Stealth[color=blue, fill=white, length=15pt]}]
        (0,0) -- (1,0);
}
\end{codeexample}

    Note that by setting |fill| to the special color |pgffillcolor|, you can
    cause the arrow tips to be filled using the color used to fill the main
    path. (This special color is always available and always set to the current
    filling color of the graphic state.):
    %
\begin{codeexample}[width=3cm,preamble={\usetikzlibrary{arrows.meta}}]
\tikz [ultra thick] \draw [draw=red, fill=red!50,
                           arrows = {-Stealth[length=15pt, fill=pgffillcolor]}]
                          (0,0) -- (1,1) -- (2,0);
\end{codeexample}
    %
\end{key}


\subsubsection{Line Styling}
\label{section-arrow-key-caps}

Arrow tips are created by drawing and possibly filling a path that makes up the
arrow tip. When \tikzname\ draws a path, there are different ways in which such
a path can be drawn (such as dashing). Three particularly important parameters
are the line join, the line cap, see Section~\ref{section-line-cap} for an
introduction, and the line width (thickness).

\tikzname\ resets the line cap and line join each time it draws an arrow tip
since you usually do not want their settings to ``spill over'' to the way the
arrow tips are drawn. You can, however, change there values explicitly for an
arrow tip:

\begin{key}{/pgf/arrow keys/line cap=\meta{|round| or |butt|}}
    Sets the line cap of all lines that are drawn in the arrow to a round cap
    or a butt cap. (Unlike for normal lines, the |rect| cap is not allowed.)
    Naturally, this key has no effect for arrows whose paths are closed.

    Each arrow tip has a default value for the line cap, which can be overruled
    using this option.

    Changing the cap should have no effect on the size of the arrow. However,
    it will have an effect on where the exact ``tip'' of the arrow is since
    this will always be exactly at the end of the arrow:
    %
\begin{codeexample}[width=3cm,preamble={\usetikzlibrary{arrows.meta}}]
\tikz [line width=2mm]
  \draw [arrows = {-Computer Modern Rightarrow[line cap=butt]}]
        (0,0) -- (1,0);
\end{codeexample}
\begin{codeexample}[width=3cm,preamble={\usetikzlibrary{arrows.meta}}]
\tikz [line width=2mm]
  \draw [arrows = {-Computer Modern Rightarrow[line cap=round]}]
        (0,0) -- (1,0);
\end{codeexample}
\begin{codeexample}[width=3cm,preamble={\usetikzlibrary{arrows.meta}}]
\tikz [line width=2mm]
  \draw [arrows = {-Bracket[reversed,line cap=butt]}]
        (0,0) -- (1,0);
\end{codeexample}
\begin{codeexample}[width=3cm,preamble={\usetikzlibrary{arrows.meta}}]
\tikz [line width=2mm]
  \draw [arrows = {-Bracket[reversed,line cap=round]}]
        (0,0) -- (1,0);
\end{codeexample}
    %
\end{key}

\begin{key}{/pgf/arrow keys/line join=\meta{|round| or |miter|}}
    Sets the line join to round or miter (|bevel| is not allowed). This time,
    the key only has an effect on paths that have ``corners'' in them. The same
    rules as for |line cap| apply: the size is not affects, but the tip end is:
    %
\begin{codeexample}[width=3cm,preamble={\usetikzlibrary{arrows.meta}}]
\tikz [line width=2mm]
  \draw [arrows = {-Computer Modern Rightarrow[line join=miter]}]
        (0,0) -- (1,0);
\end{codeexample}
\begin{codeexample}[width=3cm,preamble={\usetikzlibrary{arrows.meta}}]
\tikz [line width=2mm]
  \draw [arrows = {-Computer Modern Rightarrow[line join=round]}]
        (0,0) -- (1,0);
\end{codeexample}
\begin{codeexample}[width=3cm,preamble={\usetikzlibrary{arrows.meta}}]
\tikz [line width=2mm]
  \draw [arrows = {-Bracket[reversed,line join=miter]}]
        (0,0) -- (1,0);
\end{codeexample}
\begin{codeexample}[width=3cm,preamble={\usetikzlibrary{arrows.meta}}]
\tikz [line width=2mm]
  \draw [arrows = {-Bracket[reversed,line join=round]}]
        (0,0) -- (1,0);
\end{codeexample}
    %
\end{key}

The following keys set both of the above:

\begin{key}{/pgf/arrow keys/round}
    A shorthand for |line cap=round, line join=round|, resulting in ``rounded''
    arrow heads.
    %
\begin{codeexample}[width=3cm,preamble={\usetikzlibrary{arrows.meta}}]
\tikz [line width=2mm]
  \draw [arrows = {-Computer Modern Rightarrow[round]}] (0,0) -- (1,0);
\end{codeexample}
\begin{codeexample}[width=3cm,preamble={\usetikzlibrary{arrows.meta}}]
\tikz [line width=2mm]
  \draw [arrows = {-Bracket[reversed,round]}] (0,0) -- (1,0);
\end{codeexample}
    %
\end{key}

\begin{key}{/pgf/arrow keys/sharp}
    A shorthand for |line cap=butt, line join=miter|, resulting in ``sharp'' or
    ``pointed'' arrow heads.
    %
\begin{codeexample}[width=3cm,preamble={\usetikzlibrary{arrows.meta}}]
\tikz [line width=2mm]
  \draw [arrows = {-Computer Modern Rightarrow[sharp]}] (0,0) -- (1,0);
\end{codeexample}
\begin{codeexample}[width=3cm,preamble={\usetikzlibrary{arrows.meta}}]
\tikz [line width=2mm]
  \draw [arrows = {-Bracket[reversed,sharp]}] (0,0) -- (1,0);
\end{codeexample}
    %
\end{key}

You can also set the width of lines used inside arrow tips:

\begin{key}{/pgf/arrow keys/line width=\meta{dimension}| |\opt{\meta{line width factor}}%
        | |\opt{\meta{outer factor}}}
    This key sets the line width inside an arrow tip for drawing (out)lines of
    the arrow tip. When you set this width to |0pt|, which makes sense only for
    closed tips, the arrow tip is only filled. This can result in better
    rendering of some small arrow tips and in case of bend arrow tips (because
    the line joins will also be bend and not ``mitered''.)

    The meaning of the factors is as usual the same as for |length| or |width|.
    %
\begin{codeexample}[width=2cm,preamble={\usetikzlibrary{arrows.meta}}]
\tikz \draw [arrows = {-Latex[line width=0.1pt, fill=white, length=10pt]}] (0,0) -- (1,0);
\end{codeexample}
\begin{codeexample}[width=2cm,preamble={\usetikzlibrary{arrows.meta}}]
\tikz \draw [arrows = {-Latex[line width=1pt, fill=white, length=10pt]}] (0,0) -- (1,0);
\end{codeexample}
    %
\end{key}

\begin{key}{/pgf/arrow keys/line width'=\meta{dimension}| |\opt{\meta{length factor}}}
    Works like |line width| only the factor is with respect to the |length|.
\end{key}


\subsubsection{Bending and Flexing}
\label{section-arrow-flex}

Up to now, we have only added arrow tip to the end of straight lines, which is
in some sense ``easy''. Things get far more difficult, if the line to which we
wish to end an arrow tip is curved. In the following, we have a look at the
different actions that can be taken and how they can be configured.

To get a feeling for the difficulties involved, consider the following
situation: We have a ``gray wall'' at the $x$-coordinate of and a red line that
ends in its middle.
%
\begin{codeexample}[preamble={\usetikzlibrary{patterns}}]
\def\wall{ \fill     [fill=black!50]  (1,-.5) rectangle (2,.5);
           \pattern  [pattern=bricks] (1,-.5) rectangle (2,.5);
           \draw     [line width=1pt]  (1cm+.5pt,-.5) -- ++(0,1); }
\begin{tikzpicture}
  \wall
  % The "line"
  \draw [red,line width=1mm] (-1,0) -- (1,0);
\end{tikzpicture}
\end{codeexample}

Now we wish to add a blue open arrow tip the red line like, say,
|Stealth[length=1cm,open,blue]|:
%
\begin{codeexample}[setup code,hidden]
\usetikzlibrary{patterns}
\def\wall{ \fill     [fill=black!50]  (1,-.5) rectangle (2,.5);
           \pattern  [pattern=bricks] (1,-.5) rectangle (2,.5);
           \draw     [line width=1pt]  (1cm+.5pt,-.5) -- ++(0,1); }
\end{codeexample}
\begin{codeexample}[preamble={\usetikzlibrary{arrows.meta}}]
\begin{tikzpicture}
  \wall
  \draw [red,line width=1mm,-{Stealth[length=1cm,open,blue]}]
        (-1,0) -- (1,0);
\end{tikzpicture}
\end{codeexample}

There are several noteworthy things about the blue arrow tip:
%
\begin{enumerate}
    \item Notice that the red line no longer goes all the way to the wall.
        Indeed, the red line ends more or less exactly where it meets the
        blue line, leaving the arrow tip empty. Now, recall that the red line
        was supposed to be the path |(-2,0)--(1,0)|; however, this path has
        obviously become much shorter (by 6.25mm to be precise). This effect
        is called \emph{path shortening} in \tikzname.
    \item The very tip of the arrow just ``touches'' the wall, even we zoom
        out a lot. This point, where the original path ended and where the
        arrow tip should now lie, is called the \emph{tip end} in \tikzname.
    \item Finally, the point where the red line touches the blue line is the
        point where the original path ``visually ends''. Notice that this is
        not the same as the point that lies at a distance of the arrow's
        |length| from the wall -- rather it lies at a distance of |length|
        minus the |inset|. Let us call this point the \emph{visual
    end} of the arrow.
\end{enumerate}

As pointed out earlier, for straight lines, shortening the path and rotating
and shifting the arrow tip so that it ends precisely at the tip end and the
visual end lies on a line from the tip end to the start of the line is
relatively easy.

For curved lines, things are much more difficult and \tikzname\ copes with the
difficulties in different ways, depending on which options you add to arrows.
Here is now a curved red line to which we wish to add our arrow tip (the
original straight red line is shown in light red):
%
\begin{codeexample}[]
\begin{tikzpicture}
  \wall
  \draw [red!25,line width=1mm] (-1,0) -- (1,0);
  \draw [red,line width=1mm] (-1,-.5) .. controls (0,-.5) and (0,0) .. (1,0);
\end{tikzpicture}
\end{codeexample}

The first way of dealing with curved lines is dubbed the ``quick and dirty''
way (although the option for selecting this option is politely just called
``|quick|'' \dots):

\begin{key}{/pgf/arrow keys/quick}
    Recall that curves in \tikzname\ are actually Bézier curves, which means
    that they start and end at certain points and we specify two vectors, one
    for the start and one for the end, that provide tangents to the curve at
    these points. In particular, for the end of the curve, there is a point
    called the \emph{second support point} of the curve such that a tangent to
    the curve at the end goes through this point. In our above example, the
    second support point is at the middle of the light red line and, indeed, a
    tangent to the red line at the point touching the wall is perfectly
    horizontal.

    In order to add our arrow tip to the curved path, our first objective is to
    ``shorten'' the path by 6.25mm. Unfortunately, this is now much more
    difficult than for a straight path. When the |quick| option is added to an
    arrow tip (it is also the default if no special libraries are loaded), we
    cheat somewhat: Instead of really moving along 6.25mm along the path, we
    simply shift the end of the curve by 6.25mm \emph{along the tangent} (which
    is easy to compute). We also have to shift the second support point by the
    same amount to ensure that the line still has the same tangent at the end.
    This will result in the following:
    %
\begin{codeexample}[preamble={\usetikzlibrary{arrows.meta}}]
\begin{tikzpicture}
  \wall
  \draw [red!25,line width=1mm] (-1,0) -- (1,0);
  \draw [red,line width=1mm,-{Stealth[length=1cm,open,blue,quick]}]
        (-1,-.5) .. controls (0,-.5) and (0,0) .. (1,0);
\end{tikzpicture}
\end{codeexample}

    They main problem with the above picture is that the red line is no longer
    equal to the original red line (notice much sharper curvature near its
    end). In our example this is not such a bad thing, but it certainly ``not a
    nice thing'' that adding arrow tips to a curve changes the overall shape of
    the curves. This is especially bothersome if there are several similar
    curves that have different arrow heads. In this case, the similar curves
    now suddenly look different.

    Another big problem with the above approach is that it works only well if
    there is only a single arrow tip. When there are multiple ones, simply
    shifting them along the tangent as the |quick| option does produces
    less-than-satisfactory results:
    %
\begin{codeexample}[]
\begin{tikzpicture}
  \wall
  \draw [red!25,line width=1mm] (-1,0) -- (1,0);
  \draw [red,line width=1mm,-{[quick,sep]>>>}]
        (-1,-.5) .. controls (0,-.5) and (0,0) .. (1,0);
\end{tikzpicture}
\end{codeexample}
    %
    Note that the third arrow tip does not really lie on the curve any more.
\end{key}

Because of the shortcomings of the |quick| key, more powerful mechanisms for
shortening lines and rotating arrows tips have been implemented. To use them,
you need to load the following library:

\begin{tikzlibrary}{bending}
    Load this library to use the |flex|, |flex'|, or |bending| arrow keys. When
    this library is loaded, |flex| becomes the default mode that is used with
    all paths, unless |quick| is explicitly selected for the arrow tip.
\end{tikzlibrary}

\begin{key}{/pgf/arrow keys/flex=\opt{\meta{factor}} (default 1)}
    When the |bending| library is loaded, this key is applied to all arrow tips
    by default. It has the following effect:
    %
    \begin{enumerate}
        \item Instead of simply shifting the visual end of the arrow along
            the tangent of the curve's end, we really move it along the curve
            by the necessary distance. This operation is more expensive than
            the |quick| operation -- but not \emph{that} expensive, only
            expensive enough so that it is not selected by default for all
            arrow tips. Indeed, some compromises are made in the
            implementation where accuracy was traded for speed, so the
            distance by which the line end is shifted is not necessarily
            \emph{exactly} 6.25mm; only something reasonably close.
        \item The supports of the line are updated accordingly so that the
            shortened line will still follow \emph{exactly} the original
            line. This means that the curve deformation effect caused by the
            |quick| command does not happen here.
        \item Next, the arrow tip is rotated and shifted as follows: First,
            we shift it so that its tip is exactly at the tip end, where the
            original line ended. Then, the arrow is rotated so the \emph{the
            visual end lies on the line}:
            %
\begin{codeexample}[preamble={\usetikzlibrary{arrows.meta,bending}}]
\begin{tikzpicture}
  \wall
  \draw [red!25,line width=1mm] (-1,0) -- (1,0);
  \draw [red,line width=1mm,-{Stealth[length=1cm,open,blue,flex]}]
        (-1,-.5) .. controls (0,-.5) and (0,0) .. (1,0);
\end{tikzpicture}
\end{codeexample}
    \end{enumerate}

    As can be seen in the example, the |flex| option gives a result that is
    visually pleasing and does not deform the path.

    There is, however, one possible problem with the |flex| option: The arrow
    tip no longer points along the tangent of the end of the path. This may or
    may not be a problem, put especially for larger arrow tips readers will use
    the orientation of the arrow head to gauge the direction of the tangent of
    the line. If this tangent is important (for example, if it should be
    horizontal), then it may be necessary to enforce that the arrow tip
    ``really points in the direction of the tangent''.

    To achieve this, the |flex| option takes an optional \meta{factor}
    parameter, which defaults to |1|. This factor specifies how much the arrow
    tip should be rotated: If set to |0|, the arrow points exactly along a
    tangent to curve at its tip. If set to |1|, the arrow point exactly along a
    line from the visual end point on the curve to the tip. For values in the
    middle, we interpolate the rotation between these two extremes; so
    |flex=.5| will rotate the arrow's visual end ``halfway away from the
    tangent towards the actual position on the line''.
    %
\begin{codeexample}[preamble={\usetikzlibrary{arrows.meta,bending}}]
\begin{tikzpicture}
  \wall
  \draw [red!25,line width=1mm] (-1,0) -- (1,0);
  \draw [red,line width=1mm,-{Stealth[length=1cm,open,blue,flex=0]}]
        (-1,-.5) .. controls (0,-.5) and (0,0) .. (1,0);
\end{tikzpicture}
\end{codeexample}
\begin{codeexample}[preamble={\usetikzlibrary{arrows.meta,bending}}]
\begin{tikzpicture}
  \wall
  \draw [red!25,line width=1mm] (-1,0) -- (1,0);
  \draw [red,line width=1mm,-{Stealth[length=1cm,open,blue,flex=.5]}]
        (-1,-.5) .. controls (0,-.5) and (0,0) .. (1,0);
\end{tikzpicture}
\end{codeexample}
    %
    Note how in the above examples the red line is visible inside the open
    arrow tip. Open arrow tips do not go well with a flex value other than~|1|.
    Here is a more realistic use of the |flex=0| key:
    %
\begin{codeexample}[preamble={\usetikzlibrary{arrows.meta,bending}}]
\begin{tikzpicture}
  \wall
  \draw [red!25,line width=1mm] (-1,0) -- (1,0);
  \draw [red,line width=1mm,-{Stealth[length=1cm,flex=0]}]
        (-1,-.5) .. controls (0,-.5) and (0,0) .. (1,0);
\end{tikzpicture}
\end{codeexample}
    %
    If there are several arrow tips on a path, the |flex| option positions them
    independently, so that each of them lies optimally on the path:
    %
\begin{codeexample}[preamble={\usetikzlibrary{bending}}]
\begin{tikzpicture}
  \wall
  \draw [red!25,line width=1mm] (-1,0) -- (1,0);
  \draw [red,line width=1mm,-{[flex,sep]>>>}]
        (-1,-.5) .. controls (0,-.5) and (0,0) .. (1,0);
\end{tikzpicture}
\end{codeexample}
    %
\end{key}

\begin{key}{/pgf/arrow keys/flex'=\opt{\meta{factor}} (default 1)}
    The |flex'| key is almost identical to the |flex| key. The only difference
    is that a factor of |1| corresponds to rotating the arrow tip so that the
    instead of the visual end, the ``ultimate back end'' of the arrow tip lies
    on the red path. In the example instead of having the arrow tip at a
    distance of |6.25mm| from the tip lie on the path, we have the point at a
    distance of |1cm| from the tip lie on the path:
    %
\begin{codeexample}[preamble={\usetikzlibrary{arrows.meta,bending}}]
\begin{tikzpicture}
  \wall
  \draw [red!25,line width=1mm] (-1,0) -- (1,0);
  \draw [red,line width=1mm,-{Stealth[length=1cm,open,blue,flex']}]
        (-1,-.5) .. controls (0,-.5) and (0,0) .. (1,0);
\end{tikzpicture}
\end{codeexample}
    %
    Otherwise, the factor works as for |flex| and, indeed |flex=0| and
    |flex'=0| have the same effect.

    The main use of this option is not so much with an arrow tip like |Stealth|
    but rather with tips like the standard |>| in the context of a strongly
    curved line:
    %
\begin{codeexample}[preamble={\usetikzlibrary{arrows.meta,bending}}]
\begin{tikzpicture}
  \wall
  \draw [red!25,line width=1mm] (-1,0) -- (1,0);
  \draw [red,line width=1mm,-{Computer Modern Rightarrow[flex]}]
        (0,-.5) .. controls (1,-.5) and (0.5,0) .. (1,0);
\end{tikzpicture}
\end{codeexample}
    %
    In the example, the |flex| option does not really flex the arrow since for
    a tip like the Computer Modern arrow, the visual end is the same as the
    arrow tip -- after all, the red line does, indeed, end almost exactly where
    it used to end.

    Nevertheless, you may feel that the arrow tip looks ``wrong'' in the sense
    that it should be rotated. This is exactly what the |flex'| option does
    since it allows us to align the ``back end'' of the tip with the red line:
    %
\begin{codeexample}[preamble={\usetikzlibrary{arrows.meta,bending}}]
\begin{tikzpicture}
  \wall
  \draw [red!25,line width=1mm] (-1,0) -- (1,0);
  \draw [red,line width=1mm,-{Computer Modern Rightarrow[flex'=.75]}]
        (0,-.5) .. controls (1,-.5) and (0.5,0) .. (1,0);
\end{tikzpicture}
\end{codeexample}
    %
    In the example, I used |flex'=.75| so as not to overpronounce the effect.
    Usually, you will have to fiddle with it sometime to get the ``perfectly
    aligned arrow tip'', but a value of |.75| is usually a good start.
\end{key}

\begin{key}{/pgf/arrow keys/bend}
    \emph{Bending} an arrow tip is a radical solution to the problem of
    positioning arrow tips on a curved line: The arrow tip is no longer
    ``rigid'' but the drawing itself will now bend along the curve. This has
    the advantage that all the problems of flexing with wrong tangents and
    overflexing disappear. The downsides are longer computation times (bending
    an arrow is \emph{much} more expensive that flexing it, let alone than
    quick mode) and also the fact that excessive bending can lead to ugly arrow
    tips. On the other hand, for most arrow tips their bend version are
    visually quite pleasing and create a sophisticated look:
    %
\begin{codeexample}[preamble={\usetikzlibrary{arrows.meta,bending}}]
\begin{tikzpicture}
  \wall
  \draw [red!25,line width=1mm] (-1,0) -- (1,0);
  \draw [red,line width=1mm,-{Stealth[length=20pt,bend]}]
        (-1,-.5) .. controls (0,-.5) and (0,0) .. (1,0);
\end{tikzpicture}
\end{codeexample}
\begin{codeexample}[preamble={\usetikzlibrary{bending}}]
\begin{tikzpicture}
  \wall
  \draw [red!25,line width=1mm] (-1,0) -- (1,0);
  \draw [red,line width=1mm,-{[bend,sep]>>>}]
        (-1,-.5) .. controls (0,-.5) and (0,0) .. (1,0);
\end{tikzpicture}
\end{codeexample}
\begin{codeexample}[preamble={\usetikzlibrary{arrows.meta,bending}}]
\begin{tikzpicture}
  \wall
  \draw [red!25,line width=1mm] (-1,0) -- (1,0);
  \draw [red,line width=1mm,-{Stealth[bend,round,length=20pt]}]
        (0,-.5) .. controls (1,-.5) and (0.25,0) .. (1,0);
\end{tikzpicture}
\end{codeexample}
    %
\end{key}
% TODOsp: codeexamples: `bending` library is needed up to here
% TODOsp: codeexamples: `\def\wall` + `patterns` library are needed up to here
% TODOsp: codeexamples: `arrows.meta` library needed up to here


\subsection{Arrow Tip Specifications}
\label{section-arrow-spec}

\subsubsection{Syntax}

When you select the arrow tips for the start and the end of a path, you can
specify a whole sequence of arrow tips, each having its own local options. At
the beginning of this section, it was pointed out that the syntax for selecting
the start and end arrow tips is the following:
%
\begin{quote}
    \meta{start specification}|-|\meta{end specification}
\end{quote}

We now have a closer look at what these specifications may look like. The
general syntax of the \meta{start specification} is as follows:
%
\begin{quote}
    \opt{|[|\meta{options for all tips}|]|} \meta{first arrow tip spec}
    \meta{second arrow tip spec} \meta{third arrow tip spec} \dots
\end{quote}
%
As can be seen, an arrow tip specification may start with some options in
brackets. If this is the case, the \meta{options for all tips} will, indeed, be
applied to all arrow tips that follow. (We will see, in a moment, that there
are even more places where options may be specified and a list of the ordering
in which the options are applied will be given later.)

The main part of a specification is taken up by a sequence of individual arrow
tip specifications. Such a specification can be of three kinds:
%
\begin{enumerate}
    \item It can be of the form \meta{arrow tip kind
        name}|[|\meta{options}|]|.
    \item It can be of the form \meta{shorthand}|[|\meta{options}|]|.
    \item It can be of the form \meta{single char
        shorthand}\opt{|[|\meta{options}|]|}. Note that only for this form
        the brackets are optional.
\end{enumerate}

The easiest kind is the first one: This adds an arrow tip of the kind
\meta{arrow tip kind name} to the sequence of arrow tips with the
\meta{options} applied to it at the start (for the \meta{start specification})
or at the end (for the \meta{end specification}). Note that for the \meta{start
specification} the first arrow tip specified in this way will be at the very
start of the curve, while for the \meta{end specification} the ordering is
reversed: The last arrow tip specified will be at the very end of the curve.
This implies that a specification like
%
\begin{quote}
    |Stealth[] Latex[] - Latex[] Stealth[]|
\end{quote}
%
will give perfectly symmetric arrow tips on a line (as one would expect).

It is important that even if there are no \meta{options} for an arrow tip, the
square brackets still need to be written to indicate the end of the arrow tip's
name. Indeed, the opening brackets are used to divide the arrow tip
specification into names.

Instead of a \meta{arrow tip kind name}, you may also provide the name of a
so-called \emph{shorthand}. Shorthands look like normal arrow tip kind names
and, indeed, you will often be using shorthands without noticing that you do.
The idea is that instead of, say, |Computer Modern Rightarrow| you might wish
to just write |Rightarrow| or perhaps just |To| or even just |>|. For this, you
can create a shorthand that tells \tikzname\ that whenever this shorthand is
used, another arrow tip kind is meant. (Actually, shorthands are somewhat more
powerful, we have a detailed look at them in
Section~\ref{section-arrow-tip-macro}.) For shorthands, the same rules apply as
for normal arrow tip kinds: You \emph{need} to provide brackets so that
\tikzname\ can find the end of the name inside a longer specification.

The third kind of arrow tip specifications consist of just a single letter like
|>| or |)| or |*| or even |o| or |x| (but you may not use |[|, |]|, or |-|
since they will confuse the parser). These single letter arrow specifications
will invariably be shorthands that select some ``real'' arrow tip instead. An
important feature of single letter arrow tips is that they do \emph{not} need
to be followed by options (but they may).

Now, since we can use any letter for single letter shorthands, how can
\tikzname\ tell whether by |foo[]| we mean an arrow tip kind |foo| without any
options or whether we mean an arrow tip called |f|, followed by two arrow tips
called |o|? Or perhaps an arrow tip called |f| followed by an arrow tip called
|oo|? To solve this problem, the following rule is used to determine which of
the three possible specifications listed above applies: First, we check whether
everything from the current position up to the next opening bracket (or up to
the end) is the name of an arrow tip or of a shorthand. In our case, |foo|
would first be tested under this rule. Only if |foo| is neither the name of an
arrow tip kind nor of a shorthand does \tikzname\ consider the first letter of
the specification, |f| in our case. If this is not the name of a shorthand, an
error is raised. Otherwise the arrow tip corresponding to |f| is added to the
list of arrow tips and the process restarts with the rest. Thus, we would next
text whether |oo| is the name of an arrow tip or shorthand and, if not, whether
|o| is such a name.

All of the above rules mean that you can rather easily specify arrow tip
sequences if they either mostly consist of single letter names or of longer
names. Here are some examples:
%
\begin{itemize}
    \item |->>>| is interpreted as three times the |>| shorthand since |>>>| is
        not the name of any arrow tip kind (and neither is |>>|).
    \item |->[]>>| has the same effect as the above.
    \item |-[]>>>| also has the same effect.
    \item |->[]>[]>[]| so does this.
    \item |->Stealth| yields an arrow tip |>| followed by a |Stealth| arrow
        at the end.
    \item |-Stealth>| is illegal since there is no arrow tip |Stealth>| and
        since |S| is also not the name of any arrow tip.
    \item |-Stealth[] >| is legal and does what was presumably meant in the
        previous item.
    \item |< Stealth-| is legal and is the counterpart to |-Stealth[] >|.
    \item |-Stealth[length=5pt] Stealth[length=6pt]| selects two stealth
        arrow tips, but at slightly different sizes for the end of lines.
\end{itemize}

An interesting question concerns how flexing and bending interact with multiple
arrow tips: After all, flexing and quick mode use different ways of shortening
the path so we cannot really mix them. The following rule is used: We check,
independently for the start and the end specifications, whether at least one
arrow tip in them uses one of the options |flex|, |flex'|, or |bend|. If so,
all |quick| settings in the other arrow tips are ignored and treated as if
|flex| had been selected for them, too.


\subsubsection{Specifying Paddings}

When you provide several arrow tips in a row, all of them are added to the
start or end of the line:
%
\begin{codeexample}[]
\tikz \draw [<<<->>>>] (0,0) -- (2,0);
\end{codeexample}
%
The question now is what will be the distance between them? For this, the
following arrow key is important:

\begin{key}{/pgf/arrow keys/sep=\meta{dimension}| |\opt{\meta{line
    width factor}}| |\opt{\meta{outer factor}} (default 0.88pt .3 1)%
}
    When a sequence of arrow tips is specified in an arrow tip specification
    for the end of the line, the arrow tips are normally arranged in such a way
    that the tip of each arrow ends exactly at the ``back end'' of the next
    arrow tip (for start specifications, the ordering is inverted, of course).
    Now, when the |sep| option is set, instead of exactly touching the back end
    of the next arrow, the specified \meta{dimension} is added as additional
    space (the distance may also be negative, resulting in an overlap of the
    arrow tips). The optional factors have the same meaning as for the |length|
    key, see that key for details.

    Let us now have a look at some examples. First, we use two arrow tips with
    different separations between them:
    %
\begin{codeexample}[preamble={\usetikzlibrary{arrows.meta}}]
\tikz {
  \draw [-{>[sep=1pt]>[sep= 2pt]>}] (0,1.0) -- (1,1.0);
  \draw [-{>[sep=1pt]>[sep=-2pt]>}] (0,0.5) -- (1,0.5);
  \draw [-{>         >[sep]     >}] (0,0.0) -- (1,0.0);
}
\end{codeexample}

    You can also specify a |sep| for the last arrow tip in the sequence (for
    end specifications, otherwise for the first arrow tip). In this case, this
    first arrow tip will not exactly ``touch'' the point where the path ends,
    but will rather leave the specified amount of space. This is usually quite
    desirable.
    %
\begin{codeexample}[preamble={\usetikzlibrary{arrows.meta,positioning}}]
\tikz {
  \node [draw] (A) {A};
  \node [draw] (B) [right=of A] {B};

  \draw [-{>>[sep=2pt]}] (A) to [bend left=45] (B);
  \draw [- >>          ] (A) to [bend right=45] (B);
}
\end{codeexample}
    %
    Indeed, adding a |sep| to an arrow tip is \emph{very} desirable, so you
    will usually write something like |>={To[sep]}| somewhere near the start of
    your files.

    One arrow tip kind can be quite useful in this context: The arrow tip kind
    |_|. It draws nothing and has zero length, \emph{but} it has |sep| set as a
    default option. Since it is a single letter shorthand, you can write short
    and clean ``code'' in this way:
    %
\begin{codeexample}[]
\tikz \draw [->_>] (0,0) -- (1,0);
\end{codeexample}
\begin{codeexample}[]
\tikz \draw [->__>] (0,0) -- (1,0);
\end{codeexample}
    %
    However, using the |sep| option will be faster than using the |_| arrow tip
    and it also allows you to specify the desired length directly.
\end{key}


\subsubsection{Specifying the Line End}

In the previous examples of sequences of arrow tips, the line of the path
always ended at the last of the arrow tips (for end specifications) or at the
first of the arrow tips (for start specifications). Often, this is what you may
want, but not always. Fortunately, it is quite easy to specify the desired end
of the line: The special single char shorthand |.| is reserved to indicate that
last arrow that is still part of the line; in other words, the line will stop
at the last arrow before |.| is encountered (for end specifications) are at the
first arrow following |.| (for start specifications).
%
\begin{codeexample}[]
\tikz [very thick] \draw [<<<->>>] (0,0) -- (2,0);
\end{codeexample}
\begin{codeexample}[]
\tikz [very thick] \draw [<.<<->.>>] (0,0) -- (2,0);
\end{codeexample}
\begin{codeexample}[]
\tikz [very thick] \draw [<<.<-.>>>] (0,0) -- (2,0);
\end{codeexample}
\begin{codeexample}[]
\tikz [very thick] \draw [<<.<->.>>] (0,0) to [bend left] (2,0);
\end{codeexample}

It is permissible that there are several dots in a specification, in this case
the first one ``wins'' (for end specifications, otherwise the last one).

Note that |.| is parsed as any other shorthand. In particular, if you wish to
add a dot after a normal arrow tip kind name, you need to add brackets:
%
\begin{codeexample}[preamble={\usetikzlibrary{arrows.meta}}]
\tikz [very thick] \draw [-{Stealth[] . Stealth[] Stealth[]}] (0,0) -- (2,0);
\end{codeexample}
%
Adding options to |.| is permissible, but they have no effect. In particular,
|sep| has no effect since a dot is not an arrow.


\subsubsection{Defining Shorthands}
\label{section-arrow-tip-macro}

It is often desirable to create ``shorthands'' for the names of arrow tips that
you are going to use very often. Indeed, in most documents you will only need a
single arrow tip kind and it would be useful that you could refer to it just as
|>| in your arrow tip specifications. As another example, you might constantly
wish to switch between a filled and a non-filled circle as arrow tips and would
like to use |*| and |o| are shorthands for these case. Finally, you might just
like to shorten a long name like |Computer Modern Rightarrow| down to just, say
|To| or something similar.

All of these case can be addressed by defining appropriate shorthands. This is
done using the following handler:

\begin{handler}{{.tip}{=\meta{end specification}}}
    Defined the \meta{key} as a name that can be used inside arrow tip
    specifications. If the \meta{key} has a path before it, this path is
    ignored (so there is only one ``namespace'' for arrow tips). Whenever it is
    used, it will be replaced by the \meta{end specification}. Note that you
    must \emph{always} provide (only) an end specification; when the \meta{key}
    is used inside a start specification, the ordering and the meaning of the
    keys inside the \meta{end specification} are translated automatically.
    \todosp{remaining instance of bug \#473}
    %
\begin{codeexample}[preamble={\usetikzlibrary{arrows.meta}}]
\tikz [foo /.tip = {Stealth[sep]. >>}]
  \draw [-foo] (0,0) -- (2,0);
\end{codeexample}
\begin{codeexample}[preamble={\usetikzlibrary{arrows.meta}}]
\tikz [foo /.tip = {Stealth[sep] Latex[sep]},
       bar /.tip = {Stealth[length=10pt,open]}]
  \draw [-{foo[red] . bar}] (0,0) -- (2,0);
\end{codeexample}

    In the last of the examples, we used |foo[red]| to make the arrows red. Any
    options given to a shorthand upon use will be passed on to the actual
    arrows tip for which the shorthand stands. Thus, we could also have written
    |Stealth[sep,red]| |Latex[sep,red]| instead of |foo[red]|. In other words,
    the ``replacement'' of a shorthand by its ``meaning'' is a semantic
    replacement rather than a syntactic replacement. In particular, the
    \meta{end specification} will be parsed immediately when the shorthand is
    being defined. However, this applies only to the options inside the
    specification, whose values are evaluated immediately. In contrast, which
    actual arrow tip kind is meant by a given shorthand used inside the
    \meta{end specification} is resolved only up each use of the shorthand.
    This means that when you write
    %
    \begin{quote}
        |dup /.tip = >>|
    \end{quote}
    %
    and then later write
    %
    \begin{quote}
        |> /.tip = whatever|
    \end{quote}
    %
    then |dup| will have the effect as if you had written
    |whatever[]whatever[]|. You will find that this behaviour is what one would
    expect.

    There is one problem we have not yet addressed: The asymmetry of single
    letter arrow tips like |>| or |)|. When someone writes
    %
\begin{codeexample}[]
\tikz \draw [<->] (0,0) -- (1,0);
\end{codeexample}
    %
    we rightfully expect one arrow tip pointing left at the left end and an
    arrow tip pointing right at the right end. However, compare
    %
\begin{codeexample}[]
\tikz \draw [>->] (0,0) -- (1,0);
\end{codeexample}
\begin{codeexample}[preamble={\usetikzlibrary{arrows.meta}}]
\tikz \draw [Stealth-Stealth] (0,0) -- (1,0);
\end{codeexample}
    %
    In both cases, we have \emph{identical} text in the start and end
    specifications, but in the first case we rightfully expect the left arrow
    to be flipped.

    The solution to this problem is that it is possible to define two names for
    the same arrow tip, namely one that is used inside start specifications and
    one for end specifications. Now, we can decree that the ``name of |>|''
    inside start specifications is simply |<| and the above problems disappear.

    To specify different names for a shorthand in start and end specifications,
    use the following syntax: Instead of \meta{key}, you use \meta{name in
    start specifications}|-|\meta{name in end specifications}. Thus, to set the
    |>| key correctly, you actually need to write
    %
\begin{codeexample}[preamble={\usetikzlibrary{arrows.meta}}]
\tikz [<-> /.tip = Stealth] \draw [<->>] (0,0) -- (1,0);
\end{codeexample}
\begin{codeexample}[preamble={\usetikzlibrary{arrows.meta}}]
\tikz [<-> /.tip = Latex] \draw [>-<] (0,0) -- (1,0);
\end{codeexample}

    Note that the above also works even though we have not set |<| as an arrow
    tip name for end specifications! The reason this works is that the
    \tikzname\ (more precisely, \pgfname) actually uses the following
    definition internally:
    %
    \begin{quote}
        |>-< /.tip = >[reversed]|
    \end{quote}
    %
    Translation: ``When |<| is used in an end specification, please replace it
    by |>|, but reversed. Also, when |>| is used in a start specification, we
    also mean this inverted |>|.''

    By default, |>| is a shorthand for |To| and |To| is a shorthand for |to|
    (an arrow from the old libraries) when |arrows.meta| is not loaded library.
    When |arrows.meta| is loaded, |To| is redefined to mean the same as
    |Computer Modern Rightarrow|.
\end{handler}

\begin{key}{/tikz/>=\meta{end arrow specification}}
    This is a short way of saying |<->/.tip=|\meta{end arrow specification}.
    %
\begin{codeexample}[preamble={\usetikzlibrary{arrows.meta}}]
\begin{tikzpicture}[scale=2,ultra thick]
  \begin{scope}[>=Latex]
    \draw[>->]    (0pt,3ex) -- (1cm,3ex);
    \draw[|<->>|] (0pt,2ex) -- (1cm,2ex);
  \end{scope}
  \begin{scope}[>=Stealth]
    \draw[>->]    (0pt,1ex) -- (1cm,1ex);
    \draw[|<<.<->|] (0pt,0ex) -- (1cm,0ex);
  \end{scope}
\end{tikzpicture}
\end{codeexample}
    %
\end{key}

\begin{key}{/tikz/shorten <=\meta{length}}
    Shorten the path by \meta{length} in the direction of the starting point.
\end{key}

\begin{key}{/tikz/shorten >=\meta{length}}
    Shorten the path by \meta{length} in the direction of the end point.
\end{key}


\subsubsection{Scoping of Arrow Keys}
\label{section-arrow-scopes}

There are numerous places where you can specify keys for an arrow tip. There
is, however, one final place that we have not yet mentioned:

\begin{key}{/tikz/arrows=|[|\meta{arrow keys}|]|}
    The |arrows| key, which is normally used to set the arrow tips for the
    current scope, can also be used to set some arrow keys for the current
    scope. When the argument to |arrows| starts with an opening bracket and
    only otherwise contains one further closing bracket at the very end, this
    semantic of the |arrow| key is assumed.

    The \meta{arrow keys} will be set for the rest of current scope. This is
    useful for generally setting some design parameters or for generally
    switching on, say, bending as in:
    %
\begin{codeexample}[code only]
\tikz [arrows={[bend]}] ... % Bend all arrows
\end{codeexample}
    %
\end{key}

We can now summarize which arrow keys are applied in what order when an arrow
tip is used:
%
\begin{enumerate}
    \item First, the so-called \emph{defaults} are applied, which are values
        for the different parameters of a key. They are fixed in the
        definition of the key and cannot be changed. Since they are executed
        first, they are only the ultimate fallback.
    \item The \meta{keys} from the use of |arrows=[|\meta{keys}|]| in all
        enclosing scopes.
    \item Recursively, the \meta{keys} provided with the arrow tip inside
        shorthands.
    \item The keys provided at the beginning of an arrow tip specification in
        brackets.
    \item The keys provided directly next to the arrow tip inside the
        specification.
\end{enumerate}


\subsection{Reference: Arrow Tips}
\label{section-arrows-meta}

\begin{pgflibrary}{arrows.meta}
    This library defines a large number of standard ``meta'' arrow tips.
    ``Meta'' means that you can configure these arrow tips in many different
    ways like changing their size or their line caps and joins and many other
    details.

    The only reason this library is not loaded by default is for compatibility
    with older versions of \tikzname. You can, however, safely load and use
    this library alongside the older libraries |arrows| and |arrows.spaced|.
\end{pgflibrary}

The different arrow tip kinds defined in the |arrows.meta| library can be
classified in different groups:
%
\begin{itemize}
    \item \emph{Barbed} arrow tips consist mainly of lines that ``point
        backward'' from the tip of the arrow and which are not filled. For
        them, filling has no effect. A typical example is \tikz [baseline]
        \draw (0,.5ex) -- (1.5em,.5ex) [-Straight Barb];. Here is the list of
        defined arrow tips:
        %
        \begin{arrowexamples}
            \arrowexample Arc Barb[]
            \arrowexample Bar[]
            \arrowexample Bracket[]
            \arrowexample Hooks[]
            \arrowexample Parenthesis[]
            \arrowexample Straight Barb[]
            \arrowexample Tee Barb[]
        \end{arrowexamples}

        All of these arrow tips can be configured and resized in many different
        ways as described in the following. Above, they are shown at their
        ``natural'' sizes, which are chosen in such a way that for a line width
        of 0.4pt their width matches the height of a letter ``x'' in Computer
        Modern at 11pt (with some ``overshooting'' to create visual
        consistency).
    \item \emph{Mathematical} arrow tips are actually a subclass of the
        barbed arrow tips, but we list them separately. They contain arrow
        tips that look exactly like the tips of arrows used in mathematical
        fonts such as the |\to|-symbol $\to$ from standard \TeX.
        %
        \begin{arrowexamples}
            \arrowexample Classical TikZ Rightarrow[]
            \arrowexample Computer Modern Rightarrow[]
            \arrowexampledouble Implies[]
            \arrowexample To[]
        \end{arrowexamples}
        %
        The |To| arrow tip is a shorthand for |Computer Modern Rightarrow| when
        |arrows.meta| is loaded.
    \item \emph{Geometric} arrow tips consist of a filled shape like a kite
        or a circle or a ``stealth-fighter-like'' shape. A typical example is
        \tikz [baseline] \draw (0,.5ex) -- (1.5em,.5ex) [-Stealth];. These
        arrow tips can also be used in an ``open'' variant as in \tikz
        [baseline] \draw (0,.5ex) -- (1.5em,.5ex) [-{Stealth[open]}];.
        %
        \begin{arrowexamples}
            \arrowexample Circle[]
            \arrowexample Diamond[]
            \arrowexample Ellipse[]
            \arrowexample Kite[]
            \arrowexample Latex[]
            \arrowexample Latex[round]
            \arrowexample Rectangle[]
            \arrowexample Square[]
            \arrowexample Stealth[]
            \arrowexample Stealth[round]
            \arrowexample Triangle[]
            \arrowexample Turned Square[]
        \end{arrowexamples}

        Here are the ``open'' variants:
        %
        \begin{arrowexamples}
            \arrowexample Circle[open]
            \arrowexample Diamond[open]
            \arrowexample Ellipse[open]
            \arrowexample Kite[open]
            \arrowexample Latex[open]
            \arrowexample Latex[round,open]
            \arrowexample Rectangle[open]
            \arrowexample Square[open]
            \arrowexample Stealth[open]
            \arrowexample Stealth[round,open]
            \arrowexample Triangle[open]
            \arrowexample Turned Square[open]
        \end{arrowexamples}

        Note that ``open'' arrow tips are not the same as ``filled with
        white'', which is also available (just say |fill=white|). The
        difference is that the background will ``shine through'' an open
        arrow, while a filled arrow always obscures the background:
        %
\begin{codeexample}[preamble={\usetikzlibrary{arrows.meta}}]
\tikz {
  \shade [left color=white, right color=red!50] (0,0) rectangle (4,1);

  \draw [ultra thick,-{Triangle[open]}]       (0,2/3) -- ++ (3,0);
  \draw [ultra thick,-{Triangle[fill=white]}] (0,1/3) -- ++ (3,0);
}
\end{codeexample}

    \item \emph{Cap} arrow tips are used to add a ``cap'' to the end of a
        line. The graphic languages underlying \tikzname\ (\textsc{pdf},
        \textsc{postscript} or \textsc{svg}) all support three basic types of
        line caps on a very low level: round, rectangular, and ``butt''.
        Using cap arrow tips, you can add new caps to lines and use different
        caps for the end and the start. An example is the line \tikz
        [baseline] \draw [line width=1ex, {Round Cap[reversed]}-{Triangle
        Cap[] . Fast Triangle[] Fast Triangle[]}] (0,0.5ex) -- (2em,0.5ex);.
        %
        \begin{arrowcapexamples}
            \arrowcapexample Butt Cap[]
            \arrowcapexample Fast Round[]
            \arrowcapexample Fast Triangle[]
            \arrowcapexample Round Cap[]
            \arrowcapexample Triangle Cap[]
        \end{arrowcapexamples}
    \item \emph{Special} arrow tips are used for some specific purpose and do
        not fit into the above categories.
        %
        \begin{arrowexamples}
            \arrowexample Rays[]
            \arrowexample Rays[n=8]
        \end{arrowexamples}
\end{itemize}


\subsubsection{Barbed Arrow Tips}

\begin{arrowtip}{Arc Barb}{
    This arrow tip attaches an arc to the end of the line whose angle is given
    by the |arc| option. The |length| and |width| parameters refer to the size
    of the arrow tip for |arc| set to 180 degrees, which is why in the example
    for |arc=210| the actual length is larger than the specified |length|. The
    line width is taken into account for the computation  of the length and
    width. Use the |round| option to add round caps to the end of the arcs.
}%
{length=1.5cm,arc=210}%
{length=1.5cm,width=3cm}

    \begin{arrowexamples}
        \arrowexample[]
        \arrowexampledup[sep]
        \arrowexampledupdot[sep]
        \arrowexample[arc=120]
        \arrowexample[arc=270]
        \arrowexample[length=2pt]
        \arrowexample[length=2pt,width=5pt]
        \arrowexample[line width=2pt]
        \arrowexample[reversed]
        \arrowexample[round]
        \arrowexample[slant=.3]
        \arrowexample[left]
        \arrowexample[right]
        \arrowexample[harpoon,reversed]
        \arrowexample[red]
    \end{arrowexamples}
    %
    The following options have no effect: |open|, |fill|.

    On |double| lines, the arrow tip will not look correct.
\end{arrowtip}

\begin{arrowtipsimple}{Bar}
    A simple bar. This is a simple instance of |Tee Barb| for length zero.
\end{arrowtipsimple}

\begin{arrowtip}{Bracket}{
    This is an instance of the |Tee Barb| \todosp{no space shown here} arrow tip that results in something
    resembling a bracket. Just like the |Parenthesis| arrow tip, a |Bracket| is
    not modelled from a text square bracket, but rather its size has been
    chosen so that it fits with the other arrow tips.
}%
{}%
{}

    \begin{arrowexamples}
        \arrowexample[]
        \arrowexampledup[sep]
        \arrowexampledupdot[sep]
        \arrowexample[reversed]
        \arrowexample[round]
        \arrowexample[slant=.3]
        \arrowexample[left]
        \arrowexample[right]
        \arrowexample[harpoon,reversed]
        \arrowexample[red]
    \end{arrowexamples}
    %
    The following options have no effect: |open|, |fill|.

    On |double| lines, the arrow tip will not look correct.
\end{arrowtip}

\begin{arrowtip}{Hooks}{
    This arrow tip attaches two ``hooks'' to the end of the line. The |length|
    and |width| parameters refer to the size of the arrow tip if both arcs are
    180 degrees; in the example the arc is 210 degrees and, thus, the arrow is
    actually longer that the |length| dictates. The line width is taken into
    account for the computation of the length and width. The |arc| option is
    used to specify the angle of the arcs. Use the |round| option to add round
    caps to the end of the arcs.
}%
{length=1cm,width=3.5cm,arc=210}%
{length=1cm,width=3.5cm}

    \begin{arrowexamples}
        \arrowexample[]
        \arrowexampledup[sep]
        \arrowexampledupdot[sep]
        \arrowexample[arc=120]
        \arrowexample[arc=270]
        \arrowexample[length=2pt]
        \arrowexample[length=2pt,width=5pt]
        \arrowexample[line width=2pt]
        \arrowexample[reversed]
        \arrowexample[round]
        \arrowexample[slant=.3]
        \arrowexample[left]
        \arrowexample[right]
        \arrowexample[harpoon,reversed]
        \arrowexample[red]
    \end{arrowexamples}
    %
    The following options have no effect: |open|, |fill|.

    On |double| lines, the arrow tip will not look correct.
\end{arrowtip}

\begin{arrowtip}{Parenthesis}{
    This arrow tip is an instantiation of the |Arc Barb| \todosp{no space shown here} so that it resembles a
    parenthesis. However, the idea is not to recreate a ``real'' parenthesis as
    it is used in text, but rather a ``bow'' at a size that harmonizes with the
    other arrow tips at their default sizes.
}%
{}%
{}

    \begin{arrowexamples}
        \arrowexample[]
        \arrowexampledup[sep]
        \arrowexampledupdot[sep]
        \arrowexample[reversed]
        \arrowexample[round]
        \arrowexample[slant=.3]
        \arrowexample[left]
        \arrowexample[right]
        \arrowexample[harpoon,reversed]
        \arrowexample[red]
    \end{arrowexamples}
    %
    The following options have no effect: |open|, |fill|.

    On |double| lines, the arrow tip will not look correct.
\end{arrowtip}

\begin{arrowtip}{Straight Barb}{
    This is the ``archetypal'' arrow head, consisting of just two straight
    lines. The |length| and |width| parameters refer to the horizontal and
    vertical distances between the points on the path making up the arrow tip.
    As can be seen, the line width of the arrow tip's path is not taken into
    account. The |angle| option is particularly useful to set the opening angle
    at the tip of the arrow head. The |round| option gives a ``softer'' or
    ``rounder'' version of the arrow tip.
}%
{length=2cm,width=3cm}%
{length=2cm/-4mm,width=3cm}

    \begin{arrowexamples}
        \arrowexample[]
%        \arrowexampledouble[]
        \arrowexampledup[]
        \arrowexampledupdot[]
        \arrowexample[length=5pt]
        \arrowexample[length=5pt,width=5pt]
        \arrowexample[line width=2pt]
        \arrowexample[reversed]
        \arrowexample[angle=60:2pt 3]
        \arrowexample[round]
        \arrowexample[slant=.3]
        \arrowexample[left]
        \arrowexample[right]
        \arrowexample[harpoon,reversed]
        \arrowexample[red]
    \end{arrowexamples}
    %
    The following options have no effect: |open|, |fill|.

    On |double| lines, the arrow tip will not look correct.
\end{arrowtip}

\begin{arrowtip}{Tee Barb}{
    This arrow tip attaches a little ``T'' on both sides of the tip. The arrow
    |inset| dictates the distance from the back end to the middle of the stem
    of the T. When the inset is equal to the length, the arrow tip is drawn as
    a single line, not as three lines (this is important for the ``round''
    version since, then, the corners get rounded).
}%
{length=1.5cm,width=3cm,inset=1cm}%
{length=1.5cm,width=3cm,inset=1cm}

    \begin{arrowexamples}
        \arrowexample[]
        \arrowexampledup[sep]
        \arrowexampledupdot[sep]
        \arrowexample[inset=0pt]
        \arrowexample[inset'=0pt 1]
        \arrowexample[line width=2pt]
        \arrowexample[round]
        \arrowexample[round,inset'=0pt 1]
        \arrowexample[slant=.3]
        \arrowexample[left]
        \arrowexample[right]
        \arrowexample[harpoon,reversed]
        \arrowexample[red]
    \end{arrowexamples}
    %
    The following options have no effect: |open|, |fill|.

    On |double| lines, the arrow tip will not look correct.
\end{arrowtip}


\subsubsection{Mathematical Barbed Arrow Tips}

\begin{arrowtip}{Classical TikZ Rightarrow}{
    This arrow tip is the ``old'' or ``classical'' arrow tip that used to be
    the standard in \tikzname\ in earlier versions. It was modelled on an old
    version of the tip of \texttt{\string\rightarrow} ($\rightarrow$) of the
    Computer Modern fonts. However, this ``old version'' was really old, Donald
    Knuth (the designer of both \TeX\ and of the Computer Modern fonts)
    replaced the arrow tip of the mathematical fonts in~1992.
}%
{length=1cm,width=2cm}%
{length=1cm,width=2cm}

    The main problem with this arrow tip is that it is ``too small'' at its
    natural size. I recommend using the new \texttt{Computer Modern Rightarrow}
    arrow tip instead, which matches the current $\to$. This new version is
    also the default used as |>| and as |To|, now.
    %
    \begin{arrowexamples}
        \arrowexample[]
        \arrowexampledup[sep]
        \arrowexampledupdot[sep]
        \arrowexample[length=3pt]
        \arrowexample[sharp]
        \arrowexample[slant=.3]
        \arrowexample[left]
        \arrowexample[right]
        \arrowexample[harpoon,reversed]
        \arrowexample[red]
    \end{arrowexamples}
    %
    The following options have no effect: |open|, |fill|.

    On |double| lines, the arrow tip will not look correct.
\end{arrowtip}

\begin{arrowtip}{Computer Modern Rightarrow}{
    For a line width of 0.4pt (the default), this arrow tip looks very much
    like \texttt{\string\rightarrow} ($\to$) of the Computer Modern math fonts.
    However, it is not a ``perfect'' match: the line caps and joins of the
    ``real'' $\to$ are rounded differently from this arrow tip; but it takes a
    keen eye to notice the difference. When the |arrows.meta| library is loaded,
    this arrow tip becomes the default of |To| and, thus, is used whenever |>|
    is used (unless, of course, you redefined |>|).
}%
{length=1cm,width=2cm}%
{length=1cm,width=2cm}

    \begin{arrowexamples}
        \arrowexample[]
        \arrowexampledup[sep]
        \arrowexampledupdot[sep]
        \arrowexample[length=3pt]
        \arrowexample[sharp]
        \arrowexample[slant=.3]
        \arrowexample[left]
        \arrowexample[right]
        \arrowexample[harpoon,reversed]
        \arrowexample[red]
    \end{arrowexamples}
    %
    The following options have no effect: |open|, |fill|.

    On |double| lines, the arrow tip will not look correct.
\end{arrowtip}

\begin{arrowtipsimple}{Implies}
    This arrow tip makes only sense in conjunction with the |double| option.
    The idea is that you attach it to a double line to get something that looks
    like \TeX's \texttt{\string\implies} arrow ($\implies$). A typical use of
    this arrow tip is
    %
\begin{codeexample}[preamble={\usetikzlibrary{arrows.meta,graphs}}]
\tikz \graph [clockwise=3, math nodes,
              edges = {double equal sign distance, -Implies}] {
  "\alpha", "\beta", "\gamma";
  "\alpha" -> "\beta" -> "\gamma" -> "\alpha"
};
\end{codeexample}
    %
    \begin{arrowexamples}
        \arrowexampledouble[]
        \arrowexampledouble[red]
    \end{arrowexamples}
\end{arrowtipsimple}

\begin{arrowtipsimple}{To}
    This is a shorthand for |Computer Modern Rightarrow| when the |arrows.meta|
    library is loaded. Otherwise, it is a shorthand for the classical
    \tikzname\ rightarrow.
\end{arrowtipsimple}


\subsubsection{Geometric Arrow Tips}

\begin{arrowtip}{Circle}{
    Although this tip is called ``circle'', you can also use it to draw
    ellipses if you set the length and width to different values. Neither
    |round| nor |reversed| has any effect on this arrow tip.
}%
{length=2cm,width=2cm}%
{length=2cm,width=2cm}

    \begin{arrowexamples}
        \arrowexample[]
        \arrowexampledup[sep]
        \arrowexampledupdot[sep]
        \arrowexample[open]
        \arrowexample[length=3pt]
        \arrowexample[slant=.3]
        \arrowexample[left]
        \arrowexample[right]
        \arrowexample[red]
    \end{arrowexamples}
\end{arrowtip}

\begin{arrowtipsimple}{Diamond}
    This is an instance of |Kite| where the length is larger than the width.
    %
    \begin{arrowexamples}
        \arrowexample[]
        \arrowexampledup[]
        \arrowexampledupdot[]
        \arrowexample[open]
        \arrowexample[length=10pt]
        \arrowexample[round]
        \arrowexample[slant=.3]
        \arrowexample[left]
        \arrowexample[right]
        \arrowexample[red]
        \arrowexample[fill=red!50]
    \end{arrowexamples}
\end{arrowtipsimple}

\begin{arrowtipsimple}{Ellipse}
    This is a shorthand for a ``circle'' that is twice as wide as high.

    \begin{arrowexamples}
        \arrowexample[]
        \arrowexampledup[sep]
        \arrowexampledupdot[sep]
        \arrowexample[open]
        \arrowexample[length=10pt]
        \arrowexample[round]
        \arrowexample[slant=.3]
        \arrowexample[left]
        \arrowexample[right]
        \arrowexample[red]
        \arrowexample[fill=red!50]
    \end{arrowexamples}
\end{arrowtipsimple}

\begin{arrowtip}{Kite}{
    This arrow tip consists of four lines that form a ``kite''. The |inset|
    prescribed how far the width-axis of the kite is removed from the back end.
    Note that the inset cannot be negative, use a |Stealth| arrow tip for this.
}%
{length=3cm,width=2cm,inset=1cm}%
{length=3cm,width=2cm,inset=1cm}

    \begin{arrowexamples}
        \arrowexample[]
        \arrowexampledup[sep]
        \arrowexampledupdot[sep]
        \arrowexample[open]
        \arrowexample[length=6pt,width=4pt]
        \arrowexample[length=6pt,width=4pt,inset=1.5pt]
        \arrowexample[round]
        \arrowexample[slant=.3]
        \arrowexample[left]
        \arrowexample[right]
        \arrowexample[red]
    \end{arrowexamples}
\end{arrowtip}

\begin{arrowtip}{Latex}{
    This arrow tip is the same as the arrow tip used in \LaTeX's standard
    pictures (via the \texttt{\string\vec} command), if you set the length to
    4pt. The default size for this arrow tip was set slightly larger so that it
    fits better with the other geometric arrow tips.
}%
{length=3cm,width=2cm}%
{length=3cm,width=2cm}

    \begin{arrowexamples}
        \arrowexample[]
        \arrowexampledup[sep]
        \arrowexampledupdot[sep]
        \arrowexample[open]
        \arrowexample[length=4pt]
        \arrowexample[round]
        \arrowexample[slant=.3]
        \arrowexample[left]
        \arrowexample[right]
        \arrowexample[red]
    \end{arrowexamples}
\end{arrowtip}

\begin{arrowtipsimple}{LaTeX}
    Another spelling for the |Latex| arrow tip.
\end{arrowtipsimple}

\begin{arrowtip}{Rectangle}{
    A rectangular arrow tip. By default, it is twice as long as high.
}%
{length=3cm,width=2cm}%
{length=3cm,width=2cm}

    \begin{arrowexamples}
        \arrowexample[]
        \arrowexampledup[sep]
        \arrowexampledupdot[sep]
        \arrowexample[open]
        \arrowexample[length=4pt]
        \arrowexample[round]
        \arrowexample[slant=.3]
        \arrowexample[left]
        \arrowexample[right]
        \arrowexample[red]
    \end{arrowexamples}
\end{arrowtip}

\begin{arrowtipsimple}{Square}
    An instance of the |Rectangle| whose width is identical to the length.
    %
    \begin{arrowexamples}
        \arrowexample[]
        \arrowexampledup[sep]
        \arrowexampledupdot[sep]
        \arrowexample[open]
        \arrowexample[length=4pt]
        \arrowexample[round]
        \arrowexample[slant=.3]
        \arrowexample[left]
        \arrowexample[right]
        \arrowexample[red]
    \end{arrowexamples}
\end{arrowtipsimple}

\begin{arrowtip}{Stealth}{
    This arrow tip is similar to a |Kite|, only the |inset| now counts
    ``inwards''. Because of that sharp angles, for this arrow tip is makes
    quite a difference, visually, if use the |round| option. Also, using the
    |harpoon| option (or |left| or |right|) will \emph{lengthen} the arrow tip
    because of the even sharper corner at the tip.
}%
{length=3cm,width=2cm,inset=1cm}%
{length=3cm,width=2cm,inset=1cm}

    \begin{arrowexamples}
        \arrowexample[]
        \arrowexampledup[sep]
        \arrowexampledupdot[sep]
        \arrowexample[open]
        \arrowexample[length=6pt,width=4pt]
        \arrowexample[length=6pt,width=4pt,inset=1.5pt]
        \arrowexample[round]
        \arrowexample[slant=.3]
        \arrowexample[left]
        \arrowexample[right]
        \arrowexample[red]
    \end{arrowexamples}
\end{arrowtip}

\begin{arrowtipsimple}{Triangle}
    An instance of a |Kite| with zero inset.
    %
    \begin{arrowexamples}
        \arrowexample[]
        \arrowexampledup[sep]
        \arrowexampledupdot[sep]
        \arrowexample[open]
        \arrowexample[length=4pt]
        \arrowexample[angle=45:1pt 3]
        \arrowexample[angle=60:1pt 3]
        \arrowexample[angle=90:1pt 3]
        \arrowexample[round]
        \arrowexample[slant=.3]
        \arrowexample[left]
        \arrowexample[right]
        \arrowexample[red]
    \end{arrowexamples}
\end{arrowtipsimple}

\begin{arrowtipsimple}{Turned Square}
    An instance of a |Kite| with identical width and height and mid-inset.
    %
    \begin{arrowexamples}
        \arrowexample[]
        \arrowexampledup[sep]
        \arrowexampledupdot[sep]
        \arrowexample[open]
        \arrowexample[length=4pt]
        \arrowexample[round]
        \arrowexample[slant=.3]
        \arrowexample[left]
        \arrowexample[right]
        \arrowexample[red]
    \end{arrowexamples}
\end{arrowtipsimple}


\subsubsection{Caps}

Recall that a \emph{cap} is a way of ending a line. The graphic languages
underlying \tikzname\ (\textsc{pdf}, \textsc{postscript} or \textsc{svg}) all
support three basic types of line caps on a very low level: round, rectangular,
and ``butt''. Using cap arrow tips, you can add new caps to lines and use
different caps for the end and the start.

\begin{arrowtipsimple}{Butt Cap}
    This arrow tip ends the line ``in the normal way'' with a straight end.
    This arrow tip is only need to ``cover up'' the actual line cap, if this
    happens to differ from the normal cap. In the following example, the line
    cap is ``round'', but, nevertheless, the right end is a ``butt'' cap:
    %
\begin{codeexample}[preamble={\usetikzlibrary{arrows.meta}}]
\tikz \draw [line width=1ex, line cap=round, -Butt Cap] (0,0) -- (1,0);
\end{codeexample}
    %
\end{arrowtipsimple}

\begin{arrowcap}{Fast Round}{
    This arrow tip is not really a cap, you use it in conjunction with
    (typically) the |Round Cap|. The idea is that you end your line using the
    round cap and then add several \texttt{Fast Round}s. As for |Round Cap|,
    the |length| parameter dictates the length is the length of the ``main
    part'', the inset sets the length of a line that comes before this tip.
}%
{length=5mm,inset=1cm}%
{length=5mm,inset=-1cm}%
{-15mm}

\begin{codeexample}[preamble={\usetikzlibrary{arrows.meta}}]
\tikz \draw [line width=1ex,
             -{Round Cap []. Fast Round[] Fast Round[]}]
  (0,0) -- (1,0);
\end{codeexample}
    %
    Note that in conjunction with the |bend| option, this works even quite well
    for curves:
    %
\begin{codeexample}[preamble={\usetikzlibrary{arrows.meta,bending}}]
\tikz [f/.tip = Fast Round] % shorthand
  \draw [line width=1ex, -{[bend] Round Cap[] . f f f}]
  (0,0) to [bend left] (1,0);
\end{codeexample}

    \begin{arrowcapexamples}
        \arrowcapexample[]
        \arrowcapexample[reversed]
        \arrowcapexample[cap angle=60]
        \arrowcapexample[cap angle=60,inset=5pt]
        \arrowcapexample[length=.5ex]
        \arrowcapexample[slant=.3]
    \end{arrowcapexamples}
\end{arrowcap}

\begin{arrowcap}{Fast Triangle}{
    This arrow tip works like |Fast Round|, only for triangular caps.
}%
{length=5mm,inset=1cm}%
{length=5mm,inset=-1cm}%
{-15mm}

\begin{codeexample}[preamble={\usetikzlibrary{arrows.meta}}]
\tikz \draw [line width=1ex,
             -{Triangle Cap []. Fast Triangle[] Fast Triangle[]}]
  (0,0) -- (1,0);
\end{codeexample}
    %
    Again, this tip works well for curves:
    %
\begin{codeexample}[preamble={\usetikzlibrary{arrows.meta,bending}}]
\tikz [f/.tip = Fast Triangle] % shorthand
  \draw [line width=1ex, -{[bend] Triangle Cap[] . f f f}]
  (0,0) to [bend left] (1,0);
\end{codeexample}

    \begin{arrowcapexamples}
        \arrowcapexample[]
        \arrowcapexample[reversed]
        \arrowcapexample[cap angle=60]
        \arrowcapexample[cap angle=60,inset=5pt]
        \arrowcapexample[length=.5ex]
        \arrowcapexample[slant=.3]
    \end{arrowcapexamples}
\end{arrowcap}


\begin{arrowcap}{Round Cap}{
    This arrow tip ends the line using a half circle or, if the length has been
    modified, a half-ellipse.
  }%
{length=5mm}%
{length=5mm}%
{-5mm}

    \begin{arrowcapexamples}
        \arrowcapexample[]
        \arrowcapexample[reversed]
        \arrowcapexample[length=.5ex]
        \arrowcapexample[slant=.3]
    \end{arrowcapexamples}
\end{arrowcap}

\begin{arrowcap}{Triangle Cap}{
    This arrow tip ends the line using a triangle whose length is given by the
    |length| option.
}%
{length=5mm}%
{length=5mm}%
{-5mm}

    You can get any angle you want at the tip by specifying a length that is an
    appropriate multiple of the line width. The following options does this
    computation for you:
    %
    \begin{key}{/pgf/arrow keys/cap angle=\meta{angle}}
        Sets |length| to an appropriate multiple of the line width so that the
        angle of a |Triangle Cap| is exactly \meta{angle} at the tip.
    \end{key}

    \begin{arrowcapexamples}
        \arrowcapexample[]
        \arrowcapexample[reversed]
        \arrowcapexample[cap angle=60]
        \arrowcapexample[cap angle=60,reversed]
        \arrowcapexample[length=.5ex]
        \arrowcapexample[slant=.3]
    \end{arrowcapexamples}
\end{arrowcap}


\subsubsection{Special Arrow Tips}

\begin{arrowtip}{Rays}{
    This arrow tip attaches a ``bundle of rays'' to the tip. The number of
    evenly spaced rays is given by the |n| arrow key (see below). When the
    number is even, the rays will lie to the left and to the right of the
    direction of the arrow; when the number is odd, the rays are rotated in
    such a way that one of them points perpendicular to the direction of the
    arrow (this is to ensure that no ray points in the direction of the line,
    which would look strange). The |length| and |width| describe the length and
    width of an ellipse into which the rays fit.
}%
{length=3cm,width=3cm,n=6}%
{length=3cm,width=3cm}

    \begin{arrowexamples}
        \arrowexample[]
        \arrowexampledup[sep]
        \arrowexampledupdot[sep]
        \arrowexample[width'=0pt 2]
        \arrowexample[round]
        \arrowexample[n=2]
        \arrowexample[n=3]
        \arrowexample[n=4]
        \arrowexample[n=5]
        \arrowexample[n=6]
        \arrowexample[n=7]
        \arrowexample[n=8]
        \arrowexample[n=9]
        \arrowexample[slant=.3]
        \arrowexample[left]
        \arrowexample[right]
        \arrowexample[left,n=5]
        \arrowexample[right,n=5]
        \arrowexample[red]
    \end{arrowexamples}
\end{arrowtip}

\begin{key}{/pgf/arrow keys/n=\meta{number} (initially 4)}
    Sets the number of rays in a |Rays| arrow tip.
\end{key}

% % Copyright 2006 by Till Tantau
%
% This file may be distributed and/or modified
%
% 1. under the LaTeX Project Public License and/or
% 2. under the GNU Free Documentation License.
%
% See the file doc/generic/pgf/licenses/LICENSE for more details.

\section{Nodes and Edges}

\label{section-nodes}

\subsection{Overview}

In the present section, the usage of \emph{nodes} in
\tikzname\ is explained. A node is typically a rectangle or circle or
another simple shape with some text on it.

Nodes are added to paths using the special path
operation |node|. Nodes \emph{are not part of the path
  itself}. Rather, they are added to the picture just before or after
the path has been drawn.

In Section~\ref{section-nodes-basic} the basic syntax of the node
operation is explained, followed in Section~\ref{section-nodes-multi}
by the syntax for multi-part nodes, which are nodes that contain
several different text parts. After this, the different options for
the text in nodes are explained. In
Section~\ref{section-nodes-anchors} the concept of \emph{anchors} is
introduced along with their usage. In
Section~\ref{section-nodes-transformations} the different ways
transformations affect nodes are
studied. Sections~\ref{section-nodes-placing-1}
and~\ref{section-nodes-placing-2} are about placing nodes on or next
to straight lines and curves.
Section~\ref{section-nodes-connecting} explains how a node can
be used as a ``pseudo-coordinate.'' Section~\ref{section-nodes-edges}
introduces the |edge| operation, which
works similar to the |to| operation and also similar to the |node|
operation.


\subsection{Nodes and Their Shapes}

\label{section-nodes-basic}

In the simplest case, a node is just some text that is
placed at some coordinate. However, a node can also have a border
drawn around it or have a more complex background and
foreground. Indeed, some nodes do not have a text at all, but consist
solely of the background. You can name nodes so that you can reference
their coordinates later in the same picture or, if certain precautions
are taken as explained in Section~\ref{section-cross-picture-tikz},
also in different pictures.

There are no special \TeX\ commands for adding a node to a picture; rather,
there is path operation called |node| for this. Nodes are created
whenever \tikzname\ encounters |node| or |coordinate| at a point on a
path where it would expect a normal path operation (like |-- (1,1)| or
|rectangle (1,1)|). It is also possible to give node specifications
\emph{inside} certain path operations as explained later.

The node operation is typically followed by some options, which apply
only to the node. Then, you can optionally \emph{name} the node by
providing a name in parentheses. Lastly, for the |node| operation you
must provide some label text for the node in curly braces, while for
the |coordinate| operation you may not. The node is placed at the
current position of the path either \emph{after the path has been
  drawn} or (more seldomly and only if you add the |behind path|
option) \emph{just before the path is drawn.} Thus, all nodes are
drawn ``on top'' or ``behind'' the path and are retained until the
path is complete. If there are several nodes on a path, perhaps some
behind and some on top of the path, first come the nodes behind the
path in the order they were encountered, then comes that path, and
then come the remaining node, again in the order they are
encountered. 

\begin{codeexample}[]
\tikz \fill [fill=yellow!80!black]
     (0,0) node              {first node}  
  -- (1,1) node[behind path] {second node}
  -- (2,0) node              {third node};
\end{codeexample}

\subsubsection{Syntax of the Node Command}

The syntax for specifying nodes is the following:
\begin{pathoperation}{node}{\opt{\meta{foreach statements}}%
      \opt{|[|\meta{options}|]|}\opt{|(|\meta{name}|)|}%
    \opt{|at(|\meta{coordinate}|)|}\opt{\marg{node contents}}}
  Since this path operation is one of the most involved around, let us
  go over it step by step.

  \medskip
  \textbf{Order of the parts of the specification.}
  Everything between ``|node|'' and the opening brace of a node is
  optional. If there are \meta{foreach statements}, they must come
  first, directly following ``|node|.'' Other than that, the ordering
  of all the other elements of a node specification (the
  \meta{options}, the  \meta{name}, and \meta{coordinate}) is
  arbitrary, indeed, there can be multiple occurrences of any of these
  elements (although only for options this makes much sense).  
  
  \medskip
  \textbf{The text of a node.}
  At the end of a node, you must (normally) provide some \meta{node contents}
  in curly braces; indeed, the ``end'' of the node specification is
  detected by the opening curly brace. For normal nodes it is possible to use
  ``fragile'' stuff inside the \meta{node contents} like the |\verb|
  command (for the technically savvy: code inside the \meta{node
    contents} is allowed to change catcodes; however, this rule does
  not apply to ``nodes on a path'' to be discussed later).

  Instead of giving \meta{node contents} at the end of the node in
  curly braces, you can also use the following key:
  \begin{key}{/tikz/node contents=\meta{node contents}}
    \label{option-node-contents}%
    This key sets the contents of the node to the given text as if
    you had given it at the end in curly braces. When the option is
    used inside the options of a node, the parsing of the node stops
    immediately after the end of the option block. In particular, the
    option block cannot be followed by further option blocks or curly
    braces (or, rather, these do not count as part of the node
    specification.) Also note that the \meta{node contents} may not
    contain fragile stuff since the catcodes get fixed upon reading
    the options. Here is an example:
\begin{codeexample}[]
\tikz {
  \path (0,0) node [red]                    {A}
        (1,0) node [blue]                   {B}
        (2,0) node [green, node contents=C]
        (3,0) node [node contents=D]           ;
}
\end{codeexample}
\end{key}

  \medskip
  \textbf{Specifying the location of the node.}
  Nodes are placed at the last position mentioned on the path. The
  effect of adding ``|at|'' to a node 
  specification is that the coordinate given after |at| is used
  instead. The |at| syntax is not available when a node is given
  inside a path operation (it would not make any sense there).

  \begin{key}{/tikz/at=\meta{coordinate}}
    This is another way of specifying the |at| coordinate. Note that,
    typically, you will have to enclose the \meta{coordinate} in curly
    braces so that a comma inside the \meta{coordinate} does not
    confuse \TeX.
  \end{key}

  Another aspect of the ``location'' of a node is whether it appears
  \emph{in front of} or \emph{behind} the current path. You can change
  which of these two possibilities happens on a node-by-node basis
  using the following keys:
  \begin{key}{/tikz/behind path}
    When this key is set, either as a local option for the node or
    some surrounding scope, the node will be drawn behind the current
    path. For this, \tikzname\ collects all nodes defined on the
    current path with this option set and then inserts all of them, in
    the order they appear, just before it draws the path. Thus,
    several nodes with this option set may obscure one anther, but
    never the path itself. ``Just before it draws the path'' actually
    means that the nodes are inserted into the page output just before
    any pre-actions are applied to the path (see below for what
    pre-actions are).
\begin{codeexample}[]
\tikz \fill [fill=blue!50, draw=blue, very thick]
      (0,0)   node [behind path, fill=red!50]   {first node}
   -- (1.5,0) node [behind path, fill=green!50] {second node}
   -- (1.5,1) node [behind path, fill=brown!50] {third node}
   -- (0,1)   node [             fill=blue!30]  {fourth node};
\end{codeexample}
    
    Note that |behind path| only applies to the current path; not to
    the current scope or picture. To put a node ``behind everything''
    you need to use layers and options like |on background layer|, see
    the background library in Section~\ref{section-tikz-backgrounds}.
  \end{key}

  \begin{key}{/tikz/in front of path}
    This is the opposite of |behind path|: It causes nodes to be drawn
    on top of the path. Since this is the default behaviour, you
    usually do not need this option; it is only needed when an
    enclosing scope has used |behind path| and you now wish to
    ``switch back'' to the normal behaviour.
  \end{key}
  
  \medskip
  \textbf{The name of a node.}
  The |(|\meta{name}|)| is a name for later reference and it is
  optional. You may also add the option |name=|\meta{name} to the
  \meta{option} list; it has the same effect.

  \begin{key}{/tikz/name=\meta{node name}}
    Assigns a name to the node for later reference. Since this is a
    ``high-level'' name (drivers never know of it), you can use spaces,
    number, letters, or whatever you like when naming a node. Thus, you
    can name a node just |1| or perhaps |start of chart| or even
    |y_1|. Your node name should \emph{not} contain any punctuation like
    a dot, a comma, or a colon since these are used to detect what kind
    of coordinate you mean when you reference a node.
  \end{key}

  \begin{key}{/tikz/alias=\meta{another node name}}
    This option allows you to provide another name for the
    node. Giving this option multiple times will allow you to access
    the node via several aliases. Using the |node also| syntax,
    you can also assign an alias name to a node at a later point, see
    Section~\ref{section-node-also}. 
  \end{key}
    
  \medskip
  \textbf{The options of a node.}
  The \meta{options} is an optional list of options that \emph{apply
    only to the node} and have no effect outside. The other way round,
  most ``outside'' options also apply to the node, but not all. For
  example, the ``outside'' rotation does not apply to nodes (unless some
  special options are used, sigh). Also, the outside path action, like
  |draw| or |fill|, never applies to the node and must be given in the
  node (unless some special other options are used, deep sigh).

  \medskip
  \textbf{The shape of a node.}
  As mentioned before, we can add a border and even a background to a
  node:
\begin{codeexample}[]
\tikz \fill[fill=yellow!80!black]
      (0,0) node {first node}
   -- (1,1) node[draw, behind path] {second node}
   -- (0,2) node[fill=red!20,draw,double,rounded corners] {third node};
\end{codeexample}

  The ``border'' is actually just a special case of a much more general
  mechanism. Each node has a certain \emph{shape} which, by default, is
  a rectangle. However, we can also ask \tikzname\ to use a circle shape
  instead or an ellipse shape (you have to include one of the
  |shapes.geometric| library for the latter shape):

\begin{codeexample}[]
\tikz \fill[fill=yellow!80!black]
      (0,0) node                            {first node}
   -- (1,1) node[ellipse,draw, behind path] {second node}
   -- (0,2) node[circle,fill=red!20]        {third node};
\end{codeexample}

  There are many more shapes available such as, say, a shape for a
  resistor or a large arrow, see the |shapes| library in
  Section~\ref{section-libs-shapes} for details.

  To select the shape of a node, the following option is used:
  \begin{key}{/tikz/shape=\meta{shape name} (initially rectangle)}
    Select the shape either of the current node or, when this option is
    not given inside a node but somewhere outside, the shape of all
    nodes in the current scope.%
    \indexoption{\meta{shape name}}

    Since this option is used often, you can leave out the
    |shape=|. When \tikzname\ encounters an option like |circle|
    that it does not know, it will, after everything else has failed,
    check whether this option is the name of some shape. If so, that
    shape is selected as if you had said |shape=|\meta{shape name}.

    By default, the following shapes are available: |rectangle|,
    |circle|, |coordinate|. Details of these shapes, like their anchors
    and size options, are discussed in Section~\ref{section-the-shapes}.
  \end{key}

  \medskip
  \textbf{The foreach statement for nodes.}
  At the beginning of a node specification (and only there) you can provide multiple
  \meta{foreach statements}, each of which has the form |foreach| \meta{var}
  |in| |{|\meta{list}|}| (note that there is no slash before
  |foreach|). When they are given, instead of a single node, multiple
  nodes will be created: The \meta{var} will iterate over all values
  of \meta{list} and for each of them, a new node is created. These
  nodes are all created using all the text following the \meta{foreach
    statements}, but in each copy the \meta{var} will have the current
  value of the current element in the \meta{list}.

  As an example, the following two codes have the same effect:
\begin{codeexample}[]
\tikz \draw (0,0) node foreach \x in {1,2,3} at (\x,0) {\x};
\end{codeexample}
\begin{codeexample}[]
\tikz \draw (0,0) node at (1,0) {1} node at (2,0) {2} node at (3,0) {3};
\end{codeexample}
  When you provide several |foreach| statements, they work like
  ``nested loops'':
\begin{codeexample}[]
\tikz \node foreach \x in {1,...,4} foreach \y in {1,2,3}
            [draw] at (\x,\y) {\x,\y};
\end{codeexample}
  As the example shows, a \meta{list} can contain ellipses (three
  dots) to indicated that a larger number of numbers is meant. Indeed,
  you can use the full power of the |\foreach| command here, including
  multiple parameters and options, see Section~\ref{section-foreach}.

  \medskip
  \textbf{Styles for nodes.}
  The following styles influence how nodes are rendered:
  \begin{stylekey}{/tikz/every node (initially \normalfont empty)}
    This style is installed at the beginning of every node.
\begin{codeexample}[]
\begin{tikzpicture}[every node/.style={draw}]
  \draw (0,0) node {A} -- (1,1) node {B};
\end{tikzpicture}
\end{codeexample}
  \end{stylekey}
  \begin{stylekey}{/tikz/every \meta{shape} node (initially \normalfont empty)}
    These styles are installed at the beginning of a node of a given
    \meta{shape}. For example, |every rectangle node| is used for
    rectangle nodes, and so on.
\begin{codeexample}[]
\begin{tikzpicture}
  [every rectangle node/.style={draw},
   every circle node/.style={draw,double}]
  \draw (0,0) node[rectangle] {A} -- (1,1) node[circle] {B};
\end{tikzpicture}
\end{codeexample}
  \end{stylekey}
  
  \medskip
  \textbf{Name scopes.}
  It turns out that the name of a node can further be influenced using
  two keys:
  \begin{key}{/tikz/name prefix=\meta{text} (initially \normalfont empty)}
    The value of this key is prefixed to every node inside the
    current scope. This includes both the naming of the node (via
    the |name| key or via the implicit |(|\meta{name}|)| syntax) as
    well as any referencing of the node. Outside the scope,
    the nodes can (and need to) be referenced using ``full name''
    consisting of the prefix and the node name.
    
    The net effect of this is that you can set the name prefix at
    the beginning of a scope to some value and then use short and
    simple names for the nodes inside the scope. Later, outside the
    scope, you can reference the nodes via their full name:
\begin{codeexample}[]
\tikz {
  \begin{scope}[name prefix = top-]
    \node (A) at (0,1) {A};
    \node (B) at (1,1) {B};
    \draw (A) -- (B);
  \end{scope}
  \begin{scope}[name prefix = bottom-]
    \node (A) at (0,0) {A};
    \node (B) at (1,0) {B};
    \draw (A) -- (B);
  \end{scope}
  
  \draw [red] (top-A) -- (bottom-B);  
}
\end{codeexample}
    As can be seen, name prefixing makes it easy to write
    ``recycable'' code.
  \end{key}
  \begin{key}{/tikz/name suffix=\meta{text} (initially \normalfont empty)}
    Works as |name prefix|, only the \meta{text} is appended to
    every node name in the current scope.  
  \end{key}      
\end{pathoperation}

There is a special syntax for specifying ``light-weight'' nodes:

\begin{pathoperation}{coordinate}{\opt{|[|\meta{options}|]|}|(|\meta{name}|)|\opt{|at(|\meta{coordinate}|)|}}
  This has the same effect as

  |\node[shape=coordinate]|\verb|[|\meta{options}|](|\meta{name}|)at(|\meta{coordinate}|){}|,

  where the |at| part may be omitted.
\end{pathoperation}

Since nodes are often the only path operation on paths, there are two
special commands for creating paths containing only a node:

\begin{command}{\node}
  Inside |{tikzpicture}| this is an abbreviation for |\path node|.
\end{command}

\begin{command}{\coordinate}
  Inside |{tikzpicture}| this is an abbreviation for |\path coordinate|.
\end{command}


\subsubsection{Predefined Shapes}

\label{section-nodes-predefined}

\label{section-the-shapes}

\pgfname\ and \tikzname\ define three shapes, by default:
\begin{itemize}
\item
  |rectangle|,
\item
  |circle|, and
\item
  |coordinate|.
\end{itemize}
By loading library packages, you can define more shapes like ellipses
or diamonds; see Section~\ref{section-libs-shapes} for the complete
list of shapes.

\label{section-tikz-coordinate-shape}
The |coordinate| shape is handled in a special way by \tikzname. When
a node |x| whose shape is |coordinate| is used as a coordinate |(x)|,
this has the same effect as if you had said |(x.center)|. None  of the
special ``line shortening rules'' apply in this case. This can be
useful since, normally, the line shortening causes paths to be
segmented and they cannot be used for filling. Here is an example that
demonstrates the difference:
\begin{codeexample}[]
\begin{tikzpicture}[every node/.style={draw}]
  \path[yshift=1.5cm,shape=rectangle]
    (0,0) node(a1){} (1,0) node(a2){}
    (1,1) node(a3){} (0,1) node(a4){};
  \filldraw[fill=yellow!80!black] (a1) -- (a2) -- (a3) -- (a4);

  \path[shape=coordinate]
    (0,0) coordinate(b1) (1,0) coordinate(b2)
    (1,1) coordinate(b3) (0,1) coordinate(b4);
  \filldraw[fill=yellow!80!black] (b1) -- (b2) -- (b3) -- (b4);
\end{tikzpicture}
\end{codeexample}



\subsubsection{Common Options: Separations, Margins, Padding and
  Border Rotation}

\label{section-shape-seps}
\label{section-shape-common-options}

The exact behaviour of shapes differs, shapes defined for more
special purposes (like a, say, transistor shape) will have even more
custom behaviors. However, there are some options that apply to most
shapes:

\begin{key}{/pgf/inner sep=\meta{dimension} (initially .3333em)}
  \keyalias{tikz}
  An additional (invisible) separation space of \meta{dimension} will
  be added inside the shape, between the text and the shape's
  background path. The effect is as if you had added appropriate
  horizontal and vertical skips at the beginning and end of the text
  to make it a bit ``larger.''

  For those familiar with \textsc{css}, this is the same as
  \emph{padding}.

\begin{codeexample}[]
\begin{tikzpicture}
  \draw (0,0)     node[inner sep=0pt,draw] {tight}
        (0cm,2em) node[inner sep=5pt,draw] {loose}
        (0cm,4em) node[fill=yellow!80!black]   {default};
\end{tikzpicture}
\end{codeexample}
\end{key}

\begin{key}{/pgf/inner xsep=\meta{dimension} (initially .3333em)}
  \keyalias{tikz}
  Specifies the inner separation in the $x$-direction, only.
\end{key}

\begin{key}{/pgf/inner ysep=\meta{dimension} (initially .3333em)}
  \keyalias{tikz}
  Specifies the inner separation in the $y$-direction, only.
\end{key}

\begin{key}{/pgf/outer sep=\meta{dimension or ``auto''}}
  \keyalias{tikz}
  This option adds an additional (invisible) separation space of
  \meta{dimension} outside the background path. The main effect of
  this option is that all anchors will move a little ``to the
  outside.''

  For those familiar with \textsc{css}, this is same as \emph{margin}.

  The default for this option is half the line width. When the default
  is used and when the background path is draw, the anchors will lie
  exactly on the ``outside border'' of the path (not on the path
  itself). 
\begin{codeexample}[]
\begin{tikzpicture}
  \draw[line width=5pt]
    (0,0)  node[fill=yellow!80!black] (f) {filled}
    (2,0)  node[draw]                 (d) {drawn}
    (1,-2) node[draw,scale=2]         (s) {scaled};

  \draw[->] (1,-1) -- (f);
  \draw[->] (1,-1) -- (d);
  \draw[->] (1,-1) -- (s);
\end{tikzpicture}
\end{codeexample}
  
  As the above example demonstrates, the standard settings for the
  outer sep are not always ``correct.'' First, when a shape is filled,
  but not drawn, the outer sep should actually be |0|. Second, when a
  node is scaled, for instance by a factor of 5, the outer separation
  also gets scaled by a factor of 5, while the line width stays at its
  original width; again causing problems.

  In such cases, you can say |outer sep=auto| to make \tikzname\
  \emph{try} to compensate for the effects described above. This is
  done by, firstly, setting the outer sep to |0| when no drawing is
  done and, secondly, setting the outer separations to half the line
  width (as before) times two adjustment factors, one for the
  horizontal separations and one for the vertical  
  separations (see Section~\ref{section-adjustment-transformations}
  for details on these factors). Note, however, that these factors can
  compensate only for transformations that are either scalings plus
  rotations or scalings with different magnitudes in the horizontal
  and the vertical direction. If you apply slanting, the factors will
  only approximate the correct values.

  In general, it is a good idea to say |outer sep=auto| at some early
  stage. It is not the default mainly for compatibility with earlier
  versions.
\begin{codeexample}[]
\begin{tikzpicture}[outer sep=auto]
  \draw[line width=5pt]
    (0,0)  node[fill=yellow!80!black] (f) {filled}
    (2,0)  node[draw]                 (d) {drawn}
    (1,-2) node[draw,scale=2]         (s) {scaled};

  \draw[->] (1,-1) -- (f);
  \draw[->] (1,-1) -- (d);
  \draw[->] (1,-1) -- (s);
\end{tikzpicture}
\end{codeexample}
\end{key}


\begin{key}{/pgf/outer xsep=\meta{dimension} (initially .5\string\pgflinewidth)}
  \keyalias{tikz}
  Specifies the outer separation in the $x$-direction, only. This
  value will be overwritten when |outer sep| is set, either to the
  value given there or a computed value in case of |auto|.
\end{key}

\begin{key}{/pgf/outer ysep=\meta{dimension} (initially .5\string\pgflinewidth)}
  \keyalias{tikz}
  Specifies the outer separation in the $y$-direction, only.
\end{key}


\begin{key}{/pgf/minimum height=\meta{dimension} (initially 0pt)}
  \keyalias{tikz}
  This option ensures that the height of the shape (including the
  inner, but ignoring the outer separation) will be at least
  \meta{dimension}. Thus, if the text plus the inner separation is not
  at least as large as \meta{dimension}, the shape will be enlarged
  appropriately. However, if the text is already larger than
  \meta{dimension}, the shape will not be shrunk.
\begin{codeexample}[]
\begin{tikzpicture}
  \draw (0,0) node[minimum height=1cm,draw] {1cm}
        (2,0) node[minimum height=0cm,draw] {0cm};
\end{tikzpicture}
\end{codeexample}
\end{key}

\begin{key}{/pgf/minimum width=\meta{dimension} (initially 0pt)}
  \keyalias{tikz}
  Same as |minimum height|, only for the width.
\begin{codeexample}[]
\begin{tikzpicture}
  \draw (0,0) node[minimum height=2cm,minimum width=3cm,draw] {$3 \times 2$};
\end{tikzpicture}
\end{codeexample}
\end{key}

\begin{key}{/pgf/minimum size=\meta{dimension}}
  \keyalias{tikz}
  Sets both the minimum height and width at the same time.
\begin{codeexample}[]
\begin{tikzpicture}
  \draw (0,0)  node[minimum size=2cm,draw] {square};
  \draw (0,-2) node[minimum size=2cm,draw,circle] {circle};
\end{tikzpicture}
\end{codeexample}
\end{key}

\begin{key}{/pgf/shape aspect=\meta{aspect ratio}}
  \keyalias{tikz}
  Sets a desired aspect ratio for the shape. For the |diamond| shape,
  this option sets the ratio between width and height of the shape.
\begin{codeexample}[]
\begin{tikzpicture}
  \draw (0,0)  node[shape aspect=1,diamond,draw] {aspect 1};
  \draw (0,-2) node[shape aspect=2,diamond,draw] {aspect 2};
\end{tikzpicture}
\end{codeexample}
\end{key}

\label{section-rotating-shape-borders}

Some shapes (but not all), support a special kind of rotation. This
rotation affects only the border of a shape and is independent of the
node contents, but \emph{in addition} to any other transformations.
	
\begin{codeexample}[]
\tikzstyle{every node}=[dart, shape border uses incircle,
  inner sep=1pt, draw]
\tikz \node foreach \a/\b/\c in {A/0/0, B/45/0, C/0/45, D/45/45}
            [shape border rotate=\b, rotate=\c] at (\b/36,-\c/36) {\a};
\end{codeexample}

There are two types of rotation: restricted and unrestricted. Which
type of rotation is applied is determined by on how the shape border
is constructed. If the shape border is constructed using an incircle,
that is, a circle that tightly fits the node contents (including
the |inner sep|), then the rotation can be unrestricted. If, however,
the border is constructed using the natural dimensions of the node
contents, the rotation is restricted to integer multiples of 90
degrees.

Why should there be two kinds of rotation and border construction?
Borders constructed using the natural dimensions of the node contents
provide a much tighter fit to the node contents, but to maintain
this tight fit, the border rotation must be restricted to integer
multiples of 90 degrees. By using an incircle, unrestricted rotation
is possible, but the border will not make a very tight fit to the
node contents.
	
\begin{codeexample}[]
\tikzstyle{every node}=[isosceles triangle, draw]
\begin{tikzpicture}
  \node {abc};
  \node [shape border uses incircle] at (2,0) {abc};
\end{tikzpicture}
\end{codeexample}

There are \pgfname{} keys that determine how a shape border is
constructed, and to specify its rotation.
It should be noted that not all shapes support these keys, so
reference should be made to the documentation for individual
shapes.
	
\begin{key}{/pgf/shape border uses incircle=\opt{\meta{boolean}}
    (default true)}
  \keyalias{tikz}
  Determines if the border of a shape is constructed using the
  incircle. If no value is given \meta{boolean} will take the default
  value |true|.
\end{key}


\begin{key}{/pgf/shape border rotate=\meta{angle} (initially 0)}
  \keyalias{tikz}
  Rotates the border of a shape independently of the node contents,
  but in addition to any other transformations. If the shape
  border is not constructed using the incircle, the rotation will be
  rounded to the nearest integer multiple of 90 degrees when the
  shape is drawn.
\end{key}

Note that if the border of the shape is rotated,
the compass point anchors, and `text box' anchors (including
|mid east|, |base west|, and so on), \emph{do not rotate}, but the
other anchors do:
	
\begin{codeexample}[]
\tikzstyle{every node}=[shape=trapezium, draw, shape border uses incircle]
\begin{tikzpicture}
  \node at (0,0)  (A) {A};
  \node [shape border rotate=30] at (1.5,0) (B) {B};
  \foreach \s/\t in
    {left side/base east, bottom side/north, bottom left corner/base}{
       \fill[red]  (A.\s) circle(1.5pt) (B.\s) circle(1.5pt);
       \fill[blue] (A.\t) circle(1.5pt) (B.\t) circle(1.5pt);
  }
\end{tikzpicture}
\end{codeexample}

Finally, a somewhat unfortunate side-effect of rotating shape borders
is that the supporting shapes do not distinguish between
|outer xsep| and |outer ysep|, and typically, the larger of the
two values will be used.




\subsection{Multi-Part Nodes}

\label{section-nodes-multi}

Most nodes just have a single simple text label. However, nodes of a
more complicated shape might be made up from several \emph{node
  parts}. For example, in automata theory a so-called Moore state has
a state name, drawn in the upper part of the state circle, and an
output text, drawn in the lower part of the state circle. These two
parts are quite independent. Similarly, a \textsc{uml} class shape
would have a name part, a method part, and an attributes
part. Different molecule shapes might use parts for the different atoms
to be drawn at the different positions, and so on.

Both \pgfname\ and \tikzname\ support such multipart nodes. On the
lower level, \pgfname\ provides a system for specifying that a shape
consists of several parts. On the \tikzname\ level, you specify the
different node parts by using the following command:

\begin{command}{\nodepart\opt{|[|\meta{options}|]|}\marg{part name}}
  This command can only be used inside the \meta{text} argument of a
  |node| path operation. It works a little bit like a |\part| command
  in \LaTeX. It will stop the typesetting of whatever node part was
  typeset until now and then start putting all following text into the
  node part named \meta{part name}---until another |\partname| is
  encountered or until the node \meta{text} ends. The \meta{options}
  will be local to this part.

\begin{codeexample}[]
\begin{tikzpicture}
  \node [circle split,draw,double,fill=red!20]
  {
    % No \nodepart has been used, yet. So, the following is put in the
    % ``text'' node part by default.
    $q_1$
    \nodepart{lower} % Ok, end ``text'' part, start ``output'' part
    $00$
  }; % output part ended.
\end{tikzpicture}
\end{codeexample}

  You will have to lookup which parts are defined by a shape.

  The following styles influences node parts:
  \begin{stylekey}{/tikz/every \meta{part name} node part (initially
      \normalfont empty)}
    This style is installed at the beginning of every node part named
    \meta{part name}.
\begin{codeexample}[]
\tikz [every lower node part/.style={red}]
  \node [circle split,draw] {$q_1$ \nodepart{lower} $00$};
\end{codeexample}
  \end{stylekey}
\end{command}



\subsection{The Node Text}

\label{section-nodes-options}

\subsubsection{Text Parameters: Color and Opacity}

The simplest option for the text in nodes is its color. Normally, this
color is just the last color installed using |color=|, possibly
inherited from another scope. However, it is possible to specifically
set the color used for text using the following option:

\begin{key}{/tikz/text=\meta{color}}
  Sets the color to be used for text labels. A |color=| option
  will immediately override this option.
\begin{codeexample}[]
\begin{tikzpicture}
  \draw[red]       (0,0) -- +(1,1) node[above]     {red};
  \draw[text=red]  (1,0) -- +(1,1) node[above]     {red};
  \draw            (2,0) -- +(1,1) node[above,red] {red};
\end{tikzpicture}
\end{codeexample}
\end{key}

Just like the color itself, you may also wish to set the opacity of
the text only. For this, use the |text opacity| option, which
is detailed in Section~\ref{section-tikz-transparency}.

\subsubsection{Text Parameters: Font}

Next, you may wish to adjust the font used for the text. Naturally,
you can just use a font command like |\small| or |\rm| at the
beginning of a node. However, the following two options make it easier
to set the font used in nodes on a general basis. Let us start with:

\begin{key}{/tikz/node font=\meta{font commands}}
  This option sets the font used for all text used in a node. 
\begin{codeexample}[]
\begin{tikzpicture}
  \draw[node font=\itshape] (1,0) -- +(1,1) node[above] {italic};
\end{tikzpicture}
\end{codeexample}
  Since the \meta{font commands} are executed at a very early stage in
  the construction of the node, the font selected using this command
  will also dictate the values of dimensions defined in terms of |em|
  or |ex|. For instance, when the |minimum height| of a node is |3em|,
  the actual height will be (at least) three times the line distance
  selected by the \meta{font commands}:
\begin{codeexample}[]
\tikz \node [node font=\tiny,  minimum height=3em, draw] {tiny};
\tikz \node [node font=\small, minimum height=3em, draw] {small};
\end{codeexample}
\end{key}

The other font command is:

\begin{key}{/tikz/font=\meta{font commands}}
  Sets the font used for the text inside nodes. However, this font
  will \emph{not} (yet) be installed when any of the dimensions of the
  node are being computed, so dimensions like |1em| will be with
  respect to the font used outside the node (usually the font that was
  in force when the picture started). 
\begin{codeexample}[]
\begin{tikzpicture}
  \node [font=\itshape] {italic};
\end{tikzpicture}
\end{codeexample}

\begin{codeexample}[]
\tikz \node [font=\tiny,  minimum height=3em, draw] {tiny};
\tikz \node [font=\small, minimum height=3em, draw] {small};
\end{codeexample}

  A useful example of how the |font| option can be used is the
  following: 

\begin{codeexample}[]
\tikz [every text node part/.style={font=\itshape},
       every lower node part/.style={font=\footnotesize}]
  \node [circle split,draw] {state \nodepart{lower} output};
\end{codeexample}

  As can be seen, the font can be changed for each node part. This
  does \emph{not} work with the |node font| command since, as the name
  suggests, this command can only be used to select the ``overall''
  font for the node and this is done very early. 
\end{key}


\subsubsection{Text Parameters: Alignment and Width for Multi-Line Text}

Normally, when a node is typeset, all the text you give in the braces
is put in one long line (in an |\hbox|, to be precise) and the node
will become as wide as necessary.

From time to time you may wish to create nodes that contain multiple
lines of text. There are three different ways of achieving this:
\begin{enumerate}
\item Inside the node, you can put some standard environment that
  produces multi-line, aligned text. For instance, you can use a
  |{tabular}| inside a node:
\begin{codeexample}[width=5cm]
\tikz \node [draw] {
  \begin{tabular}{cc}
    upper left & upper right\\
    lower left & lower right
  \end{tabular}
};
\end{codeexample}
  This approach offers the most flexibility in the sense that it
  allows you to use all of the alignment commands offered by your
  format of choice.
\item You use |\\| inside your node to mark the end of lines and then
  request \tikzname\ to arrange these lines in some manner. This will
  only be done, however, if the |align| option has been given.
\begin{codeexample}[]
\tikz[align=left] \node[draw] {This is a\\demonstration.};
\end{codeexample}
\begin{codeexample}[]
\tikz[align=center] \node[draw] {This is a\\demonstration.};
\end{codeexample}
  The |\\| command takes an optional extra space as an argument in square
  brackets.
\begin{codeexample}[]
\tikz \node[fill=yellow!80!black,align=right]
  {This is a\\[-2pt] demonstration text for\\[1ex] alignments.};
\end{codeexample}
\item You can request that \tikzname\ does an automatic line-breaking
  for you inside the node by specifying a fixed |text width| for the
  node. In this case, you can still use |\\| to enforce a
  line-break. Note that when you specify a text width, the node will
  have this width, independently of whether the text actually
  ``reaches the end'' of the node.
\end{enumerate}

Let us now first have a look at the |text width| command.
\begin{key}{/tikz/text width=\meta{dimension}}
  This option will put the text of a node in a box of the given width
  (something akin to a |{minipage}| of this width, only portable
  across formats). If the node text is not as wide as
  \meta{dimension}, it will nevertheless be put in a box of this
  width. If it is larger, line breaking will be done.

  By default, when this option is given, a ragged right border will be
  used (|align=left|). This is sensible since, typically, these boxes
  are narrow and justifying the text looks ugly. You can, however,
  change the alignment using |align| or directly using commands line
  |\centering|.
\begin{codeexample}[]
\tikz \draw (0,0) node[fill=yellow!80!black,text width=3cm]
  {This is a demonstration text for showing how line breaking works.};
\end{codeexample}
  Setting \meta{dimension} to an empty string causes the automatic
  line breaking to be disabled.
\end{key}

\begin{key}{/tikz/align=\meta{alignment option}}
  This key is used to set up an alignment for multi-line text inside a
  node. If |text width| is set to some width (let us call this
  \emph{alignment with line breaking}), the |align| key will
  setup the |\leftskip| and the |\rightskip| in such a way that the
  text is broken and aligned according to \meta{alignment option}. If |text width|
  is not set (that is, set to the empty string; let us call this
  \emph{alignment without line breaking}), then a different
  mechanism is used internally, namely the key |node halign header|, is
  set to an appropriate value. While this key, which is documented
  below, is not to be used by beginners, the net effect is simple:
  When |text width| is not set, you can use |\\| to break lines and
  align them according to \meta{alignment option} and the resulting node's width
  will be minimal to encompass the resulting lines.

  In detail, you can set \meta{alignment option} to one of the following values:
  \begin{description}
  \item[|align=|\declare{|left|}]
    For alignment without line breaking, the different lines are simply
    aligned such that their left borders are below one another.
\begin{codeexample}[]
\tikz \node[fill=yellow!80!black,align=left]
  {This is a\\ demonstration text for\\ alignments.};
\end{codeexample}
    For alignment with line breaking, the same will happen only
    the lines will now, additionally, be broken automatically:
\begin{codeexample}[]
\tikz \node[fill=yellow!80!black,text width=3cm,align=left]
  {This is a demonstration text for showing how line breaking works.};
\end{codeexample}

  \item[|align=|\declare{\texttt{flush left}}]
    For alignment without line breaking this option has exactly the
    same effect as |left|. However, for alignment with line breaking,
    there is a difference: While |left| uses the
    original plain \TeX\ definition of a ragged right border, in which
    \TeX\ will try to balance the right border as well as possible,
    |flush left| causes the right border to be ragged in the
    \LaTeX-style, in which no balancing occurs. This looks ugly, but
    it may be useful for very narrow boxes and when you wish to avoid
    hyphenations.
\begin{codeexample}[]
\tikz \node[fill=yellow!80!black,text width=3cm,align=flush left]
  {This is a demonstration text for showing how line breaking works.};
\end{codeexample}

  \item[|align=|\declare{|right|}]
    Works like |left|, only for right alignment.
\begin{codeexample}[]
\tikz \node[fill=yellow!80!black,align=right]
  {This is a\\ demonstration text for\\ alignments.};
\end{codeexample}
\begin{codeexample}[]
\tikz \node[fill=yellow!80!black,text width=3cm,align=right]
  {This is a demonstration text for showing how line breaking works.};
\end{codeexample}

  \item[|align=|\declare{\texttt{flush right}}]
    Works like |flush left|, only for right alignment.
\begin{codeexample}[]
\tikz \node[fill=yellow!80!black,text width=3cm,align=flush right]
  {This is a demonstration text for showing how line breaking works.};
\end{codeexample}

  \item[|align=|\declare{|center|}]
    Works like |left| or |right|, only for centered alignment.
\begin{codeexample}[]
\tikz \node[fill=yellow!80!black,align=center]
  {This is a\\ demonstration text for\\ alignments.};
\end{codeexample}
\begin{codeexample}[]
\tikz \node[fill=yellow!80!black,text width=3cm,align=center]
  {This is a demonstration text for showing how line breaking works.};
\end{codeexample}

    There is one annoying problem with the |center|
    alignment (but not with |flush center| and the other options): If
    you specify a large line width and the node text  
    fits on a single line and is, in fact, much shorter than the
    specified |text width|, an underfull horizontal box will
    result. Unfortunately, this cannot be avoided, due to the way
    \TeX\ works (more precisely, I have thought long and hard about this
    and have not been able to figure out a sensible way to avoid this).
    For this reason, \tikzname\ switches off horizontal badness
    warnings inside boxes with |align=center|. Since this will also
    suppress some ``wanted'' warnings, there is also an option for
    switching the warnings on once more:

    \begin{key}{/tikz/badness warnings for centered text=\meta{true or
          false} (initially false)}
      If set to true, normal badness warnings will be issued for
      centered boxes. Note that you may get annoying warnings for
      perfectly normal boxes, namely whenever the box is very large
      and the contents is not long enough to fill the box
      sufficiently. 
    \end{key}

  \item[|align=|\declare{\texttt{flush center}}]
    Works like |flush left| or |flush right|, only for center
    alignment. Because of all the trouble that results from the
    |center| option in conjunction with narrow lines, I suggest picking
    this option rather than  |center| \emph{unless} you have longer
    text, in which case |center| will give the typographically better
    results. 
\begin{codeexample}[]
\tikz \node[fill=yellow!80!black,text width=3cm,align=flush center]
  {This is a demonstration text for showing how line breaking works.};
\end{codeexample}

  \item[|align=|\declare{|justify|}]
    For alignment without line breaking, this has the same effect as
    |left|. For alignment with line breaking, this causes the text to
    be ``justified.'' Use this only with rather broad nodes.
{%
\hbadness=10000
\begin{codeexample}[]
\tikz \node[fill=yellow!80!black,text width=3cm,align=justify]
  {This is a demonstration text for showing how line breaking works.};
\end{codeexample}
}
  In the above example, \TeX\ complains (rightfully) about three very
  badly typeset lines. (For this manual I asked \TeX\ to stop
  complaining by using |\hbadness=10000|, but this is a foul deed,
  indeed.)

  \item[|align=|\declare{|none|}]
    Disables all alignments and |\\| will not be redefined.
  \end{description}
\end{key}


\begin{key}{/tikz/node halign header=\meta{macro storing a header} (initially
    \normalfont empty)}
  This is the key that is used by |align| internally for alignment
  without line breaking. Read the following only if you are familiar
  with the |\halign| command.

  This key only has an effect if |text width|
  is empty, otherwise it is ignored. Furthermore, if \meta{macro storing a header} is
  empty, then this key also has no effect. So, suppose |text width| is
  empty, but \meta{header} is not. In this case the following happens:

  When the node text is parsed, the command |\\| is redefined
  internally. This redefinition is done in such a way that the text
  from the start of the node to the first occurrence of |\\| is put in
  an |\hbox|. Then the text following |\\| up to the next |\\| is put
  in another |\hbox|. This goes on until the text between the last
  |\\| and the closing |}| is also put in an |\hbox|.

  The \meta{macro storing a header} should be a macro that contains
  some text suitable for use as a header for the |\halign|
  command. For instance, you might define
\begin{codeexample}[code only]
\def\myheader{\hfil\hfil##\hfil\cr}
\tikz [node halign header=\myheader] ...
\end{codeexample}
  You cannot just say |node halign header=\hfil\hfil#\hfil\cr| because
  this confuses \TeX\ inside matrices, so this detour via a macro is
  needed.

  Next, conceptually, all these boxes are recursively put inside an
  |\halign| command. Assuming that \meta{first} is the first of the
  above boxes, the command |\halign{|\meta{header} |\box|\meta{first}
    |\cr}| is used to create a new box, which we will call the
  \meta{previous box}. Then, the following box is created, where
  \meta{second} is the second input box:
  |\halign{|\meta{header} |\box|\meta{previous box} |\cr|
    |\box|\meta{second}|\cr}|. Let us call the resulting box the
  \meta{previous box} once more. Then the next box that is created is
  |\halign{|\meta{header} |\box|\meta{previous box} |\cr|
    |\box|\meta{third}|\cr}|.

  All of this means that if \meta{header} is an |\halign| header
  like |\hfil#\hfil\cr|, then all boxes will be centered relative to
  one another. Similarly, a \meta{header} of |\hfil#\cr| causes the
  text to be flushed right.

  Note that this mechanism is not flexible enough to all multiple
  columns inside \meta{header}. You will have to use a |tabular| or a
  |matrix| in such cases.

  One further note: Since the text of each line is placed in a box,
  settings will be local to each ``line.'' This is very similar to the
  way a cell in a |tabular| or a |matrix| behaves.
\end{key}


\subsubsection{Text Parameters: Height and Depth of Text}

In addition to changing the width of nodes, you can also change the
height of nodes. This can be done in two ways: First, you can use the
option |minimum height|, which ensures that the height of the whole
node is at least the given height (this option is described in more
detail later). Second, you can use the option |text height|, which
sets the height of the text itself, more precisely, of the \TeX\ text
box of the text. Note that the |text height| typically is not the
height of the shape's box: In addition to the |text height|, an
internal |inner sep| is added as extra space and the text depth is
also taken into account.

I recommend using |minimum size| instead of |text height| except for
special situations.

\begin{key}{/tikz/text height=\meta{dimension}}
  Sets the height of the text boxes in shapes. Thus, when you write
  something like |node {text}|, the |text| is first typeset, resulting
  in some box of a certain height. This height is then replaced by the
  height |text height|. The resulting box is then used to determine
  the size of the shape, which will typically be larger. When you
  write |text height=| without specifying anything, the ``natural''
  size of the text box remains unchanged.
\begin{codeexample}[]
\tikz \node[draw]                  {y};
\tikz \node[draw,text height=10pt] {y};
\end{codeexample}
\end{key}

\begin{key}{/tikz/text depth=\meta{dimension}}
  This option works like |text height|, only for the depth of the text
  box. This option is mostly useful when you need to ensure a uniform
  depth of text boxes that need to be aligned.
\end{key}



\subsection{Positioning Nodes}

\label{section-nodes-anchors}

When you place a node at some coordinate, the node is centered on this
coordinate by default. This is often undesirable and it would be
better to have the node to the right or above the actual coordinate.


\subsubsection{Positioning Nodes Using Anchors}

\pgfname\ uses a so-called anchoring mechanism to give you a very fine
control over the placement. The idea is simple: Imagine a node of
rectangular shape of a certain size. \pgfname\ defines numerous anchor
positions in the shape. For example to upper right corner is called,
well, not ``upper right anchor,'' but the |north east| anchor of the
shape. The center of the shape has an anchor called |center| on top of
it, and so on. Here are some examples (a complete list is given in
Section~\ref{section-the-shapes}).

\medskip\noindent
\begin{tikzpicture}
  \path node[minimum height=2cm,minimum width=5cm,fill=blue!25](x) {Big node};
  \fill (x.north)      circle (2pt) node[above] {|north|}
        (x.north east) circle (2pt) node[above] {|north east|}
        (x.north west) circle (2pt) node[above] {|north west|}
        (x.west) circle (2pt)       node[left]  {|west|}
        (x.east) circle (2pt)       node[right] {|east|}
        (x.base) circle (2pt)       node[below] {|base|};
\end{tikzpicture}

Now, when you place a node at a certain coordinate, you can ask \tikzname\
to place the node shifted around in such a way that a certain
anchor is at the coordinate. In the following example, we ask \tikzname\
to shift the first node such that its  |north east| anchor is at
coordinate |(0,0)| and that the |west| anchor of the second node is at
coordinate |(1,1)|.

\begin{codeexample}[]
\tikz \draw           (0,0) node[anchor=north east] {first node}
            rectangle (1,1) node[anchor=west] {second node};
\end{codeexample}

Since the default anchor is |center|, the default behaviour is to
shift the node in such a way that it is centered on the current
position.

\begin{key}{/tikz/anchor=\meta{anchor name}}
  Causes the node to be shifted such that it's anchor \meta{anchor
  name} lies on the current coordinate.

  The only anchor that is present in all shapes is |center|. However,
  most shapes will at least define anchors in all ``compass
  directions.'' Furthermore, the standard shapes also define a |base|
  anchor, as well as |base west| and |base east|, for placing things on
  the baseline of the text.

  The standard shapes also define a |mid| anchor (and |mid west| and
  |mid east|). This anchor is half the height of the character ``x''
  above the base line. This anchor is useful for vertically centering
  multiple nodes that have different heights and depth. Here is an
  example:
\begin{codeexample}[]
\begin{tikzpicture}[scale=3,transform shape]
  % First, center alignment -> wobbles
  \draw[anchor=center] (0,1)  node{x} -- (0.5,1)  node{y} -- (1,1)  node{t};
  % Second, base alignment -> no wobble, but too high
  \draw[anchor=base]   (0,.5) node{x} -- (0.5,.5) node{y} -- (1,.5) node{t};
  % Third, mid alignment
  \draw[anchor=mid]    (0,0)  node{x} -- (0.5,0)  node{y} -- (1,0)  node{t};
\end{tikzpicture}
\end{codeexample}
\end{key}



\subsubsection{Basic Placement Options}

Unfortunately, while perfectly logical, it is often rather
counter-intuitive that in order to place a node \emph{above} a given
point, you need to specify the |south| anchor. For this reason, there
are some useful options that allow you to select the standard anchors
more intuitively:

\begin{key}{/tikz/above=\meta{offset} (default 0pt)}
  Does the same as |anchor=south|. If the \meta{offset} is specified,
  the node is additionally shifted upwards by the given
  \meta{offset}.
\begin{codeexample}[]
\tikz \fill (0,0) circle (2pt) node[above] {above};
\end{codeexample}
\begin{codeexample}[]
\tikz \fill (0,0) circle (2pt) node[above=2pt] {above};
\end{codeexample}
\end{key}

\begin{key}{/tikz/below=\meta{offset} (default 0pt)}
  Similar to |above|.
\end{key}

\begin{key}{/tikz/left=\meta{offset} (default 0pt)}
  Similar to |above|.
\end{key}

\begin{key}{/tikz/right=\meta{offset} (default 0pt)}
  Similar to |above|.
\end{key}

\begin{key}{/tikz/above left}
  Does the same as |anchor=south east|. Note that giving both |above|
  and |left| options does not have the same effect as |above left|,
  rather only the last |left| ``wins.'' Actually, this option also
  takes an \meta{offset} parameter, but using this parameter without
  using the |positioning| library is deprecated. (The |positioning|
  library changes the meaning of this parameter to something more
  sensible.)
\begin{codeexample}[]
\tikz \fill (0,0) circle (2pt) node[above left] {above left};
\end{codeexample}
\end{key}

\begin{key}{/tikz/above right}
  Similar to  |above left|.
\begin{codeexample}[]
\tikz \fill (0,0) circle (2pt) node[above right] {above right};
\end{codeexample}
\end{key}

\begin{key}{/tikz/below left}
  Similar to |above left|.
\end{key}
\begin{key}{/tikz/below right}
  Similar to |above left|.
\end{key}

\begin{key}{/tikz/centered}
  A shorthand for |anchor=center|. 
\end{key}

% A second set of options behaves similarly, namely the |above of|,
% |below of|, and so on options. They cause the same anchors to be set
% as the options without |of|, however, their parameter is different:
% You must provide the name of another node. The current node will then
% be placed, say, above this specified node at a distance given by the
% option |node distance|.
% \begin{key}{/tikz/above of=\meta{node}}
%   This option causes the node to be placed at the distance
%   |node distance| above of \meta{node}. The anchor is |center|.
% \begin{codeexample}[]
% \begin{tikzpicture}[node distance=1cm]
%   \draw[help lines] (0,0) grid (3,2);
%   \node (a)                    {a};
%   \node (b) [above of=a]       {b};
%   \node (c) [above of=b]       {c};
%   \node (d) [right of=c]       {d};
%   \node (e) [below right of=d] {e};
% \end{tikzpicture}
% \end{codeexample}
% \end{key}

% \begin{key}{/tikz/above left of=\meta{node}}
%   Works like |above of|, only the node is now put above and left. The
%   |node distance| is the Euclidean distance between the two nodes, not
%   the $L_1$-distance.
% \end{key}

% \begin{key}{/tikz/above right of=\meta{node}}
%   Works similarly.
% \end{key}
% \begin{key}{/tikz/left of=\meta{node}}
%   Works similarly.
% \end{key}
% \begin{key}{/tikz/right of=\meta{node}}
%   Works similarly.
% \end{key}
% \begin{key}{/tikz/below of=\meta{node}}
%   Works similarly.
% \end{key}
% \begin{key}{/tikz/below left of=\meta{node}}
%   Works similarly.
% \end{key}
% \begin{key}{/tikz/below right of=\meta{node}}
%   Works similarly.
% \end{key}
% \begin{key}{/tikz/node distance=\meta{dimension}}
%   Sets the distance between nodes that are placed using the
%   |... of| options. Note that this distance is the distance between
%   the centers of the nodes, not the distance between their borders.
% \end{key}



\subsubsection{Advanced Placement Options}

While the standard placement options suffice for simple cases, the
|positioning| library offers more convenient placement options.

\begin{tikzlibrary}{positioning}
  The library defines additional options for placing nodes
  conveniently. It also redefines the standard options like |above| so
  that they give you better control of node placement.
\end{tikzlibrary}

When this library is loaded, the options like |above| or |above left| behave
differently.

\begin{key}{/tikz/above=\opt{\meta{specification}} (default 0pt)}
  With the |positioning| library loaded, the |above| option does not
  take a simple \meta{dimension} as its parameter. Rather, it can
  (also) take a more elaborate \meta{specification} as parameter. This
  \meta{specification} has the following general form: It starts with
  an optional \meta{shifting part} and is followed by an optional
  \meta{of-part}. Let us start with the \meta{shifting part}, which
  can have three forms:
  \begin{enumerate}
  \item It can simply be a \declare{\meta{dimension}} (or a mathematical
    expression that evaluates to a dimension) like |2cm| or
    |3cm/2+4cm|. In this case, the following happens: the node's
    anchor is set to |south| and the node is vertically shifted
    upwards by \meta{dimension}.
\begin{codeexample}[]
\begin{tikzpicture}
  \draw[help lines] (0,0) grid (2,2);
  \node at (1,1) [above=2pt+3pt,draw] {above};
\end{tikzpicture}
\end{codeexample}
    This use of the |above| option is the same as if the |positioning|
    library were not loaded.
  \item It can be a \declare{\meta{number}} (that is, any mathematical
    expression that does not include a unit like |pt| or
    |cm|). Examples are |2| or |3+sin(60)|. In this case, the anchor
    is also set to |south| and the node is vertically shifted by
    the vertical component of the coordinate |(0,|\meta{number}|)|.
\begin{codeexample}[]
\begin{tikzpicture}
  \draw[help lines] (0,0) grid (2,2);
  \node at (1,1) [above=.2,draw] {above};
  % south border of the node is now 2mm above (1,1)
\end{tikzpicture}
\end{codeexample}
  \item It can be of the form \declare{\meta{number or
        dimension 1}| and |\meta{number or dimension 2}}. This specification
    does not make particular sense for the |above| option, it is much
    more useful for options like |above left|. The reason it is
    allowed for the |above| option is that it is sometimes
    automatically used, as explained later.

    The effect of this option is the following. First, the point
    |(|\meta{number or dimension 2}|,|\meta{number or dimension 1}|)|
    is computed (note the inversed order), using the normal rules for
    evaluating such a coordinate, yielding some position. Then, the
    node is shifted by the vertical component of this point. The
    anchor is set to |south|.
\begin{codeexample}[]
\begin{tikzpicture}
  \draw[help lines] (0,0) grid (2,2);
  \node at (1,1) [above=.2 and 3mm,draw] {above};
  % south border of the node is also 2mm above (1,1)
\end{tikzpicture}
\end{codeexample}
  \end{enumerate}
  The \meta{shifting part} can optionally be followed by a
  \meta{of-part}, which has one of the following forms:
  \begin{enumerate}
  \item The \meta{of-part} can be \declareandlabel{of}| |\meta{coordinate},
    where \meta{coordinate} is \emph{not} in parentheses and it is
    \emph{not} just a node name. An example would be
    |of somenode.north| or |of 2,3|. In this case, the
    following happens: First, the node's |at| parameter is set to the
    \meta{coordinate}. Second, the node is shifted according to the
    \meta{shift-part}. Third, the anchor is set to |south|.

    Here is a basic example:
\begin{codeexample}[]
\begin{tikzpicture}[every node/.style=draw]
  \draw[help lines] (0,0) grid (2,2);
  \node (somenode) at (1,1) {some node};

  \node [above=5mm of somenode.north east] {\tiny 5mm of somenode.north east};
  \node [above=1cm of somenode.north]      {\tiny 1cm of somenode.north};
\end{tikzpicture}
\end{codeexample}
    As can be seen the |above=5mm of somenode.north east| option does,
    indeed, place the node 5mm above the north east anchor of
    |somenode|. The same effect could have been achieved writing
    |above=5mm| followed by |at=(somenode.north east)|.

    If the \meta{shifting-part} is missing, the shift is not zero, but
    rather the value of the |node distance| key is used, see below.
  \item The \meta{of-part} can be |of |\meta{node name}. An
    example would be |of somenode|. In this case, the following
    usually happens:
    \begin{itemize}
    \item The anchor is set to |south|.
    \item The node is shifted according to the \meta{shifting part}
      or, if it is missing, according to the value of |node distance|.
    \item The node's |at| parameter is set to \meta{node
        name}|.north|.
    \end{itemize}
    The net effect of all this is that the new node will be placed in
    such a way that the distance between is south border and
    \meta{node name}'s north border is exactly the given distance.
\begin{codeexample}[]
\begin{tikzpicture}[every node/.style=draw]
  \draw[help lines] (0,0) grid (2,2);
  \node (some node) at (1,1) {some node};

  \node (other node) [above=1cm of some node] {\tiny above=1cm of some node};

  \draw [<->] (some node.north) -- (other node.south)
                                node [midway,right,draw=none] {1cm};
\end{tikzpicture}
\end{codeexample}
    It is possible to change the behaviour of this
    \meta{specification} rather drastically, using the following key:
    \begin{key}{/tikz/on grid=\meta{boolean} (initially false)}
      When this key is set to |true|, an \meta{of-part} of the
      current form behaves differently: The anchors set for the
      current node as well as the anchor used for the other \meta{node name} are set to |center|.

      This has the following effect: When you say
      |above=1cm of somenode| with |on grid| set to true, the new node
      will be placed in such a way that its center is 1cm above the
      center of |somenode|. Repeatedly placing nodes in this way will
      result in nodes that are centered on ``grid coordinate,'' hence
      the name of the option.
\begin{codeexample}[]
\begin{tikzpicture}[every node/.style=draw]
  \draw[help lines] (0,0) grid (2,3);

  % Not gridded
  \node (a1) at (0,0) {not gridded};
  \node (b1) [above=1cm of a1] {fooy};
  \node (c1) [above=1cm of b1] {a};

  % gridded
  \node (a2) at (2,0) {gridded};
  \node (b2) [on grid,above=1cm of a2] {fooy};
  \node (c2) [on grid,above=1cm of b2] {a};
\end{tikzpicture}
\end{codeexample}
    \end{key}
  \end{enumerate}

  \begin{key}{/tikz/node distance=\meta{shifting part} (initially 1cm and 1cm)}
    The value of this key is used as \meta{shifting part} is used if
    and only if a \meta{of-part} is present, but no \meta{shifting
      part}.
\begin{codeexample}[]
\begin{tikzpicture}[every node/.style=draw,node distance=5mm]
  \draw[help lines] (0,0) grid (2,3);

  % Not gridded
  \node (a1) at (0,0) {not gridded};
  \node (b1) [above=of a1] {fooy};
  \node (c1) [above=of b1] {a};

  % gridded
  \begin{scope}[on grid]
    \node (a2) at (2,0) {gridded};
    \node (b2) [above=of a2] {fooy};
    \node (c2) [above=of b2] {a};
  \end{scope}
\end{tikzpicture}
\end{codeexample}
  \end{key}
\end{key}

\begin{key}{/tikz/below=\opt{\meta{specification}}}
  This key is redefined in the same manner as |above|.
\end{key}

\begin{key}{/tikz/left=\opt{\meta{specification}}}
  This key is redefined in the same manner as |above|, only all
  vertical shifts are replaced by horizontal shifts.
\end{key}

\begin{key}{/tikz/right=\opt{\meta{specification}}}
  This key is redefined in the same manner as |left|.
\end{key}

\begin{key}{/tikz/above left=\opt{\meta{specification}}}
  This key is also redefined in a manner similar to the above, but
  behaviour of the \meta{shifting part} is more complicated:
  \begin{enumerate}
  \item When the \meta{shifting part} is of the form \meta{number or
      dimension}| and |\meta{number or dimension}, it has
    (essentially) the effect of shifting the node vertically upwards
    by the first \meta{number or dimension} and to the left by the
    second. To be more precise, the coordinate |(|\meta{second number
      or dimension}|,|\meta{first number or dimension}|)| is computed
    and then the node is shifted vertically by the $y$-part of
    the resulting coordinate and horizontally be the negated $x$-part
    of the result. (This is exactly what you expect, except possibly
    when you have used the |x| and |y| options to modify the
    |xy|-coordinate system so that the unit  vectors no longer point
    in the expected directions.)
  \item When the \meta{shifting part} is of the form \meta{number or
      dimension}, the node is shifted by this \meta{number or
      dimension} in the direction of $135^\circ$. This means that
    there is a difference between a \meta{shifting part} of |1cm| and
    of |1cm and 1cm|: In the second case, the node is shifted by 1cm
    upward and 1cm to the left; in the first case it is shifted by
    $\frac{1}{2}\sqrt{2}$cm upward and by the same amount to the
    left. A more mathematical way of phrasing this is the following: A
    plain \meta{dimension} is measured in the $l_2$-norm, while a
    \meta{dimension}| and |\meta{dimension} is measured in the
    $l_1$-norm.
  \end{enumerate}
  The following example should help to illustrate the difference:
\begin{codeexample}[]
\begin{tikzpicture}[every node/.style={draw,circle}]
  \draw[help lines] (0,0) grid (2,5);
  \begin{scope}[node distance=5mm and 5mm]
    \node (b) at (1,4) {b};
    \node [left=of b] {1};       \node [right=of b] {2};
    \node [above=of b] {3};      \node [below=of b] {4};
    \node [above left=of b] {5}; \node [above right=of b] {6};
    \node [below left=of b] {7}; \node [below right=of b] {8};
  \end{scope}
  \begin{scope}[node distance=5mm]
    \node (a) at (1,1) {a};
    \node [left=of a] {1};       \node [right=of a] {2};
    \node [above=of a] {3};      \node [below=of a] {4};
    \node [above left=of a] {5}; \node [above right=of a] {6};
    \node [below left=of a] {7}; \node [below right=of a] {8};
  \end{scope}
\end{tikzpicture}
\end{codeexample}
\begin{codeexample}[]
\begin{tikzpicture}[every node/.style={draw,rectangle}]
  \draw[help lines] (0,0) grid (2,5);
  \begin{scope}[node distance=5mm and 5mm]
    \node (b) at (1,4) {b};
    \node [left=of b] {1};       \node [right=of b] {2};
    \node [above=of b] {3};      \node [below=of b] {4};
    \node [above left=of b] {5}; \node [above right=of b] {6};
    \node [below left=of b] {7}; \node [below right=of b] {8};
  \end{scope}
  \begin{scope}[node distance=5mm]
    \node (a) at (1,1) {a};
    \node [left=of a] {1};       \node [right=of a] {2};
    \node [above=of a] {3};      \node [below=of a] {4};
    \node [above left=of a] {5}; \node [above right=of a] {6};
    \node [below left=of a] {7}; \node [below right=of a] {8};
  \end{scope}
\end{tikzpicture}
\end{codeexample}
\begin{codeexample}[]
\begin{tikzpicture}[every node/.style={draw,rectangle},on grid]
  \draw[help lines] (0,0) grid (4,4);
  \begin{scope}[node distance=1]
    \node (a) at (2,3) {a};
    \node [left=of a] {1};       \node [right=of a] {2};
    \node [above=of a] {3};      \node [below=of a] {4};
    \node [above left=of a] {5}; \node [above right=of a] {6};
    \node [below left=of a] {7}; \node [below right=of a] {8};
  \end{scope}
  \begin{scope}[node distance=1 and 1]
    \node (b) at (2,0) {b};
    \node [left=of b] {1};       \node [right=of b] {2};
    \node [above=of b] {3};      \node [below=of b] {4};
    \node [above left=of b] {5}; \node [above right=of b] {6};
    \node [below left=of b] {7}; \node [below right=of b] {8};
  \end{scope}
\end{tikzpicture}
\end{codeexample}
\end{key}

\begin{key}{/tikz/below left=\opt{\meta{specification}}}
  Works similar to |above left|.
\end{key}
\begin{key}{/tikz/above right=\opt{\meta{specification}}}
  Works similar to |above left|.
\end{key}
\begin{key}{/tikz/below right=\opt{\meta{specification}}}
  Works similar to |above left|.
\end{key}

The |positioning| package also introduces the following new placement
keys:
\begin{key}{/tikz/base left=\opt{\meta{specification}}}
  This key works like the |left| key, only instead of the |east| anchor,
  the |base east| anchor is used and, when the second form of an
  \meta{of-part} is used, the corresponding |base west| anchor.

  This key is useful for chaining together nodes so that their base
  lines are aligned.
\begin{codeexample}[]
\begin{tikzpicture}[node distance=1ex]
  \draw[help lines] (0,0) grid (3,1);
  \huge
  \node (X) at (0,1)     {X};
  \node (a) [right=of X] {a};
  \node (y) [right=of a] {y};

  \node (X) at (0,0)          {X};
  \node (a) [base right=of X] {a};
  \node (y) [base right=of a] {y};
\end{tikzpicture}
\end{codeexample}
\end{key}
\begin{key}{/tikz/base right=\opt{\meta{specification}}}
  Works like |base left|.
\end{key}
\begin{key}{/tikz/mid left=\opt{\meta{specification}}}
  Works like |base left|, but with |mid east| and |mid west| anchors
  instead of |base east| and |base west|.
\end{key}
\begin{key}{/tikz/mid right=\opt{\meta{specification}}}
  Works like |mid left|.
\end{key}




\subsubsection{Advanced Arrangements of  Nodes}

The simple |above| and |right| options may not always suffice for
arranging a large number of nodes. For such situations \tikzname\
offers libraries that make positioning easier: The |graphdrawing|
library and the |matrix| library. These libraries for positioning
nodes are described in two separate Sections~\ref{section-matrices}
and~\ref{section-intro-gd}. 


\subsection{Fitting Nodes to a Set of Coordinates}

\label{section-nodes-fitting}

It is sometimes desirable that the size and position of a node is not
given using anchors and size parameters, rather one would sometimes
have a box be placed and be sized such that it ``is just large enough
to contain this, that, and that point.'' This situation typically
arises when a picture has been drawn and, afterwards, parts of the
picture are supposed to be encircled or highlighted.

In this situation the |fit| option from the |fit| library is useful,
see Section~\ref{section-library-fit} for the details. The idea is
that you may give the |fit| option to a node. The |fit| option expects
a list of coordinates (one after the other without commas) as its
parameter. The effect will be that the node's text area has exactly
the necessary size so that it contains all the given coordinates. Here
is an example:

\begin{codeexample}[]
\begin{tikzpicture}[level distance=8mm]
  \node (root) {root}
    child { node (a) {a} }
    child { node (b) {b}
      child { node (d) {d} }
      child { node (e) {e} } }
    child { node (c) {c} };

  \node[draw=red,inner sep=0pt,thick,ellipse,fit=(root) (b) (d) (e)] {};
  \node[draw=blue,inner sep=0pt,thick,ellipse,fit=(b) (c) (e)] {};
\end{tikzpicture}
\end{codeexample}

If you want to fill the fitted node you will usually have to place it
on a background layer.

\begin{codeexample}[]
\begin{tikzpicture}[level distance=8mm]
  \node (root) {root}
    child { node (a) {a} }
    child { node (b) {b}
      child { node (d) {d} }
      child { node (e) {e} } }
    child { node (c) {c} };

  \begin{scope}[on background layer]
    \node[fill=red!20,inner sep=0pt,ellipse,fit=(root) (b) (d) (e)] {};
    \node[fill=blue!20,inner sep=0pt,ellipse,fit=(b) (c) (e)] {};
  \end{scope}
\end{tikzpicture}
\end{codeexample}

\subsection{Transformations}

\label{section-nodes-transformations}


It is possible to transform nodes, but, by default, transformations do
not apply to nodes. The reason is that you usually do \emph{not} want
your text to be scaled or rotated even if the main graphic is
transformed. Scaling text is evil, rotating slightly less so.

However, sometimes you \emph{do} wish to transform a node, for
example, it certainly sometimes makes sense to rotate a node by
90 degrees. There are two ways to achieve this:

\begin{enumerate}
\item
  You can use the following option:
  \begin{key}{/tikz/transform shape}
    Causes the current ``external'' transformation matrix to be
    applied to the shape. For example, if you said
    |\tikz[scale=3]| and then say |node[transform shape] {X}|, you
    will get a ``huge'' X in your graphic.
  \end{key}
\item
  You can give transformation options \emph{inside} the option list of
  the node. \emph{These} transformations always apply to the node.
\begin{codeexample}[]
\begin{tikzpicture}[every node/.style={draw}]
  \draw[help lines](0,0) grid (3,2);
  \draw            (1,0) node{A}
                   (2,0) node[rotate=90,scale=1.5] {B};
  \draw[rotate=30] (1,0) node{A}
                   (2,0) node[rotate=90,scale=1.5] {B};
  \draw[rotate=60] (1,0) node[transform shape] {A}
                   (2,0) node[transform shape,rotate=90,scale=1.5] {B};
\end{tikzpicture}
\end{codeexample}
\end{enumerate}


Even though \tikzname\ currently does not allow you to configure
so-called \emph{nonlinear transformations,} see
Section~\ref{section-nonlinear-transformations}, there is an option
that influences how nodes are transformed when nonlinear
transformations are in force:

\begin{key}{/tikz/transform shape nonlinear=\opt{\meta{true or false}}  (initially false)}
  When set to true, \tikzname\ will try to apply any current nonlinear
  transformation also to nodes. Typically, for the text in nodes this
  is not possible in general, in such cases a linear approximation of
  the nonlinear transformation is used. For more details, see
  Section~\ref{section-nonlinear-transformations}.
\makeatletter  
\begin{codeexample}[]
\begin{tikzpicture}
   % Install a nonlinear transformation:
   \pgfsetcurvilinearbeziercurve
      {\pgfpoint{0mm}{20mm}}
      {\pgfpoint{10mm}{20mm}}
      {\pgfpoint{10mm}{10mm}}
      {\pgfpoint{20mm}{10mm}}
   \pgftransformnonlinear{\pgfpointcurvilinearbezierorthogonal\pgf@x\pgf@y}%

   % Draw something:
   \draw [help lines] (0,-30pt) grid [step=10pt] (80pt,30pt);
   
   \foreach \x in {0,20,...,80}
     \node [fill=red!20]  at (\x pt, -20pt) {\x};
  
   \foreach \x in {0,20,...,80}
     \node [fill=blue!20, transform shape nonlinear] at (\x pt, 20pt) {\x};
\end{tikzpicture}
\end{codeexample}
\end{key}



\subsection{Placing Nodes on a Line or Curve Explicitly}

\label{section-nodes-placing-1}

Until now, we always placed node on a coordinate that is mentioned in
the path. Often, however, we wish to place nodes on ``the middle'' of
a line and we do not wish to compute these coordinates ``by hand.''
To facilitate such placements, \tikzname\ allows you to specify that a
certain node should be somewhere ``on'' a line. There are two ways of
specifying this: Either explicitly by using the |pos| option or
implicitly by placing the node ``inside'' a path operation. These two
ways are described in the following.

\label{section-pos-option}

\begin{key}{/tikz/pos=\meta{fraction}}
  When this option is given, the node is not anchored on the last
  coordinate. Rather, it is anchored on some point on the line from
  the previous coordinate to the current point. The \meta{fraction}
  dictates how ``far'' on the line the point should be. A
  \meta{fraction} of 0 is the previous coordinate, 1 is the current
  one, everything else is in between. In particular, 0.5 is the
  middle.

  Now, what is ``the previous line''? This depends on the previous
  path construction operation.

  In the simplest case, the previous path operation was a ``line-to''
  operation, that is, a  |--|\meta{coordinate} operation:
\begin{codeexample}[]
\tikz \draw (0,0) -- (3,1)
    node[pos=0]{0} node[pos=0.5]{1/2} node[pos=0.9]{9/10};
\end{codeexample}

  For the |arc| operation, the position is simply the corresponding
  position on the arc:
\begin{codeexample}[]
\tikz {
  \draw [help lines] (0,0) grid (3,2);
  \draw (2,0) arc [x radius=1, y radius=2, start angle=0, end angle=180]
              node foreach \t in {0,0.125,...,1} [pos=\t,auto] {\t};
}
\end{codeexample}

  The next case is the curve-to operation (the |..| operation). In this
  case, the ``middle'' of the curve, that is, the position |0.5| is
  not necessarily the point at the exact half distance on the
  line. Rather, it is some point at ``time'' 0.5 of a point traveling
  from the start of the curve, where it is at time 0, to the end of
  the curve, which it reaches at time 0.5. The ``speed'' of the point
  depends on the length of the support vectors (the vectors that
  connect the start and end points to the control points). The exact
  math is a bit complicated (depending on your point of view, of
  course); you may wish to consult a good book on computer graphics
  and B\'ezier curves if you are intrigued.
\begin{codeexample}[]
\tikz \draw (0,0) .. controls +(right:3.5cm) and +(right:3.5cm) .. (0,3)
  node foreach \p in {0,0.125,...,1} [pos=\p]{\p};
\end{codeexample}

  Another interesting case are the horizontal/vertical line-to operations
  \verb!|-! and \verb!-|!. For them, the position (or time) |0.5| is
  exactly the corner point.

\begin{codeexample}[]
\tikz \draw (0,0) |- (3,1)
  node[pos=0]{0} node[pos=0.5]{1/2} node[pos=0.9]{9/10};
\end{codeexample}

\begin{codeexample}[]
\tikz \draw (0,0) -| (3,1)
  node[pos=0]{0} node[pos=0.5]{1/2} node[pos=0.9]{9/10};
\end{codeexample}

  For all other path construction operations, \emph{the position
  placement does not work}, currently.
\end{key}

\begin{key}{/tikz/auto=\opt{\meta{direction}} (default \normalfont is scope's setting)}
  This option causes an anchor positions to be calculated
  automatically according to the following rule. Consider a line
  between to points. If the \meta{direction} is |left|, then the
  anchor is chosen such that the node is to the left of this line. If
  the \meta{direction} is |right|, then the node is to the right of
  this line. Leaving out \meta{direction} causes automatic placement
  to be enabled with the last value of |left| or |right| used. A
  \meta{direction} of |false| disables automatic placement. This
  happens also  whenever an anchor is given explicitly by the
  |anchor| option or by one of the |above|, |below|, etc.\ options.

  This option only has an effect for nodes that are placed on lines or
  curves.

\begin{codeexample}[]
\begin{tikzpicture}
  [scale=.8,auto=left,every node/.style={circle,fill=blue!20}]
  \node (a) at (-1,-2) {a};
  \node (b) at ( 1,-2) {b};
  \node (c) at ( 2,-1) {c};
  \node (d) at ( 2, 1) {d};
  \node (e) at ( 1, 2) {e};
  \node (f) at (-1, 2) {f};
  \node (g) at (-2, 1) {g};
  \node (h) at (-2,-1) {h};

  \foreach \from/\to in {a/b,b/c,c/d,d/e,e/f,f/g,g/h,h/a}
    \draw [->] (\from) -- (\to)
               node[midway,fill=red!20] {\from--\to};
\end{tikzpicture}
\end{codeexample}
\end{key}

\begin{key}{/tikz/swap}
  This option exchanges the roles of |left| and |right| in automatic
  placement. That is, if |left| is the current |auto| placement,
  |right| is set instead and the other way round.
\begin{codeexample}[]
\begin{tikzpicture}[auto]
  \draw[help lines,use as bounding box] (0,-.5) grid (4,5);

  \draw (0.5,0) .. controls (9,6) and (-5,6) .. (3.5,0)
    node foreach \pos in {0,0.1,0.2,0.3,0.4,0.5,0.6,0.7,0.8,0.9,1}
         [pos=\pos,swap,fill=red!20] {\pos}
    node foreach \pos in {0.025,0.2,0.4,0.6,0.8,0.975}
         [pos=\pos,fill=blue!20] {\pos};
\end{tikzpicture}
\end{codeexample}
\begin{codeexample}[]
\begin{tikzpicture}[shorten >=1pt,node distance=2cm,auto]
  \draw[help lines] (0,0) grid (3,2);

  \node[state] (q_0)                      {$q_0$};
  \node[state] (q_1) [above right of=q_0] {$q_1$};
  \node[state] (q_2) [below right of=q_0] {$q_2$};
  \node[state] (q_3) [below right of=q_1] {$q_3$};

  \path[->] (q_0) edge              node        {0} (q_1)
                  edge              node [swap] {1} (q_2)
            (q_1) edge              node        {1} (q_3)
                  edge [loop above] node        {0} ()
            (q_2) edge              node [swap] {0} (q_3)
                  edge [loop below] node        {1} ();
\end{tikzpicture}
\end{codeexample}
\end{key}

\begin{key}{/tikz/'}
  This is a very short alias for |swap|.  
\end{key}

\begin{key}{/tikz/sloped}
  This option causes the node to be rotated such that a horizontal
  line becomes a tangent to the curve. The rotation is normally
  done in such a way that text is never ``upside down.'' To get
  upside-down text, use can use |[rotate=180]| or
  |[allow upside down]|, see below.
\begin{codeexample}[]
\tikz \draw (0,0) .. controls +(up:2cm) and +(left:2cm) .. (1,3)
    node foreach \p in {0,0.25,...,1} [sloped,above,pos=\p]{\p};
\end{codeexample}
\begin{codeexample}[]
\begin{tikzpicture}[->]
  \draw (0,0)   -- (2,0.5) node[midway,sloped,above] {$x$};
  \draw (2,-.5) -- (0,0)   node[midway,sloped,below] {$y$};
\end{tikzpicture}
\end{codeexample}
\end{key}


\begin{key}{/tikz/allow upside down=\meta{boolean} (default true, initially false)}
  If set to |true|, \tikzname\ will not ``righten'' upside down text.
\begin{codeexample}[]
\tikz [allow upside down]
  \draw (0,0) .. controls +(up:2cm) and +(left:2cm) .. (1,3)
    node foreach \p in {0,0.25,...,1} [sloped,above,pos=\p]{\p};
\end{codeexample}
\begin{codeexample}[]
\begin{tikzpicture}[->,allow upside down]
  \draw (0,0)   -- (2,0.5) node[midway,sloped,above] {$x$};
  \draw (2,-.5) -- (0,0)   node[midway,sloped,below] {$y$};
\end{tikzpicture}
\end{codeexample}
\end{key}


There exist styles for specifying positions a bit less ``technically'':
\begin{stylekey}{/tikz/midway}
  This has the same effect as |pos=0.5|.
\begin{codeexample}[]
\tikz \draw (0,0) .. controls +(up:2cm) and +(left:3cm) .. (1,5)
       node[at end]          {\texttt{at end}}
       node[very near end]   {\texttt{very near end}}
       node[near end]        {\texttt{near end}}
       node[midway]          {\texttt{midway}}
       node[near start]      {\texttt{near start}}
       node[very near start] {\texttt{very near start}}
       node[at start]        {\texttt{at start}};
\end{codeexample}
\end{stylekey}

\begin{stylekey}{/tikz/near start}
  Set to |pos=0.25|.
\end{stylekey}

\begin{stylekey}{/tikz/near end}
  Set to |pos=0.75|.
\end{stylekey}

\begin{stylekey}{/tikz/very near start}
  Set to |pos=0.125|.
\end{stylekey}

\begin{stylekey}{/tikz/very near end}
  Set to |pos=0.875|.
\end{stylekey}

\begin{stylekey}{/tikz/at start}
  Set to |pos=0|.
\end{stylekey}

\begin{stylekey}{/tikz/at end}
  Set to |pos=1|.
\end{stylekey}


\subsection{Placing Nodes on a Line or Curve Implicitly}

\label{section-nodes-placing-2}

When you wish to place a node on the line |(0,0) -- (1,1)|,
it is natural to specify the node not following the |(1,1)|, but
``somewhere in the middle.'' This is, indeed, possible and you can
write |(0,0) -- node{a} (1,1)| to place a node midway between |(0,0)| and
|(1,1)|.

What happens is the following: The syntax of the line-to path
operation is actually |--|
\opt{|node|\meta{node specification}}\meta{coordinate}. (It is even
possible to give multiple nodes in this way.) When the optional
|node| is encountered, that is,
when the |--| is directly followed by |node|, then the
specification(s) are read and ``stored away.'' Then, after the
\meta{coordinate} has finally been reached, they are inserted again,
but with the |pos| option set.

There are two things to note about this: When a node specification is
``stored,'' its catcodes become fixed. This means that you cannot use
overly complicated verbatim text in them. If you really need, say, a
verbatim text, you will have to put it in a normal node following the
coordinate and add the |pos| option.

Second, which |pos| is chosen for the node? The position is inherited
from the surrounding scope. However, this holds only for nodes
specified in this implicit way. Thus, if you add the option
|[near end]| to a scope, this does not mean that \emph{all} nodes given
in this scope will be put on near the end of lines. Only the nodes
for which an implicit |pos| is added will be placed near the
end. Typically, this is what you want. Here are some examples that
should make this clearer:

\begin{codeexample}[]
\begin{tikzpicture}[near end]
  \draw (0cm,4em) -- (3cm,4em) node{A};
  \draw (0cm,3em) --           node{B}          (3cm,3em);
  \draw (0cm,2em) --           node[midway] {C} (3cm,2em);
  \draw (0cm,1em) -- (3cm,1em) node[midway] {D} ;
\end{tikzpicture}
\end{codeexample}

Like the line-to operation, the curve-to operation |..| also allows you to
specify nodes ``inside'' the operation. After both the first |..| and
also after the second |..| you can place node specifications. Like for
the |--| operation, these will be collected and then reinserted after
the operation with the |pos| option set.


\subsection{The Label and Pin Options}

\subsubsection{Overview}

In addition to the |node| path operation, the two options |label| and
|pin| can be used to ``add a node next to another node''. As an
example, suppose we want to draw a graph in which the nodes are
small circles:
\begin{codeexample}[]
\tikz [circle] {
  \node [draw] (s) {};
  \node [draw] (a) [right=of s] {} edge (s);
  \node [draw] (b) [right=of a] {} edge (a);
  \node [draw] (t) [right=of b] {} edge (b);
}
\end{codeexample}

Now, in the above example, suppose we wish to indicate that the first
node is the start node and the last node is the target node. We could
write |\node (s) {$s$};|, but this would enlarge the first
node. Rather, we want the ``$s$'' to be placed next to the node. For
this, we need to create \emph{another} node, but next to the existing
node. The |label| and |pin| option allow us to do exactly this without
having to use the cumbersome |node| syntax:

\begin{codeexample}[]
\tikz [circle] {
  \node [draw] (s) [label=$s$]  {};
  \node [draw] (a) [right=of s] {} edge (s);
  \node [draw] (b) [right=of a] {} edge (a);
  \node [draw] (t) [right=of b, label=$t$] {} edge (b);
}
\end{codeexample}

\subsubsection{The Label Option}

\begin{key}{/tikz/label=\opt{|[|\meta{options}|]|\meta{angle}|:|}\meta{text}}
  \label{label-option}%
  When this option is given to a |node| operation, it causes
  \emph{another} node to be added to the path after the current node
  has been finished. This extra node will have the text
  \meta{text}. It is placed, in principle, in the direction
  \meta{angle} relative to the main node, but the exact rules are a
  bit complex. Suppose the |node| currently under construction is called
  |main node| and let us call the label node |label node|. Then the
  following happens:
  \begin{enumerate}
  \item The \meta{angle} is used to determine a position on the border
    of the |main node|. If the \meta{angle} is missing, the value of
    the following key is used instead:
    \begin{key}{/tikz/label position=\meta{angle} (initially above)}
      Sets the default position for labels.
    \end{key}
    The \meta{angle} determines the position on the border of the
    shape in two different ways. Normally, the border position is
    given by |main node.|\meta{angle}. This means that the
    \meta{angle} can either be a number like |0| or |-340|, but it can
    also be an anchor like |north|. Additionally, the special angles
    |above|, |below|, |left|, |right|, |above left|, and so on are
    automatically replaced by the corresponding angles |90|, |270|,
    |180|, |0|, |135|, and so on.

    A special case arises when the following key is set:
    \begin{key}{/tikz/absolute=\meta{true or false} (default true)}
      When this key is set, the \meta{angle} is interpreted
      differently: We still use a point on the border of the
      |main node|, but the angle is measured ``absolutely,'' that is,
      an angle of |0| refers to the point on the border that lies on a
      straight line from the |main node|'s center to the right
      (relative to the paper, not relative to the local coordinate
      system of either the node or the scope).

      The difference can be seen in the following example:
\begin{codeexample}[]
\tikz [rotate=-80,every label/.style={draw,red}]
  \node [transform shape,rectangle,draw,label=right:label] {main node};
\end{codeexample}
\begin{codeexample}[]
\tikz [rotate=-80,every label/.style={draw,red},absolute]
  \node [transform shape,rectangle,draw,label=right:label] {main node};
\end{codeexample}
    \end{key}
  \item Then, an anchor point for the |label node| is computed. It is determined
    in such a way that the |label node| will ``face away'' from the
    border of the |main node|. The anchor that is chosen depends on
    the position of the border point that is chosen and its position
    relative to the center of the |main node| and on whether the
    |transform shape| option is set. In detail, when the computed
    border point is at $0^\circ$, the anchor |west| will be
    used. Similarly, when the border point is at $90^\circ$, the
    anchor |south| will be used, and so on for $180^\circ$ and
    $270^\circ$.

    For angles between these ``major'' angles, like
    $30^\circ$ or $110^\circ$, combined anchors, like |south west| for
    $30^\circ$ or |south east| for $110^\circ$, are used. However, for
    angles close to the major angles, (differing by up to $2^\circ$ 
    from the major angle), the anchor for the major angle is
    used. Thus, a label at a border point for $2^\circ$ will have the
    anchor |west|, while a label for $3^\circ$ will have the anchor
    |south west|, resulting in a ``jump'' of the anchor. You can set
    the anchor ``by hand'' using the |anchor| key or indirect keys
    like |left|.
\begin{codeexample}[]
\tikz 
  \node [circle, draw,
         label=default,
         label=60:$60^\circ$,
         label=below:$-90^\circ$,
         label=3:$3^\circ$,
         label=2:$2^\circ$,
         label={[below]180:$180^\circ$},
         label={[centered]135:$115^\circ$}] {my circle};
\end{codeexample}
  \item
    One \meta{angle} is special: If you set the \meta{angle} to
    |center|, then the label will be placed on the center of the main
    node. This is mainly useful for adding a label text to an existing
    node, especially if it has been rotated.
\begin{codeexample}[]
\tikz \node [transform shape,rotate=90,
             rectangle,draw,label={[red]center:R}] {main node};
\end{codeexample}
  \end{enumerate}

  You can pass \meta{options} to the node |label node|. For this, you
  provide the options in square brackets before the \meta{angle}. If you
  do so, you need to add braces around the whole argument of the
  |label| option and this is also the case if you have brackets or
  commas or semicolons or anything special in the \meta{text}.
\begin{codeexample}[]
\tikz \node [circle,draw,label={[red]above:X}] {my circle};
\end{codeexample}

\begin{codeexample}[]
\begin{tikzpicture}
  \node [circle,draw,label={[name=label node]above left:$a,b$}] {};
  \draw (label node) -- +(1,1);
\end{tikzpicture}
\end{codeexample}

  If you provide multiple |label| options, then multiple extra label
  nodes are added in the order they are given.

  The following styles influence how labels are drawn:
  \begin{key}{/tikz/label distance=\meta{distance} (initially 0pt)}
    The \meta{distance} is additionally inserted between the main node
    and the label node.
\begin{codeexample}[]
\tikz[label distance=5mm]
  \node [circle,draw,label=right:X,
                     label=above right:Y,
                     label=above:Z]       {my circle};
\end{codeexample}
  \end{key}
  \begin{stylekey}{/tikz/every label (initially \normalfont empty)}
    This style is used in every node created by the |label|
    option. The default is |draw=none,fill=none|.
  \end{stylekey}
\end{key}

See Section~\ref{section-label-quotes} for an easier
syntax for specifying nodes.

\subsubsection{The Pin Option}

\begin{key}{/tikz/pin=\opt{|[|\meta{options}|]|}\meta{angle}|:|\meta{text}}
  This option is quite similar to the |label| option, but there is
  one difference: In addition to adding an extra node to the picture,
  it also adds an edge from this node to the main node. This causes
  the node to look like a pin that has been added to the main node:
\begin{codeexample}[]
\tikz \node [circle,fill=blue!50,minimum size=1cm,pin=60:$q_0$] {};
\end{codeexample}

  The meaning of the \meta{options} and the \meta{angle} and the
  \meta{text} is exactly the same as for the |node| option. Only, the
  options and styles the influence the way pins look are different:
  \begin{key}{/tikz/pin distance=\meta{distance} (initially 3ex)}
    This \meta{distance} is used instead of the |label distance| for
    the distance between the main node and the label node.
\begin{codeexample}[]
\tikz[pin distance=1cm]
  \node [circle,draw,pin=right:X,
                     pin=above right:Y,
                     pin=above:Z]       {my circle};
\end{codeexample}
  \end{key}
  \begin{stylekey}{/tikz/every pin (initially {draw=none,fill=none})}
    This style is used in every node created by the |pin|
    option.
  \end{stylekey}
  \begin{key}{/tikz/pin position=\meta{angle} (initially above)}
    The default pin position. Works like |label position|.
  \end{key}
  \begin{stylekey}{/tikz/every pin edge (initially help lines)}
    This style is used in every edge created by the |pin| options.
\begin{codeexample}[]
\tikz [pin distance=15mm,
       every pin edge/.style={<-,shorten <=1pt,decorate,
                              decoration={snake,pre length=4pt}}]
  \node [circle,draw,pin=right:X,
                     pin=above right:Y,
                     pin=above:Z]       {my circle};
\end{codeexample}
  \end{stylekey}

  \begin{key}{/tikz/pin edge=\meta{options} (initially \normalfont empty)}
    This option can be used to set the options that are to be used
    in the edge created by the |pin| option.
\begin{codeexample}[]
\tikz[pin distance=10mm]
  \node [circle,draw,pin={[pin edge={blue,thick}]right:X},
                     pin=above:Z]       {my circle};
\end{codeexample}
\begin{codeexample}[]
\tikz [every pin edge/.style={},
       initial/.style={pin={[pin distance=5mm,
                             pin edge={<-,shorten <=1pt}]left:start}}]
  \node [circle,draw,initial] {my circle};
\end{codeexample}
  \end{key}
\end{key}


\subsubsection{The Quotes Syntax}
\label{section-label-quotes}

The |label| and |pin| option provide a syntax for
creating nodes next to existing nodes, but this syntax is often a
bit too verbose. By including the following library, you get access to
an even more concise syntax:

\begin{tikzlibrary}{quotes}
  Enables the quotes syntax for labels, pins, edge nodes, and pic
  texts. 
\end{tikzlibrary}

Let us start with the basics of what this library does: Once loaded,
inside the options of a |node| command, instead of the usual
\meta{key}|=|\meta{value} pairs, you may also provide strings of the
following form (the actual syntax slightly more general, see the
detailed descriptions later on):
\begin{quote}
  |"|\meta{text}|"|\opt{\meta{options}}
\end{quote}
The \meta{options} must be surrounded in curly braces when they
contain a comma, otherwise the curly braces are optional. The may be
preceded by an optional space.

When a \meta{string} of the above form is encountered inside the
options of a |node|, then it is internally transformed to
\begin{quote}
  |label={[|\meta{options}|]|\meta{text}|}|
\end{quote}

Let us have a look at an example:
\begin{codeexample}[]
\tikz \node ["my label" red, draw] {my node};
\end{codeexample}
The above has the same effect as the following:
\begin{codeexample}[]
\tikz \node [label={[red]my label}, draw] {my node};
\end{codeexample}

Here are further examples, one where no \meta{options} are added
to the |label|, one where a position is specified, and examples with
more complicated options in curly braces:

\begin{codeexample}[]
\begin{tikzpicture}
  \matrix [row sep=5mm] {
    \node [draw, "label"]                  {A}; \\
    \node [draw, "label" left]             {B}; \\
    \node [draw, "label" centered]         {C}; \\
    \node [draw, "label" color=red]        {D}; \\
    \node [draw, "label" {red,draw,thick}] {E}; \\
  };  
\end{tikzpicture}
\end{codeexample}

Let us now have a more detailed look at what which commands this
library provides:

\begin{key}{/tikz/quotes mean label}
  When this option is used (which is the default when this library is
  loaded), then, as described above, inside the options of a node a
  special syntax check is done.
  
  \medskip
  \noindent\textbf{The syntax.}
  For each string in the list of options
  it is tested whether it starts with a quotation mark (note that this
  will never happen for normal keys since the normal keys of
  \tikzname\ do not start with quotation marks). When this happens,
  the \meta{string} should not be a key--value pair, but, rather, must
  have the form:
  \begin{quote}
    |"|\meta{text}|"|\opt{|'|}\opt{\meta{options}}
  \end{quote}

  (We will discuss the optional apostrophe in a moment. It is not
  really important for the current option, but only for
  edge labels, which are discussed later).

  \medskip
  \noindent\textbf{Transformation to a label option.}
  When a \meta{string} has the above form, it is treated (almost) as
  if you had written
  \begin{quote}
    |label={[|\meta{options}|]|\meta{text}|}|
  \end{quote}
  instead. The ``almost'' refers to the following additional feature:
  In reality, before the \meta{options} are executed inside the
  |label| command, the direction keys |above|, |left|, |below right|
  and so on are redefined so that |above| is a shorthand for
  |label position=90| and similarly for the other keys. The net effect
  is that in order to specify the position of the \meta{text} relative
  to the main node you can just put something like |left| or
  |above right| inside the \meta{options}:
\begin{codeexample}[]
\tikz 
  \node ["$90^\circ$" above, "$180^\circ$" left, circle, draw] {circle};
\end{codeexample}

  Alternatively, you can also use \meta{direction}|:|\meta{actual
    text} as your \meta{text}. This works since the |label| command
  allows you to specify a direction at the beginning when it is
  separated by a colon:
\begin{codeexample}[]
\tikz 
  \node ["90:$90^\circ$", "left:$180^\circ$", circle, draw] {circle};
\end{codeexample}
  Arguably, placing |above| or |left| behind the \meta{text} seems
  more natural than having it inside the \meta{text}.

  In addition to the above, before the \meta{options} are executed,
  the following style is also executed:
  \begin{stylekey}{/tikz/every label quotes}
\begin{codeexample}[]
\tikz [every label quotes/.style=red]
  \node ["90:$90^\circ$", "left:$180^\circ$", circle, draw] {circle};
\end{codeexample}
  \end{stylekey}
  
  \medskip
  \noindent\textbf{Handling commas and colons inside the text.}
  The \meta{text} may not contain a comma, unless it is inside curly
  braces. The reason is that the key handler separates the total
  options of a |node| along the commas it finds. So, in order to have
  text containing a comma, just add curly braces around either the
  comma or just around the whole \meta{text}:
\begin{codeexample}[]
\tikz \node ["{yes, we can}", draw] {foo};
\end{codeexample}
  The same is true for a colon, only in this case you may need to
  surround specifically the colon by curly braces to stop the |label|
  option from interpreting everything before the colon as a direction:
\begin{codeexample}[]
\tikz \node ["yes{:} we can", draw] {foo};
\end{codeexample}

  \medskip
  \noindent\textbf{The optional apostrophe.}
  Following the closing quotation marks in a \meta{string} there may
  (but need  
  not) be a single quotation mark (an apostrophe), possibly surrounded
  by whitespaces. If it is present, it is simply added to the
  \meta{options} as another option (and, indeed, a single apostrophe
  is a legal option in \tikzname, it is a shorthand for |swap|):

  \begin{tabular}{ll}
    String & has the same effect as \\\hline
    |"foo"'| & |"foo" {'}| \\
    |"foo"' red| & |"foo" {',red}| \\
    |"foo"'{red}| & |"foo" {',red}| \\
    |"foo"{',red}| & |"foo" {',red}| \\
    |"foo"{red,'}| & |"foo" {red,'}| \\
    |"foo"{'red}| & |"foo" {'red}| (illegal; there is no key |'red|)\\
    |"foo" red'| & |"foo" {red'}| (illegal; there is no key |red'|)\\
  \end{tabular}
\end{key}

\begin{key}{/tikz/quotes mean pin}
  This option has exactly the same effect as |quotes mean label|, only
  instead of transforming quoted text to the |label| option, they get
  transformed to the |pin| option:
\begin{codeexample}[]
\tikz [quotes mean pin]
  \node ["$90^\circ$" above, "$180^\circ$" left, circle, draw] {circle};
\end{codeexample}  
  Instead of |every label quotes|, the following style is executed
  with each such pin:
  \begin{stylekey}{/tikz/every pin quotes}
  \end{stylekey}
\end{key}

If instead of |label|s or |pin|s you would like quoted strings to be
interpreted in a different manner, you can also define your own
handlers:

\begin{key}{/tikz/node quotes mean=\meta{replacement}}
  This key allows you to define your own handler for quotes
  options. Inside the options of a |node|, whenever a key--value pair
  with the syntax
  \begin{quote}
    |"|\meta{text}|"|\opt{|'|}\opt{\meta{options}}
  \end{quote}
  is encountered, the following happens: The above string gets
  replaced by \meta{replacement} where inside the \meta{replacement}
  the parameter |#1| is \meta{text} and |#2| is \meta{options}. If the
  apostrophe is present (see also the discussion of
  |quotes mean label|), the \meta{options} start with |',|.

  The \meta{replacement} is then parsed normally as options (using
  |\pgfkeys|).

  Here is an example, where the quotes are used to define labels that
  are automatically named according to the |text|:
\begin{codeexample}[]
\tikzset{node quotes mean={label={[#2,name={#1}]#1}}}

\tikz {
  \node ["1", "2" label position=left, circle, draw] {circle};
  \draw (1) -- (2);
}
\end{codeexample}
\end{key}

Some further options provided by the |quotes| library concern labels
next to edges rather than nodes and they are described in
Section~\ref{section-edge-quotes}. 

\subsection{Connecting Nodes: Using Nodes as Coordinates}

\label{section-nodes-connecting}

Once you have defined a node and given it a name, you can use this
name to reference it. This can be done in two ways, see also
Section~\ref{section-node-coordinates}. Suppose you have said
|\path(0,0) node(x) {Hello World!};| in order to define a node named |x|.
\begin{enumerate}
\item
  Once the node |x| has been defined, you can use
  |(x.|\meta{anchor}|)| wherever you would normally use a normal
  coordinate. This will yield the position at which the given
  \meta{anchor} is in the picture. Note that transformations do not
  apply to this coordinate, that is, |(x.north)| will be the northern
  anchor of |x| even if you have said |scale=3| or |xshift=4cm|. This
  is usually what you would expect.
\item
  You can also just use |(x)| as a coordinate. In most cases, this
  gives the same coordinate as |(x.center)|. Indeed, if the |shape| of
  |x| is |coordinate|, then |(x)| and |(x.center)| have exactly the
  same effect.

  However, for most other shapes, some path construction operations like
  |--| try to be ``clever'' when they are asked to draw a line
  from such a coordinate or to such a coordinate. When you say
  |(x)--(1,1)|, the |--| path operation will not draw a line from the center
  of |x|, but \emph{from the border} of |x| in the direction going
  towards |(1,1)|. Likewise, |(1,1)--(x)| will also have the line
  end on the border in the direction coming from |(1,1)|.

  In addition to |--|, the curve-to path operation |..| and the path
  operations \verb!-|! and \verb!|-! will also handle nodes without
  anchors correctly. Here is an example, see also
  Section~\ref{section-node-coordinates}:
\begin{codeexample}[]
\begin{tikzpicture}
  \path (0,0) node             (x) {Hello World!}
        (3,1) node[circle,draw](y) {$\int_1^2 x \mathrm d x$};

  \draw[->,blue]   (x) -- (y);
  \draw[->,red]    (x) -| node[near start,below] {label} (y);
  \draw[->,orange] (x) .. controls +(up:1cm) and +(left:1cm) .. node[above,sloped] {label} (y);
\end{tikzpicture}
\end{codeexample}
\end{enumerate}




\subsection{Connecting Nodes: Using the Edge Operation}

\label{section-nodes-edges}

\subsubsection{Basic Syntax of the Edge Operation}

The |edge| operation works like a |to| operation that is added after
the main path has been drawn, much like a node is added after the main
path has been drawn. This allows each |edge| to have a
different appearance. As the |node| operation, an |edge| temporarily
suspends the construction of the current path and a new path $p$ is
constructed. This new path $p$ will be drawn after the main path has
been drawn. Note that $p$ can be totally different from the main
path with respect to its options. Also note that if there are
several |to| and/or |node| operations in the main path, each
creates its own path(s) and they are drawn in the order that they
are encountered on the main path.

\begin{pathoperation}{edge}{\opt{|[|\meta{options}|]|}
    \opt{\meta{nodes}} |(|\meta{coordinate}|)|}
  The effect of the |edge| operation is that after the main path the
  following path is added to the picture:
  \begin{quote}
    |\path[every edge,|\meta{options}|] (\tikztostart) |\meta{path}|;|
  \end{quote}
  Here, \meta{path} is the |to path|. Note that, unlike the path added
  by the |to| operation, the |(\tikztostart)| is added before the
  \meta{path} (which is unnecessary for the |to| operation, since this
  coordinate is already part of the main path).

  The |\tikztostart| is the last coordinate on the path just before
  the |edge| operation, just as for the |node| or |to| operations.
  However, there is one exception to this rule: If the |edge|
  operation is directly preceded by a |node| operation, then this
  just-declared node is the start coordinate (and not, as would
  normally be the case, the coordinate where this just-declared node
  is placed -- a small, but subtle difference). In this regard, |edge|
  differs from both |node| and |to|.

  If there are several |edge| operations in a row, the start coordinate
  is the same for all of them as their target coordinates are not,
  after all, part of the main path. The start coordinate is, thus, the
  coordinate preceding the first |edge| operation. This is
  similar to nodes insofar as the |edge| operation does not modify the
  current path at all. In particular, it does not change the last
  coordinate visited, see the following example:

\begin{codeexample}[]
\begin{tikzpicture}
  \node (a) at   (0:1) {$a$};
  \node (b) at  (90:1) {$b$} edge [->]     (a);
  \node (c) at (180:1) {$c$} edge [->]     (a)
                             edge [<-]     (b);
  \node (d) at (270:1) {$d$} edge [->]     (a)
                             edge [dotted] (b)
                             edge [<-]     (c);
\end{tikzpicture}
\end{codeexample}

  A different way of specifying the above graph using the |edge|
  operation is the following:

\begin{codeexample}[]
\begin{tikzpicture}
  \node foreach \name/\angle in {a/0,b/90,c/180,d/270}
        (\name) at (\angle:1) {$\name$};

  \path[->] (b) edge (a)
                edge (c)
                edge [-,dotted] (d)
            (c) edge (a)
                edge (d)
            (d) edge (a);
\end{tikzpicture}
\end{codeexample}

  As can be seen, the path of the |edge| operation inherits the
  options from the main path, but you can locally overrule them.

\begin{codeexample}[]
\begin{tikzpicture}
  \node foreach \name/\angle in {a/0,b/90,c/180,d/270}
        (\name) at (\angle:1.5) {$\name$};

  \path[->] (b) edge            node[above right]  {$5$}     (a)
                edge                                         (c)
                edge [-,dotted] node[below,sloped] {missing} (d)
            (c) edge                                         (a)
                edge                                         (d)
            (d) edge [red]      node[above,sloped] {very}
                                node[below,sloped] {bad}     (a);
\end{tikzpicture}
\end{codeexample}

  Instead of |every to|, the style |every edge| is installed at the
  beginning of the main path.
  \begin{stylekey}{/tikz/every edge (initially draw)}                 
    Executed for each |edge|.
\begin{codeexample}[]
\begin{tikzpicture}[every to/.style={draw,dashed}]
  \path (0,0) edge (3,2);
\end{tikzpicture}
\end{codeexample}
  \end{stylekey}
\end{pathoperation}


\subsubsection{Nodes on Edges: Quotes Syntax}
\label{section-edge-quotes}

The standard way of specifying nodes that are placed ``on'' an edge
(or on a to-path; all of the following is also true for to--paths) is 
to put node specifications after the |edge| keyword, but before the
target coordinate. Another way is to use the |edge node| option and
its friends. Yet another way is to use the quotes syntax.

The syntax is essentially the same as for labels added to nodes as
described in Section~\ref{section-label-quotes} and you also need to
load the |quotes| library.

In detail, when the |quotes| library is loaded, each time a key--value
pair in a list of options passed to an |edge| or a |to| path command
starts with |"|, the key--value pair must actually be a string of the
following form:
\begin{quote}
  |"|\meta{text}|"|\opt{|'|}\opt{\meta{options}}
\end{quote}
This string is transformed into the following:
\begin{quote}
  |edge node=node [every edge quotes,|\meta{options}|]{|\meta{text}|}|
\end{quote}
As described in Section~\ref{section-label-quotes}, the apostrophe
becomes part of the \meta{options}, when present.

The following style is important for the placement of the labels:
\begin{stylekey}{/tikz/every edge quotes (initially auto)}
  This style is |auto| by default, which causes labels specified using
  the quotes-syntax to be placed next to the edges. Unless the setting
  of |auto| has been changed, they will be placed to the left.
\begin{codeexample}[]
\tikz \draw (0,0) edge ["left", ->] (2,0);
\end{codeexample}

  In order to place all labels to the right by default, change this
  style to |auto=right|:
\begin{codeexample}[]
\tikz [every edge quotes/.style={auto=right}]
  \draw (0,0) edge ["right", ->] (2,0);
\end{codeexample}

  To place all nodes ``on'' the edge, just make this style empty (and,
  possibly, make your labels opaque):
\begin{codeexample}[]
\tikz [every edge quotes/.style={fill=white,font=\footnotesize}]
  \draw (0,0) edge ["mid", ->] (2,1);
\end{codeexample}
\end{stylekey}

You may often wish to place some edge nodes to the right of edges and
some to the left. For this, the special treatment of the apostrophe is
particularly convenient: Recall that in \tikzname\ there is an option
just called |'|, which is a shorthand for |swap|. Now, following the
closing quotation mark come the options of an edge node. Thus, if the
closing quotation mark is followed by an apostrophe, the |swap| option
will be added to the edge label, causing it is be placed on the other
side. Because of the special treatment, you can even add another
option like |near end| after the apostrophe without having to add
curly braces and commas:

\begin{codeexample}[]
\tikz 
  \draw (0,0) edge ["left", "right"',
                    "start" near start,
                    "end"' near end] (4,0);
\end{codeexample}

In order to modify the distance between the edge labels and the edge,
you should consider introducing some styles:

\begin{codeexample}[]
\tikz [tight/.style={inner sep=1pt}, loose/.style={inner sep=.7em}]
  \draw (0,0) edge ["left"   tight,
                    "right"' loose,
                    "start"  near start] (4,0);
\end{codeexample}


\subsection{Referencing Nodes Outside the Current Picture}

\label{section-cross-picture-tikz}

\subsubsection{Referencing a Node in a Different Picture}

It is possible (but not quite trivial) to reference nodes in pictures
other than the current one. This means that you can create a picture
and a node therein and, later, you can draw a line from some other
position to this node.

To reference nodes in different pictures, proceed as follows:
\begin{enumerate}
\item You need to add the |remember picture| option to all pictures
  that contain nodes that you wish to reference and also to all
  pictures from which you wish to reference a node in another
  picture.
\item You need to add the |overlay| option to paths or to whole
  pictures that contain references to nodes in different
  pictures. (This option switches the computation of the
  bounding box off.)
\item You need to use a driver that supports picture remembering and
  you need to run \TeX\ twice.
\end{enumerate}
(For more details on what is going on behind the scenes, see
Section~\ref{section-cross-pictures-pgf}.)

Let us have a look at the effect of these options.
\begin{key}{/tikz/remember picture=\meta{boolean} (initially false)}
  This option tells \tikzname\ that it should attempt to remember the
  position of the current picture on the page. This attempt may fail
  depending on which backend driver is used. Also, even if remembering
  works, the position may only be available on a second run of \TeX.

  Provided that remembering works, you may consider saying
\begin{codeexample}[code only]
\tikzstyle{every picture}+=[remember picture]
\end{codeexample}
  to make \tikzname\ remember all pictures. This will add one line in
  the |.aux| file for each picture in your document -- which typically
  is not very much. Then, you do not have to worry about remembered
  pictures at all.
\end{key}

\begin{key}{/tikz/overlay=\meta{boolean} (default true)}
  This option is mainly intended for use when nodes in other pictures
  are referenced, but you can also use it in other situations. The
  effect of this option is that everything within the current scope is
  not taken into consideration when the bounding box of the current
  picture is computed.

  You need to specify this option on all paths (or at least on all
  parts of paths) that contain a reference to a node in another
  picture. The reason is that, otherwise, \tikzname\ will attempt to
  make the current picture large enough to encompass \emph{the node in
    the other picture}. However, on a second run of \TeX\ this will
  create an even bigger picture, leading to larger and larger
  pictures. Unless you know what you are doing, I suggest specifying
  the |overlay| option with all pictures that contain references to
  other pictures.
\end{key}

Let us now have a look at a few examples. These examples work only if
this document is processed with a driver that supports picture
remembering.
\medskip

\noindent\begin{minipage}{\textwidth}
Inside the current text we place two pictures, containing nodes named
|n1| and |n2|, using
\begin{codeexample}[code only]
\tikz[remember picture] \node[circle,fill=red!50] (n1) {};
\end{codeexample}
which yields \tikz[remember picture] \node[circle,fill=red!50] (n1)
{};, and
\begin{codeexample}[code only]
\tikz[remember picture] \node[fill=blue!50] (n2) {};
\end{codeexample}
yielding the node \tikz[remember picture] \node[fill=blue!50] (n2)
{};. To connect these nodes, we create another picture using the
|overlay| option and also the |remember picture| option.
\begin{codeexample}[]
\begin{tikzpicture}[remember picture,overlay]
  \draw[->,very thick] (n1) -- (n2);
\end{tikzpicture}
\end{codeexample}
Note that the last picture is seemingly empty. What happens is that it
has zero size and contains an arrow that lies well outside its bounds.
As a last example, we connect a node in another picture to the first
two nodes. Here, we provide the |overlay| option only with the line
that we do not wish to count as part of the picture.
\begin{codeexample}[]
\begin{tikzpicture}[remember picture]
  \node (c) [circle,draw] {Big circle};

  \draw [overlay,->,very thick,red,opacity=.5]
    (c) to[bend left] (n1) (n1) -| (n2);
\end{tikzpicture}
\end{codeexample}
\end{minipage}


\subsubsection{Referencing the Current Page Node -- Absolute Positioning}

There is a special node called |current page| that can be used to
access the current page. It is a node of shape rectangle whose
|south west| anchor is the lower left corner of the page and whose
|north east| anchor is the upper right corner of the page. While this
node is handled in a special way internally, you can reference it as
if it were defined in some remembered picture other than the current
one. Thus, by giving the |remember picture| and the |overlay|
options to a picture, you can position nodes \emph{absolutely} on a
page.

The first example places some text in the lower left corner of the
current page:
\begin{codeexample}[]
\begin{tikzpicture}[remember picture,overlay]
  \node [xshift=1cm,yshift=1cm] at (current page.south west)
        [text width=7cm,fill=red!20,rounded corners,above right]
  {
    This is an absolutely positioned text in the
    lower left corner. No shipout-hackery is used.
  };
\end{tikzpicture}
\end{codeexample}

The next example adds a circle in the middle of the page.
\begin{codeexample}[]
\begin{tikzpicture}[remember picture,overlay]
  \draw [line width=1mm,opacity=.25]
    (current page.center) circle (3cm);
\end{tikzpicture}
\end{codeexample}

The final example overlays some text over the page (depending on where
this example is found on the page, the text may also be behind the
page).
\begin{codeexample}[]
\begin{tikzpicture}[remember picture,overlay]
  \node [rotate=60,scale=10,text opacity=0.2]
    at (current page.center) {Example};
\end{tikzpicture}
\end{codeexample}



\subsection{Late Code and Late Options}
\label{section-node-also}

All options given to a node only locally affect this one node. While
this is a blessing in most cases, you may sometimes want to cause
options to have effects ``later'' on. The other way round, you may
sometimes note ``only later'' that some options should be added to the
options of a node. For this, the following version of the |node| path
command can be used:

\begin{pathoperation}{node also}{\opt{|[|\meta{late options}|]|}|(|\meta{name}|)|}
  Note that the \meta{name} is compulsory and that \emph{no} text may
  be given. Also, the ordering of options and node label must be as above.
  
  The effect of the above is the following effect: The node
  \meta{name} must already  be existing. Now, the \meta{late options} are
  executed in a local scope. Most of these options will have no effect
  since you \emph{cannot change the appearance of the node,} that is,
  you cannot change a red node into a green node using these ``late''
  options. However, giving the |append after command| and
  |prefix after command| options inside
  the \meta{late options} (directly or indirectly) does have the desired
  effect: The given path gets executed with the |\tikzlastnode|
  set to the determined node.

  The net effect of all this is that you can provide, say, the |label|
  option inside the \meta{options} to a add a label to a node that has
  already been constructed.
  
\begin{codeexample}[]
\begin{tikzpicture}
  \node      [draw,circle]       (a) {Hello};
  \node also [label=above:world] (a);
\end{tikzpicture}
\end{codeexample}
\end{pathoperation}

As explained in Section~\ref{section-paths}, you
can use the options |append after command| and
|prefix after command| to add a path after a node. The following macro may be
useful there:
\begin{command}{\tikzlastnode}
  Expands to the last node on the path.
\end{command}

Instead of the |node also| syntax, you can also the following option:

\begin{key}{/tikz/late options=\meta{options}}
  This option can be given on a path (but not as an argument to a
  |node| path command) and has the same effect as the |node also| path
  command. Inside the \meta{options}, you should use the |name| option
  to specify the node for which you wish to add late options:
 
\begin{codeexample}[]
\begin{tikzpicture}
  \node      [draw,circle]       (a) {Hello};
  \path [late options={name=a, label=above:world}];
\end{tikzpicture}
\end{codeexample}
\end{key}


%%% Local Variables:
%%% mode: latex
%%% TeX-master: "pgfmanual"
%%% End:

% % Copyright 2013 by Till Tantau
%
% This file may be distributed and/or modified
%
% 1. under the LaTeX Project Public License and/or
% 2. under the GNU Free Documentation License.
%
% See the file doc/generic/pgf/licenses/LICENSE for more details.

\section{Pics: Small Pictures on Paths}

\label{section-pics}

\subsection{Overview}

A ``pic'' is a ``short picture'' (hence the short name\dots) that can
be inserted anywhere in \tikzname\ picture where you could also insert
a node. Similarly to nodes, pics have a ``shape'' (called \emph{type}
to avoid confusion) that someone has defined. Each time a pic of a
specified type is used, the type's code is executed, resulting in some
drawings to be added to the current picture. The syntax for adding
nodes and adding pics to a picture are also very similar. The core
difference is that pics are typically more complex than nodes and may
consist of a whole bunch of nodes themselves together with complex
paths joining them.

As a very simple example, suppose we want to define a pic type
|seagull| that just draw ``two bumps.'' The code for this definition
is quite easy:
\begin{codeexample}[code only]
\tikzset{ 
  seagull/.pic={
    % Code for a "seagull". Do you see it?...
    \draw (-3mm,0) to [bend left] (0,0) to [bend left] (3mm,0);
  }
}
\end{codeexample}

The first line just tells \TeX\ that you set some \tikzname\ options
for the current scope (which is the whole document); you could put
|seagull/.pic=...| anywhere else where \tikzname\ options are allowed
(which is just about anywhere). We have now defined a |seagull| pic
type and can use it as follows:

\tikzset{ 
  seagull/.pic={
    % Code for a "seagull". Do you see it?...
    \draw (-3mm,0) to [bend left] (0,0) to [bend left] (3mm,0);
  }
}
\begin{codeexample}[]
\tikz \fill [fill=blue!20]
     (1,1)
  -- (2,2) pic             {seagull}
  -- (3,2) pic             {seagull}
  -- (3,1) pic [rotate=30] {seagull}
  -- (2,1) pic [red]       {seagull};
\end{codeexample}

As can be see, defining new types of pics is much easier than defining
new shapes for nodes; but see Section~\ref{section-new-pic-types}
for the fine details.

Since defining new pics types is easier than defining new node shapes
and since using pics is as easy as using nodes, why should you use
nodes at all? There are chiefly two reasons:

\begin{enumerate}
\item Unlike nodes, pics cannot be referenced later on. You \emph{can}
  reference nodes that are inside a pic, but not ``the pic itself.''
  In particular, you cannot draw lines between pics the way you
  can draw them between nodes. In general, whenever it makes sense
  that some drawing could conceivably be connected to other
  node-like-things, then a node is better than a pic. 
\item If pics are used to emulate the full power of a node (which is
  possible, in principle), they will be slower to construct and take
  up more memory than a node achieving the same effect.
\end{enumerate}

Despite these drawbacks, pics are an excellent choice for creating
highly configurable reusable pieces of drawings that can be inserted
into larger contexts.


\subsection{The Pic Syntax}

\begin{command}{\pic}
  Inside |{tikzpicture}| this is an abbreviation for |\path pic|.
\end{command}

The syntax for adding a pic to a picture is very similar to the syntax
used for nodes (indeed, internally the same parser code is used). The
main difference is that instead of a node contents you provide the
picture's type between the  braces:

\begin{pathoperation}{pic}{\opt{\meta{foreach statements}}%
      \opt{|[|\meta{options}|]|}\opt{|(|\meta{prefix}|)|}%
    \opt{|at(|\meta{coordinate}|)|}\opt{\marg{pic type}}}
  
  Adds a pic to the current \tikzname\ picture of the specified
  \meta{pic type}. The effect is, basically, that some code associated
  with the \meta{pic type} is executed (how this works, exactly, is explained
  later). This code can consist of arbitrary \tikzname\ code. As for
  nodes, the current path will not be modified by this path command,
  all drawings produced by the code are ``external'' to the path the
  same way neither a node nor its border are part of the path on which
  they are specified. 
  
  Just like the |node| command, this path operation is somewhat
  complex and we go over it step by step.

  \medskip
  \textbf{Order of the parts of the specification.}
  Just like for nodes, everything between ``|pic|'' and the opening
  brace of the \meta{pic type} is optional and can be given in any
  order. If there are \meta{foreach statements}, they must come
  first, directly following ``|pic|.'' As for nodes, the ``end'' of
  the pic specification is normally detected by the presence of the
  opening brace. You can, however, use the |pic type|
  option to specify the pic type as an option.
  
  \begin{key}{/tikz/pic type=\meta{pic type}}
    This key sets the pic type of the current~|pic|. When this option
    is used inside an option block of a |pic|, the parsing of the
    |pic| ends immediately and no pic type in braces is expected. (In
    other words, this option behaves exactly like the |node contents|
    option and, indeed, the two are interchangeable.) 
\begin{codeexample}[]
\tikz {
  \path (0,0) pic [pic type = seagull]
        (1,0) pic                      {seagull};
}
\end{codeexample}
  \end{key}

  \medskip
  \textbf{The location of a pic.}
  Just like nodes, pics are placed at the last position mentioned on
  the path or, when |at| is used, at a specified position. ``Placing''
  a pic somewhere actually means that the coordinate system is
  translated (shifted) to this last position. This means that inside
  of the pic type's code any mentioning of the origin refers to the
  last position used on the path or to the specified |at|.

\begin{codeexample}[]
\tikz { % different ways of placing pics
  \draw [help lines] (0,0) grid (3,2);
  \pic  at (1,0)     {seagull};
  \path (2,1)    pic {seagull};
  \pic  [at={(3,2)}] {seagull};
}
\end{codeexample}

  As for nodes, except for the described shifting, the coordinate
  system of a pic is reset prior to executing the pic type's
  code. This can be changed using the |transform shape| option, which
  has the same effect as for nodes: The ``outer'' transformation gets
  applied to the node:
\begin{codeexample}[]
\tikz [scale=2] { 
  \pic at (0,0)                   {seagull};
  \pic at (1,0) [transform shape] {seagull};
}
\end{codeexample}

  When the \meta{options} contain transformation commands like |scale|
  or |rotate|, these transformations always apply to the pic:
\begin{codeexample}[]
\tikz [rotate=30] { 
  \pic at (0,0)             {seagull};
  \pic at (1,0) [rotate=90] {seagull};
}
\end{codeexample}

  Just like nodes, pics can also be positioned implicitly and,
  somewhat unsurprisingly, the same rules concerning positioning and
  sloping apply:
\begin{codeexample}[]
\tikz \draw
  (0,0) to [bend left] 
           pic [near start]       {seagull} 
           pic                    {seagull}
           pic [sloped, near end] {seagull} (4,0);
\end{codeexample}
  
  \medskip
  \textbf{The options of a node.}
  As always, any given \meta{options} apply only to the pic and have
  no effect outside. As for nodes, most ``outside'' options also apply
  to the pics, but not the ``action'' options like |draw| or
  |fill|. These must be given in the \meta{options} of the pic.

  \medskip
  \textbf{The code of a pic.}
  As stated earlier, the main job of a pic is to execute some code in
  a scope that is shifted according to the last point on the path or
  to the |at| position specified in the pic. It was also claimed that
  this code is specified by the \meta{pic type}. However, this
  specification is somewhat indirect. What really happens is the
  following: When a |pic| is encountered, the current path is
  suspended and a new internal scope is started. The \meta{options}
  are executed and also the \meta{pic type} (as explained
  in a moment). After all this is done, the code stored in the following key
  gets executed:

  \begin{key}{/tikz/pics/code=\meta{code}}
    This key stores the \meta{code} that should be drawn in the
    current pic. Normally, setting this key is done by the \meta{pic
      type}, but you can also set it in the \meta{options} and leave
    the \meta{pic type} empty:
\begin{codeexample}[]
\tikz \pic [pics/code={\draw (-3mm,0) to[bend left] (0,0)
                                      to[bend left] (3mm,0);}]
      {}; % no pic type specified
\end{codeexample}
  \end{key}
  
  Now, how does the \meta{pic type} set |pics/code|? It turns out that
  the \meta{pic type} is actually just a list of keys that are
  executed with the prefix |/tikz/pics/|. In the above examples, this
  ``list of keys'' just consisted of the single key ``|seagull|'' that
  did not take any arguments, but, in principle, you could provide any
  arbitrary text understood by |\pgfkeys| here. This means that when
  we write |pic{seagull}|, \tikzname\ will execute the key
  |/tikz/pics/seagull|. It turns out, see
  Section~\ref{section-new-pic-types}, that this key is just a style
  set to |code={\draw(-3mm,0)...;}|. Thus, |pic{seagull}| will cause
  the |pics/code| key to be set to the text needed to draw the
  seagull.
  
  Indeed, you can also use the \meta{pic type} simply to set the
  |code| of the pic. This is useful for cases when you have some code
  that you ``just want to execute, but do not want to define a new pic
  type.'' Here is a typical example where we use pics to add some
  markings to a path:
\begin{codeexample}[]
\tikz \draw (0,0) .. controls(1,0) and (2,1) .. (3,1)
  foreach \t in {0, 0.1, ..., 1} {
    pic [pos=\t] {code={\draw circle [radius=2pt];}}
  };
\end{codeexample}

  
  In our seagull example, we always explicitly used |\draw| to draw
  the seagull. This implies that when a user writes something
  |pic[fill]{seagull}| in the hope of having a ``filled'' seagull,
  nothing special will happen: The |\draw| inside the pic explicitly
  states that the path should be drawn, not filled, and the fact that
  in the surrounding scope the |fill| option is set has no effect.
  The following key can be used to change this:
  \begin{key}{/tikz/pic actions}
    This key is a style that can be used (only) inside the code of a
    pic. There, it will set the ``action'' keys set inside the
    \meta{options} of the pic (``actions'' are drawing, filling,
    shading, and clipping or any combination thereof).

    To see how this key works, let us define the following pic:
\begin{codeexample}[code only]
\tikzset{
  my pic/.pic = {
    \path [pic actions] (0,0) circle[radius=3mm];
    \draw (-3mm,-3mm) rectangle (3mm,3mm);
  }
}
\end{codeexample}
    In the code, whether or not the circle gets drawn/filled/shaded
    depends on which options where given to the |pic| command when it
    is used. In contrast, the rectangle will always (just) be drawn.
\tikzset{
  my pic/.pic = {
    \path [pic actions] (0,0) circle[radius=3mm];
    \draw (-3mm,-3mm) rectangle (3mm,3mm);
  }
}
\begin{codeexample}[width=6cm]
\tikz \pic                      {my pic}; \space
\tikz \pic [red]                {my pic}; \space
\tikz \pic [draw]               {my pic}; \space
\tikz \pic [draw=red]           {my pic}; \space
\tikz \pic [draw, shading=ball] {my pic}; \space
\tikz \pic [fill=red!50]        {my pic};
\end{codeexample}
  \end{key}
  
  \medskip
  \textbf{Code executed behind or in front of the path.}
  As for nodes, a pic can be ``behind'' the current path or ``in front
  of it'' and, just as for nodes, the two options |behind path| and
  |in front of path| are used to specify which is meant. In detail, if
  |node| and |pic| are both used repeatedly on a path, in the
  resulting picture we first see all nodes and pics with the
  |behind path| option set in the order they appear on the path (nodes
  and pics are interchangeable in this regard), then comes the path,
  and then come all nodes and pics that are in front of the path in
  the order they appeared.
\begin{codeexample}[]
\tikz \fill [fill=blue!20]
     (1,1)
  -- (2,2) pic [behind path]      {seagull}
  -- (3,2) pic                    {seagull}
  -- (3,1) pic [rotate=30]        {seagull}
  -- (2,1) pic [red, behind path] {seagull};
\end{codeexample}
  In contrast to nodes, a pic need not only be completely behind
  the path or in front of the path as specified by the user. Instead,
  a pic type may also specify that a certain part of the drawing
  should always be behind the path and it may specify that a certain
  other part should always be before the path. For this, the values of
  the following keys are relevant:
  
  \begin{key}{/tikz/pics/foreground code=\meta{code}}
    This key stores \meta{code} that will always be drawn in front of
    the current path, even when |behind path| is used. If
    |behind path| is not used and |code| is (also) set, the code of
    |code| is 
    drawn first, following by the foreground \meta{code}.
  \end{key}
  
  \begin{key}{/tikz/pics/background code=\meta{code}}
    Like |foreground code|, only that the \meta{code} is always put
    behind the path, never in front of it.
  \end{key}
  
  \medskip
  \textbf{The foreach statement for pics.}
  As for nodes, a pic specification may start with |foreach|. The
  effect and semantics are the same as for nodes.
\begin{codeexample}[]
\tikz \pic foreach \x in {1,2,3} at (\x,0) {seagull};
\end{codeexample}

  \medskip
  \textbf{Styles for pics.}
  The following styles influence how nodes are rendered:
  \begin{stylekey}{/tikz/every pic (initially \normalfont empty)}
    This style is installed at the beginning of every pic.
\begin{codeexample}[]
\begin{tikzpicture}[every node/.style={draw}]
  \draw (0,0) node {A} -- (1,1) node {B};
\end{tikzpicture}
\end{codeexample}
  \end{stylekey}

  
  \medskip
  \textbf{Name scopes.}
  You can specify a \meta{name} for a pic using the key
  |name=|\meta{name} or by giving the name in parenthesis inside the
  pic's specification. The effect of this is, for once, quite
  different from what happens for nodes: All that happens is that
  |name prefix| is set to \meta{name} at the beginning of the pic.
  
  The |name prefix| key was already introduced in the description of
  the |node| command: It allows you to set some text that is prefixed
  to all nodes in a scope. For pics this makes particular sense: All
  nodes defined by a pic's code can be referenced from outside the pic
  with the prefix provided.

  To see how this works, let us add some nodes to the code of the
  seagull:
\begin{codeexample}[code only]
\tikzset{ 
  seagull/.pic={
    % Code for a "seagull". Do you see it?...
    \coordinate (-left wing) at (-3mm,0);
    \coordinate (-head)      at (0,0);
    \coordinate (-right wing) at (3mm,0);
    
    \draw (-left wing) to [bend left] (0,0) (-head) to [bend left] (-right wing);
  }
}
\end{codeexample}
\tikzset{ 
  seagull/.pic={
    % Code for a "seagull". Do you see it?...
    \coordinate (-left wing) at (-3mm,0);
    \coordinate (-head)      at (0,0);
    \coordinate (-right wing) at (3mm,0);
    
    \draw (-left wing) to [bend left] (0,0) (-head) to [bend left] (-right wing);
  }
}

  Now, we can use it as follows:
\begin{codeexample}[code only]
\tikz {
  \pic (Emma)               {seagull};
  \pic (Alexandra) at (0,1) {seagull};

  \draw (Emma-left wing) -- (Alexandra-right wing);
}
\end{codeexample}

  Sometimes, you may also wish your pic to access nodes outside the
  pic (typically, because they are given as parameters). In this case,
  the name prefix gets in the way since the nodes outside the picture
  do not have this prefix. The trick is to locally reset the name
  prefix to the value it had outside the picture, which is achieved
  using the following style:

  \begin{key}{/tikz/name prefix ..}
    This key is available only inside the code of a pic. There, it
    (locally) changes the name prefix to the value it had outside the
    pic. This allows you to access nodes outside the current pic.
  \end{key}

\end{pathoperation}


There are two general purpose keys that pics may find useful:

\begin{key}{/tikz/pic text=\meta{text}}
  This macro stores the \meta{text} in the macro |\tikzpictext|, which
  is |\let| to |\relax| by default. Setting the |pic text| to some
  value is the ``preferred'' way of communicating a (single) piece of
  text that should become part of a pic (typically of a node). In
  particular, the |quotes| library maps quoted parameters to this key.
\end{key}

\begin{key}{/tikz/pic text options=\meta{options}}
  This macro stores the \meta{options} in the macro
  |\tikzpictextoptions|, which is |\let| to the empty string by
  default. The |quotes| library maps options for quoted parameters to
  this key. 
\end{key}



\subsubsection{The Quotes Syntax}
\label{section-pic-quotes}

When you load the |quotes| library, you can use the ``quotes syntax''
inside the options of a pic. Recall that for nodes this syntax is used
to add a label to a node. For pics, the quotes syntax is used to set
the |pic text| key. Whether or not the pic type's code takes this key
into consideration is, however, up to the key.

In detail, when the |quotes| library is loaded, each time a key--value 
pair in a list of options passed to an |pic| starts with |"|, the
key--value pair must actually be a string of the following form:
\begin{quote}
  |"|\meta{text}|"|\opt{|'|}\opt{\meta{options}}
\end{quote}
This string is transformed into the following:
\begin{quote}
  |every pic quotes/.try,pic text=|\meta{text}|, pic text options={|\meta{options}|}|
\end{quote}

As example of a pic type that takes these values into account is the
|angle| pic type:
\begin{codeexample}[]
\tikz \draw (3,0) coordinate (A)
         -- (0,1) coordinate (B)
         -- (1,2) coordinate (C)
            pic [draw, "$\alpha$"] {angle};
\end{codeexample}

As described in Section~\ref{section-label-quotes}, the apostrophe
becomes part of the \meta{options}, when present. As can be seen
above, the following style is executed:
\begin{stylekey}{/tikz/every pic quotes (initially \normalfont empty)}
\end{stylekey}

\subsection{Defining New Pic Types}
\label{section-new-pic-types}

As explained in the description of the |pic| command, in order to
define a new pic type you need to
\begin{enumerate}
\item define a key with the path prefix |/tikz/pics| that
\item sets the key |/tikz/pics/code| to the code of the pic. 
\end{enumerate}

It turns out that this is easy enough to achieve using styles:

\begin{codeexample}[code only]
\tikzset{
  pics/seagull/.style ={
  	% Ok, this is the key that should, when
    % executed, set the code key:
    code = { %
      \draw (...) ... ;
    }
  }
}
\end{codeexample}

Even though the above pattern is easy enough, there is a special
so-called key handler that allows us to write even simpler code,
namely:

\begin{codeexample}[code only]
\tikzset{
  seagull/.pic = {
    \draw (...) ... ;
  }
}
\end{codeexample}

\begin{handler}{{.pic}|=|\meta{some code}}
  This handler can only be used with a key with the prefix |/tikz/|,
  so just should normally use it only as an option to a \tikzname\
  command or to the |\tikzset| command. It takes the \meta{key}'s path
  and, inside that path, it replaces |/tikz/| by |/tikz/pics/| (so,
  basically, it adds the ``missing'' |pics| part of the path). Then,
  it sets up things so that the resulting name to key is a style that
  executes |code=some code|.
\end{handler}

In almost all cases, the |.pic| key handler will suffice to setup
keys. However, there are cases where you really need to use the first
version using  |.style| and |code=|:
\begin{itemize}
\item Whenever your pic type needs to set the foreground or the
  background code.
\item In case of complicated arguments given to the keys.
\end{itemize}

As an example, let us define a simple pic that draws a filled circle
behind the path. Furthermore, we make the circle's radius a parameter
of the pic:

\begin{codeexample}[]
\tikzset{
  pics/my circle/.style = {
    background code = { \fill circle [radius=#1]; }
  }
}
\tikz [fill=blue!30]
  \draw (0,0) pic {my circle=2mm} -- (1,1) pic {my circle=5mm};
\end{codeexample}

% % Copyright 2010 by Till Tantau
% Copyright 2011 by Jannis Pohlmann
%
% This file may be distributed and/or modified
%
% 1. under the LaTeX Project Public License and/or
% 2. under the GNU Free Documentation License.
%
% See the file doc/generic/pgf/licenses/LICENSE for more details.


\section{Specifying Graphs}
\label{section-library-graphs}

\subsection{Overview}

\tikzname\ offers a powerful path command for specifying how the nodes in a
graph are connected by edges and arcs: The |graph| path command, which becomes
available when you load the |graphs| library.

\begin{tikzlibrary}{graphs}
    The package must be loaded to use the |graph| path command.
\end{tikzlibrary}

In this section, by \emph{graph} we refer to a set of nodes together with some
edges (sometimes also called arcs, in case they are directed) such as the
following:
%
\begin{codeexample}[preamble={\usetikzlibrary{graphs}}]
\tikz \graph { a -> {b, c} -> d };
\end{codeexample}

\begin{codeexample}[preamble={\usetikzlibrary{graphs.standard}}]
\tikz \graph {
  subgraph I_nm [V={a, b, c}, W={1,...,4}];

  a -> { 1, 2, 3 };
  b -> { 1, 4 };
  c -> { 2 [>green!75!black], 3, 4 [>red]}
};
\end{codeexample}

\begin{codeexample}[preamble={\usetikzlibrary{graphs}}]
\tikz
  \graph [nodes={draw, circle}, clockwise, radius=.5cm, empty nodes, n=5] {
    subgraph I_n [name=inner] --[complete bipartite]
    subgraph I_n [name=outer]
  };
\end{codeexample}

\begin{codeexample}[
    preamble={\usetikzlibrary{graphs}},
    pre={\definecolor{graphicbackground}{rgb}{0.96,0.96,0.8}},
]
\tikz
  \graph [nodes={draw, circle}, clockwise, radius=.75cm, empty nodes, n=8] {
    subgraph C_n [name=inner] <->[shorten <=1pt, shorten >=1pt]
    subgraph C_n [name=outer]
  };
\end{codeexample}

\begin{codeexample}[width=6.6cm,preamble={\usetikzlibrary{graphs}}]
\tikz [>={To[sep]}, rotate=90, xscale=-1,
       mark/.style={fill=black!50}, mark/.default=]
  \graph [trie, simple,
          nodes={circle,draw},
          edges={nodes={
              inner sep=1pt, anchor=mid,
              fill=graphicbackground}}, % yellowish background
          put node text on incoming edges]
    {
      root[mark] -> {
        a -> n -> {
          g [mark],
          f -> a -> n -> g [mark]
        },
        f -> a -> n -> g [mark],
        g[mark],
        n -> {
          g[mark],
          f -> a -> n -> g[mark]
        }
      },
      { [edges=red] % highlight one path
        root -> f -> a -> n
      }
    };
\end{codeexample}

The nodes of a graph are normal \tikzname\ nodes, the edges are normal lines
drawn between nodes. There is nothing in the |graphs| library that you cannot
do using the normal |\node| and the |edge| commands. Rather, its purpose is to
offer a concise and powerful way of \emph{specifying} which nodes are present
and how they are connected. The |graphs| library only offers simple methods for
specifying \emph{where} the nodes should be shown, its main strength is in
specifying which nodes and edges are present in principle. The problem of
finding ``good positions on the canvas'' for the nodes of a graph is left to
\emph{graph drawing algorithms}, which are covered in Part~\ref{part-gd} of
this manual and which are not part of the |graphs| library; indeed, these
algorithms can be used also with graphs specified using |node| and |edge|
commands.
%
\ifluatex
As an example, consider the above drawing of a trie, which is drawn without
using the graph drawing libraries. Its layout can be somewhat improved by
loading the |layered| graph drawing library, saying |\tikz[layered layout,...|,
and then using Lua\TeX, resulting in the following drawing of the same graph:
\medskip

\tikz [layered layout, >={To[sep]}, rotate=90, xscale=-1,
       mark/.style={fill=black!50}, mark/.default=]
  \graph [trie, simple, sibling distance=8mm,
          nodes={circle,draw},
          edges={nodes={
              inner sep=1pt, anchor=mid, fill=white}},
          put node text on incoming edges]
    {
      root[mark] -> {
        a -> n -> {
          g [mark],
          f -> a -> n -> g [mark]
        },
        f -> a -> n -> g [mark],
        g[mark],
        n -> {
          g[mark],
          f -> a -> n -> g[mark]
        }
      },
      { [edges=red] % highlight one path
        root -> f -> a -> n
      }
    };
\medskip
\fi

The |graphs| library uses a syntax that is quite different from the normal
\tikzname\ syntax for specifying nodes. The reason for this is that for many
medium-sized graphs it can become quite cumbersome to specify all the nodes
using |\node| repeatedly and then using a great number of |edge| command;
possibly with complicated |\foreach| statements. Instead, the syntax of the
|graphs| library is loosely inspired by the \textsc{dot} format, which is quite
useful for specifying medium-sized graphs, with some extensions on top.


\subsection{Concepts}

The present section aims at giving a quick overview of the main concepts behind
the |graph| command. The exact syntax is explained in more detail in later
sections.


\subsubsection{Concept: Node Chains}

The basic way of specifying a graph is to write down a \emph{node chain} as in
the following example:
%
\begin{codeexample}[preamble={\usetikzlibrary{graphs}}]
\tikz [every node/.style = draw]
  \graph { foo -> bar -> blub };
\end{codeexample}

As can be seen, the text |foo -> bar -> my node| creates three nodes, one with
the text |foo|, one with |bar| and one with the text |blub|. These nodes are
connected by arrows, which are caused by the |->| between the node texts. Such
a sequence of node texts and arrows between them is called a \emph{chain} in
the following.

Inside a graph there can be more than one chain:
%
\begin{codeexample}[preamble={\usetikzlibrary{graphs}}]
\tikz \graph {
  a -> b -> c;
  d -> e -> f;
  g -> f;
};
\end{codeexample}

Multiple chains are separated by a semicolon or a comma (both have exactly the
same effect). As the example shows, when a node text is seen for the second
time, instead of creating a new node, a connection is created to the already
existing node.

When a node like |f| is created, both the node name and the node text are
identical by default. This is not always desirable and can be changed by using
the |as| key or by providing another text after a slash:
%
\begin{codeexample}[preamble={\usetikzlibrary{graphs}}]
\tikz \graph {
  x1/$x_1$ -> x2 [as=$x_2$, red] -> x34/{$x_3,x_4$};
  x1 -> [bend left] x34;
};
\end{codeexample}

When you wish to use a node name that contains special symbols like commas or
dashes, you must surround the node name by quotes. This allows you to use quite
arbitrary text as a ``node name'':
%
\begin{codeexample}[preamble={\usetikzlibrary{graphs}}]
\tikz \graph {
  "$x_1$" -> "$x_2$"[red] -> "$x_3,x_4$";
  "$x_1$" ->[bend left] "$x_3,x_4$";
};
\end{codeexample}


\subsubsection{Concept: Chain Groups}

Multiple chains that are separated by a semicolon or a comma and that are
surrounded by curly braces form what will be called a \emph{chain group} or
just a \emph{group}. A group in itself has no special effect. However, things
get interesting when you write down a node or even a whole group and connect it
to another group. In this case, the ``exit points'' of the first node or group
get connected to the ``entry points'' of the second node or group:
%
\begin{codeexample}[preamble={\usetikzlibrary{graphs}}]
\tikz \graph {
  a -> {
    b -> c,
    d -> e
  } -> f
};
\end{codeexample}

Chain groups make it easy to create tree structures:
%
\begin{codeexample}[width=10cm,preamble={\usetikzlibrary{graphs}}]
\tikz
  \graph [grow down,
          branch right=2.5cm] {
  root -> {
    child 1,
    child 2 -> {
      grand child 1,
      grand child 2
    },
    child 3 -> {
      grand child 3
    }
  }
};
\end{codeexample}

As can be seen, the placement is not particularly nice by default, use the
algorithms from the graph drawing libraries to get a better layout. For
instance, adding |tree layout| to the above code results in the following
somewhat more pleasing rendering:
%
\ifluatex
\medskip

\tikz \graph [grow down, branch right=2.5cm, tree layout] {
  root -> {
    child 1,
    child 2 -> {
      grand child 1,
      grand child 2
    },
    child 3 -> {
      grand child 3
    }
  }
};
\else
    (You need to use Lua\TeX\ to typeset this graphic.)
\fi


\subsubsection{Concept: Edge Labels and Styles}

When connectors like |->| or |--| are used to connect nodes or whole chain
groups, one or more edges will typically be created. These edges can be styles
easily by providing options in square brackets directly after these connectors:
%
\begin{codeexample}[preamble={\usetikzlibrary{graphs}}]
\tikz \graph {
  a ->[red] b --[thick] {c, d};
};
\end{codeexample}

Using the quotes syntax, see Section~\ref{section-label-quotes}, you can even
add labels to the edges easily by putting the labels in quotes:
%
\begin{codeexample}[preamble={\usetikzlibrary{graphs,quotes}}]
\tikz \graph {
  a ->[red, "foo"] b --[thick, "bar"] {c, d};
};
\end{codeexample}

For the first edge, the effect is as desired, however between |b| and the group
|{c,d}| two edges are inserted and the options |thick| and the label option
|"bar"| is applied to both of them. While this is the correct and consistent
behaviour, we typically might wish to specify different labels for the edge
going from |b| to |c| and the edge going from |b| to |d|. To achieve this
effect, we can no longer specify the label as part of the options of |--|.
Rather, we must pass the desired label to the nodes |c| and |d|, but we must
somehow also indicate that these options actually ``belong'' to the edge
``leading to'' to nodes. This is achieved by preceding the options with a
greater-than sign:
%
\begin{codeexample}[preamble={\usetikzlibrary{graphs,quotes}}]
\tikz \graph {
  a -> b -- {c [> "foo"], d [> "bar"']};
};
\end{codeexample}

Symmetrically, preceding the options by |<| causes the options and labels to
apply to the ``outgoing'' edges of the node:
%
\begin{codeexample}[preamble={\usetikzlibrary{graphs,quotes}}]
\tikz \graph {
  a [< red] -> b -- {c [> blue], d [> "bar"']};
};
\end{codeexample}

This syntax allows you to easily create trees with special edge labels as in
the following example of a treap:
%
\begin{codeexample}[preamble={\usetikzlibrary{graphs,quotes}}]
\tikz
  \graph [edge quotes={fill=white,inner sep=1pt},
          grow down, branch right, nodes={circle,draw}] {
    "" -> h [>"9"] -> {
      c [>"4"] -> {
        a [>"2"],
        e [>"0"]
      },
      j [>"7"]
    }
  };
\end{codeexample}


\subsubsection{Concept: Node Sets}

When you write down some node text inside a |graph| command, a new node is
created by default unless this node has already been created inside the same
|graph| command. In particular, if a node has already been declared outside of
the current |graph| command, a new node of the same name gets created.

This is not always the desired behaviour. Often, you may wish to make nodes
part of a graph than have already been defined prior to the use of the |graph|
command. For this, simply surround a node name by parentheses. This will cause
a reference to be created to an already existing node:
%
\begin{codeexample}[preamble={\usetikzlibrary{graphs}}]
\tikz {
  \node (a) at (0,0) {A};
  \node (b) at (1,0) {B};
  \node (c) at (2,0) {C};

  \graph { (a) -> (b) -> (c) };
}
\end{codeexample}

You can even go a step further: A whole collection of nodes can all be flagged
to belong to a \emph{node set} by adding the option |set=|\meta{node set name}.
Then, inside a |graph| command, you can collectively refer to these nodes by
surrounding the node set name in parentheses:
%
\begin{codeexample}[preamble={\usetikzlibrary{graphs,shapes.geometric}}]
\tikz [new set=my nodes] {
  \node [set=my nodes, circle,    draw] at (1,1)   {A};
  \node [set=my nodes, rectangle, draw] at (1.5,0) {B};
  \node [set=my nodes, diamond,   draw] at (1,-1)  {C};
  \node (d)           [star,      draw] at (3,0)   {D};

  \graph { X -> (my nodes) -> (d) };
}
\end{codeexample}


\subsubsection{Concept: Graph Macros}

Often, a graph will consist -- at least in parts -- of standard parts. For
instance, a graph might contain a cycle of certain size or a path or a clique.
To facilitate specifying such graphs, you can define a \emph{graph macro}. Once
a graph macro has been defined, you can use the name of the graph to make a
copy of the graph part of the graph currently being specified:
%
\begin{codeexample}[preamble={\usetikzlibrary{graphs.standard}}]
\tikz \graph { subgraph K_n [n=6, clockwise] };
\end{codeexample}

\begin{codeexample}[preamble={\usetikzlibrary{graphs.standard}}]
\tikz \graph { subgraph C_n [n=5, clockwise] -> mid };
\end{codeexample}

The library |graphs.standard| defines a number of such graphs, including the
complete clique $K_n$ on $n$ nodes, the complete bipartite graph $K_{n,m}$ with
shores sized $n$ and $m$, the cycle $C_n$ on $n$ nodes, the path $P_n$ on $n$
nodes, and the independent set $I_n$ on $n$ nodes.


\subsubsection{Concept: Graph Expressions and Color Classes}

When a graph is being constructed using the |graph| command, it is constructed
recursively by uniting smaller graphs to larger graphs. During this recursive
union process the nodes of the graph get implicitly \emph{colored}
(conceptually) and you can also explicitly assign colors to individual nodes
and even change the colors as the graph is being specified. All nodes having
the same color form what is called a \emph{color class}.

The power of color class is that special \emph{connector operators} allow you
to add edges between nodes having certain colors. For instance, saying
|clique=red| at the beginning of a group will cause all nodes that have been
flagged as being (conceptually) ``red'' to be connected as a clique. Similarly,
saying |complete bipartite={red}{green}| will cause edges to be added between
all red and all green nodes. More advanced connectors, like the |butterfly|
connector, allow you to add edges between color classes in a fancy manner.
%
\begin{codeexample}[preamble={\usetikzlibrary{graphs}}]
\tikz [x=8mm, y=6mm, circle]
  \graph [nodes={fill=blue!70}, empty nodes, n=8] {
    subgraph I_n [name=A] --[butterfly={level=4}]
    subgraph I_n [name=B] --[butterfly={level=2}]
    subgraph I_n [name=C] --[butterfly]
    subgraph I_n [name=D] --
    subgraph I_n [name=E]
  };
\end{codeexample}


\subsection{Syntax of the Graph Path Command}

\subsubsection{The Graph Command}

In order to construct a graph, you should use the |graph| path command, which
can be used anywhere on a path at any place where you could also use a command
like, say, |plot| or |--|.

\begin{command}{\graph}
    Inside a |{tikzpicture}| this is an abbreviation for |\path graph|.
\end{command}

\begin{pathoperation}{graph}{\opt{\oarg{options}}\meta{group specification}}
    When this command is encountered on a path, the construction of the current
    path is suspended (similarly to an |edge| command or a |node| command). In
    a local scope, the \meta{options} are first executed with the key path
    |/tikz/graphs| using the following command:
    %
    \begin{command}{\tikzgraphsset\marg{options}}
        Executes the \meta{options} with the path prefix |/tikz/graphs|.
    \end{command}
    %
    Apart from the keys explained in the following, further permissible keys
    will be listed during the course of the rest of this section.

    \begin{stylekey}{/tikz/graphs/every graph}
        This style is executed at the beginning of every |graph| path command
        prior to the \meta{options}.
    \end{stylekey}

    Once the scope has been set up and once the \meta{options} have been
    executed, a parser starts to parse the \meta{group specification}. The
    exact syntax of such a group specification in explained in detail in
    Section~\ref{section-library-graphs-group-spec}. Basically, a group
    specification is a list of chain specifications, separated by commas or
    semicolons.

    Depending on the content of the \meta{group specification}, two things will
    happen:
    %
    \begin{enumerate}
        \item A number of new nodes may be created. These will be inserted into
            the picture in the same order as if they had been created using
            multiple |node| path commands at the place where the |graph| path
            command was used. In other words, all nodes created in a |graph|
            path command will be painted on top of any nodes created earlier in
            the path and behind any nodes created later in the path. Like
            normal nodes, the newly created nodes always lie on top of the path
            that is currently being created (which is often empty, for instance
            when the |\graph| command is used).
        \item Edges between the nodes may be added. They are added in the same
            order as if the |edge| command had been used at the position where
            the |graph| command is being used.
    \end{enumerate}

    Let us now have a look at some common keys that may be used inside the
    \meta{options}:
    %
    \begin{key}{/tikz/graphs/nodes=\meta{options}}
        This option causes the \meta{options} to be applied to each newly
        created node inside the \meta{group specification}.
        %
\begin{codeexample}[preamble={\usetikzlibrary{graphs}}]
\tikz \graph [nodes=red] { a -> b -> c };
\end{codeexample}
        %
        Multiple uses of this key accumulate.
    \end{key}
    %
    \begin{key}{/tikz/graphs/edges=\meta{options}}
        This option causes the \meta{options} to be applied to each newly
        created edge inside the \meta{group specification}.
        %
\begin{codeexample}[preamble={\usetikzlibrary{graphs}}]
\tikz \graph [edges={red,thick}] { a -> b -> c };
\end{codeexample}
        %
        Again, multiple uses of this key accumulate.
    \end{key}
    %
    \begin{key}{/tikz/graphs/edge=\meta{options}}
        This is an alias for |edges|.
    \end{key}

    \begin{key}{/tikz/graphs/edge node=\meta{node specification}}
        This key specifies that the \meta{node specification} should be added
        to each newly created edge as an implicitly placed node.
        %
\begin{codeexample}[preamble={\usetikzlibrary{graphs}}]
\tikz \graph [edge node={node [red, near end] {X}}] { a -> b -> c };
\end{codeexample}
        %
        Again, multiple uses of this key accumulate.
        %
\begin{codeexample}[preamble={\usetikzlibrary{graphs}}]
\tikz \graph [edge node={node [near end] {X}},
              edge node={node [near start] {Y}}] { a -> b -> c };
\end{codeexample}
    \end{key}

    \begin{key}{/tikz/graphs/edge label=\meta{text}}
        This key is an abbreviation for |edge node=node[auto]{|\meta{text}|}|.
        The net effect is that the |text| is placed next to the newly created
        edges.
        %
\begin{codeexample}[preamble={\usetikzlibrary{graphs}}]
\tikz \graph [edge label=x] { a -> b -> {c,d} };
\end{codeexample}
    \end{key}

    \begin{key}{/tikz/graphs/edge label'=\meta{text}}
        This key is an abbreviation for |edge node=node[auto,swap]{|\meta{text}|}|.
        %
\begin{codeexample}[preamble={\usetikzlibrary{graphs.standard}}]
\tikz \graph [edge label=out, edge label'=in]
  { subgraph C_n [clockwise, n=5] };
\end{codeexample}
    \end{key}
\end{pathoperation}


\subsubsection{Syntax of Group Specifications}
\label{section-library-graphs-group-spec}

A \meta{group specification} inside a |graph| path command has the following
syntax:
%
\begin{quote}
    |{|\opt{\oarg{options}}\meta{list of chain specifications}|}|
\end{quote}
%
The \meta{chain specifications} must contain chain specifications, whose syntax
is detailed in the next section, separated by either commas or semicolons; you
can freely mix them. It is permissible to use empty lines (which are mapped to
|\par| commands internally) to structure the chains visually, they are simply
ignored by the parser.

In the following example, the group specification consists of three chain
specifications, namely of |a -> b|, then |c| alone, and finally |d -> e -> f|:
%
\begin{codeexample}[preamble={\usetikzlibrary{graphs}}]
\tikz \graph {
  a -> b,
  c;

  d -> e -> f
};
\end{codeexample}
%
The above has the same effect as the more compact group specification
|{a->b,c,d->e->f}|.

Commas are used to detect where chain specifications end. However, you will
often wish to use a comma also inside the options of a single node like in the
following example:
%
\begin{codeexample}[preamble={\usetikzlibrary{graphs}}]
\tikz \graph {
  a [red, draw] -> b [blue, draw],
  c [brown, draw, circle]
};
\end{codeexample}

Note that the above example works as expected: The first comma inside the
option list of |a| is \emph{not} interpreted as the end of the chain
specification ``|a [red|''. Rather, commas inside square brackets are
``protected'' against being interpreted as separators of group specifications.

The \meta{options} that can be given at the beginning of a group specification
are local to the group. They are executed with the path prefix |/tikz/graphs|.
Note that for the outermost group specification of a graph it makes no
difference whether the options are passed to the |graph| command or whether
they are given at the beginning of this group. However, for groups nested
inside other groups, it does make a difference:
%
\begin{codeexample}[preamble={\usetikzlibrary{graphs}}]
\tikz \graph {
  a -> { [nodes=red] % the option is local to these nodes:
    b, c
  } ->
  d
};
\end{codeexample}

\medskip
\textbf{Using foreach.}
There is special support for the |\foreach| statement inside groups: You may
use the statement inside a group specification at any place where a \meta{chain
specification} would normally go. In this case, the |\foreach| statement is
executed and for each iteration the content of the statement's body is treated
and parsed as a new chain specification.
%
\begin{codeexample}[preamble={\usetikzlibrary{graphs}}]
\tikz \graph [math nodes, branch down=5mm] {
  a -> {
    \foreach \i in {1,2,3} {
      a_\i -> { x_\i, y_\i }
    },
    b
  }
};
\end{codeexample}

\medskip
\textbf{Using macros.}
In some cases you may wish to use macros and \TeX\ code to compute which nodes
and edges are present in a group. You cannot use macros in the normal way
inside a graph specification since the parser does not expand macros as it
scans for the start and end of groups and node names. Rather, only after
commas, semicolons, and hyphens have already been detected and only after all
other parsing decisions have been made will macros be expanded. At this point,
when a macro expands to, say |a,b|, this will not result in two nodes to be
created since the parsing is already done. For these reasons, a special key is
needed to make it possible to ``compute'' which nodes should be present in a
group.

\begin{key}{/tikz/graph/parse=\meta{text}}
    This key can only be used inside the \meta{options} of a \meta{group
    specification}. Its effect is that the \meta{text} is inserted at the
    beginning of the current group as if you had entered it there. Naturally,
    it makes little sense to just write down some static \meta{text} since you
    could just as well directly place it at the beginning of the group. The
    real power of this command stems from the fact that the keys mechanism
    allows you to say, for instance, |parse/.expand once| to insert the text
    stored in some macro into the group.
    %
\begin{codeexample}[preamble={\usetikzlibrary{graphs}}]
\def\mychain{ a -> b -> c; }
\tikz \graph { [parse/.expand once=\mychain] d -> e };
\end{codeexample}
    %
    In the following, more fancy example we use a loop to create a chain of
    dynamic length.
    %
\begin{codeexample}[preamble={\usetikzlibrary{graphs}}]
\def\mychain#1{
  \def\mytext{1}
  \foreach \i in {2,...,#1} {
    \xdef\mytext{\mytext -> \i}
  }
}
\tikzgraphsset{my chain/.style={
    /utils/exec=\mychain{#1},
    parse/.expand once=\mytext}
}
\tikz \graph { [my chain=4] };
\end{codeexample}
    %
    Multiple uses of this key accumulate, that is, all the \text{text}s given
    in the different uses is inserted in the order it is given.
\end{key}


\subsubsection{Syntax of Chain Specifications}

A \meta{chain specification} has the following syntax: It consists of a
sequence of \meta{node specifications}, where subsequent node specifications
are separated by \meta{edge specifications}. Node specifications, which
typically consist of some text, are discussed in the next section in more
detail. They normally represent a single node that is either newly created or
exists already, but they may also specify a whole set of nodes.

An \meta{edge specification} specifies \emph{which} of the node(s) to the left
of the edge specification should be connected to which node(s) to the right of
it and it also specifies in which direction the connections go. In the
following, we only discuss how the direction is chosen, the powerful mechanism
behind choosing which nodes should be connect is detailed in
Section~\ref{section-library-graphs-color-classes}.

The syntax of an edge specification is always one of the following five
possibilities:
%
\begin{quote}
    |->| \opt{\oarg{options}}\\
    |--| \opt{\oarg{options}}\\
    |<-| \opt{\oarg{options}}\\
    |<->| \opt{\oarg{options}}\\
    |-!-| \opt{\oarg{options}}
\end{quote}

The first four correspond to a directed edge, an undirected edge, a
``backward'' directed edge, and a bidirected edge, respectively. The fifth edge
specification means that there should be no edge (this specification can be
used together with the |simple| option to remove edges that have previously
been added, see Section~\ref{section-library-graphs-simple}).

Suppose the nodes \meta{left nodes} are to the left of the \meta{edge
specification} and \meta{right nodes} are to the right and suppose we have
written |->| between them. Then the following happens:
%
\begin{enumerate}
    \item The \meta{options} are executed (inside a local scope) with the path
        |/tikz/graphs|.  These options may setup the connector algorithm (see
        below) and may also use keys like |edge| or |edge label| to specify how
        the edge should look like. As a convenience, whenever an unknown key is
        encountered for the path |/tikz/graphs|, the key is passed to the
        |edge| key. This means that you can directly use options like |thick|
        or |red| inside the \meta{options} and they will apply to the edge as
        expected.
    \item The chosen connector algorithm, see
        Section~\ref{section-library-graphs-color-classes}, is used to compute
        from which of the \meta{left nodes} an edge should lead to which of the
        \meta{right nodes}. Suppose that $(l_1,r_1)$, \dots, $(l_n,r_n)$ is the
        list of node pairs that result (so there should be an edge between
        $l_1$ and $r_1$ and another edge between $l_2$ and $r_2$ and so on).
    \item For each pair $(l_i,r_i)$ an edge is created. This is done by calling
        the following key (for the edge specification |->|, other keys are
        executed for the other kinds of specifications):
        %
        \begin{key}{/tikz/graphs/new ->=\marg{left node}\marg{right node}\marg{edge options}\marg{edge nodes}}
            This key will be called for a |->| edge specification with the
            following four parameters:
            %
            \begin{enumerate}
                \item \meta{left node} is the name of the ``left'' node, that
                    is, the name of $l_i$.
                \item \meta{right node} is the name of the right node.
                \item \meta{edge options} are the accumulated options from all
                    calls of |/tikz/graph/edges| in groups that surround the
                    edge specification.
                \item \meta{edge nodes} is text like |node {A} node {B}| that
                    specifies some nodes that should be put as labels on the
                    edge using \tikzname's implicit positioning mechanism.
            \end{enumerate}
            %
            By default, the key executes the following code:
            %
            \begin{quote}
                |\path [->,every new ->]|\\
                \hbox{}\quad|(|\meta{left node}|\tikzgraphleftanchor) edge [|%
                \meta{edge options}|]| \meta{edge nodes}||\\
                \hbox{}\quad|(|\meta{right node}|\tikzgraphrightanchor);|
            \end{quote}
            %
            You are welcome to change the code underlying the key.
            %
            \begin{stylekey}{/tikz/every new ->}
                This key gets executed by default for a |new ->|.
            \end{stylekey}
        \end{key}
        %
        \begin{key}{/tikz/graphs/left anchor=\meta{anchor}}
            This anchor is used for the node that is to the left of an edge
            specification. Setting this anchor to the empty string means that
            no special anchor is used (which is the default). The \meta{anchor}
            is stored in the macro |\tikzgraphleftanchor| with a leading dot.
            %
\begin{codeexample}[preamble={\usetikzlibrary{graphs}}]
\tikz \graph {
  {a,b,c} -> [complete bipartite] {e,f,g}
};
\end{codeexample}
            %
\begin{codeexample}[preamble={\usetikzlibrary{graphs}}]
\tikz \graph [left anchor=east, right anchor=west] {
  {a,b,c} -- [complete bipartite] {e,f,g}
};
\end{codeexample}
        \end{key}
        %
        \begin{key}{/tikz/graphs/right anchor=\meta{anchor}}
            Works like |left anchor|, only for |\tikzgraphrightanchor|.
        \end{key}
        %
        For the other three kinds of edge specifications, the following keys
        will be called:
        %
        \begin{key}{/tikz/graphs/new --=\marg{left node}\marg{right node}\marg{edge options}\marg{edge nodes}}
            This key is called for |--| with the same parameters as above. The
            only difference in the definition is that in the |\path| command
            the |->| gets replaced by |-|.
            %
            \begin{stylekey}{/tikz/every new --}
            \end{stylekey}
        \end{key}
        %
        \begin{key}{/tikz/graphs/new <->=\marg{left node}\marg{right node}\marg{edge options}\marg{edge nodes}}
            Called for |<->| with the same parameters as above. The |->| is
            replaced by |<-|
            %
            \begin{stylekey}{/tikz/every new <->}
            \end{stylekey}
        \end{key}
        %
        \begin{key}{/tikz/graphs/new <-=\marg{left node}\marg{right node}\marg{edge options}\marg{edge nodes}}
            Called for |<-| with the same parameters as above.%
            \footnote{%
                You might wonder why this key is needed: It seems more logical
                at first sight to just call |new edge directed| with swapped
                first parameters. However, a positioning algorithm might wish
                to take the fact into account that an edge is ``backward''
                rather than ``forward'' in order to improve the layout. Also,
                different arrow heads might be used.
            }
            %
            \begin{stylekey}{/tikz/every new <-}
            \end{stylekey}
        \end{key}
        %
        \begin{key}{/tikz/graphs/new -\protect\exclamationmarktext-=\marg{left node}\marg{right node}\marg{edge options}\marg{edge nodes}}
            Called for |-!-| with the same parameters as above. Does nothing by
            default.
        \end{key}
\end{enumerate}

Here is an example that shows the default rendering of the different edge
specifications:
%
\begin{codeexample}[preamble={\usetikzlibrary{graphs}}]
\tikz \graph [branch down=5mm] {
  a -> b;
  c -- d;
  e <- f;
  g <-> h;
  i -!- j;
};
\end{codeexample}


\subsubsection{Syntax of Node Specifications}
\label{section-library-graphs-node-spec}

Node specifications are the basic building blocks of a graph specification.
There are three different possible kinds of node specifications, each of which
has a different syntax:
%
\begin{description}
    \item[Direct Node Specification]
        \ \\
        \opt{|"|}\meta{node name}\opt{|"|}\opt{|/|\opt{|"|}\meta{text}\opt{|"|}} \opt{\oarg{options}}\\
        (note that the quotation marks are optional and only needed when the
        \meta{node name} contains special symbols)
    \item[Reference Node Specification]
        \ \\
        |(|\meta{node name or node set name}|)|
    \item[Group Node Specification]
        \ \\
        \meta{group specification}
\end{description}

The rule for determining which of the possible kinds is meant is as follows: If
the node specification starts with an opening parenthesis, a reference node
specification is meant; if it starts with an opening curly brace, a group
specification is meant; and in all other cases a direct node specification is
meant.

\medskip
\textbf{Direct Node Specifications.} If after reading the first symbol of a
node specification is has been detected to be \emph{direct}, \tikzname\ will
collect all text up to the next edge specification and store it as the
\meta{node name}; however, square brackets are used to indicate options and a
slash ends the \meta{node name} and start a special \meta{text} that is used as
a rendering text instead of the original \meta{node name}.

Due to the way the parsing works and due to the restrictions on node names,
most special characters are forbidding inside the \meta{node name}, including
commas, semicolons, hyphens, braces, dots, parentheses, slashes, dashes, and
more (but spaces, single underscores, and the hat character \emph{are}
allowed). To use special characters in the name of a node, you can optionally
surround the \meta{node name} and/or the \meta{text} by quotation marks. In
this case, you can use all of the special symbols once more. The details of
what happens, exactly, when the \meta{node name} is surrounded by quotation
marks is explained later; surrounding the \meta{text} by quotation marks has
essentially the same effect as surrounding it by curly braces.

Once the node name has been determined, it is checked whether the same node
name was already used inside the current graph. If this is the case, then we
say that the already existing node is \emph{referenced}; otherwise we say that
the node is \emph{fresh}.
%
\begin{codeexample}[preamble={\usetikzlibrary{graphs}}]
\tikz \graph {
  a -> b; % both are fresh
  c -> a; % only c is fresh, a is referenced
};
\end{codeexample}

This behaviour of deciding whether a node is fresh or referenced can, however,
be modified by using the following keys:
%
\begin{key}{/tikz/graphs/use existing nodes=\opt{\meta{true or false}} (default true)}
    When this key is set to |true|, all nodes will be considered to the
    referenced, no node will be fresh. This option is useful if you have
    already created all the nodes of a graph prior to using the |graph| command
    and you now only wish to connect the nodes. It also implies that an error
    is raised if you reference a node which has not been defined previously.
\end{key}

\begin{key}{/tikz/graphs/fresh nodes=\opt{\meta{true or false}} (default true)}
    When this key is set to |true|, all nodes will be considered to be fresh.
    This option is useful when you create for instance a tree with many
    identical nodes.

    When a node name is encountered that was already used previously, a new
    name is chosen is follows: An apostrophe (|'|) is appended repeatedly until
    a node name is found that has not yet been used:
    %
\begin{codeexample}[preamble={\usetikzlibrary{graphs}}]
\tikz \graph [branch down=5mm] {
  { [fresh nodes]
    a -> {
      b -> {c, c},
      b -> {c, c},
      b -> {c, c},
    }
  },
  b' -- b''
};
\end{codeexample}
    %
\end{key}

\begin{key}{/tikz/graphs/number nodes=\opt{\meta{start number}} (default 1)}
    When this key is used in a scope, each encountered node name will get
    appended a new number, starting with \meta{start}. Typically, this ensures
    that all node names are different. Between the original node name and the
    appended number, the setting of the following will be inserted:
    %
    \begin{key}{/tikz/graphs/number nodes sep=\meta{text} (initially \normalfont space)}
    \end{key}
    %
\begin{codeexample}[preamble={\usetikzlibrary{graphs}}]
\tikz \graph [branch down=5mm] {
  { [number nodes]
    a -> {
      b -> {c, c},
      b -> {c, c},
      b -> {c, c},
    }
  },
  b 2 -- b 5
};
\end{codeexample}
    %
\end{key}

When a fresh node has been detected, a new node is created in the inside a
protecting scope. For this, the current placement strategy is asked to compute
a default position for the node, see
Section~\ref{section-library-graphs-placement} for details. Then, the command
%
\begin{quote}
    |\node (|\meta{full node name}|) [|\meta{node options}|] {|\meta{text}|};|
\end{quote}
%
is called. The different parameters are as follows:
%
\begin{itemize}
    \item The \meta{full node name} is normally the \meta{node name} that has
        been determined as described before. However, there are two exceptions:

        First, if the \meta{node name} is empty (which happens when there is no
        \meta{node name} before the slash), then a fresh internal node name is
        created and used as \meta{full node name}. This name is guaranteed to
        be different from all node names used in this or any other graph. Thus,
        a direct node starting with a slash represents an anonymous fresh node.

        Second, you can use the following key to prefix the \meta{node name}
        inside the \meta{full node name}:

        \begin{key}{/tikz/graphs/name=\meta{text}}
            This key prepends the \meta{text}, followed by a separating symbol
            (a space by default), to all \meta{node name}s inside a \meta{full
            node name}. Repeated calls of this key accumulate, leading to
            ever-longer ``name paths'':
            %
\begin{codeexample}[preamble={\usetikzlibrary{graphs}}]
\begin{tikzpicture}
  \graph {
    { [name=first]  1, 2, 3} --
    { [name=second] 1, 2, 3}
  };
  \draw [red] (second 1) circle [radius=3mm];
\end{tikzpicture}
\end{codeexample}
            %
            Note that, indeed, in the above example six nodes are created even
            though the first and second set of nodes have the same \meta{node
            name}. The reason is that the full names of the six nodes are all
            different. Also note that only the \meta{node name} is used as the
            node text, not the full name. This can be changed as described
            later on.

            This key can be used repeatedly, leading to ever longer node names.
        \end{key}

        \begin{key}{/tikz/graphs/name separator=\meta{symbols} (initially \string\space)}
            Changes the symbol that is used to separate the \meta{text} from
            the \meta{node name}. The default is |\space|, resulting in a
            space.
            %
\begin{codeexample}[preamble={\usetikzlibrary{graphs}}]
\begin{tikzpicture}
  \graph [name separator=] { % no separator
    { [name=first]  1, 2, 3} --
    { [name=second] 1, 2, 3}
  };
  \draw [red] (second1) circle [radius=3mm];
\end{tikzpicture}
\end{codeexample}
            %
\begin{codeexample}[preamble={\usetikzlibrary{graphs}}]
\begin{tikzpicture}
  \graph [name separator=-] {
    { [name=first]  1, 2, 3} --
    { [name=second] 1, 2, 3}
  };
  \draw [red] (second-1) circle [radius=3mm];
\end{tikzpicture}
\end{codeexample}
        \end{key}
    \item The \meta{node options} are
        %
        \begin{enumerate}
            \item The options that have accumulated in calls to |nodes| from
                the surrounding scopes.
            \item The local \meta{options}.
        \end{enumerate}
        %
        The options are executed with the path prefix |/tikz/graphs|, but any
        unknown key is executed with the prefix |/tikz|. This means, in
        essence, that some esoteric keys are more difficult to use inside the
        options and that any key with the prefix |/tikz/graphs| will take
        precedence over a key with the prefix |/tikz|.
    \item The \meta{text} that is passed to the |\node| command is computed as
        follows: First, you can use the following key to directly set the
        \meta{text}:
        %
        \begin{key}{/tikz/graphs/as=\meta{text}}
            The \meta{text} is used as the text of the node. This allows you to
            provide a text for the node that differs arbitrarily from the name
            of the node.
            %
\begin{codeexample}[preamble={\usetikzlibrary{graphs}}]
\tikz \graph { a [as=$x$] -- b [as=$y_5$] -> c [red, as={a--b}] };
\end{codeexample}
            %
            This key always takes precedence over all of the mechanisms
            described below.
        \end{key}
        %
        In case the |as| key is not used, a default text is chosen as follows:
        First, when a direct node specification contains a slash (or, for
        historical reasons, a double underscore), the text to the right of the
        slash (or double underscore) is stored in the macro
        |\tikzgraphnodetext|; if there is no slash, the \meta{node name} is
        stored in |\tikzgraphnodetext|, instead. Then, the current value of the
        following key is used as \meta{text}:
        %
        \begin{key}{/tikz/graphs/typeset=\meta{code}}
            The macro or code stored in this key is used as the \meta{text} if
            the node. Inside the \meta{code}, the following macros are
            available:
            %
            \begin{command}{\tikzgraphnodetext}
                This macro expands to the \meta{text} to the right of the
                double underscore or slash in a direct node specification or,
                if there is no slash, to the \meta{node name}.
            \end{command}
            %
            \begin{command}{\tikzgraphnodename}
                This macro expands to the name of the current node without the
                path.
            \end{command}
            %
            \begin{command}{\tikzgraphnodepath}
                This macro expands to the current path of the node. These paths
                result from the use of the |name| key as described above.
            \end{command}
            %
            \begin{command}{\tikzgraphnodefullname}
                This macro contains the concatenation of the above two.
            \end{command}
        \end{key}
        %
        By default, the typesetter is just set to |\tikzgraphnodetext|, which
        means that the default text of a node is its name. However, it may be
        useful to change this: For instance, you might wish that the text of
        all graph nodes is, say, surrounded by parentheses:
        %
\begin{codeexample}[preamble={\usetikzlibrary{graphs}}]
\tikz \graph [typeset=(\tikzgraphnodetext)]
  { a -> b -> c };
\end{codeexample}
        %
        A more advanced macro might take apart the node text and render it
        differently:
        %
\begin{codeexample}[preamble={\usetikzlibrary{graphs}}]
\def\mytypesetter{\expandafter\myparser\tikzgraphnodetext\relax}
\def\myparser#1 #2 #3\relax{%
  $#1_{#2,\dots,#3}$
}
\tikz \graph [typeset=\mytypesetter, grow down]
  { a 1 n -> b 2 m -> c 4 nm };
\end{codeexample}
        %
        The following styles install useful predefined typesetting macros:
        %
        \begin{key}{/tikz/graphs/empty nodes}
            Just sets |typeset| to nothing, which causes all nodes to have an
            empty text (unless, of course, the |as| option is used):
            %
\begin{codeexample}[preamble={\usetikzlibrary{graphs}}]
\tikz \graph [empty nodes, nodes={circle, draw}] { a -> {b, c} };
\end{codeexample}
        \end{key}
        %
        \begin{key}{/tikz/graphs/math nodes}
            Sets |typeset| to |$\tikzgraphnodetext$|, which causes all nodes
            names to be typeset in math mode:
            %
\begin{codeexample}[preamble={\usetikzlibrary{graphs}}]
\tikz \graph [math nodes, nodes={circle, draw}] { a_1 -> {b^2, c_3^n} };
\end{codeexample}
        \end{key}
\end{itemize}

If a node is referenced instead of fresh, then this node becomes the node that
will be connected by the preceding or following edge specification to other
nodes. The \meta{options} are executed even for a referenced node, but they
cannot be used to change the appearance of the node (because the node exists
already). Rather, the \meta{options} can only be used to change the logical
coloring of the node, see Section~\ref{section-library-graphs-color-classes}
for details.

\medskip
\textbf{Quoted Node Names.} When the \meta{node name} and/or the \meta{text} of
a node is surrounded by quotation marks, you can use all sorts of special
symbols as part of the text that are normally forbidden:
%
\begin{codeexample}[preamble={\usetikzlibrary{graphs}}]
\begin{tikzpicture}
  \graph [grow right=2cm] {
    "Hi, World!"       -> "It's \emph{important}!"[red,rotate=-45];
    "name"/actual text -> "It's \emph{important}!";
  };
  \draw (name) circle [radius=3pt];
\end{tikzpicture}
\end{codeexample}

In detail, for the following happens when quotation marks are encountered at
the beginning of a node name or its text:
%
\begin{itemize}
    \item Everything following the quotation mark up to the next single
        quotation mark is collected into a macro \meta{collected}. All sorts of
        special characters, including commas, square brackets, dashes, and even
        backslashes are allowed here. Basically, the only restriction is that
        braces must be balanced.
    \item A double quotation mark (|""|) does not count as the ``next single
        quotation mark''. Rather, it is replaced by a single quotation mark.
        For instance, |"He said, ""Hello world."""| would be stored inside
        \meta{collected} as |He said, "Hello world."| However, this rule
        applies only on the outer-most level of braces. Thus, in
        %
\begin{codeexample}[code only]
"He {said, ""Hello world.""}"
\end{codeexample}
        %
        we would get |He {said, ""Hello world.""}| as \meta{collected}.
    \item ``The next single quotation mark'' refers to the next quotation mark
        on the current level of braces, so in |"hello {"} world"|, the next
        quotation mark would be the one following |world|.
\end{itemize}

Now, once the \meta{collected} text has been gather, it is used as follows:
When used as \meta{text} (what is actually displayed), it is just used ``as
is''. When it is used as \meta{node name}, however, the following happens:
Every ``special character'' in \meta{collected} is replaced by its Unicode
name, surrounded by |@|-signs. For instance, if \meta{collected} is
|Hello, world!|, the \meta{node name} is the somewhat longer text
|Hello@COMMA@ world@EXCLAMATION MARK@|. Admittedly, referencing such a node
from outside the graph is cumbersome, but when you use exactly the same
\meta{collected} text once more, the same \meta{node name} will result. The
following characters are considered ``special'':
%
\begin{quote}
    \texttt{\char`\|}|$&^~_[](){}/.-,+*'`!":;<=>?@#%\{}|%$
\end{quote}
%
These are exactly the Unicode character with a decimal code number between 33
and 126 that are neither digits nor letters.

\medskip
\textbf{Reference Node Specifications.} A reference node specification is a
node specification that starts with an opening parenthesis. In this case,
parentheses must surround a \meta{name} as in |(foo)|, where |foo| is the
\meta{name}. The following will now happen:
%
\begin{enumerate}
    \item It is tested whether \meta{name} is the name of a currently active
        \emph{node set}. This case will be discussed in a moment.
    \item Otherwise, the \meta{name} is interpreted and treated as a referenced
        node, but independently of whether the node has already been fresh in
        the current graph or not. In other words, the node must have been
        defined either already inside the graph (in which case the parenthesis
        are more or less superfluous) or it must have been defined outside the
        current picture.

        The way the referenced node is handled is the same way as for a direct
        node that is a referenced node.

        If the node does not already exist, an error message is printed.
\end{enumerate}

Let us now have a look at node sets. Inside a |{tikzpicture}| you can locally
define a \emph{node set} by using the following key:
%
\begin{key}{/tikz/new set=\meta{set name}}
    This will setup a node set named \meta{set name} within the current scope.
    Inside the scope, you can add nodes to the node set using the |set| key. If
    a node set of the same name already exists in the current scope, it will be
    reset and made empty for the current scope.

    Note that this command has the path |/tikz| and is normally used
    \emph{outside} the |graph| command.
\end{key}
%
\begin{key}{/tikz/set=\meta{set name}}
    This key can be used as an option with a |node| command. The \meta{set
    name} must be the name of a node set that has previously been created
    inside some enclosing scope via the |new set| key. The effect is that the
    current node is added to the node set.
\end{key}

When you use a |graph| command inside a scope where some node set called
\meta{set name} is defined, then inside this |graph| command you use
|(|\meta{set name}|)| to reference \emph{all} of the nodes in the node set. The
effect is the same as if instead of the reference to the set name you had
created a group specification containing a list of references to all the nodes
that are part of the node set.
%
\begin{codeexample}[preamble={\usetikzlibrary{graphs}}]
\begin{tikzpicture}[new set=red, new set=green, shorten >=2pt]
  \foreach \i in {1,2,3} {
    \node [draw, red!80,         set=red]   (r\i) at (\i,1) {$r_\i$};
    \node [draw, green!50!black, set=green] (g\i) at (\i,2) {$g_\i$};
  }
  \graph {
    root [xshift=2cm] ->
    (red)             -> [complete bipartite, right anchor=south]
    (green)
  };
\end{tikzpicture}
\end{codeexample}

There is an interesting caveat with referencing node sets: Suppose that at the
beginning of a graph you just say |(foo);| where |foo| is a set name. Unless
you have specified special options, this will cause the following to happen: A
group is created whose members are all the nodes of the node set |foo|. These
nodes become referenced nodes, but otherwise nothing happens since, by default,
the nodes of a group are not connected automatically. However, the referenced
nodes have now been referenced inside the graph, you can thus subsequently
access them as if they had been defined inside the graph. Here is an example
showing how you can create nodes outside a |graph| command and then connect
them inside as if they had been declared inside:
%
\begin{codeexample}[preamble={\usetikzlibrary{graphs}}]
\begin{tikzpicture}[new set=import nodes]
  \begin{scope}[nodes={set=import nodes}] % make all nodes part of this set
    \node [red] (a) at (0,1) {$a$};
    \node [red] (b) at (1,1) {$b$};
    \node [red] (d) at (2,1) {$d$};
  \end{scope}

  \graph {
    (import nodes);         % "import" the nodes

    a -> b -> c -> d -> e;  % only c and e are new
  };
\end{tikzpicture}
\end{codeexample}

\medskip
\textbf{Group Node Specifications.} At a place where a node specification
should go, you can also instead provide a group specification. Since nodes
specifications are part of chain specifications, which in turn are part of
group specifications, this is a recursive definition.
%
\begin{codeexample}[preamble={\usetikzlibrary{graphs}}]
\tikz \graph { a -> {b,c,d} -> {e -> {f,g}, h} };
\end{codeexample}

As can be seen in the above example, when two groups of nodes are connected via
an edge specification, it is not immediately obvious which connecting edges are
added. This is detailed in Section~\ref{section-library-graphs-color-classes}.


\subsubsection{Specifying Tries}

In computer science, a \emph{trie} is a special kind of tree, where for each
node and each symbol of an alphabet, there is at most one child of the node
labeled with this symbol.

The |trie| key is useful for drawing tries, but it can also be used in other
situations. What it does, essentially, is to prepend the node names of all
nodes \emph{before} the current node of the current chain to the node's name.
This will often make it easier or more natural to specify graphs in which
several nodes have the same label.

\begin{key}{/tikz/graphs/trie=\opt{\meta{true or false}} (default true, initially false)}
    If this key is set to |true|, after a node has been created on a chain, the
    |name| key is executed with the node's \meta{node name}. Thus, all nodes
    later on this chain have the ``path'' of nodes leading to this node as
    their name. This means, in particular, that
    %
    \begin{enumerate}
        \item two nodes of the same name but in different parts of a chain will
            be different,
        \item while if another chain starts with the same nodes, no new nodes
            get created.
    \end{enumerate}
    %
    In total, this is exactly the behaviour you would expect of a trie:
    %
\begin{codeexample}[preamble={\usetikzlibrary{graphs}}]
\tikz \graph [trie] {
  a -> {
    a,
    c -> {a, b},
    b
  }
};
\end{codeexample}
    %
    You can even ``reiterate'' over a path in conjunction with the |simple|
    option. However, in this case, the default placement strategies will not
    work and you will need options like |layered layout| from the graph drawing
    libraries, which need Lua\TeX.
    %
\ifluatex
\begin{codeexample}[preamble={\usetikzlibrary{graphs,graphdrawing}\usegdlibrary{layered}}]
\tikz \graph [trie, simple, layered layout] {
  a -> b -> a,
  a -> b -> c,
  a -> {d,a}
};
\end{codeexample}
    %
    In the following example, we setup the |typeset| key so that it shows the
    complete names of the nodes:
    %
\begin{codeexample}[preamble={\usetikzlibrary{graphs,graphdrawing}\usegdlibrary{layered}}]
\tikz \graph [trie, simple, layered layout,
              typeset=\tikzgraphnodefullname] {
  a -> b -> a,
  a -> b -> c,
  a -> {d,a}
};
\end{codeexample}
\fi
    %
    You can also use the |trie| key locally and later reference nodes using
    their full name:
    %
\begin{codeexample}[preamble={\usetikzlibrary{graphs}}]
\tikz \graph {
  { [trie, simple]
    a -> {
      b,
      c -> a
    }
  },
  a b ->[red] a c a
};
\end{codeexample}
    %
\end{key}


\subsection{Quick Graphs}
\label{section-library-graphs-quick}

The graph syntax is powerful, but this power comes at a price: parsing the
graph syntax, which is done by \TeX, can take some time. Normally, the parsing
is fast enough that you will not notice it, but it can be bothersome when you
have graphs with hundreds of nodes as happens frequently when nodes are
generated algorithmically by some other program. Fortunately, when another
program generated a graph specification, we typically do not need the full
power of the graph syntax. Rather, a small subset of the graph syntax would
suffice that allows to specify nodes and edges. For these reasons, the is a
special ``quick'' version of the graph syntax.

Note, however, that using this syntax will usually at most halve the time
needed to parse a graph. Thus, it really mostly makes sense in conjunction with
large, algorithmically generated graphs.

\begin{key}{/tikz/graphs/quick}
    When you provide this key with a graph, the syntax of graph specifications
    gets restricted. You are no longer allowed to use certain features of the
    graph syntax; but all features that are still allowed are also allowed in
    the same way when you do not provide the |quick| option. Thus, leaving out
    the |quick| option will never hurt.

    Since the syntax is so severely restricted, it is easier to explain which
    aspects of the graph syntax \emph{will} still work:
    %
    \begin{enumerate}
        \item A quick graph consists of a sequence of either nodes, edges
            sequences, or groups. These are separated by commas or semicolons.
        \item Every node is of the form
            %
            \begin{quote}
                |"|\meta{node name}|"|\opt{|/"|\meta{node text}|"[|\meta{options}|]|}
            \end{quote}

            The quotation marks are mandatory. The part |/"|\meta{node text}|"|
            may  be missing, in which case the node name is used as the node
            text. The \meta{options} may also be missing. The \meta{node name}
            may not contain any ``funny'' characters (unlike in the normal
            graph command).
        \item Every chain is of the form
            %
            \begin{quote}
                \meta{node spec} \meta{connector} \meta{node spec}
                \meta{connector} \dots \meta{connector} \meta{node spec}|;|
            \end{quote}

            Here, the \meta{node spec} are node specifications as described
            above, the \meta{connector} is one of the four connectors |->|,
            |<-|, |--|, and |<->| (the connector |-!-| is not allowed since the
            |simple| option is also not allowed). Each connector may be
            followed by options in square brackets. The semicolon may be
            replaced by a comma.
        \item Every group is of the form
            %
            \begin{quote}
                |{ [|\meta{options}|]| \meta{chains and groups} |};|
            \end{quote}
            %
            The \meta{options} are compulsory. The semicolon can, again, be
            replaced by a comma.
        \item The |number nodes| option will work as expected.
    \end{enumerate}

    Here is a typical way this syntax might be used:
    %
\begin{codeexample}[preamble={\usetikzlibrary{graphs,quotes}}]
\tikz \graph [quick] { "a" --["foo"] "b"[x=1] };
\end{codeexample}

\begin{codeexample}[preamble={\usetikzlibrary{graphs}}]
\tikz \graph [quick] {
  "a"/"$a$" -- "b"[x=1] --[red] "c"[x=2];
  { [nodes=blue] "a" -- "d"[y=1]; };
};
\end{codeexample}

    Let us now have a look at the most important things that will \emph{not}
    work when the |quick| option is used:

    \begin{itemize}
        \item Connecting a node and a group as in |a->{b,c}|.
        \item Node names without quotation marks as in |a--b|.
        \item Everything described in subsequent subsections, which includes
            subgraphs (graph macros), graph sets, graph color classes,
            anonymous nodes, the |fresh nodes| option, sublayouts, simple
            graphs, edge annotations.
        \item Placement strategies -- you either have to define all node
            positions explicitly using |at=| or |x=| and |y=| or you must use a
            graph drawing algorithm like |layered layout|.
    \end{itemize}
\end{key}


\subsection{Simple Versus Multi-Graphs}
\label{section-library-graphs-simple}

The |graphs| library allows you to construct both simple graphs and
multi-graphs. In a simple graph there can be at most one edge between any two
vertices, while in a multi-graph there can be multiple edges (hence the name).
The two keys |multi| and |simple| allow you to switch (even locally inside on
of the graph's scopes) between which kind of graph is being constructed. By
default, the |graph| command produces a multi-graph since these are faster to
construct.

\begin{key}{/tikz/graphs/multi}
    When this edge is set for a whole graph (which is the default) or just for
    a group (which is useful if the whole graph is simple in general, but a
    part is a multi-graph), then when you specify an edge between two nodes
    several times, several such edges get created:
    %
\begin{codeexample}[preamble={\usetikzlibrary{graphs}}]
\tikz \graph [multi] { % "multi" is not really necessary here
  a ->[bend left,  red]  b;
  a ->[bend right, blue] b;
};
\end{codeexample}
    %
    In case |multi| is used for a scope inside a larger scope where the
    |simple| option is specified, then inside the local |multi| scope edges are
    immediately created and they are completely ignored when it comes to
    deciding which kind of edges should be present in the surrounding simple
    graph. From the surrounding scope's point of view it is as if the local
    |multi| graph contained no edges at all.

    This means, in particular, that you can use the |multi| option with a
    single edge to ``enforce'' this edge to be present in a simple graph.
\end{key}

\begin{key}{/tikz/graphs/simple}
    In contrast a multi-graph, in a simple graph, at most one edge gets created
    for every pair of vertices:
    %
\begin{codeexample}[preamble={\usetikzlibrary{graphs}}]
\tikz \graph [simple]{
  a ->[bend left,  red]  b;
  a ->[bend right, blue] b;
};
\end{codeexample}
    %
    As can be seen, the second edge ``wins'' over the first edge. The general
    rule is as follows: In a simple graph, whenever an edge between two
    vertices is specified multiple times, only the very last specification and
    its options will actually be executed.

    The real power of the |simple| option lies in the fact that you can first
    create a complicated graph and then later redirect and otherwise modify
    edges easily:
    %
\begin{codeexample}[preamble={\usetikzlibrary{graphs}}]
\tikz \graph [simple, grow right=2cm] {
  {a,b,c,d} ->[complete bipartite] {e,f,g,h};

  { [edges={red,thick}] a -> e -> d -> g -> a };
};
\end{codeexample}

    One particularly interesting kind of edge specification for a simple graph
    is |-!-|. Recall that this is used to indicate that ``no edge'' should be
    added between certain nodes. In a multi-graph, this key usually has no
    effect (unless the key |new -!-| has been redefined) and is pretty
    superfluous. In a simple graph, however, it counts as an edge kind and you
    can thus use it to remove an edge that been added previously:
    %
\begin{codeexample}[preamble={\usetikzlibrary{graphs.standard}}]
\tikz \graph [simple] {
  subgraph K_n [n=8, clockwise];
  % Get rid of the following edges:
  1 -!- 2;
  3 -!- 4;
  6 -!- 8;
  % And make one edge red:
  1 --[red] 3;
};
\end{codeexample}

    Creating a graph such as the above in other fashions is pretty awkward.

    For every unordered pair $\{u,v\}$ of vertices at most one edge will be
    created in a simple graph. In particular, when you say |a -> b| and later
    also |a <- b|, then only the edge |a <- b| will be created. Similarly, when
    you say |a -> b| and later |b -> a|, then only the edge |b -> a| will be
    created.

    The power of the |simple| command comes at a certain cost: As the graph is
    being constructed, a (sparse) array is created that keeps track for each
    edge of the last edge being specified. Then, at the end of the scope
    containing the |simple| command, for every pair of vertices the edge is
    created. This is implemented by two nested loops iterating over all
    possible pairs of vertices -- which may take quite a while in a graph of,
    say, 1000 vertices. Internally, the |simple| command is implemented as an
    operator that adds the edges when it is called, but this should be
    unimportant in normal situations.
\end{key}


\subsection{Graph Edges: Labeling and Styling}

When the |graphs| library creates an edge between two nodes in a graph, the
appearance (called ``styling'' in \tikzname) can be specified in different
ways. Sometimes you will simply wish to say ``the edges between these two
groups of node should be red'', but sometimes you may wish to say ``this
particular edge going into this node should be red''. In the following,
different ways of specifying such styling requirements are discussed. Note that
adding labels to edges is, from \tikzname's point of view, almost the same as
styling edges, since they are also specified using options.


\subsubsection{Options For All Edges Between Two Groups}

When you write |... ->[options] ...| somewhere inside your graph specification,
this typically cause one or more edges to be created between the nodes in the
chain group before the |->| and the nodes in the chain group following it. The
|options| are applied to all of them. In particular, if you use the |quotes|
library and you write some text in quotes inside the |options|, this text will
be added as a label to each edge:
%
\begin{codeexample}[preamble={\usetikzlibrary{graphs,quotes}}]
\tikz
  \graph [edge quotes=near start] {
    { a, b } -> [red, "x", complete bipartite] { c, d };
  };
\end{codeexample}

As documented in the |quotes| library in more detail, you can easily modify the
appearance of edge labels created using the quotes syntax by adding options
after the closing quotes:
%
\begin{codeexample}[preamble={\usetikzlibrary{graphs,quotes}}]
\tikz \graph {
  a ->["x"] b ->["y"'] c ->["z" red] d;
};
\end{codeexample}

The following options make it easy to setup the styling of nodes created in
this way:
%
\begin{key}{/tikz/graphs/edge quotes=\opt{\meta{options}}}
    A shorthand for setting the style |every edge quotes| to \meta{options}.
    %
\begin{codeexample}[preamble={\usetikzlibrary{graphs,quotes}}]
  \tikz \graph [edge quotes={blue,auto}] {
  a ->["x"] b ->["y"'] c ->["b" red] d;
};
\end{codeexample}
    %
\end{key}

\begin{key}{/tikz/graphs/edge quotes center}
    A shorthand for |edge quotes| to |anchor=center|.
    %
\begin{codeexample}[preamble={\usetikzlibrary{graphs,quotes}}]
\tikz \graph [edge quotes center] {
  a ->["x"] b ->["y"] c ->["z" red] d;
};
\end{codeexample}
    %
\end{key}

\begin{key}{/tikz/graphs/edge quotes mid}
    A shorthand for |edge quotes| to |anchor=mid|.
    %
\begin{codeexample}[preamble={\usetikzlibrary{graphs,quotes}}]
\tikz \graph [edge quotes mid] {
  a ->["x"] b ->["y"] c ->["z" red] d;
};
\end{codeexample}
    %
\end{key}


\subsubsection{Changing Options For Certain Edges}

Consider the following tree-like graph:
%
\begin{codeexample}[preamble={\usetikzlibrary{graphs}}]
\tikz \graph { a -> {b,c} };
\end{codeexample}

Suppose we wish to specify that the edge from |a| to |b| should be red, while
the edge from |a| to |c| should be blue. The difficulty lies in the fact that
\emph{both} edges are created by the single |->| operator and we can only add
one of these option |red| or |blue| to the operator.

There are several ways to solve this problem. First, we can simply split up the
specification and specify the two edges separately:
%
\begin{codeexample}[preamble={\usetikzlibrary{graphs}}]
\tikz \graph {
  a -> [red]  b;
  a -> [blue] c;
};
\end{codeexample}
%
While this works quite well, we can no longer use the nice chain group syntax
of the |graphs| library. For the rather simple graph |a->{b,c}| this is not a
big problem, but if you specify a tree with, say, 30 nodes it is really
worthwhile being able to specify the tree ``in its natural form in the \TeX\
code'' rather than having to list all of the edges explicitly. Also, as can be
seen in the above example, the node placement is changed, which is not always
desirable.

One can sidestep this problem using the |simple| option: This option allows you
to first specify a graph and then, later on, replace edges by other edges and,
thereby, provide new options:
%
\begin{codeexample}[preamble={\usetikzlibrary{graphs}}]
\tikz \graph [simple] {
  a -> {b,c};
  a -> [red]  b;
  a -> [blue] c;
};
\end{codeexample}

The first line is the original specification of the tree, while the following
two lines replace some edges of the tree (in this case, all of them) by edges
with special options. While this method is slower and in the above example
creates even longer code, it is very useful if you wish to, say, highlight a
path in a larger tree: First specify the tree normally and, then, ``respecify''
the path or paths with some other edge options in force. In the following
example, we use this to highlight a whole subtree of a larger tree:
    %
\begin{codeexample}[preamble={\usetikzlibrary{graphs}}]
\tikz \graph [simple] {
  % The larger tree, no special options in force
  a -> {
    b -> {c,d},
    e -> {f,g},
    h
  },
  { [edges=red] % Now highlight a part of the tree
    a -> e -> {f,g}
  }
};
\end{codeexample}


\subsubsection{Options For Incoming and Outgoing Edges}

When you use the syntax |... ->[options] ...| to specify options, you specify
options for the ``connections between two sets of nodes''. In many cases,
however, it will be more natural to specify options ``for the edges lead to or
coming from a certain node'' and you will want to specify these options ``at
the node''. Returning to the example of the graph |a->{b,c}| where we want a
red edge between |a| and |b| and a blue edge between |a| and |c|, this could
also be phrased as follows: ``Make the edge leading to |b| red and make the
edge leading to |c| blue''.

For this situation, the |graphs| library offers a number of special keys, which
are documented in the following. However, most of the time you will not use
these keys directly, but, rather, use a special syntax explained in
Section~\ref{section-syntax-outgoing-incoming}.

\begin{key}{/tikz/graphs/target edge style=\meta{options}}
    This key can (only) be used with a \emph{node} inside a graph
    specification. When used, the \meta{options} will be added to every edge
    that is created by a connector like |->| in which the node is a
    \emph{target}. Consider the following example:
    %
\begin{codeexample}[preamble={\usetikzlibrary{graphs}}]
\tikz \graph {
  { a, b } ->
  { c [target edge style=red], d } ->
  { e, f }
};
\end{codeexample}
    %
    In the example, only when the edge from |a| to |c| is created, |c| is the
    ``target'' of the edge. Thus, only this edge becomes red.

    When an edge already has options set directly, the \meta{options} are
    executed after these direct options, thus, they ``overrule'' them:
    %
\begin{codeexample}[preamble={\usetikzlibrary{graphs}}]
\tikz \graph {
  { a, b } -> [blue, thick]
  { c [target edge style=red], d } ->
  { e, f }
};
\end{codeexample}

    The \meta{options} set in this way will stay attached to the node, so also
    for edges created later on that lead to the node will have these options
    set:
    %
\begin{codeexample}[preamble={\usetikzlibrary{graphs}}]
\tikz \graph {
  { a, b } ->
  { c [target edge style=red], d } ->
  { e, f },
  b -> c
};
\end{codeexample}

    Multiple uses of this key accumulate. However, you may sometimes also wish
    to ``clear'' these options for a key since at some later point you no
    longer wish the \meta{options} to be added when some further edges are
    added. This can be achieved using the following key:
    %
    \begin{key}{/tikz/graphs/target edge clear}
        Clears all \meta{options} for edges with the node as a target and
        also edge labels (see below) for this node.
    \end{key}
    %
\begin{codeexample}[preamble={\usetikzlibrary{graphs}}]
\tikz \graph {
  { a, b } ->
  { c [target edge style=red], d },
  b -> c[target edge clear]
};
\end{codeexample}
    %
\end{key}

\begin{key}{/tikz/graphs/target edge node=\meta{node specification}}
    This key works like |target edge style|, only the \meta{node specification}
    will not be added as options to any newly created edges with the current
    node as their target, but rather it will be added as a node specification.
    %
\begin{codeexample}[preamble={\usetikzlibrary{graphs}}]
\tikz \graph {
  { a, b } ->
  { c [target edge node=node{X}], d } ->
  { e, f }
};
\end{codeexample}
    %
    As for |target edge style| multiple uses of this key accumulate and the key
    |target edge clear| will (also) clear all target edge nodes that have been
    set for a node earlier on.
\end{key}

\begin{key}{/tikz/graphs/source edge style=\meta{options}}
    Works exactly like |target edge style|, only now the \meta{options} are
    only added when the node is a source of a newly created edge:
    %
\begin{codeexample}[preamble={\usetikzlibrary{graphs}}]
\tikz \graph {
  { a, b } ->
  { c [source edge style=red], d } ->
  { e, f }
};
\end{codeexample}
    %
    If both for the source and also for the target of an edge \meta{options}
    have been specified, the options are applied in the following order:
    %
    \begin{enumerate}
        \item First come the options from the edge itself.
        \item Then come the options contributed by the source node using this
            key.
        \item Then come the options contributed by the target node using
            |target node style|.
    \end{enumerate}
    %
\begin{codeexample}[preamble={\usetikzlibrary{graphs}}]
\tikz \graph {
  a [source edge style=red] ->[green]
  b [target edge style=blue]  % blue wins
};
\end{codeexample}
    %
\end{key}

\begin{key}{/tikz/graphs/source edge node=\meta{node specification}}
    Works like |source edge style| and |target edge node|.
\end{key}

\begin{key}{/tikz/graphs/source edge clear=\meta{node specification}}
    Works like |target edge clear|.
\end{key}


\subsubsection{Special Syntax for Options For Incoming and Outgoing Edges}
\label{section-syntax-outgoing-incoming}

The keys |target node style| and its friends are powerful, but a bit cumbersome
to write down. For this reason, the |graphs| library introduces a special
syntax that is based on what I call the ``first-char syntax'' of keys. Inside
the options of a node inside a graph, the following special rules apply:
%
\begin{enumerate}
    \item Whenever an option starts with |>|, the rest of the options are
        passed to |target edge style|. For instance, when you write |a[>red]|,
        then this has the same effect as if you had written
        %
\begin{codeexample}[code only]
a[target edge style={red}]
\end{codeexample}
        %
    \item Whenever an options starts with |<|, the rest of the options are
        passed to |source edge style|.
    \item In both of the above case, in case the options following the |>| or
        |<| sign start with a quote, the created edge label is passed to
        |source edge node| or |target edge node|, respectively.

        This is exactly what you want to happen.
\end{enumerate}
%
Additionally, the following styles provide shorthands for ``clearing'' the
target and source options:
%
\begin{key}{/tikz/graphs/clear >}
    A more easy-to-remember shorthand for |target edge clear|.
\end{key}
%
\begin{key}{/tikz/graphs/clear <}
    A more easy-to-remember shorthand for |source edge clear|.
\end{key}

These mechanisms make it especially easy to create trees in which the edges are
labeled in some special way:
%
\begin{codeexample}[preamble={\usetikzlibrary{graphs,quotes}}]
\tikz
  \graph [edge quotes={fill=white,inner sep=1pt},
          grow down, branch right] {
    / -> h [>"9"] -> {
      c [>"4" text=red,] -> {
        a [>"2", >thick],
        e [>"0"]
      },
      j [>"7"]
    }
  };
\end{codeexample}


\subsubsection{Placing Node Texts on Incoming Edges}

Normally, the text of a node is shown (only) inside the node. In some case, for
instance when drawing certain kind of trees, the nodes themselves should not
get any text, but rather the edge leading to the node should be labeled as in
the following example:
%
\begin{codeexample}[preamble={\usetikzlibrary{graphs,quotes}}]
\tikz \graph [empty nodes]
{
  root -> {
    a [>"a"],
    b [>"b"] -> {
      c [>"c"],
      d [>"d"]
    }
  }
};
\end{codeexample}
%
As the example shows, it is a bit cumbersome that we have to label the nodes
and then specify the same text once more using the incoming edge syntax.

For these cases, it would be better if the text of the node where not used with
the node but, rather, be passed directly to the incoming or the outgoing edge.
The following styles do exactly this:

\begin{key}{/tikz/graphs/put node text on incoming edges=\opt{\meta{options}}}
    When this key is used with a node or a group, the following happens:
    %
    \begin{enumerate}
        \item The command
            |target edge node={node[|\meta{options}|]{\tikzgraphnodetext}}| is
            executed. This means that all incoming edges of the node get a
            label with the text that would usually be displayed in the node.
            You can use keys like |math nodes| normally.
        \item The command |as={}| is executed. This means that the node itself
            will display nothing.
    \end{enumerate}
    %
    Here is an example that show how this command is used.
    %
\begin{codeexample}[preamble={\usetikzlibrary{graphs}}]
\tikz \graph [put node text on incoming edges,
              math nodes, nodes={circle,draw}]
  { a -> b -> {c, d} };
\end{codeexample}
    %
\end{key}

\begin{key}{/tikz/graphs/put node text on outgoing edges=\opt{\meta{options}}}
    Works like the previous key, only with |target| replaced by |source|.
\end{key}


\subsection{Graph Operators, Color Classes, and Graph Expressions}
\label{section-library-graphs-color-classes}

\tikzname's |graph| command employs a powerful mechanism for adding edges
between nodes and sets of nodes. To a graph theorist, this mechanism may be
known as a \emph{graph expression}: A graph is specified by starting with small
graphs and then applying \emph{operators} to them that form larger graphs and
that connect and recolor colored subsets of the graph's node in different ways.


\subsubsection{Color Classes}
\label{section-library-graph-coloring}

\tikzname\ keeps track of a \emph{(multi)coloring} of the graph as it is being
constructed. This does not mean that the actual color of the nodes on the page
will be different, rather, in the following we refer to ``logical'' colors in
the way graph theoreticians do. These ``logical'' colors are only important
while the graph is being constructed and they are ``thrown away'' at the end of
the construction. The actual (``physical'') colors of the nodes are set
independently of these logical colors.

As a graph is being constructed, each node can be part of one or more
overlapping \emph{color classes}. So, unlike what is sometimes called a
\emph{legal coloring}, the logical colorings that \tikzname\ keeps track of may
assign multiple colors to the same node and two nodes connected by an edge may
well have the same color.

Color classes must be declared prior to use. This is done using the following
key:
%
\begin{key}{/tikz/graphs/color class=\meta{color class name}}
    This sets up a new color class called \meta{color class name}. Nodes and
    whole groups of nodes can now be colored with \meta{color class name}. This
    is done using the following keys, which become
    available inside the current scope:
    %
    \begin{key}{/tikz/graphs/\meta{color class name}}
        This key internally uses the |operator| command to setup an operator
        that will cause all nodes of the current group to get the ``logical
        color'' \meta{color class name}. Nodes retain this color in all
        encompassing scopes, unless it is explicitly changed (see below) or
        unset (again, see below).
        %
\begin{codeexample}[preamble={\usetikzlibrary{graphs}}]
\tikz \graph [color class=red] {
  [cycle=red]  % causes all "logically" red nodes to be connected in
               % a cycle
  a,
  b [red],
  { [red] c ->[bend right] d },
  e
};
\end{codeexample}
        %
\begin{codeexample}[preamble={\usetikzlibrary{graphs}}]
\tikz \graph [color class=red, color class=green,
              math nodes, clockwise, n=5] {
  [complete bipartite={red}{green}]
  { [red]   r_1, r_2 },
  { [green] g_1, g_2, g_3 }
};
\end{codeexample}
    \end{key}
    %
    \begin{key}{/tikz/graphs/not \meta{color class name}}
        Sets up an operator for the current scope so that all nodes in it loose
        the color \meta{color class name}. You can also use |!|\meta{color
        class name} as an alias for this key.
        %
\begin{codeexample}[preamble={\usetikzlibrary{graphs}}]
\tikz \graph [color class=red, color class=green,
              math nodes, clockwise, n=5] {
  [complete bipartite={red}{green}]
  { [red]   r_1, r_2 },
  { [green] g_1, g_2, g_3 },
  g_2 [not green]
};
\end{codeexample}
    \end{key}
    %
    \begin{key}{/tikz/graphs/recolor \meta{color class name} by=\meta{new color}}
        Causes all keys having color \meta{color class name} to get \meta{new
        color} instead. They loose having color \meta{color class name}, but
        other colors are not affected.
        %
\begin{codeexample}[preamble={\usetikzlibrary{graphs}}]
\tikz \graph [color class=red, color class=green,
              math nodes, clockwise, n=5] {
  [complete bipartite={red}{green}]
  { [red]   r_1, r_2 },
  { [green] g_1, g_2, g_3 },
  g_2 [recolor green by=red]
};
\end{codeexample}
    \end{key}
\end{key}

The following color classes are available by default:
%
\begin{itemize}
    \item Color class |all|. Every node is part of this class by default. This
        is useful to access all nodes of a (sub)graph, since you can simply
        access all nodes of this color class.
    \item Color classes |source| and |target|. These classes are used to
        identify nodes that lead ``into'' a group of nodes and nodes from which
        paths should ``leave'' the group. Details on how these colors are
        assigned are explained in Section~\ref{section-library-graphs-join}. By
        saying |not source| or |not target| with a node, you can influence how
        it is connected:
        %
\begin{codeexample}[preamble={\usetikzlibrary{graphs}}]
\tikz \graph { a -> { b, c, d } -> e };
\end{codeexample}
        %
\begin{codeexample}[preamble={\usetikzlibrary{graphs}}]
\tikz \graph { a -> { b[not source], c, d[not target] } -> e };
\end{codeexample}
        %
    \item Color classes |source'| and |target'|. These are temporary colors
        that are also explained in Section~\ref{section-library-graphs-join}.
\end{itemize}


\subsubsection{Graph Operators on Groups of Nodes}

Recall that the |graph| command constructs graphs recursively from nested
\meta{group specifications}. Each such \meta{group specification} describes a
subset of the nodes of the final graph. A \emph{graph operator} is an algorithm
that gets the nodes of a group as input and (typically) adds edges between
these nodes in some sensible way. For instance, the |clique| operator will
simply add edges between all nodes of the group.

\begin{key}{/tikz/graphs/operator=\meta{code}}
    This key has an effect in three places:
    %
    \begin{enumerate}
        \item It can be used in the \meta{options} of a \meta{direct node
            specification}.
        \item It can be used in the \meta{options} of a \meta{group
            specification}.
        \item It can be used in the \meta{options} of an \meta{edge
            specification}.
    \end{enumerate}
    %
    The first case is a special case of the second, since it is treated like a
    group specification containing a single node. The last case is more
    complicated and discussed in the next section. So, let us focus on the
    second case.

    Even though the \meta{options} of a group are given at the beginning of the
    \meta{group specification}, the \meta{code} is only executed when the group
    has been parsed completely and all its nodes have been identified. If you
    use the |operator| multiple times in the \meta{options}, the effect
    accumulates, that is, all code passed to the different calls of |operator|
    gets executed in the order it is encountered.

    The \meta{code} can do ``whatever it wants'', but it will typically add
    edges between certain nodes. You can configure what kind of edges
    (directed, undirected, etc.) are created by using the following keys:
    %
    \begin{key}{/tikz/graphs/default edge kind=\meta{value} (initially -\/-)}
        This key stores one of the five edge kinds |--|, |<-|, |->|, |<->|, and
        |-!-|. When an operator wishes to create a new edge, it should
        typically set
        %
\begin{codeexample}[code only]
\tikzgraphsset{new \pfkeysvalueof{/tikz/graphs/default edge kind}=...}
\end{codeexample}
        %
        While this key can be set explicitly, it may be more convenient to use
        the abbreviating keys listed below. Also, this key is automatically set
        to the current value of \meta{edge specification} when a joining
        operator is called, see the discussion of joining operators in
        Section~\ref{section-library-graphs-join}.
    \end{key}
    %
    \begin{key}{/tikz/graphs/--}
        Sets the |default edge kind| to |--|.
        %
\begin{codeexample}[preamble={\usetikzlibrary{graphs.standard}}]
\tikz \graph { subgraph K_n [--, n=5, clockwise, radius=6mm] };
\end{codeexample}
    \end{key}
    %
    \begin{key}{/tikz/graphs/->}
        Sets the |default edge kind| to |->|.
        %
\begin{codeexample}[preamble={\usetikzlibrary{graphs.standard}}]
\tikz \graph { subgraph K_n [->, n=5, clockwise, radius=6mm] };
\end{codeexample}
    \end{key}
    %
    \begin{key}{/tikz/graphs/<-}
        Sets the |default edge kind| to |<-|.
        %
\begin{codeexample}[preamble={\usetikzlibrary{graphs.standard}}]
\tikz \graph { subgraph K_n [<-, n=5, clockwise, radius=6mm] };
\end{codeexample}
    \end{key}
    %
    \begin{key}{/tikz/graphs/<->}
        Sets the |default edge kind| to |<->|.
        %
\begin{codeexample}[preamble={\usetikzlibrary{graphs.standard}}]
\tikz \graph { subgraph K_n [<->, n=5, clockwise, radius=6mm] };
\end{codeexample}
    \end{key}
    %
    \begin{key}{/tikz/graphs/-\protect\exclamationmarktext-}
        Sets the |default edge kind| to |-!-|.
    \end{key}

    When the \meta{code} of an operator is executed, the following commands can
    be used to find the nodes that should be connected:
    %
    \begin{command}{\tikzgraphforeachcolorednode\marg{color name}\marg{macro}}
        When this command is called inside \meta{code}, the following will
        happen: \tikzname\ will iterate over all nodes inside the
        just-specified group that have the color \meta{color name}. The order
        in which they are iterated over is the order in which they appear
        inside the group specification (if a node is encountered several times
        inside the specification, only the first occurrence counts). Then, for
        each node the \meta{macro} is executed with the node's name as the only
        argument.

        In the following example we use an operator to connect every node
        colored |all| inside the subgroup to he node |root|.
        %
\begin{codeexample}[preamble={\usetikzlibrary{graphs}}]
\def\myconnect#1{\tikzset{graphs/new ->={root}{#1}{}{}}}

\begin{tikzpicture}
  \node (root) at (-1,-1) {root};

  \graph {
    x,
    {
      [operator=\tikzgraphforeachcolorednode{all}{\myconnect}]
      a, b, c
    }
  };
\end{tikzpicture}
\end{codeexample}
    \end{command}

    \begin{command}{\tikzgraphpreparecolor\marg{color name}\marg{counter}\marg{prefix}}
        This command is used to ``prepare'' the nodes of a certain color for
        random access. The effect is the following: It is counted how many
        nodes there are having color \meta{color name} in the current group and
        the result is stored in \meta{counter}. Next, macros named
        \meta{prefix}|1|, \meta{prefix}|2|, and so on are defined, that store
        the names of the first, second, third, and so on node having the color
        \meta{color name}.

        The net effect is that after you have prepared a color, you can quickly
        iterate over them. This is especially useful when you iterate over
        several color at the same time.

        As an example, let us create an operator then adds a zig-zag path
        between two color classes:
        %
\begin{codeexample}[preamble={\usetikzlibrary{graphs}}]
\newcount\leftshorecount   \newcount\rightshorecount
\newcount\mycount          \newcount\myothercount
\def\zigzag{
  \tikzgraphpreparecolor{left shore}\leftshorecount{left shore prefix}
  \tikzgraphpreparecolor{right shore}\rightshorecount{right shore prefix}
  \mycount=0\relax
  \loop
    \advance\mycount by 1\relax%
    % Add the "forward" edge
    \tikzgraphsset{new ->=
      {\csname left shore prefix\the\mycount\endcsname}
      {\csname right shore prefix\the\mycount\endcsname}{}{}}
    \myothercount=\mycount\relax%
    \advance\myothercount by1\relax%
    \tikzgraphsset{new <-=
      {\csname left shore prefix\the\myothercount\endcsname}
      {\csname right shore prefix\the\mycount\endcsname}{}{}}
  \ifnum\myothercount<\leftshorecount\relax
  \repeat
}
\begin{tikzpicture}
  \graph [color class=left shore, color class=right shore]
  { [operator=\zigzag]
    { [left shore, Cartesian placement]                      a, b, c },
    { [right shore, Cartesian placement, nodes={xshift=1cm}] d, e, f }
  };
\end{tikzpicture}
\end{codeexample}
        %
        Naturally, in order to turn the above code into a usable operator, some
        more code would be needed (like default values and taking care of
        shores of different sizes).
    \end{command}
\end{key}

There are a number of predefined operators, like |clique| or |cycle|, see the
reference Section~\ref{section-library-graphs-reference} for a complete list.


\subsubsection{Graph Operators for Joining Groups}
\label{section-library-graphs-join}

When you join two nodes |foo| and |bar| by the edge specification |->|, it is
fairly obvious, what should happen: An edge from |(foo)| to |(bar)| should be
created. However, suppose we use an edge specification between two node sets
like |{a,b,c}| and |{d,e,f}|. In this case, it is not so clear which edges
should be created. One might argue that all possible edges from any node in the
first set to any node in the second set should be added. On the other hand, one
might also argue that only a matching between these two sets should be created.
Things get even more muddy when a longer chain of node sets are joined.

Instead of fixing how edges are created between two node sets, \tikzname\ takes
a somewhat more general, but also more complicated approach, which can be
broken into two parts. In the following, assume that the following chain
specification is given:
%
\begin{quote}
    \meta{spec$_1$} \meta{edge specification} \meta{spec$_2$}
\end{quote}
%
An example might be |{a,b,c} -> {d, e->f}|.

\medskip
\textbf{The source and target vertices.} Let us start with the question of
which vertices of the first node set should be connected to vertices in the
second node set.

There are two predefined special color classes that are used for this: |source|
and |target|. For every group specification, some vertices are colored as
|source| vertices and some vertices are |target| vertices (a node can both be a
target and a source). Initially, every vertex is both a source and a target,
but that can change as we will see in a moment.

The intuition behind source and target vertices is that, in some sense, edges
``from the outside'' lead into the group via the source vertices and lead out
of the group via the target vertices. To be more precise, the following
happens:
%
\begin{enumerate}
    \item The target vertices of the first group are connected to the source
        vertices of the second group.
    \item In the group resulting from the union of the nodes from
        \meta{spec$_1$} and \meta{spec$_2$}, the source vertices are only those
        from the first group, and the target vertices are only those from the
        second group.
\end{enumerate}

Let us go over the effect of these rules for the example
|{a,b,c} -> {d, e->f}|. First, each individual node is initially both a
|source| and a |target| vertex. Then, in |{a,b,c}| all nodes are still both
source and target vertices since just grouping vertices does not change their
colors. Now, in |e->f| something interesting happens for the first time: the
target vertices of the ``group'' |e| (which is just the node |e|) are connected
to the source vertices of the ``group'' |f|. This means, that an edge is added
from |e| to |f|. Then, in the resulting group |e->f| the only source vertex is
|e| and the only target vertex is |f|. This implies that in the group
|{d,e->f}| the sources are |d| and |e| and the targets are |d| and~|f|.

Now, in |{a,b,c} -> {d,e->f}| the targets  of |{a,b,c}| (which are all three of
them) are connected to the sources of |{d,e->f}| (which are just |d| and~|e|).
Finally, in the whole graph only |a|, |b|, and |c| are sources while only  |d|
and |f| are targets.
%
\begin{codeexample}[preamble={\usetikzlibrary{graphs}}]
\def\hilightsource#1{\fill [green, opacity=.25] (#1) circle [radius=2mm]; }
\def\hilighttarget#1{\fill [red,   opacity=.25] (#1) circle [radius=2mm]; }
\tikz \graph
  [operator=\tikzgraphforeachcolorednode{source}{\hilightsource},
   operator=\tikzgraphforeachcolorednode{target}{\hilighttarget}]
  { {a,b,c} -> {d, e->f} };
\end{codeexample}

The next objective is to make more precise what it means that ``the targets of
the first graph'' and the ``sources of the second graph'' should be connected.
We know already of a general way of connecting nodes of a graph: operators!
Thus, we use an operator for this job. For instance, the |complete bipartite|
operator adds an edge from every node having a certain color to every node have
a certain other color. This is exactly what we need here: The first color is
``the color |target| restricted to the nodes of the first graph'' and the
second color is ``the color |source| restricted to the nodes of the second
graph''.

However, we cannot really specify that only nodes from a certain subgraph are
meant -- the |operator| machinery only operates on all nodes of the current
graph. For this reason, what really happens is the following: When the |graph|
command encounters \meta{spec$_1$} \meta{edge specification} \meta{spec$_2$},
it first computes and colors the nodes of the first and the second
specification independently. Then, the |target| nodes of the first graph are
recolored to |target'| and the |source| nodes of the second graph are recolored
to |source'|. Then, the two graphs are united into one graph and a
\emph{joining operator} is executed, which should add edges between |target'|
and |source'|. Once this is done, the colors |target'| and |source'| get
erased. Note that in the resulting graph only the |source| nodes from the first
graph are still |source| nodes and likewise for the |target| nodes of the
second graph.

\medskip
\textbf{The joining operators.} The job of a joining operator is to add edges
between nodes colored |target'| and |source'|. The following rule is used to
determine which operator should be chosen for performing this job:
%
\begin{enumerate}
    \item If the \meta{edge specification} explicitly sets the |operator| key
        to something non-empty (and also not to |\relax|), then the \meta{code}
        of this |operator| call is used.
    \item Otherwise, the current value of the following key is used:
        %
        \begin{key}{/tikz/graphs/default edge operator=\meta{key} (initially matching and star)}
            This key stores the name of a \meta{key} that is executed for every
            \meta{edge specification} whose \meta{options} do not contain the
            |operator| key.
            %
\begin{codeexample}[preamble={\usetikzlibrary{graphs}}]
\tikz \graph [default edge operator=matching] {
  {a, b}    ->[matching and star]
  {c, d, e} --[complete bipartite]
  {f, g, h} --
  {i, j, k}
};
\end{codeexample}
        \end{key}
\end{enumerate}

A typical joining operator is |complete bipartite|. It takes the names of two
color classes as input and adds edges from all vertices of the first class to
all vertices of the second class. Now, the trick is that the default value for
the |complete bipartite| key is |{target'}{source'}|. Thus, if you just write
|->[complete bipartite]|, the same happens as if you had written
%
\begin{quote}
    |->[complete bipartite={target'}{source'}]|
\end{quote}
%
This is exactly what we want to happen. The same default values are also set
for other joining operators like |matching| or |butterfly|.

Even though an operator like |complete bipartite| is typically used together
with an edge specification, it can also be used as a normal operator together
with a group specification. In this case, however, the color classes must be
named explicitly:
%
\begin{codeexample}[preamble={\usetikzlibrary{graphs}}]
\begin{tikzpicture}
  \graph [color class=red, color class=green, math nodes]
  { [complete bipartite={red}{green}]
    { [red,   Cartesian placement]                      r_1, r_2, r_3 },
    { [green, Cartesian placement, nodes={xshift=1cm}]  g_1, g_2, g_3 }
  };
\end{tikzpicture}
\end{codeexample}

A list of predefined joining operators can be found in the reference
Section~\ref{section-library-graphs-reference}.

The fact that joining operators can also be used as normal operators leads to a
subtle problem: A normal operator will typically use the current value of
|default edge kind| to decide which kind of edges should be put between the
identified vertices, while a joining operator should, naturally, use the kind
of edge specified by the \meta{edge specification}. This problem is solved as
follows: Like a normal operator, a joining operator should also use the current
value of |default edge kind| for the edges it produces. The trick is that this
will automatically be set to the current \meta{edge specification} when the
operator explicitly in the \meta{options} of the edge specification or
implicitly in the |default edge operator|.


\subsection{Graph Macros}
\label{section-library-graphs-macros}

A \emph{graph macro} is a small graph that is inserted at some point into the
graph that is currently being constructed. There is special support for such
graph macros in \tikzname. You might wonder why this is necessary -- can't one
use \TeX's normal macro mechanism? The answer is ``no'': one cannot insert new
nodes into a graph using normal macros because the chains, groups, and nodes
are determined prior to macro expansion. Thus, any macro encountered where some
node text should go will only be expanded when this node is being named and
typeset.

A graph macro is declared using the following key:

\begin{key}{/tikz/graphs/declare=\marg{graph name}\marg{specification}}
    This key declares that \meta{graph name} can subsequently be used as a
    replacement for a \meta{node name}. Whenever the \meta{graph name} is used
    in the following, a graph group will be inserted instead whose content is
    exactly \meta{specification}. In case \meta{graph name} is used together
    with some \meta{options}, they are executed prior to inserting the
    \meta{specification}.
    %
\begin{codeexample}[preamble={\usetikzlibrary{graphs}}]
\tikz \graph [branch down=4mm, declare={claw}{1 -- {2,3,4}}] {
  a;
  claw;
  b;
};
\end{codeexample}
    %
    In the next example, we use a key to configure a subgraph:
    %
\begin{codeexample}[preamble={\usetikzlibrary{graphs}}]
\tikz \graph [ n/.code=\def\n{#1}, branch down=4mm,
               declare={star}{root -- { \foreach \i in {1,...,\n} {\i} }}]
{ star [n=5]; };
\end{codeexample}
    %
    Actually, the |n| key is already defined internally for a similar purpose.

    As a last example, let us define a somewhat more complicated graph macro.
    %
\begin{codeexample}[preamble={\usetikzlibrary{graphs}}]
\newcount\mycount
\tikzgraphsset{
  levels/.store in=\tikzgraphlevel,
  levels=1,
  declare={bintree}{%
    [/utils/exec={%
      \ifnum\tikzgraphlevel=1\relax%
        \def\childtrees{ / }%
      \else%
        \mycount=\tikzgraphlevel%
        \advance\mycount by-1\relax%
        \edef\childtrees{
          / -> {
            bintree[levels=\the\mycount],
            bintree[levels=\the\mycount]
          }}
      \fi%
    },
    parse/.expand once=\childtrees
    ]
    % Everything is inside the \childtrees...
  }
}
\tikz \graph [grow down=5mm, branch right=5mm] { bintree [levels=5] };
\end{codeexample}
    %
\end{key}

Note that when you use a graph macro several time inside the same graph, you
will typically have to use the |name| option so that different copies of the
subgraph are created:
%
\begin{codeexample}[preamble={\usetikzlibrary{graphs}}]
\tikz \graph [branch down=4mm, declare={claw}{1 -- {2,3,4}}] {
  claw [name=left],
  claw [name=right]
};
\end{codeexample}

You will find a list of useful graph macros in the reference section,
Section~\ref{section-library-graphs-reference-macros}.


\subsection{Online Placement Strategies}
\label{section-library-graphs-placement}

The main job of the |graphs| library is to make it easy to specify which nodes
are present in a graph and how they are connected. In contrast, it is
\emph{not} the primary job of the library to compute good positions for nodes
in a graph -- use for instance a |\matrix|, specify good positions ``by hand''
or use the graph drawing facilities. Nevertheless, some basic support for
automatic node placement is provided for simple cases. The |graphs| library
will provide you with information about the position of nodes inside their
groups and chains.

As a graph is being constructed, a \emph{placement strategy} is used to
determine a (reasonably good) position for the nodes as they are created. These
placement strategies get some information about what \tikzname\ has already
seen concerning the already constructed nodes, but it gets no information
concerning the upcoming nodes. Because of this lack of information concerning
the future, the strategies need to be what is called an \emph{online strategy}
in computer science. (The opposite are \emph{offline strategies}, which get
information about the whole graph and all the sizes of the nodes in it. The
graph drawing libraries employ such offline strategies.)

Strategies are selected using keys like |no placement| or
|Cartesian placement|. It is permissible to use different strategies inside
different parts of a graph, even though the different strategies do not always
work together in perfect harmony.


\subsubsection{Manual Placement}
\label{section-graphs-xy}

\begin{key}{/tikz/graphs/no placement}
    This strategy simply ``switches off'' the whole placement mechanism,
    causing all nodes to be placed at the origin by default. You need to use
    this strategy if you position nodes ``by hand''. For this, you can use the
    |at| key, the |shift| keys:
    %
\begin{codeexample}[preamble={\usetikzlibrary{graphs}}]
\tikz \graph [no placement]
{
  a[at={(0:0)}] -> b[at={(1,0)}] -> c[yshift=1cm];
};
\end{codeexample}
    %
    Since the syntax and the many braces and parentheses are a bit cumbersome,
    the following two keys might also be useful:
    %
    \begin{key}{/tikz/graphs/x=\meta{x dimension}}
        When you use this key, it will have the same effect as if you had
        written |at={(|\meta{x dimension}|,|\meta{y dimension}|)}|, where
        \meta{y dimension} is a value set using the |y| key:
        %
\begin{codeexample}[preamble={\usetikzlibrary{graphs}}]
\tikz \graph [no placement]
{
  a[x=0,y=0] -> b[x=1,y=0] -> c[x=0,y=1];
};
\end{codeexample}
        %
        Note that you can specify an |x| or a |y| key for a whole scope and
        then vary only the other key:
        %
\begin{codeexample}[preamble={\usetikzlibrary{graphs}}]
\tikz \graph [no placement]
{
  a ->
  { [x=1] % group option
    b [y=0] -> c[y=1]
  };
};
\end{codeexample}
        %
        Note that these keys have the path |/tikz/graphs/|, so they will be
        available inside |graph|s and will not clash with the usual |x| and |y|
        keys of \tikzname, which are used to specify the basic lengths of
        vectors.
    \end{key}
    %
    \begin{key}{/tikz/graphs/y=\meta{y dimension}}
        See above.
    \end{key}
\end{key}


\subsubsection{Placement on a Grid}

\begin{key}{/tikz/graphs/Cartesian placement}
    This strategy is the default strategy. It works, roughly, as follows: For
    each new node on a chain, advance a ``logical width'' counter and for each
    new node in a group, advance a ``logical depth'' counter. When a chain
    contains a whole group, then the ``logical width'' taken up by the group is
    the maximum over the logical widths taken up by the chains inside the
    group; and symmetrically the logical depth of a chain is the maximum of the
    depths of the groups inside it.

    This slightly confusing explanation is perhaps best exemplified. In the
    below example, the two numbers indicate the two logical width and depth of
    each node as computed by the |graphs| library. Just ignore the arcane code
    that is used to print these numbers.
    %
\begin{codeexample}[preamble={\usetikzlibrary{graphs}}]
\tikz
  \graph [nodes={align=center, inner sep=1pt}, grow right=7mm,
          typeset={\tikzgraphnodetext\\[-4pt]
                   \tiny\mywidth\\[-6pt]\tiny\mydepth},
          placement/compute position/.append code=
            \pgfkeysgetvalue{/tikz/graphs/placement/width}{\mywidth}
            \pgfkeysgetvalue{/tikz/graphs/placement/depth}{\mydepth}]
{
  a,
  b,
  c -> d -> {
    e -> f -> g,
    h -> i
  } -> j,
  k -> l
};
\end{codeexample}
    %
    You will find a detailed description of how these logical units are
    computed, exactly, in Section~\ref{section-library-graphs-new-online}.

    Now, even though we talk about ``widths'' and ``depths'' and even though by
    default a graph ``grows'' to the right and down, this is by no means fixed.
    Instead, you can use the following keys to change how widths and heights
    are interpreted:
    %
    \begin{key}{/tikz/graphs/chain shift=\meta{coordinate} (initially {(1,0)})}
        Under the regime of the |Cartesian placement| strategy, each node is
        shifted by the current logical width times this \meta{coordinate}.
        %
\begin{codeexample}[preamble={\usetikzlibrary{graphs}}]
\tikz \graph [chain shift=(45:1)] {
  a -> b -> c;
  d -> e;
  f -> g -> h;
};
\end{codeexample}
    \end{key}
    %
    \begin{key}{/tikz/graphs/group shift=\meta{coordinate} (initially {(0,-1)})}
        Like for |chain shift|, each node is shifted by the current logical
        depth times this \meta{coordinate}.
        %
\begin{codeexample}[preamble={\usetikzlibrary{graphs}}]
\tikz \graph [chain shift=(45:7mm), group shift=(-45:7mm)] {
  a -> b -> c;
  d -> e;
  f -> g -> h;
};
\end{codeexample}
    \end{key}
\end{key}

\begin{key}{/tikz/graphs/grow up=\meta{distance} (default 1)}
    Sets the |chain shift| to |(0,|\meta{distance}|)|, so that chains ``grow
    upward''. The distance by which the center of each new element is removed
    from the center of the previous one is \meta{distance}.
    %
\begin{codeexample}[preamble={\usetikzlibrary{graphs}}]
\tikz \graph [grow up=7mm] { a -> b -> c};
\end{codeexample}
    %
\end{key}

\begin{key}{/tikz/graphs/grow down=\meta{distance} (default 1)}
    Like |grow up|.
    %
\begin{codeexample}[preamble={\usetikzlibrary{graphs}}]
\tikz \graph [grow down=7mm] { a -> b -> c};
\end{codeexample}
    %
\end{key}

\begin{key}{/tikz/graphs/grow left=\meta{distance} (default 1)}
    Like |grow up|.
    %
\begin{codeexample}[preamble={\usetikzlibrary{graphs}}]
\tikz \graph [grow left=7mm] { a -> b -> c};
\end{codeexample}
    %
\end{key}

\begin{key}{/tikz/graphs/grow right=\meta{distance} (default 1)}
    Like |grow up|.
    %
\begin{codeexample}[preamble={\usetikzlibrary{graphs}}]
\tikz \graph [grow right=7mm] { a -> b -> c};
\end{codeexample}
    %
\end{key}

\begin{key}{/tikz/graphs/branch up=\meta{distance} (default 1)}
    Sets the |group shift| so that groups ``branch upward''.  The distance by
    which the center of each new element is removed from the center of the
    previous one is \meta{distance}.
    %
\begin{codeexample}[preamble={\usetikzlibrary{graphs}}]
\tikz \graph [branch up=7mm] { a -> b -> {c, d, e} };
\end{codeexample}
    %
    Note that when you draw a tree, the |branch ...| keys specify how siblings
    (or adjacent branches) are arranged, while the |grow ...| keys specify in
    which direction the branches ``grow''.
\end{key}

\begin{key}{/tikz/graphs/branch down=\meta{distance} (default 1)}
%
\begin{codeexample}[preamble={\usetikzlibrary{graphs}}]
\tikz \graph [branch down=7mm] { a -> b -> {c, d, e}};
\end{codeexample}
%
\end{key}

\begin{key}{/tikz/graphs/branch left=\meta{distance} (default 1)}
%
\begin{codeexample}[preamble={\usetikzlibrary{graphs}}]
\tikz \graph [branch left=7mm, grow down=7mm] { a -> b -> {c, d, e}};
\end{codeexample}
%
\end{key}

\begin{key}{/tikz/graphs/branch right=\meta{distance} (default 1)}
%
\begin{codeexample}[preamble={\usetikzlibrary{graphs}}]
\tikz \graph [branch right=7mm, grow down=7mm] { a -> b -> {c, d, e}};
\end{codeexample}
%
\end{key}

The following keys place nodes in a $N\times M$ grid.
%
\begin{key}{/tikz/graphs/grid placement}
    This key works similar to |Cartesian placement|. As for that placement
    strategy, a node has logical width and depth 1. However, the computed total
    width and depth are mapped to a $N\times M$ grid. The values of $N$ and $M$
    depend on the size of the graph and the value of |wrap after|. The number
    of columns $M$ is either set to |wrap after| explicitly or computed
    automatically as $\sqrt{\texttt{\string|V\string|}}$. $N$ is the number of
    rows needed to lay out the graph in a grid with $M$ columns.
    %
\begin{codeexample}[preamble={\usetikzlibrary{graphs.standard}}]
% An example with 6 nodes, 3 columns and therefor 2 rows
\tikz \graph [grid placement] { subgraph I_n[n=6, wrap after=3] };
\end{codeexample}
    %
\begin{codeexample}[preamble={\usetikzlibrary{graphs.standard}}]
% An example with 9 nodes with columns and rows computed automatically
\tikz \graph [grid placement] { subgraph Grid_n [n=9] };
\end{codeexample}
    %
\begin{codeexample}[preamble={\usetikzlibrary{graphs.standard}}]
% Directions can be changed
\tikz \graph [grid placement, branch up, grow left] { subgraph Grid_n [n=9] };
\end{codeexample}
    %
    In case a user-defined graph instead of a pre-defined |subgraph| is to be
    laid out using |grid placement|, |n| has to be specified explicitly:
    %
\begin{codeexample}[preamble={\usetikzlibrary{graphs}}]
\tikz \graph [grid placement] {
  [n=6, wrap after=3]
  a -- b -- c -- d -- e -- f
};
\end{codeexample}
    %
\end{key}


\subsubsection{Placement Taking Node Sizes Into Account}

Options like |grow up| or |branch right| do not take the sizes of the
to-be-positioned nodes into account -- all nodes are placed quite ``dumbly'' at
grid positions. It turns out that the |Cartesian placement| can also be used to
place notes in such a way that their height and/or width is taken into account.
Note, however, that while the following options may yield an adequate placement
in many situations, when you need advanced alignments you should use a |matrix|
or advanced offline strategies to place the nodes.

\begin{key}{/tikz/graphs/grow right sep=\meta{distance} (default 1em)}
    This key has several effects, but let us start with the bottom line: Nodes
    along a chain are placed in such a way that the left end of a new node is
    \meta{distance} from the right end of the previous node:
    %
\begin{codeexample}[preamble={\usetikzlibrary{graphs}}]
\tikz \graph [grow right sep, left anchor=east, right anchor=west] {
  start -- {
    long text -- {short, very long text} -- more text,
    long -- longer -- longest
  } -- end
};
\end{codeexample}
    %
    What happens internally is the following: First, the |anchor| of the nodes
    is set to |west| (or |north west| or |south west|, see below). Second, the
    logical width of a node is no longer |1|, but set to the actual width of
    the node (which we define as the horizontal difference between the |west|
    anchor and the |east| anchor) in points. Third, the |chain shift| is set to
    |(1pt,0pt)|.
\end{key}

\begin{key}{/tikz/graphs/grow left sep=\meta{distance} (default 1em)}
%
\begin{codeexample}[preamble={\usetikzlibrary{graphs}}]
\tikz \graph [grow left sep] { long -- longer -- longest };
\end{codeexample}
%
\end{key}

\begin{key}{/tikz/graphs/grow up sep=\meta{distance} (default 1em)}
%
\begin{codeexample}[preamble={\usetikzlibrary{graphs}}]
\tikz \graph [grow up sep] {
  a / $a=x$ --
  b / {$b=\displaystyle \int_0^1 x dx$} --
  c [draw, circle, inner sep=7mm]
};
\end{codeexample}
%
\end{key}

\begin{key}{/tikz/graphs/grow down sep=\meta{distance} (default 1em)}
    As above.
\end{key}

\begin{key}{/tikz/graphs/branch right sep=\meta{distance} (default 1em)}
    This key works like |grow right sep|, only it affects groups rather than
    chains.
    %
\begin{codeexample}[preamble={\usetikzlibrary{graphs}}]
\tikz \graph [grow down, branch right sep] {
  start -- {
    an even longer text -- {short, very long text} -- more text,
    long -- longer -- longest,
    some text -- a -- b
  } -- end
};
\end{codeexample}
    %
    When both this key and, say, |grow down sep| are set, instead of the |west|
    anchor, the |north west| anchor will be selected automatically.
\end{key}

\begin{key}{/tikz/graphs/branch left sep=\meta{distance} (default 1em)}
%
\begin{codeexample}[preamble={\usetikzlibrary{graphs}}]
\tikz \graph [grow down sep, branch left sep] {
  start -- {
    an even longer text -- {short, very long text} -- more text,
    long -- longer,
    some text -- a -- b
  } -- end
};
\end{codeexample}
%
\end{key}

\begin{key}{/tikz/graphs/branch up sep=\meta{distance} (default 1em)}
%
\begin{codeexample}[preamble={\usetikzlibrary{graphs}}]
\tikz \graph [branch up sep] { a, b, c[draw, circle, inner sep=7mm] };
\end{codeexample}
%
\end{key}

\begin{key}{/tikz/graphs/branch down sep=\meta{distance} (default 1em)}
\end{key}


\subsubsection{Placement On a Circle}

The following keys place nodes on circles. Note that, typically, you do not use
|circular placement| directly, but rather use one of the two keys |clockwise|
or |counterclockwise|.

\begin{key}{/tikz/graphs/circular placement}
    This key works quite similar to |Cartesian placement|. As for that
    placement strategy, a node has logical width and depth |1|. However, the
    computed total width and depth are mapped to polar coordinates rather than
    Cartesian coordinates.

    \begin{key}{/tikz/graphs/chain polar shift=|(|\meta{angle}|:|\meta{radius}|)| (initially {(0:1)})}
        Under the regime of the |circular placement| strategy, each node on a
        chain is shifted by
        |(|\meta{logical width}\meta{angle}|:|\meta{logical width}\meta{angle}|)|.
        %
\begin{codeexample}[preamble={\usetikzlibrary{graphs}}]
\tikz \graph [circular placement] {
  a -> b -> c;
  d -> e;
  f ->  g -> h;
};
\end{codeexample}
        %
    \end{key}
    %
    \begin{key}{/tikz/graphs/group polar shift=|(|\meta{angle}|:|\meta{radius}|)| (initially {(45:0)})}
        Like for |group shift|, each node on a chain is shifted by
        |(|\meta{logical depth}\meta{angle}|:|\meta{logical depth}\meta{angle}|)|.
        %
\begin{codeexample}[preamble={\usetikzlibrary{graphs}}]
\tikz \graph [circular placement, group polar shift=(30:0)] {
  a -> b -> c;
  d -> e;
  f -> g -> h;
};
\end{codeexample}
        %
\begin{codeexample}[preamble={\usetikzlibrary{graphs}}]
\tikz \graph [circular placement,
              chain polar shift=(30:0),
              group polar shift=(0:1cm)] {
  a -- b -- c;
  d -- e;
  f -- g -- h;
};
\end{codeexample}
    \end{key}
    %
    \begin{key}{/tikz/graphs/radius=\meta{dimension} (initially 1cm)}
        This is an initial value that is added to the total computed radius
        when the polar shift of a node has been calculated. Essentially, this
        key allows you to set the \meta{radius} of the innermost circle.
        %
\begin{codeexample}[preamble={\usetikzlibrary{graphs}}]
\tikz \graph [circular placement, radius=5mm] { a, b, c, d };
\end{codeexample}
        %
\begin{codeexample}[preamble={\usetikzlibrary{graphs}}]
\tikz \graph [circular placement, radius=1cm] { a, b, c, d };
\end{codeexample}
    \end{key}
    %
    \begin{key}{/tikz/graphs/phase=\meta{angle} (initially 90)}
        This is an initial value that is added to the total computed angle when
        the polar shift of a node has been calculated.
        %
\begin{codeexample}[preamble={\usetikzlibrary{graphs}}]
\tikz \graph [circular placement] { a, b, c, d };
\end{codeexample}
        %
\begin{codeexample}[preamble={\usetikzlibrary{graphs}}]
\tikz \graph [circular placement, phase=0] { a, b, c, d };
\end{codeexample}
    \end{key}
\end{key}

\label{key-graphs-clockwise}%
\begin{key}{/tikz/graphs/clockwise=\meta{number} (default \string\tikzgraphVnum)}
    This key sets the |group shift| so that if there are exactly \meta{number}
    many nodes in a group, they will form a complete circle. If you do not
    provide a \meta{number}, the current value of |\tikzgraphVnum| is used,
    which is exactly what you want when you use predefined graph macros like
    |subgraph K_n|.
    %
\begin{codeexample}[preamble={\usetikzlibrary{graphs}}]
\tikz \graph [clockwise=4] { a, b, c, d };
\end{codeexample}
    %
\begin{codeexample}[preamble={\usetikzlibrary{graphs.standard}}]
\tikz \graph [clockwise] { subgraph K_n [n=5] };
\end{codeexample}
    %
\end{key}

\label{key-graphs-counterclockwise}%
\begin{key}{/tikz/graphs/counterclockwise=\meta{number} (default \string\tikzgraphVnum)}
    Works like |clockwise|, only the direction is inverted.
\end{key}


\subsubsection{Levels and Level Styles}

As a graph is being parsed, the |graph| command keeps track of a parameter
called the \emph{level} of a node. Provided that the graph is actually
constructed in a tree-like manner, the level is exactly equal to the level of
the node inside this tree.

\begin{key}{/tikz/graphs/placement/level}
    This key stores a number that is increased for each element on a chain, but
    gets reset at the end of a group:
    %
\begin{codeexample}[preamble={\usetikzlibrary{graphs}}]
\tikz \graph [ branch down=5mm, typeset=
    \tikzgraphnodetext:\pgfkeysvalueof{/tikz/graphs/placement/level}]
{
  a -> {
    b,
    c -> {
      d,
      e -> {f,g},
      h
    },
    j
  }
};
\end{codeexample}
    %
    Unlike the parameters |depth| and |width| described in the next section,
    the key |level| is always available.
\end{key}

In addition to keeping track of the value of the |level| key, the |graph|
command also executes the following keys whenever it creates a node:

\begin{stylekey}{/tikz/graph/level=\meta{level}}
    This key gets executed for each newly created node with \meta{level} set to
    the current level of the node. You can use this key to, say, reconfigure
    the node distance or the node color.
\end{stylekey}

\begin{stylekey}{/tikz/graph/level \meta{level}}
    This key also gets executed for each newly created node with \meta{level}
    set to the current level of the node.
    %
\begin{codeexample}[preamble={\usetikzlibrary{graphs}}]
\tikz \graph [
  branch down=5mm,
  level 1/.style={nodes=red},
  level 2/.style={nodes=green!50!black},
  level 3/.style={nodes=blue}]
{
  a -> {
    b,
    c -> {
      d,
      e -> {f,g},
      h
    },
    j
  }
};
\end{codeexample}
    %
\begin{codeexample}[preamble={\usetikzlibrary{graphs}}]
\tikz \graph [
  branch down=5mm,
  level 1/.style={grow right=2cm},
  level 2/.style={grow right=1cm},
  level 3/.style={grow right=5mm}]
{
  a -> {
    b,
    c -> {
      d,
      e -> {f,g},
      h
    },
    j
  }
};
\end{codeexample}
    %
\end{stylekey}


\subsubsection{Defining New Online Placement Strategies}
\label{section-library-graphs-new-online}

In the following the details of how to define a new placement strategy are
explained. Most readers may wish to skip this section.

As a graph specification is being parsed, the |graphs| library will keep track
of different numbers that identify the positions of the nodes. Let us start
with what happens on a chain. First, the following counter is increased for
each element of the chain:
%
\begin{key}{/tikz/graphs/placement/element count}
    This key stores a number that tells us the position of the node on the
    current chain. However, you only have access to this value inside the code
    passed to the macro |compute position|, explained later on.
    %
\begin{codeexample}[preamble={\usetikzlibrary{graphs}}]
\tikz \graph [
  grow right sep, typeset=\tikzgraphnodetext:\mynum,
  placement/compute position/.append code=
    \pgfkeysgetvalue{/tikz/graphs/placement/element count}{\mynum}]
{
  a -> b -> c,
  d -> {e, f->h} -> j
};
\end{codeexample}
    %
    As can be seen, each group resets the element counter.
\end{key}

The second value that is computed is more complicated to explain, but it also
gives more interesting information:
%
\begin{key}{/tikz/graphs/placement/width}
    This key stores the ``logical width'' of the nodes parsed up to now in the
    current group or chain (more precisely, parsed since the last call of
    |place| in an enclosing group). This is not necessarily the ``total
    physical width'' of the nodes, but rather a number representing how ``big''
    the elements prior to the current element were. This \emph{may} be their
    width, but it may also be their height or even their number (which,
    incidentally, is the default). You can use the |width| to perform shifts or
    rotations of to-be-created nodes (to be explained later).

    The logical width is defined recursively as follows. First, the width of a
    single node is computed by calling the following key:
    %
    \begin{key}{/tikz/graphs/placement/logical node width=\meta{full node name}}
        This key is called to compute a physical or logical width of the node
        \meta{full node name}. You can change the code of this key. The code
        should return the computed value in the macro |\pgfmathresult|. By
        default, this key returns |1|.
    \end{key}
    %
    The width of a chain is the sum of the widths of its elements. The width of
    a group is the maximum of the widths of its elements.

    To get a feeling what the above rules imply in practice, let us first have
    a look at an example where each node has logical width and height |1|
    (which is the default). The arcane options at the beginning of the code
    just setup things so that the computed width and depth of each node is
    displayed at the bottom of each node.
    %
\begin{codeexample}[preamble={\usetikzlibrary{graphs}}]
\tikz
  \graph [nodes={align=center, inner sep=1pt}, grow right=7mm,
          typeset={\tikzgraphnodetext\\[-4pt]
                   \tiny\mywidth\\[-6pt]\tiny\mydepth},
          placement/compute position/.append code=
            \pgfkeysgetvalue{/tikz/graphs/placement/width}{\mywidth}
            \pgfkeysgetvalue{/tikz/graphs/placement/depth}{\mydepth}]
{
  a,
  b,
  c -> d -> {
    e -> f -> g,
    h -> i
  } -> j,
  k -> l
};
\end{codeexample}
    %
    In the next example the ``logical'' width and depth actually match the
    ``physical'' width and height. This is caused by the |grow right sep|
    option, which internally sets the |logical node width| key so that it
    returns the width of its parameter in points.
    %
\begin{codeexample}[preamble={\usetikzlibrary{graphs}}]
\tikz
  \graph [grow right sep, branch down sep, nodes={align=left, inner sep=1pt},
          typeset={\tikzgraphnodetext\\[-4pt] \tiny Width: \mywidth\\[-6pt] \tiny Depth: \mydepth},
          placement/compute position/.append code=
            \pgfkeysgetvalue{/tikz/graphs/placement/width}{\mywidth}
            \pgfkeysgetvalue{/tikz/graphs/placement/depth}{\mydepth}]
{
  a,
  b,
  c -> d -> {
    e -> f -> g,
    h -> i
  } -> j,
  k -> l
};
\end{codeexample}
    %
\end{key}

Symmetrically to chains, as a group is being constructed, counters are
available for the number of chains encountered so far in the current group and
for the logical depth of the current group:
%
\begin{key}{/tikz/graphs/placement/chain count}
    This key stores a number that tells us the sequence number of the
    chain in the current group.
    %
\begin{codeexample}[preamble={\usetikzlibrary{graphs}}]
\tikz \graph [
  grow right sep, branch down=5mm, typeset=\tikzgraphnodetext:\mynum,
  placement/compute position/.append code=
    \pgfkeysgetvalue{/tikz/graphs/placement/chain count}{\mynum}]
{
  a -> b -> {c,d,e},
  f,
  g -> h
};
\end{codeexample}
    %
\end{key}

\begin{key}{/tikz/graphs/placement/depth}
    Similarly to the |width| key, this key stores the ``logical depth'' of the
    nodes parsed up to now in the current group or chain and, also similarly,
    this key may or may not be related to the actual depth/height of the
    current node. As for the |width|, the exact definition is as follows: For a
    single node, the depth is computed by the following key:
    %
    \begin{key}{/tikz/graphs/placement/logical node depth=\meta{full node name}}
        The code behind this key should return the ``logical height'' of the
        node \meta{full node name} in the macro |\pgfmathresult|.
    \end{key}
    %
    Second, the depth of a group is the sum of the depths of its elements.
    Third, the depth of a chain is the maximum of the depth of its elements.
\end{key}

The |width|, |depth|, |element count|, and |chain count| keys get updated
automatically, but do not have an effect by themselves. This is to the
following two keys:

\begin{key}{/tikz/graphs/placement/compute position=\meta{code}}
    The \meta{code} is called by the |graph| command just prior to creating a
    new node (the exact moment when this key is called is detailed in the
    description of the |place| key). When the \meta{code} is called, all of the
    keys described above will hold numbers computed in the way described above.

    The job of the \meta{code} is to setup node options appropriately so that
    the to-be-created node will be placed correctly. Thus, the \meta{code}
    should typically set the key |nodes={shift=|\meta{coordinate}|}| where
    \meta{coordinate} is the computed position for the node. The \meta{code}
    could also set other options like, say, the color of a node depending on
    its depth.

    The following example appends some code to the standard code of
    |compute position| so that ``deeper'' nodes of a tree are lighter.
    (Naturally, the same effect could be achieved much more easily using the
    |level| key.)
    %
\begin{codeexample}[preamble={\usetikzlibrary{graphs}}]
\newcount\mycount
\def\lightendeepernodes{
  \pgfmathsetcount{\mycount}{
    100-20*\pgfkeysvalueof{/tikz/graphs/placement/width}
  }
  \edef\mydepth{\the\mycount}
  \tikzset{nodes={fill=red!\mydepth,circle,text=white}}
}
\tikz
  \graph [placement/compute position/.append code=\lightendeepernodes]
   {
     a -> {
       b -> c -> d,
       e -> {
         f,
         g
       },
       h
     }
   };
\end{codeexample}
    %
\end{key}

\begin{key}{/tikz/graphs/placement/place}
    Executing this key has two effects: First, the key |compute position| is
    called to compute a good position for future nodes (usually, these ``future
    nodes'' are just a single node that is created immediately). Second, all of
    the above counters like |depth| or |width| are reset (but not |level|).

    There are two places where this key is sensibly called: First, just prior
    to creating a node, which happens automatically. Second, when you change
    the online strategy. In this case, the computed width and depth values from
    one strategy typically make no sense in the other strategy, which is why
    the new strategy should proceed ``from a fresh start''. In this case, the
    implicit call of |compute position| ensures that the new strategy gets the
    last place the old strategy would have used as its starting point, while
    the computation of its positions is now relative to this new starting
    point.

    For these reasons, when an online strategy like |Cartesian placement| is
    called, this key gets called implicitly. You will rarely need to call this
    key directly, except when you define a new online strategy.
\end{key}


\subsection{Reference: Predefined Elements}
\label{section-library-graphs-reference}

\subsubsection{Graph Macros}
\label{section-library-graphs-reference-macros}

\begin{tikzlibrary}{graphs.standard}
    This library defines a number of graph macros that are often used in the
    literature. When new graphs are added to this collection, they will follow
    the definitions in the Mathematica program, see
    \url{mathworld.wolfram.com/topics/SimpleGraphs.html}.
\end{tikzlibrary}

\begin{graph}{subgraph I\_n}
    This graph consists just of $n$ unconnected vertices. The following key is
    used to specify the set of these vertices:
    %
    \begin{key}{/tikz/graphs/V=\marg{list of vertices}}
        Sets a list of vertex names for use with graphs like |subgraph I_n| and
        also other graphs. This list is available in the macro |\tikzgraphV|.
        The number of elements of this list is available in |\tikzgraphVnum|.
    \end{key}
    %
    \begin{key}{/tikz/graphs/n=\meta{number}}
        This is an abbreviation for
        |V={1,...,|\meta{number}|}, name shore V/.style={name=V}|.
    \end{key}
    %
\begin{codeexample}[preamble={\usetikzlibrary{graphs.standard}}]
\tikz \graph [branch right, nodes={draw, circle}]
  { subgraph I_n [V={a,b,c}] };
\end{codeexample}
    %
    This graph is not particularly exciting by itself. However, it is often
    used to introduce nodes into a graph that are then connected as in the
    following example:
    %
\begin{codeexample}[preamble={\usetikzlibrary{graphs.standard}}]
\tikz \graph [clockwise, clique] { subgraph I_n [n=4] };
\end{codeexample}
    %
\end{graph}

\begin{graph}{subgraph I\_nm}
    This graph consists of two sets of once $n$ unconnected vertices and then
    $m$ unconnected vertices. The first set consists of the vertices set by the
    key |V|, the other set consists of the vertices set by the key |W|.
    %
\begin{codeexample}[preamble={\usetikzlibrary{graphs.standard}}]
\tikz \graph { subgraph I_nm [V={1,2,3}, W={a,b,c}] };
\end{codeexample}
    %
    In order to set the graph path name of the two sets, the following keys get
    executed:
    %
    \begin{stylekey}{/tikz/graphs/name shore V (initially \normalfont empty)}
        Set this style to, say, |name=my V set| in order to set a name for the
        |V| set.
    \end{stylekey}
    %
    \begin{stylekey}{/tikz/graphs/name shore W (initially \normalfont empty)}
        Same as for |name shore V|.
    \end{stylekey}
    %
    \begin{key}{/tikz/graphs/W=\marg{list of vertices}}
        Sets the list of vertices for the |W| set. The elements and their
        number are available in the macros |\tikzgraphW| and |\tikzgraphWnum|,
        respectively.
    \end{key}
    %
    \begin{key}{/tikz/graphs/m=\meta{number}}
        This is an abbreviation for
        |W={1,...,|\meta{number}|}, name shore W/.style={name=W}|.
    \end{key}
    %
    The main purpose of this subgraph is to setup the nodes in a bipartite
    graph:
    %
\begin{codeexample}[preamble={\usetikzlibrary{graphs.standard}}]
\tikz \graph {
  subgraph I_nm [n=3, m=4];

  V 1 -- { W 2, W 3 };
  V 2 -- { W 1, W 3 };
  V 3 -- { W 1, W 4 };
};
\end{codeexample}
    %
\end{graph}

\begin{graph}{subgraph K\_n}
    This graph is the complete clique on the vertices from the |V| key.
    %
\begin{codeexample}[preamble={\usetikzlibrary{graphs.standard}}]
\tikz \graph [clockwise] { subgraph K_n [n=7] };
\end{codeexample}
    %
\end{graph}

\begin{graph}{subgraph K\_nm}
    This graph is the complete bipartite graph with the two shores |V| and |W|
    as in |subgraph I_nm|.
    %
\begin{codeexample}[preamble={\usetikzlibrary{graphs.standard}}]
\tikz \graph [branch right, grow down]
  { subgraph K_nm [V={6,...,9}, W={b,...,e}] };
\end{codeexample}
    %
\begin{codeexample}[preamble={\usetikzlibrary{graphs.standard}}]
\tikz \graph [simple, branch right, grow down]
{
  subgraph K_nm [V={1,2,3}, W={a,b,c,d}, ->];
  subgraph K_nm [V={2,3},   W={b,c},     <-];
};
\end{codeexample}
    %
\end{graph}

\begin{graph}{subgraph P\_n}
    This graph is the path on the vertices in |V|.
    %
\begin{codeexample}[preamble={\usetikzlibrary{graphs.standard}}]
\tikz \graph [branch right] { subgraph P_n [n=3] };
\end{codeexample}
    %
\end{graph}

\begin{graph}{subgraph C\_n}
    This graph is the cycle on the vertices in |V|.
    %
\begin{codeexample}[preamble={\usetikzlibrary{graphs.standard}}]
\tikz \graph [clockwise] { subgraph C_n [n=7, ->] };
\end{codeexample}
    %
\end{graph}

\begin{graph}{subgraph Grid\_n}
    This graph is a grid of the vertices in |V|.
    %
    \begin{key}{/tikz/graphs/wrap after=\meta{number}}
        Defines the number of nodes placed in a single row of the grid. This
        value implicitly defines the number of grid columns as well. In the
        following example a |grid placement| is used to visualize the edges
        created between the nodes of a |Grid_n| |subgraph| using different
        values for |wrap after|.
        %
\begin{codeexample}[preamble={\usetikzlibrary{graphs.standard}}]
\tikz \graph [grid placement] { subgraph Grid_n [n=3,wrap after=1] };
\tikz \graph [grid placement] { subgraph Grid_n [n=3,wrap after=3] };
\end{codeexample}
        %
\begin{codeexample}[preamble={\usetikzlibrary{graphs.standard}}]
\tikz \graph [grid placement] { subgraph Grid_n [n=4,wrap after=2] };
\tikz \graph [grid placement] { subgraph Grid_n [n=4] };
\end{codeexample}
  \end{key}
\end{graph}

% TODO: Implement the Grid_nm subgraph described here:
%
%\begin{graph}{subgraph Grid\_nm}
%  This graph is a grid built from the cartesian product of the two node
%  sets |V| and |W| which are either defined using the keys
%  |/tikz/graphs/V| and |/tikz/graphs/W| or |/tikz/graphs/n| and
%  |/tikz/graphs/m| or a mixture of both.
%
%  The resulting |Grid_nm| subgraph has $n$ ``rows'' and $m$ ``columns'' and
%  the nodes are named |V i W j| with $1\le i\le n$ and $1\le j\le n$.
%  The names of the two shores |V| and |W| can be changed as described in
%  the documentation of the keys |/tikz/graphs/name shore V| and
%  |/tikz/graphs/name shore W|.
%  \begin{codeexample}[preamble={\usetikzlibrary{graphs}}]
%\tikz \graph [grid placement] { subgraph Grid_nm [V={1,2,3}, W={4, 5, 6}] };
%  \end{codeexample}
%\end{graph}


\subsubsection{Group Operators}

The following keys use the |operator| key to setup operators that connect the
vertices of the current group having a certain color in a specific way.

\begin{key}{/tikz/graphs/clique=\meta{color} (default all)}
    Adds an edge between all vertices of the current group having the (logical)
    color \meta{color}. Since, by default, this color is set to |all|, which is
    a color that all nodes get by default, when you do not specify anything,
    all nodes will be connected.
    %
\begin{codeexample}[preamble={\usetikzlibrary{graphs}}]
\tikz \graph [clockwise, n=5] {
  a,
  b,
  {
    [clique]
    c, d, e
  }
};
\end{codeexample}
    %
\begin{codeexample}[preamble={\usetikzlibrary{graphs}}]
\tikz \graph [color class=red, clockwise, n=5] {
  [clique=red, ->]
  a, b[red], c[red], d, e[red]
};
\end{codeexample}
    %
\end{key}

\begin{key}{/tikz/graphs/induced independent set=\meta{color} (default all)}
    This key is the ``opposite'' of a |clique|: It removes all edges in the
    current group having belonging to color class \meta{color}. More precisely,
    an edge of kind |-!-| is added for each pair of vertices. This means that
    edge only get removed if you specify the |simple| option.
    %
\begin{codeexample}[preamble={\usetikzlibrary{graphs.standard}}]
\tikz \graph [simple] {
  subgraph K_n [<->, n=7, clockwise]; % create lots of edges

  { [induced independent set] 1, 3, 4, 5, 6 }
};
\end{codeexample}
    %
\end{key}

\begin{key}{/tikz/graphs/cycle=\meta{color} (default all)}
    Connects the nodes colored \meta{color} is a cyclic fashion. The ordering
    is the ordering in which they appear in the whole graph specification.
    %
\begin{codeexample}[preamble={\usetikzlibrary{graphs}}]
\tikz \graph [clockwise, n=6, phase=60] {
  { [cycle, ->] a, b, c },
  { [cycle, <-] d, e, f }
};
\end{codeexample}
    %
\end{key}

\begin{key}{/tikz/graphs/induced cycle=\meta{color} (default all)}
    While the |cycle| command will only add edges, this key will also remove
    all other edges between the nodes of the cycle, provided we are
    constructing a |simple| graph.
    %
\begin{codeexample}[preamble={\usetikzlibrary{graphs.standard}}]
\tikz \graph [simple] {
  subgraph K_n [n=7, clockwise]; % create lots of edges

  { [induced cycle, ->, edge=red] 2, 3, 4, 6, 7 },
};
\end{codeexample}
    %
\end{key}

\begin{key}{/tikz/graphs/path=\meta{color} (default all)}
    Works like |cycle|, only there is no edge from the last to the first
    vertex.
    %
\begin{codeexample}[preamble={\usetikzlibrary{graphs}}]
\tikz \graph [clockwise, n=6] {
  { [path, ->] a, b, c },
  { [path, <-] d, e, f }
};
\end{codeexample}
    %
\end{key}

\begin{key}{/tikz/graphs/induced path=\meta{color} (default all)}
    Works like |induced cycle|, only there is no edge from the last to the
    first vertex.
    %
\begin{codeexample}[preamble={\usetikzlibrary{graphs.standard}}]
\tikz \graph [simple] {
  subgraph K_n [n=7, clockwise]; % create lots of edges

  { [induced path, ->, edges=red] 2, 3, 4, 6, 7 },
};
\end{codeexample}
    %
\end{key}


\subsubsection{Joining Operators}

The following keys are typically used as options of an \meta{edge
specification}, but can also be called in a group specification (however, then,
the colors need to be set explicitly).

\begin{key}{/tikz/graphs/complete bipartite=\meta{from color}\meta{to color} (default \char`\{source'\char`\}\char`\{target'\char`\})}
    Adds all possible edges from every node having color \meta{from color} to
    every node having color \meta{to color}:
    %
\begin{codeexample}[preamble={\usetikzlibrary{graphs}}]
\tikz \graph { {a, b}       ->[complete bipartite]
               {c, d, e}    --[complete bipartite]
               {g, h, i, j} --[complete bipartite]
               k };
\end{codeexample}
    %
\begin{codeexample}[preamble={\usetikzlibrary{graphs}}]
\tikz \graph [color class=red, color class=green, clockwise, n=6] {
  [complete bipartite={red}{green}, ->]
  a [red], b[red], c[red], d[green], e[green], f[green]
};
\end{codeexample}
    %
\end{key}

\begin{key}{/tikz/graphs/induced complete bipartite}
    Works like the |complete bipartite| operator, but in a |simple| graph any
    edges between the vertices in either shore are removed (more precisely,
    they get replaced by |-!-| edges).
    %
\begin{codeexample}[preamble={\usetikzlibrary{graphs.standard}}]
\tikz \graph [simple] {
  subgraph K_n [n=5, clockwise];  % Lots of edges

  {2, 3} ->[induced complete bipartite] {4, 5}
};
\end{codeexample}
    %
\end{key}

\begin{key}{/tikz/graphs/matching=\meta{from color}\meta{to color} (default \char`\{source'\char`\}\char`\{target'\char`\})}
    This joining operator forms a maximum \emph{matching} between the nodes of
    the two sets of nodes having colors \meta{from color} and \meta{to color},
    respectively. The first node of the from set is connected to the first node
    of to set, the second node of the from set is connected to the second node
    of the to set, and so on. If the sets have the same size, what results is
    what graph theoreticians call a \emph{perfect matching}, otherwise only a
    maximum, but not perfect matching results.
    %
\begin{codeexample}[preamble={\usetikzlibrary{graphs}}]
\tikz \graph {
  {a, b, c} ->[matching]
  {d, e, f} --[matching]
  {g, h}    --[matching]
  {i, j, k}
};
\end{codeexample}
    %
\end{key}

\begin{key}{/tikz/graphs/matching and star=\meta{from color}\meta{to color} (default \char`\{source'\char`\}\char`\{target'\char`\})}
    The |matching and star| connector works like the |matching| connector, only
    it behaves differently when the two to-be-connected sets have different
    size. In this case, all the surplus nodes get connected to the last node of
    the other set, resulting in what is known as a \emph{star} in graph theory.
    This simple rule allows for some powerful effects (since this connector is
    the one initially set, there is no need to add it here):
    %
\begin{codeexample}[preamble={\usetikzlibrary{graphs}}]
\tikz \graph { a -> {b, c} -> {d, e} -- f};
\end{codeexample}
    %
    The |matching and star| connector also makes it easy to create trees and
    series-parallel graphs.
\end{key}

\begin{key}{/tikz/graphs/butterfly=\opt{\meta{options}}}
    The |butterfly| connector is used to create the kind of connections present
    between layers of a so-called \emph{butterfly network}. As for other
    connectors, two sets of nodes are connected, which are the nodes having
    color |target'| and |source'| by default. In a \emph{level $l$} connection,
    the first $l$ nodes of the first set are connected to the second $l$ nodes
    of the second set, while the second $l$ nodes of the first set get
    connected to the first $l$ nodes of the second set. Then, for next $2l$
    nodes of both sets a similar kind of connection is installed. Additionally,
    each node gets connected to the corresponding node in the other set with
    the same index (as in a |matching|):
    %
\begin{codeexample}[preamble={\usetikzlibrary{graphs.standard}}]
\tikz \graph [left anchor=east, right anchor=west,
              branch down=4mm, grow right=15mm] {
  subgraph I_n [n=12, name=A] --[butterfly={level=3}]
  subgraph I_n [n=12, name=B] --[butterfly={level=2}]
  subgraph I_n [n=12, name=C]
};
\end{codeexample}
    %
    Unlike most joining operators, the colors of the nodes in the first and the
    second set are not passed as parameters to the |butterfly| key. Rather,
    they can be set using the \meta{options}, which are executed with the path
    prefix |/tikz/graphs/butterfly|.
    %
    \begin{key}{/tikz/graphs/butterfly/level=\meta{level} (initially 1)}
        Sets the level $l$ for the connections.
    \end{key}
    %
    \begin{key}{/tikz/graphs/butterfly/from=\meta{color} (initially target')}
        Sets the color class of the from nodes.
    \end{key}
    %
    \begin{key}{/tikz/graphs/butterfly/to=\meta{color} (initially source')}
        Sets the color class of the to nodes.
    \end{key}
\end{key}


%%% Local Variables:
%%% mode: latex
%%% TeX-master: "pgfmanual-pdftex-version"
%%% End:

% % Copyright 2006 by Till Tantau
%
% This file may be distributed and/or modified
%
% 1. under the LaTeX Project Public License and/or
% 2. under the GNU Free Documentation License.
%
% See the file doc/generic/pgf/licenses/LICENSE for more details.


\section{Matrices and Alignment}
\label{section-matrices}

\subsection{Overview}

When creating pictures, one often faces the problem of correctly aligning parts
of the picture. For example, you might wish that the |baseline|s of certain
nodes should be on the same line and some further nodes should be below these
nodes with, say, their centers on a vertical lines. There are different ways of
solving such problems. For example, by making clever use of anchors, nearly all
such alignment problems can be solved. However, this often leads to complicated
code. An often simpler way is to use \emph{matrices}, the use of which is
explained in the current section.

A \tikzname\ matrix is similar to \LaTeX's |{tabular}| or |{array}|
environment, only instead of text each cell contains a little picture or a
node. The sizes of the cells are automatically adjusted such that they are
large enough to contain all the cell contents.

Matrices are a powerful tool and they need to be handled with some care. For
impatient readers who skip the rest of this section: you \emph{must} end
\emph{every} row with |\\|. In particular, the last row \emph{must} be ended
with |\\|.

Many of the ideas implemented in \tikzname's matrix support are due to Mark
Wibrow -- many thanks to Mark at this point!


\subsection{Matrices are Nodes}

Matrices are special in many ways, but for most purposes matrices are treated
like nodes. This means, that you use the |node| path command to create a matrix
and you only use a special option, namely the |matrix| option, to signal that
the node will contain a matrix. Instead of the usual \TeX-box that makes up the
|text| part of the node's shape, the matrix is used. Thus, in particular, a
matrix can have a shape, this shape can be drawn or filled, it can be used in a
tree, and so on. Also, you can refer to the different anchors of a matrix.

\begin{key}{/tikz/matrix=\meta{true or false} (default true)}
    This option can be passed to a |node| path command. It signals that the
    node will contain a matrix.
    %
\begin{codeexample}[]
\begin{tikzpicture}
  \draw[help lines] (0,0) grid (4,2);
  \node [matrix,fill=red!20,draw=blue,very thick] (my matrix) at (2,1)
  {
    \draw (0,0)   circle (4mm); & \node[rotate=10] {Hello};        \\
    \draw (0.2,0) circle (2mm); & \fill[red]   (0,0) circle (3mm); \\
  };

  \draw [very thick,->] (0,0) |- (my matrix.west);
\end{tikzpicture}
\end{codeexample}
    %
    The exact syntax of the matrix is explained in the course of this section.
    %
    \begin{stylekey}{/tikz/every matrix (initially \normalfont empty)}
        This style is used in every matrix.
    \end{stylekey}
    %
    \begin{stylekey}{/tikz/every outer matrix (initially \normalfont empty)}
        While the |every matrix| key also applies to the matrix contents, this
        only applies to the outer node which holds the matrix.
    \end{stylekey}
\end{key}

Even more so than nodes, matrices will often be the only object on a path.
Because of this, there is a special abbreviation for creating matrices:

\begin{command}{\matrix}
    Inside |{tikzpicture}| this is an abbreviation for |\path node[matrix]|.
\end{command}

Even though matrices are nodes, some options do not have the same effect as for
normal nodes:
%
\begin{enumerate}
    \item Rotations and scaling have no effect on a matrix as a whole (however,
        you can still transform the contents of the cells normally). Before the
        matrix is typeset, the rotational and scaling part of the
        transformation matrix is reset.
    \item For multi-part shapes you can only set the |text| part of the node.
    \item All options starting with |text| such as |text width| have no effect.
    \item If you place a matrix on a path, the matrix contents will be
        collected into a macro, which tokenizes them.  This means that |&| will
        lose its meaning as an alignment character, resulting in an error.  If
        you need to place a matrix on a path, use |ampersand replacement| to
        work around that problem.
\end{enumerate}


\subsection{Cell Pictures}
\label{section-tikz-cell-pictures}

A matrix consists of rows of \emph{cells}. Each row (including the last one!)
is ended by the command |\\|. The character |&| is used to separate cells.
Inside each cell, you must place commands for drawing a picture, called the
\emph{cell picture} in the following. (However, cell pictures are not enclosed
in a complete |{pgfpicture}| environment, they are a bit more light-weight. The
main difference is that cell pictures cannot have layers.) It is not necessary
to specify beforehand how many rows or columns there are going to be and if a
row contains less cell pictures than another line, empty cells are
automatically added as needed.


\subsubsection{Alignment of Cell Pictures}

For each cell picture a bounding box is computed. These bounding boxes and the
origins of the cell pictures determine how the cells are aligned. Let us start
with the rows: Consider the cell pictures on the first row. Each has a bounding
box and somewhere inside this bounding box the origin of the cell picture can
be found (the origin might even lie outside the bounding box, but let us ignore
this problem for the moment). The cell pictures are then shifted around such
that all origins lie on the same horizontal line. This may make it necessary to
shift some cell pictures upwards and others downwards, but it can be done and
this yields the vertical alignment of the cell pictures this row. The top of
the row is then given by the top of the ``highest'' cell picture in the row,
the bottom of the row is given by the bottom of the lowest cell picture. (To be
more precise, the height of the row is the maximum $y$-value of any of the
bounding boxes and the depth of the row is the negated minimum $y$-value of the
bounding boxes).
%
\begin{codeexample}[]
\begin{tikzpicture}
  [every node/.style={draw=black,anchor=base,font=\huge}]

  \matrix [draw=red]
  {
    \node {a}; \fill[blue] (0,0) circle (2pt); &
    \node {X}; \fill[blue] (0,0) circle (2pt); &
    \node {g}; \fill[blue] (0,0) circle (2pt); \\
  };
\end{tikzpicture}
\end{codeexample}

Each row is aligned in this fashion: For each row the cell pictures
are vertically aligned such that the origins lie on the same
line. Then the second row is placed below the first row such that the
bottom of the first row touches the top of the second row (unless a
|row sep| is used to add a bit of space). Then the bottom of the
second row touches the top of the third row, and so on. Typically,
each row will have an individual height and depth.
%
\begin{codeexample}[]
\begin{tikzpicture}
  [every node/.style={draw=black,anchor=base}]

  \matrix [draw=red]
  {
    \node {a}; & \node {X}; & \node {g}; \\
    \node {a}; & \node {X}; & \node {g}; \\
  };

  \matrix [row sep=3mm,draw=red] at (0,-2)
  {
    \node {a}; & \node {X}; & \node {g}; \\
    \node {a}; & \node {X}; & \node {g}; \\
  };
\end{tikzpicture}
\end{codeexample}

Let us now have a look at the columns. The rules for how the pictures on any
given column are aligned are very similar to the row alignment: Consider all
cell pictures in the first column. Each is shifted horizontally such that the
origins lie on the same vertical line. Then, the left end of the column is at
the left end of the bounding box that protrudes furthest to the left. The right
end of the column is at the right end of the bounding box that protrudes
furthest to the right. This fixes the horizontal alignment of the cell pictures
in the first column and the same happens the cell pictures in the other
columns. Then, the right end of the first column touches the left end of the
second column (unless |column sep| is used). The right end of the second column
touches the left end of the third column, and so on. (Internally, two columns
are actually used to achieve the desired horizontal alignment, but that is only
an implementation detail.)
%
\begin{codeexample}[]
\begin{tikzpicture}[every node/.style={draw}]
  \matrix [draw=red]
  {
    \node[left]  {Hallo}; \fill[blue] (0,0) circle (2pt); \\
    \node        {X};     \fill[blue] (0,0) circle (2pt); \\
    \node[right] {g};     \fill[blue] (0,0) circle (2pt); \\
  };
\end{tikzpicture}
\end{codeexample}

\begin{codeexample}[]
\begin{tikzpicture}[every node/.style={draw}]
  \matrix [draw=red,column sep=1cm]
  {
    \node {8}; & \node{1}; & \node {6}; \\
    \node {3}; & \node{5}; & \node {7}; \\
    \node {4}; & \node{9}; & \node {2}; \\
  };
\end{tikzpicture}
\end{codeexample}


\subsubsection{Setting and Adjusting Column and Row Spacing}

There are different ways of setting and adjusting the spacing between columns
and rows. First, you can use the options |column sep| and |row sep| to set a
default spacing for all rows and all columns. Second, you can add options to
the |&| character and the |\\| command to adjust the spacing between two
specific columns or rows. Additionally, you can specify whether the space
between two columns or rows should be considered between the origins of cells
in the column or row or between their borders.

\begin{key}{/tikz/column sep=\meta{spacing list}}
    This option sets a default space that is added between every two columns.
    This space can be positive or negative and is zero by default. The
    \meta{spacing list} normally contains a single dimension like |2pt|.
    %
\begin{codeexample}[]
\begin{tikzpicture}
  \matrix [draw,column sep=1cm,nodes=draw]
  {
    \node(a) {123}; & \node (b) {1};   & \node {1}; \\
    \node    {12};  & \node     {12};  & \node {1}; \\
    \node(c) {1};   & \node (d) {123}; & \node {1}; \\
  };
  \draw [red,thick]  (a.east) -- (a.east |- c)
                     (d.west) -- (d.west |- b);
  \draw [<->,red,thick] (a.east) -- (d.west |- b)
    node [above,midway] {1cm};
\end{tikzpicture}
\end{codeexample}
    %
    More generally, the \meta{spacing list} may contain a whole list of
    numbers, separated by commas, and occurrences of the two key words
    |between origins| and |between borders|. The effect of specifying such a
    list is the following: First, all numbers occurring in the list are simply
    added to compute the final spacing. Second, concerning the two keywords,
    the last occurrence of one of the keywords is important. If the last
    occurrence is |between borders| or if neither occurs, then the space is
    inserted between the two columns normally. However, if the last occurs is
    |between origins|, then the following happens: The distance between the
    columns is adjusted such that the difference between the origins of all the
    cells in the first column (remember that they all lie on straight line) and
    the origins of all the cells in the second column is exactly the given
    distance.

    \emph{The }|between origins|\emph{ option can only be used for columns
    mentioned in the first row, that is, you cannot specify this option for
    columns introduced only in later rows.}
    %
\begin{codeexample}[]
\begin{tikzpicture}
  \matrix [draw,column sep={1cm,between origins},nodes=draw]
  {
    \node(a) {123}; & \node (b) {1};   & \node {1}; \\
    \node    {12};  & \node     {12};  & \node {1}; \\
    \node    {1};   & \node     {123}; & \node {1}; \\
  };
  \draw [<->,red,thick] (a.center) -- (b.center) node [above,midway] {1cm};
\end{tikzpicture}
\end{codeexample}
    %
\end{key}

\begin{key}{/tikz/row sep=\meta{spacing list}}
    This option works like |column sep|, only for rows. Here, too, you can
    specify whether the space is added between the lower end of the first row
    and the upper end of the second row, or whether the space is computed
    between the origins of the two rows.
    %
\begin{codeexample}[]
\begin{tikzpicture}
  \matrix [draw,row sep=1cm,nodes=draw]
  {
    \node (a) {123}; & \node {1};   & \node {1}; \\
    \node (b) {12};  & \node {12};  & \node {1}; \\
    \node     {1};   & \node {123}; & \node {1}; \\
  };
  \draw [<->,red,thick] (a.south) -- (b.north) node [right,midway] {1cm};
\end{tikzpicture}
\end{codeexample}
    %
\begin{codeexample}[]
\begin{tikzpicture}
  \matrix [draw,row sep={1cm,between origins},nodes=draw]
  {
    \node (a) {123}; & \node {1};   & \node {1}; \\
    \node (b) {12};  & \node {12};  & \node {1}; \\
    \node     {1};   & \node {123}; & \node {1}; \\
  };
  \draw [<->,red,thick] (a.center) -- (b.center) node [right,midway] {1cm};
\end{tikzpicture}
\end{codeexample}
    %
\end{key}

The row-end command |\\| allows you to provide an optional argument, which must
be a dimension. This dimension will be added to the list in |row sep|. This
means that, firstly, any numbers you list in this argument will be added as an
extra row separation between the line being ended and the next line and,
secondly, you can use the keywords |between origins| and |between borders| to
locally overrule the standard setting for this line pair.
%
\begin{codeexample}[]
\begin{tikzpicture}
  \matrix [row sep=1mm]
  {
    \draw (0,0) circle (2mm); & \draw (0,0) circle (2mm); \\
    \draw (0,0) circle (2mm); & \draw (0,0) circle (2mm); \\[-1mm]
    \draw (0,0) coordinate (a) circle (2mm); &
    \draw (0,0) circle (2mm); \\[1cm,between origins]
    \draw (0,0) coordinate (b) circle (2mm); &
    \draw (0,0) circle (2mm); \\
  };
  \draw [<->,red,thick] (a.center) -- (b.center) node [right,midway] {1cm};
\end{tikzpicture}
\end{codeexample}

The cell separation character |&| also takes an optional argument, which must
also be a spacing list. This spacing list is added to the |column sep| having a
similar effect as the option for the |\\| command for rows.

This optional spacing list can only be given the first time a new column is
started (usually in the first row), subsequent usages of this option in later
rows have no effect.
%
\begin{codeexample}[]
\begin{tikzpicture}
  \matrix [draw,nodes=draw,column sep=1mm]
  {
    \node {8}; &[2mm] \node{1}; &[-1mm] \node {6}; \\
    \node {3}; &      \node{5}; &       \node {7}; \\
    \node {4}; &      \node{9}; &       \node {2}; \\
  };
\end{tikzpicture}
\end{codeexample}
%
\begin{codeexample}[]
\begin{tikzpicture}
  \matrix [draw,nodes=draw,column sep=1mm]
  {
    \node {8}; &[2mm] \node(a){1}; &[1cm,between origins] \node(b){6}; \\
    \node {3}; &      \node   {5}; &                      \node   {7}; \\
    \node {4}; &      \node   {9}; &                      \node   {2}; \\
  };
  \draw [<->,red,thick] (a.center) -- (b.center) node [above,midway] {11mm};
\end{tikzpicture}
\end{codeexample}
%
\begin{codeexample}[]
\begin{tikzpicture}
  \matrix [draw,nodes=draw,column sep={1cm,between origins}]
  {
    \node (a) {8}; & \node (b) {1}; &[between borders] \node (c) {6}; \\
    \node     {3}; & \node     {5}; &                  \node     {7}; \\
    \node     {4}; & \node     {9}; &                  \node     {2}; \\
  };
  \draw [<->,red,thick] (a.center) -- (b.center) node [above,midway] {10mm};
  \draw [<->,red,thick] (b.east) -- (c.west) node [above,midway] {10mm};
\end{tikzpicture}
\end{codeexample}


\subsubsection{Cell Styles and Options}

The following styles and options are useful for changing the appearance of all
cell pictures:

\begin{stylekey}{/tikz/every cell=\marg{row}\marg{column} (initially \normalfont empty)}
    This style is installed at the beginning of each cell picture with the two
    parameters being the current \meta{row} and \meta{column} of the cell. Note
    that setting this style to |draw| will \emph{not} cause all nodes to be
    drawn since the |draw| option has to be passed to each node individually.

    Inside this style (and inside all cells), the current \meta{row} and
    \meta{column} number are also accessible via the counters
    |\pgfmatrixcurrentrow| and |\pgfmatrixcurrentcolumn|.
\end{stylekey}

\begin{key}{/tikz/cells=\meta{options}}
    This key adds the \meta{options} to the style |every cell|. It is mainly
    just a shorthand for the code |every cell/.append style=|\meta{options}.
\end{key}

\begin{key}{/tikz/nodes=\meta{options}}
    This key adds the \meta{options} to the style |every node|. It is mainly
    just a shorthand for the code |every node/.append style=|\meta{options}.

    The main use of this option is the install some options for the nodes
    \emph{inside} the matrix that should not apply to the matrix \emph{itself}.
    %
\begin{codeexample}[]
\begin{tikzpicture}
  \matrix [nodes={fill=blue!20,minimum size=5mm}]
  {
    \node {8}; & \node{1}; & \node {6}; \\
    \node {3}; & \node{5}; & \node {7}; \\
    \node {4}; & \node{9}; & \node {2}; \\
  };
\end{tikzpicture}
\end{codeexample}
    %
\end{key}

The next set of styles can be used to change the appearance of certain rows,
columns, or cells. If more than one of these styles is defined, they are
executed in the below order (the |every cell| style is executed before all of
the below).
    %
\begin{stylekey}{/tikz/column \meta{number}}
    This style is used for every cell in column \meta{number}.
\end{stylekey}

\begin{stylekey}{/tikz/every odd column}
    This style is used for every cell in an odd column.
\end{stylekey}

\begin{stylekey}{/tikz/every even column}
    This style is used for every cell in an even column.
\end{stylekey}

\begin{stylekey}{/tikz/row \meta{number}}
    This style is used for every cell in row \meta{number}.
\end{stylekey}

\begin{stylekey}{/tikz/every odd row}
    This style is used for every cell in an odd row.
\end{stylekey}

\begin{stylekey}{/tikz/every even row}
    This style is used for every cell in an even row.
\end{stylekey}

\begin{stylekey}{/tikz/row \meta{row number} column \meta{column number}}
    This style is used for the cell in row \meta{row number} and column
    \meta{column number}.
\end{stylekey}
%
\begin{codeexample}[]
\begin{tikzpicture}
  [row 1/.style={red},
   column 2/.style={green!50!black},
   row 3 column 3/.style={blue}]

  \matrix
  {
    \node {8}; & \node{1}; & \node {6}; \\
    \node {3}; & \node{5}; & \node {7}; \\
    \node {4}; & \node{9}; & \node {2}; \\
  };
\end{tikzpicture}
\end{codeexample}

You can use the |column |\meta{number} option to change the alignment for
different columns.
%
\begin{codeexample}[]
\begin{tikzpicture}
  [column 1/.style={anchor=base west},
   column 2/.style={anchor=base east},
   column 3/.style={anchor=base}]
  \matrix
  {
    \node {123}; & \node{456}; & \node {789}; \\
    \node {12}; & \node{45}; & \node {78}; \\
    \node {1}; & \node{4}; & \node {7}; \\
  };
\end{tikzpicture}
\end{codeexample}

In many matrices all cell pictures have nearly the same code. For example,
cells typically start with |\node{| and end |};|. The following options allow
you to execute such code in all cells:

\begin{key}{/tikz/execute at begin cell=\meta{code}}
    The code will be executed at the beginning of each nonempty cell.
\end{key}
%
\begin{key}{/tikz/execute at end cell=\meta{code}}
    The code will be executed at the end of each nonempty cell.
\end{key}
%
\begin{key}{/tikz/execute at empty cell=\meta{code}}
    The code will be executed inside each empty cell.
\end{key}
%
\begin{codeexample}[]
\begin{tikzpicture}
  [matrix of nodes/.style={
     execute at begin cell=\node\bgroup,
     execute at end cell=\egroup;%
   }]
  \matrix [matrix of nodes]
  {
    8 & 1 & 6 \\
    3 & 5 & 7 \\
    4 & 9 & 2 \\
  };
\end{tikzpicture}
\end{codeexample}
%
\begin{codeexample}[]
\begin{tikzpicture}
  [matrix of nodes/.style={
     execute at begin cell=\node\bgroup,
     execute at end cell=\egroup;,%
     execute at empty cell=\node{--};%
   }]
  \matrix [matrix of nodes]
  {
    8 & 1 &   \\
    3 &   & 7 \\
      &   & 2 \\
  };
\end{tikzpicture}
\end{codeexample}

The |matrix| library defines a number of styles that make use of the above
options.


\subsection{Anchoring a Matrix}

Since matrices are nodes, they can be anchored in the usual fashion using the
|anchor| option. However, there are two ways to influence this placement
further. First, the following option is often useful:

\begin{key}{/tikz/matrix anchor=\meta{anchor}}
    This option has the same effect as |anchor|, but the option applies only to
    the matrix itself, not to the cells inside. If you just say |anchor=north|
    as an option to the matrix node, all nodes inside matrix will also have
    this anchor, unless it is explicitly set differently for each node. By
    comparison, |matrix anchor| sets the anchor for the matrix, but for the
    nodes inside the value of |anchor| remain unchanged.
    %
\begin{codeexample}[]
\begin{tikzpicture}
  \matrix [matrix anchor=west] at (0,0)
  {
    \node {123}; \\ % still center anchor
    \node {12}; \\
    \node {1}; \\
  };
  \matrix [anchor=west] at (0,-2)
  {
    \node {123}; \\ % inherited west anchor
    \node {12}; \\
    \node {1}; \\
  };
\end{tikzpicture}
\end{codeexample}
    %
\end{key}

The second way to anchor a matrix is to use \emph{an anchor of a node inside
the matrix}. For this, the |anchor| option has a special effect when given as
an argument to a matrix:

\begin{key}{/tikz/anchor=\meta{anchor or node.anchor}}
    Normally, the argument of this option refers to anchor of the matrix node,
    which is the node that includes all of the stuff of the matrix. However,
    you can also provide an argument of the form \meta{node}|.|\meta{anchor}
    where \meta{node} must be node defined inside the matrix and \meta{anchor}
    is an anchor of this node. In this case, the whole matrix is shifted around
    in such a way that this particular anchor of this particular node lies at
    the |at| position of the matrix. The same is true for |matrix anchor|.
    %
\begin{codeexample}[]
\begin{tikzpicture}
  \draw[help lines] (0,0) grid (3,2);
  \matrix[matrix anchor=inner node.south,anchor=base,row sep=3mm] at (1,1)
  {
    \node {a}; & \node             {b}; & \node {c}; & \node {d}; \\
    \node {a}; & \node(inner node) {b}; & \node {c}; & \node {d}; \\
    \node {a}; & \node             {b}; & \node {c}; & \node {d}; \\
  };
  \draw (inner node.south) circle (1pt);
\end{tikzpicture}
\end{codeexample}
    %
\end{key}


\subsection{Considerations Concerning Active Characters}

Even though \tikzname\ seems to use |&| to separate cells, \pgfname\ actually
uses a different command to separate cells, namely the command
|\pgfmatrixnextcell| and using a normal |&| character will normally fail. What
happens is that, \tikzname\ makes |&| an active character and then defines this
character to be equal to |\pgfmatrixnextcell|. In most situations this will
work nicely, but sometimes |&| cannot be made active; for instance because the
matrix is used in an argument of some macro or the matrix contains nodes that
contain normal |{tabular}| environments. In this case you can use the following
option to avoid having to type |\pgfmatrixnextcell| each time:

\begin{key}{/tikz/ampersand replacement=\meta{macro name or empty}}
    If a macro name is provided, this macro will be defined to be equal to
    |\pgfmatrixnextcell| inside matrices and |&| will not be made active. For
    instance, you could say |ampersand replacement=\&| and then use |\&| to
    separate columns as in the following example:
    %
\begin{codeexample}[]
\tikz
  \matrix [ampersand replacement=\&]
  {
    \draw (0,0)   circle (4mm); \& \node[rotate=10] {Hello};        \\
    \draw (0.2,0) circle (2mm); \& \fill[red]   (0,0) circle (3mm); \\
  };
\end{codeexample}
    %
\end{key}


\subsection{Examples}

The following examples are adapted from code by Mark Wibrow. The first two
redraw pictures from Timothy van Zandt's PStricks documentation:
%
{\catcode`\|=12
\begin{codeexample}[preamble={\usetikzlibrary{matrix}}]
\begin{tikzpicture}
  \matrix [matrix of math nodes,row sep=1cm]
  {
    |(U)| U &[2mm]                       &[8mm]    \\
            &      |(XZY)| X \times_Z Y  &      |(X)| X \\
            &      |(Y)|   Y             &      |(Z)| Z \\
  };
  \begin{scope}[every node/.style={midway,auto,font=\scriptsize}]
    \draw [double, dashed] (U)   -- node {$x$} (X);
    \draw                  (X)   -- node {$p$} (X -| XZY.east)
                           (X)   -- node {$f$} (Z)
                                 -- node {$g$} (Y)
                                 -- node {$q$} (XZY)
                                 -- node {$y$} (U);
   \end{scope}
\end{tikzpicture}
\end{codeexample}

\begin{codeexample}[
    preamble={\usetikzlibrary{matrix}},
    pre={\definecolor{graphicbackground}{rgb}{0.96,0.96,0.8}},
]
\begin{tikzpicture}[>=stealth,->,shorten >=2pt,looseness=.5,auto]
  \matrix [matrix of math nodes,
           column sep={2cm,between origins},
           row sep={3cm,between origins},
           nodes={circle, draw, minimum size=7.5mm}]
  {
            & |(A)| A &         \\
    |(B)| B & |(E)| E & |(C)| C \\
            & |(D)| D           \\
  };
  \begin{scope}[every node/.style={font=\small\itshape}]
    \draw (A) to [bend left] node [midway]   {g} (B);
    \draw (B) to [bend left] node [midway]   {f} (A);
    \draw (D) --             node [midway]   {c} (B);
    \draw (E) --             node [midway]   {b} (B);
    \draw (E) --             node [near end] {a} (C);
    \draw [-,line width=8pt,draw=graphicbackground]
          (D) to [bend right, looseness=1] (A);
    \draw (D) to [bend right, looseness=1]
            node [near start] {b} node [near end] {e} (A);
  \end{scope}
\end{tikzpicture}
\end{codeexample}

\begin{codeexample}[preamble={\usetikzlibrary{matrix}}]
\begin{tikzpicture}
  \matrix (network)
    [matrix of nodes,%
     nodes in empty cells,
     nodes={outer sep=0pt,circle,minimum size=4pt,draw},
     column sep={1cm,between origins},
     row sep={1cm,between origins}]
  {
                  &                &                 & \\
                  &                &                 & \\
    |[draw=none]| & |[xshift=1mm]| & |[xshift=-1mm]|   \\
  };
  \foreach \a in {1,...,4}{
    \draw (network-3-2) -- (network-2-\a);
    \draw (network-3-3) -- (network-2-\a);
    \draw [-stealth] ([yshift=5mm]network-1-\a.north) -- (network-1-\a);
    \foreach \b in {1,...,4}
      \draw (network-1-\a) -- (network-2-\b);
  }
  \draw [stealth-] ([yshift=-5mm]network-3-2.south) -- (network-3-2);
  \draw [stealth-] ([yshift=-5mm]network-3-3.south) -- (network-3-3);
\end{tikzpicture}
\end{codeexample}

The following example is adapted from code written by Kjell Magne Fauske, which
is based on the following paper: K.~Bossley, M.~Brown, and C.~Harris,
Neurofuzzy identification of an autonomous underwater vehicle,
\emph{International Journal of Systems Science}, 1999, 30, 901--913.
%
\begin{codeexample}[preamble={\usetikzlibrary{arrows,shapes.geometric}}]
\begin{tikzpicture}
  [auto,
   decision/.style={diamond, draw=blue, thick, fill=blue!20,
                    text width=4.5em,align=flush center,
                    inner sep=1pt},
   block/.style   ={rectangle, draw=blue, thick, fill=blue!20,
                    text width=5em,align=center, rounded corners,
                    minimum height=4em},
   line/.style    ={draw, thick, -latex',shorten >=2pt},
   cloud/.style   ={draw=red, thick, ellipse,fill=red!20,
                    minimum height=2em}]

  \matrix [column sep=5mm,row sep=7mm]
  {
    % row 1
      \node [cloud] (expert)   {expert}; &
      \node [block] (init)     {initialize model}; &
      \node [cloud] (system)   {system}; \\
    % row 2
      & \node [block] (identify) {identify candidate model}; & \\
    % row 3
      \node [block] (update)   {update model};  &
      \node [block] (evaluate) {evaluate candidate models}; & \\
    % row 4
      & \node [decision] (decide) {is best candidate}; & \\
    % row 5
      & \node [block] (stop)      {stop}; & \\
  };
  \begin{scope}[every path/.style=line]
    \path          (init)     -- (identify);
    \path          (identify) -- (evaluate);
    \path          (evaluate) -- (decide);
    \path          (update)   |- (identify);
    \path          (decide)   -| node [near start] {yes} (update);
    \path          (decide)   -- node [midway] {no} (stop);
    \path [dashed] (expert)   -- (init);
    \path [dashed] (system)   -- (init);
    \path [dashed] (system)   |- (evaluate);
  \end{scope}
\end{tikzpicture}
\end{codeexample}
}


%%% Local Variables:
%%% mode: latex
%%% TeX-master: "pgfmanual"
%%% End:

% % Copyright 2006 by Till Tantau
%
% This file may be distributed and/or modified
%
% 1. under the LaTeX Project Public License and/or
% 2. under the GNU Free Documentation License.
%
% See the file doc/generic/pgf/licenses/LICENSE for more details.

\section{Making Trees Grow}

\label{section-trees}


\subsection{Introduction to the  Child Operation}

\emph{Trees} are a common way of visualizing hierarchical
structures. A simple tree looks like this:
\begin{codeexample}[]
\begin{tikzpicture}
  \node {root}
    child {node {left}}
    child {node {right}
      child {node {child}}
      child {node {child}}
    };
\end{tikzpicture}
\end{codeexample}

Admittedly, in reality trees are more likely to grow \emph{upward} and
not downward as above. You can tell whether the author of a paper is a
mathematician or a computer scientist by looking at the direction
their trees grow. A computer scientist's trees will grow downward
while a mathematician's tree will grow upward. Naturally, the
\emph{correct} way is the mathematician's way, which can be specified as
follows:
\begin{codeexample}[]
\begin{tikzpicture}
  \node {root} [grow'=up]
    child {node {left}}
    child {node {right}
      child {node {child}}
      child {node {child}}
    };
\end{tikzpicture}
\end{codeexample}

In \tikzname, there are two ways of specifying trees: Using either the
|graph| path operation, which is covered in
Section~\ref{section-library-graphs}, or using the |child| path
operation, which is covered in the present section. Both methods have
their advantages.

In \tikzname, trees are specified by adding \emph{children} to a
node on a path using the |child| operation:

\begin{pathoperation}{child}{\opt{\oarg{options}}%
    \opt{|foreach|\meta{variables}|in|\marg{values}}\opt{\marg{child path}}}
  This operation should directly follow a completed |node| operation
  or another |child| operation, although it is permissible that the
  first |child| operation is preceded by options (we will come to
  that).

  When a |node| operation like |node {X}| is followed by |child|,
  \tikzname\ starts counting the number of child nodes that follow the
  original |node {X}|. For this, it scans the input and stores away each
  |child| and its arguments until it reaches a path operation that is
  not a |child|. Note that this will fix the character codes of all
  text inside the child arguments, which means, in essence, that you
  cannot use verbatim text inside the nodes inside a |child|. Sorry.

  Once the children have been collected and counted, \tikzname\ starts
  generating the child nodes. For each child of a parent node
  \tikzname\ computes an appropriate position where the child is
  placed. For each child, the coordinate system is transformed so that
  the origin is at this position. Then the \meta{child path} is
  drawn. Typically, the child path just consists of a |node|
  specification, which results in a node being drawn at the child's
  position. Finally, an edge is drawn from the first node in the
  \meta{child path} to the parent node.

  The optional |foreach| part (note that there is no backslash before
  |foreach|) allows you to specify multiple children in a single
  |child| command. The idea is the following: A |\foreach| statement
  is (internally) used to iterate over the list of \meta{values}. For
  each value in this list, a new |child| is added to the node. The
  syntax for \meta{variables} and for \meta{values} is the same as for
  the |\foreach| statement, see Section~\ref{section-foreach}. For
  example, when you say
\begin{codeexample}[code only]
node {root} child [red] foreach \name in {1,2} {node {\name}}
\end{codeexample}
  the effect will be the same as if you had said
\begin{codeexample}[code only]
node {root} child[red] {node {1}} child[red] {node {2}}
\end{codeexample}
  When you write
\begin{codeexample}[code only]
node {root} child[\pos] foreach \name/\pos in {1/left,2/right} {node[\pos] {\name}}
\end{codeexample}
  the effect will be the same as for
\begin{codeexample}[code only]
node {root} child[left] {node[left] {1}} child[right] {node[right] {2}}
\end{codeexample}

  You can nest things as in the following example:
\begin{codeexample}[]
\begin{tikzpicture}
  [level distance=4mm,level/.style={sibling distance=8mm/#1}]
  \coordinate
    child foreach \x in {0,1}
      {child foreach \y in {0,1}
        {child foreach \z in {0,1}}};
\end{tikzpicture}
\end{codeexample}

  The details and options for this operation are described in the rest
  of this present section.
\end{pathoperation}



\subsection{Child Paths and Child Nodes}

For each |child| of a root node, its \meta{child path} is inserted at
a specific location in the picture (the placement rules are discussed
in Section~\ref{section-tree-placement}). The first node in the
\meta{child path}, if it exists, is special and called the \emph{child
  node}. If there is no first node in the \meta{child path}, that is,
if the \meta{child path} is missing (including the curly braces) or if
it does not start with |node| or with |coordinate|, then an empty
child node of shape |coordinate| is automatically added.

Consider the example |\node {x} child {node {y}} child;|. For the
first child, the \meta{child path} has the child node |node {y}|. For
the second child, no child node is specified and, thus, it is just
|coordinate|.

As for any normal node, you can give the child node a name, shift it
around, or use options to influence how it is rendered.
\begin{codeexample}[]
\begin{tikzpicture}[sibling distance=15mm]
  \node[rectangle,draw] {root}
    child {node[circle,draw,yshift=-5mm] (left node) {left}}
    child {node[ellipse,draw] (right node) {right}};
  \draw[dashed,->] (left node) -- (right node);
\end{tikzpicture}
\end{codeexample}

In many cases, the \meta{child path} will just consist of a
specification of a child node and, possibly, children of this child
node. However, the node specification may be followed by arbitrary
other material that will be added to the picture, transformed to the
child's coordinate system. For your convenience, a move-to |(0,0)|
operation is inserted automatically at the beginning of the path. Here
is an example:

\begin{codeexample}[]
\begin{tikzpicture}
  \node {root}
    child {[fill] circle (2pt)}
    child {[fill] circle (2pt)};
\end{tikzpicture}
\end{codeexample}


At the end of the \meta{child path} you may add a special path
operation called |edge from parent|. If this operation is not given by
yourself somewhere on the path, it will be automatically added at the
end. This option causes a connecting edge from the parent node to the
child node to be added to the path. By giving options to this
operation you can influence how the edge is rendered. Also, nodes
following the |edge from parent| operation will be placed on this
edge, see Section~\ref{section-edge-from-parent} for details.

To sum up:
\begin{enumerate}
\item
  The child path starts with a node specification. If it is not there,
  it is added automatically.
\item
  The child path ends with a |edge from parent| operation, possibly
  followed by nodes to be put on this edge. If the operation is not
  given at the end, it is added automatically.
\end{enumerate}



\subsection{Naming Child Nodes}

Child nodes can be named like any other node using either the |name|
option or the special syntax in which the name of the node is placed
in round parentheses between the |node| operation and the node's
text.

If you do not assign a name to a child node, \tikzname\ will
automatically assign a name as follows: Assume that the name of the
parent node is, say, |parent|. (If you did not assign a
name to the parent, \tikzname\ will do so itself, but that name will
not be user-accessible.) The first child
of |parent| will be named |parent-1|, the second child is named
|parent-2|, and so on.

This naming convention works recursively. If the second child
|parent-2| has children, then the first of these children will be
called |parent-2-1| and the second |parent-2-2| and so on.

If you assign a name to a child node yourself, no name is generated
automatically (the node does not have two names). However, ``counting
continues,'' which means that the third child of |parent| is called
|parent-3| independently of whether you have assigned names to the
first and/or second child of |parent|.

Here is an example:

\begin{codeexample}[]
\begin{tikzpicture}[sibling distance=15mm]
  \node (root) {root}
    child
    child {
      child {coordinate (special)}
      child
    };
  \node at (root-1) {root-1};
  \node at (root-2) {root-2};
  \node at (special) {special};
  \node at (root-2-2) {root-2-2};
\end{tikzpicture}
\end{codeexample}

\subsection{Specifying Options for Trees and Children}
\label{section-tree-options}

Each |child| may have its own \meta{options}, which apply to ``the
whole child,'' including all of its grandchildren. Here is an
example:

\begin{codeexample}[]
\begin{tikzpicture}
  [thick,level 1/.style={sibling distance=15mm},
         level 2/.style={sibling distance=10mm}]
  \coordinate
    child[red]   {child child}
    child[green] {child child[blue]};
\end{tikzpicture}
\end{codeexample}

The options of the root node have no effect on the children since
the options of a node are always ``local'' to that node. Because of
this, the edges in the following tree are black, not red.

\begin{codeexample}[]
\begin{tikzpicture}[thick]
  \node [red] {root}
    child
    child;
\end{tikzpicture}
\end{codeexample}
  This raises the problem of how to set options for \emph{all}
  children. Naturally, you could always set options for the whole path
  as in |\path [red] node {root} child child;| but this is bothersome
  in some situations. Instead, it is easier to give the options
  \emph{before the first child} as follows:
\begin{codeexample}[]
\begin{tikzpicture}[thick]
  \node [red] {root}
    [green] % option applies to all children
    child
    child;
\end{tikzpicture}
\end{codeexample}

Here is the set of rules:
\begin{enumerate}
\item
  Options for the whole tree are given before the root node.
\item
  Options for the root node are given directly to the |node| operation
  of the root.
\item
  Options for all children can be given between the root node and the
  first child.
\item
  Options applying to a specific child path are given as options to
  the |child| operation.
\item
  Options applying to the node of a child, but not to the whole child
  path, are given as options to the |node| command inside the
  \meta{child path}.
\end{enumerate}

\begin{codeexample}[code only]
\begin{tikzpicture}
  \scoped
    [...]              % Options apply to the whole tree
    \node[...] {root}  % Options apply to the root node only
       [...]           % Options apply to all children
       child[...]      % Options apply to this child and all its children
       {
         node[...] {}  % Options apply to the child node only
         ...
       }
       child[...]      % Options apply to this child and all its children
    ;
\end{tikzpicture}
\end{codeexample}

There are additional styles that influence how children are rendered:
\begin{stylekey}{/tikz/every child (initially \normalfont empty)}
  This style is used at the beginning of each child, as if you had
  given the style's contents as options to the |child| operation.
\end{stylekey}

\begin{stylekey}{/tikz/every child node (initially \normalfont empty)}
  This style is used at the beginning of each child node in addition
  to the |every node| style.
\end{stylekey}

\begin{stylekey}{/tikz/level=\meta{number} (initially \normalfont empty)}
  This style is executed at the beginning of each set of children, where
  \meta{number} is the current level in the current tree. For example,
  when you say |\node {x} child child;|, then |level=1| is
  used before the first |child|. The style or code of this key will be
  passed \meta{number} as its first parameter. If this first |child|
  has children itself, then |level=2| would be used for them.

\begin{codeexample}[]
\begin{tikzpicture}[level/.style={sibling distance=20mm/#1}]
  \node {root}
    child { child child }
    child { child child child };
\end{tikzpicture}
\end{codeexample}
\end{stylekey}

\begin{stylekey}{/tikz/level \meta{number} (initially \normalfont empty)}
  This style is used in addition to the |level| style. So, when you
  say |\node {x} child child;|, then the following key list is
  executed: |level=1,level 1|.

\begin{codeexample}[]
\begin{tikzpicture}
  [level 1/.style={sibling distance=20mm},
   level 2/.style={sibling distance=5mm}]
  \node {root}
    child { child child }
    child { child child child };
\end{tikzpicture}
\end{codeexample}
\end{stylekey}



\subsection{Placing Child Nodes}

\label{section-tree-placement}

\subsubsection{Basic Idea}

Perhaps the most difficult part in drawing a tree is the correct
layout of the children. Typically, the children have different sizes
and it is not easy to arrange them in such a manner that not too much
space is wasted, the children do not overlap, and they are either
evenly spaced or their centers are evenly distributed. Calculating
good positions is especially difficult since a good position for the
first child may depend on the size of the last child.

In basic \tikzname, when you do not make use of the graph drawing
facilities explained in Part~\ref{part-gd}, a comparatively simple
approach is taken to placing the 
children. In order to compute a child's position, all that is taken
into account is the number of the current child in the list of
children and the number of children in this list. Thus, if a node has
five children, then there is a fixed position for the first child, a
position for the second child, and so on. These positions \emph{do not
  depend on the size of the children} and, hence, children can easily
overlap. However, since you can use options to shift individual
children a bit, this is not as great a problem as it may seem.

Although the placement of the children only depends on their number in
the list of children and the total number of children, everything else
about the placement is highly configurable. You can change the
distance between children (appropriately called the
|sibling distance|) and the distance between levels of the tree. These
distances may change from level to level. The direction in which the
tree grows can be changed globally and for parts of the tree. You can
even specify your own ``growth function'' to arrange children on a
circle or along special lines or curves.

\subsubsection{Default Growth Function}

The default growth function works as follows: Assume that we are given
a node and five children. These children will be placed on a line with
their centers (or, more generally, with their anchors) spaced apart by
the current |sibling distance|. The line is
orthogonal to the current \emph{direction of growth}, which is set
with the |grow| and |grow'| option (the latter option reverses the
ordering of the children). The distance from the line to the parent node
is given by the |level distance|.

{\catcode`\|=12
\begin{codeexample}[]
\begin{tikzpicture}[sibling distance=15mm, level distance=15mm]
  \path [help lines]
    node (root) {root}
    [grow=-10]
    child {node {1}}
    child {node {2}}
    child {node {3}}
    child {node {4}};

  \draw[|<->|,thick] (root-1.center)
    -- node[above,sloped] {sibling distance} (root-2.center);

  \draw[|<->|,thick] (root.center)
    -- node[above,sloped] {level distance} +(-10:\tikzleveldistance);
\end{tikzpicture}
\end{codeexample}
}

\begin{key}{/tikz/level distance=\meta{distance} (initially 15mm)}
  This key determines the distance between different levels of the
  tree, more precisely, between the parent and the line
  on which its children are arranged. When given to a single child,
  this will set the distance for this child only.

\begin{codeexample}[]
\begin{tikzpicture}
  \node {root}
    [level distance=20mm]
    child
    child {
      [level distance=5mm]
      child
      child
      child
    }
    child[level distance=10mm];
\end{tikzpicture}
\end{codeexample}

\begin{codeexample}[]
\begin{tikzpicture}
  [level 1/.style={level distance=10mm},
   level 2/.style={level distance=5mm}]
  \node {root}
    child
    child {
      child
      child[level distance=10mm]
      child
    }
    child;
\end{tikzpicture}
\end{codeexample}
\end{key}

\begin{key}{/tikz/sibling distance=\meta{distance} (initially 15mm)}
  This key specifies the distance between the anchors of the
  children of a parent node.

\begin{codeexample}[]
\begin{tikzpicture}
  [level distance=4mm,
   level 1/.style={sibling distance=8mm},
   level 2/.style={sibling distance=4mm},
   level 3/.style={sibling distance=2mm}]
  \coordinate
     child {
       child {child child}
       child {child child}
     }
     child {
       child {child child}
       child {child child}
     };
\end{tikzpicture}
\end{codeexample}

\begin{codeexample}[]
\begin{tikzpicture}
  [level distance=10mm,
   every node/.style={fill=red!60,circle,inner sep=1pt},
   level 1/.style={sibling distance=20mm,nodes={fill=red!45}},
   level 2/.style={sibling distance=10mm,nodes={fill=red!30}},
   level 3/.style={sibling distance=5mm,nodes={fill=red!25}}]
  \node {31}
     child {node {30}
       child {node {20}
         child {node {5}}
         child {node {4}}
       }
       child {node {10}
         child {node {9}}
         child {node {1}}
       }
     }
     child {node {20}
       child {node {19}
         child {node {1}}
         child[missing]
       }
       child {node {18}}
     };
\end{tikzpicture}
\end{codeexample}
\end{key}


\begin{key}{/tikz/grow=\meta{direction}}
  This key is used to define the \meta{direction} in which the tree
  will grow. The \meta{direction} can either be an angle in degrees or
  one of the following special text strings: |down|, |up|, |left|,
  |right|, |north|, |south|, |east|, |west|, |north east|,
  |north west|, |south east|, and |south west|. All of these have
  ``their obvious meaning,'' so, say, |south west| is the same as the
  angle $-135^\circ$.

  As a side effect, this option installs the default growth function.

  In addition to setting the direction, this option also has a
  seemingly strange effect: It sets the sibling distance for the
  current level to |0pt|, but leaves the sibling distance for later
  levels unchanged.

  This somewhat strange behaviour has a highly desirable effect: If
  you give this option before the list of children of a node starts,
  the ``current level'' is still the parent level. Each child will be
  on a later level and, hence, the sibling distance will be as
  specified originally. This will cause the children to be neatly
  aligned in a line orthogonal to the given \meta{direction}. However,
  if you give this option locally to a single child, then ``current
  level'' will be the same as the child's level. The zero sibling
  distance will then cause the child to be placed exactly at a point
  at distance |level distance| in the direction
  \meta{direction}. However, the children of the child will be placed
  ``normally'' on a line orthogonal to the \meta{direction}.

  These placement effects are best demonstrated by some examples:
\begin{codeexample}[]
\tikz \node {root} [grow=right] child child;
\end{codeexample}

\begin{codeexample}[]
\tikz \node {root} [grow=south west] child child;
\end{codeexample}

\begin{codeexample}[]
\begin{tikzpicture}[level distance=10mm,sibling distance=5mm]
  \node {root}
    [grow=down]
    child
    child
    child[grow=right] {
      child child child
    };
\end{tikzpicture}
\end{codeexample}

\begin{codeexample}[]
\begin{tikzpicture}[level distance=2em]
  \node {C}
    child[grow=up]    {node {H}}
    child[grow=left]  {node {H}}
    child[grow=down]  {node {H}}
    child[grow=right] {node {C}
        child[grow=up]    {node {H}}
        child[grow=right] {node {H}}
        child[grow=down]  {node {H}}
      edge from parent[double]
        coordinate (wrong)
    };
  \draw[<-,red] ([yshift=-2mm]wrong) -- +(0,-1)
    node[below]{This is wrong!};
\end{tikzpicture}
\end{codeexample}

\begin{codeexample}[]
\begin{tikzpicture}
  \node[rectangle,draw] (a) at (0,0) {start node};
  \node[rectangle,draw] (b) at (2,1) {end};

  \draw (a) -- (b)
    node[coordinate,midway] {}
      child[grow=100,<-] {node[above] {the middle is here}};
\end{tikzpicture}
\end{codeexample}
\end{key}

\begin{key}{/tikz/grow'=\meta{direction}}
  This key has the same effect as |grow|, only the children are
  arranged in the opposite order.
\end{key}


\subsubsection{Missing Children}

Sometimes one or more of the children of a node are ``missing.'' Such
a missing child will count as a child with respect to the total number
of children and also with respect to the current child count, but it
will not be rendered.

\begin{key}{/tikz/missing=\meta{true or false} (default true)}
  If this option is given to a child, the current child counter is
  increased, but the child is otherwise ignored. In particular, the
  normal contents of the child is completely ignored.

\begin{codeexample}[]
\begin{tikzpicture}[level distance=10mm,sibling distance=5mm]
  \node {root} [grow=down]
    child          { node {1} }
    child          { node {2} }
    child          { node {3} }
    child[missing] { node {4} }
    child          { node {5} }
    child          { node {6} };
\end{tikzpicture}
\end{codeexample}
\end{key}


\subsubsection{Custom Growth Functions}

\begin{key}{/tikz/growth parent anchor=\meta{anchor} (initially center)}
  This key allows you to specify which anchor of the parent node is
  to be used for computing the children's position. For example, when
  there is only one child and the |level distance| is |2cm|, then the
  child node will be placed two centimeters below the \meta{anchor} of
  the parent node. ``Being placed'' means that the child node's
  anchor (which is the anchor specified using the |anchor=| option in
  the |node| command of the child) is two centimeters below the parent
  node's \meta{anchor}.

  In the following example, the two red lines both have length |1cm|.
\begin{codeexample}[]
\begin{tikzpicture}[level distance=1cm]
  \node [rectangle,draw] (a) at (0,0) {root}
  [growth parent anchor=south] child;

  \node [rectangle,draw] (b) at (2,0) {root}
  [growth parent anchor=north east] child;

  \draw [red,thick,dashed] (a.south) -- (a-1);
  \draw [red,thick,dashed] (b.north east) -- (b-1);
\end{tikzpicture}
\end{codeexample}

  In the next example, the top and bottom nodes are aligned at the top
  and the bottom, respectively.
\begin{codeexample}[]
\begin{tikzpicture}
  [level distance=2cm,growth parent anchor=north,
   every node/.style={anchor=north,rectangle,draw}
   every child node/.style={anchor=south}]

  \node at (0,0) {root} child {node {small}};

  \node at (2,0) {big root} child {node {\large big}};
\end{tikzpicture}
\end{codeexample}
\end{key}

\begin{key}{/tikz/growth function=\meta{macro name} (initially
    \normalfont an internal function)}
  This rather low-level option allows you to set a new growth
  function. The \meta{macro name} must be the name of a macro without
  parameters. This macro will be called for each child of a node. The
  initial function is an internal function that corresponds to
  downward growth.

  The effect of executing the macro should be the following: It should
  transform the coordinate system in such a way that the origin
  becomes the place where the current child should be anchored. When
  the macro is called, the current coordinate system will be set up
  such that the anchor of the parent node is in the origin. Thus, in
  each call, the \meta{macro name} must essentially do a shift to the
  child's origin. When the macro is called, the \TeX\ counter
  |\tikznumberofchildren| will be set to the total number of children
  of the parent node and the counter |\tikznumberofcurrentchild| will
  be set to the number of the current child.

  The macro may, in addition to shifting the coordinate system, also
  transform the coordinate system further. For example, it could be
  rotated or scaled.

  Additional growth functions are defined in the library, see
  Section~\ref{section-tree-library}.
\end{key}



\subsection{Edges From the Parent Node}

\label{section-edge-from-parent}

Every child node is connected to its parent node via a special kind of
edge called the |edge from parent|. This edge is added to the
\meta{child path} when the following path operation is encountered:

\begin{pathoperation}{edge from parent}{\opt{\oarg{options}}}
  This path operation can only be used inside \meta{child paths} and
  should be given at the end, possibly followed by \meta{node
    specifications} like |node {a}|. If a \meta{child path} does not
  contain this operation, it will be added at the end of the
  \meta{child path} automatically.

  By default, this operation does the following:
  \begin{enumerate}
  \item The following style is executed:
    \begin{stylekey}{/tikz/edge from parent (initially draw)}
      This style is inserted right before the |edge from parent path| and
      before the \meta{options} are inserted.
\begin{codeexample}[]
\begin{tikzpicture}
  [edge from parent/.style={draw,red,thick}]
  \node {root}
    child {node {left} edge from parent[dashed]}
    child {node {right}
      child {node {child}}
      child {node {child} edge from parent[draw=none]}
    };
\end{tikzpicture}
\end{codeexample}
    \end{stylekey}
  \item Next, the \meta{options} are executed.
  \item Next, the text stored in the following key is inserted:
  \begin{key}{/tikz/edge from parent path=\meta{path} (initially
      \normalfont code shown below)}
    This option allows you to set the |edge from parent path| to a new
    path. Initially, this path is the following:
\begin{codeexample}[code only]
(\tikzparentnode\tikzparentanchor) -- (\tikzchildnode\tikzchildanchor)
\end{codeexample}
    The |\tikzparentnode| is a macro that will expand to the name of
    the parent node. This works even when you have not assigned a name
    to the parent node, in this case an internal name is automatically
    generated. The |\tikzchildnode| is a macro that expands to the
    name of the child node. The two |...anchor| macros are empty by
    default. So, what is essentially inserted is just the path segment
    |(\tikzparentnode) -- (\tikzchildnode)|; which is exactly an edge
    from the parent to the child.

    You can modify this edge from parent path to achieve all sorts of
    effects. For example, we could replace the straight line by a
    curve as follows:
\begin{codeexample}[]
\begin{tikzpicture}[level distance=15mm, sibling distance=15mm,
  edge from parent path=
  {(\tikzparentnode.south) .. controls +(0,-1) and +(0,1)
                           .. (\tikzchildnode.north)}]
  \node {root}
    child {node {left}}
    child {node {right}
      child {node {child}}
      child {node {child}}
    };
\end{tikzpicture}
\end{codeexample}

    Further useful |edge from parent path|s are defined in the tree
    library, see Section~\ref{section-tree-library}.
    
    The nodes in a \meta{node specification} following the
    |edge from parent| path command get executed as if the |pos| option
    had been added to all these nodes, see also Section~\ref{section-pos-option}.

    As an example, consider the following code:
\begin{codeexample}[code only]
\node (root) {} child {node (child) {} edge to parent node {label}};
\end{codeexample}
  The |edge to parent| operation and the following |node| operation
  will, together, have the same effect as if we had said:
\begin{codeexample}[code only]
(root) -- (child) node [pos=0.5] {label}
\end{codeexample}

    Here is a more complicated example:
\begin{codeexample}[]
\begin{tikzpicture}
  \node {root}
    child {
      node {left}
      edge from parent
        node[left] {a}
        node[right] {b}
    }
    child {
      node {right}
        child {
          node {child}
          edge from parent
            node[left] {c}
        }
        child {node {child}}
      edge from parent
        node[near end] {x}
    };
\end{tikzpicture}
\end{codeexample}
    
    As said before, the anchors in the default |edge from parent path|
    are empty. However, you can set them using the following keys:
  \begin{key}{/tikz/child anchor=\meta{anchor} (initially border)}
    Specifies the anchor where the edge from parent meets the child
    node by setting the macro |\tikzchildanchor| to
    |.|\meta{anchor}.

    If you specify |border| as the \meta{anchor}, then the macro
    |\tikzchildanchor| is set to the empty string. The effect of
    this is that the edge from the parent will meet the child on the
    border at an automatically calculated position.
\begin{codeexample}[]
\begin{tikzpicture}
  \node {root}
    [child anchor=north]
    child {node {left} edge from parent[dashed]}
    child {node {right}
      child {node {child}}
      child {node {child} edge from parent[draw=none]}
    };
\end{tikzpicture}
\end{codeexample}
  \end{key}

  \begin{key}{/tikz/parent anchor=\meta{anchor} (initially border)}
    This option works the same way as the |child anchor|, only for
    the parent.
  \end{key}  
  \end{key}    
  \end{enumerate}
  
  All of the above describes the standard functioning of the
  |edge from parent| command. You may, however, sometimes need even
  more fine-grained control (the graph drawing engine needs it, for
  instance). In such cases the following key gives you complete
  control:
  \begin{key}{/tikz/edge from parent macro=\meta{macro}}
    The \meta{macro} gets expanded each time the |edge from parent|
    path operation is used. This \meta{macro} must take two parameters
    and must expand to some text that is subsequently parsed by the
    parser. The first parameter will be the set of \meta{options} that 
    where passed to the |edge from parent| command, the second
    parameter will be the \meta{node specifications} that following
    the command.

    The standard behaviour of drawing a straight line from the parent
    node to the child node could be achieved by setting the
    \meta{macro} to the following:
\begin{codeexample}[code only]
\def\mymacro#1#2{
  [style=edge from parent, #1]
  (\tikzparentnode\tikzparentanchor) -- #2 (\tikzchildnode\tikzchildanchor)
}
\end{codeexample}
    Note that |#2| is placed between |--| and the node to ensure that
    nodes are put ``on top'' of the line.
  \end{key}
\end{pathoperation}



%%% Local Variables:
%%% mode: latex
%%% TeX-master: "pgfmanual-pdftex-version"
%%% End:

% % Copyright 2007 by Till Tantau
%
% This file may be distributed and/or modified
%
% 1. under the LaTeX Project Public License and/or
% 2. under the GNU Free Documentation License.
%
% See the file doc/generic/pgf/licenses/LICENSE for more details.

\section{Plots of Functions}

\label{section-tikz-plots}

A warning before we get started: \emph{If you are looking for an easy
  way to create a normal plot of a function with scientific axes,
  ignore this section and instead look at the |pgfplots| package or at
  the |datavisualization| command from Part~\ref{part-dv}.}

\subsection{Overview}

\label{section-why-pgname-for-plots}

\tikzname\ can be used to create plots of functions, a job that is
normally handled by powerful programs like \textsc{gnuplot} or
\textsc{mathematica}. These programs can produce
two different kinds of output: First, they can output a complete plot
picture in a certain format (like \pdf) that includes all low-level
commands necessary for drawing the complete plot (including axes and
labels). Second, they can usually also produce ``just plain data'' in
the form of a long list of coordinates. Most of the powerful programs
consider it a to be ``a bit boring'' to just output tabled data and
very much prefer to produce fancy pictures. Nevertheless, when coaxed,
they can also provide the plain data.

The advantage of creating plots directly using \tikzname\ is
\emph{consistency:} Plots created using \tikzname\ will automatically
have the same styling and fonts as those used in the rest of a
document -- something that is hard to do right when an external
program gets involved. Other problems people encounter with external
programs include: Formulas will look different, if they can be
rendered at all; line widths will usually be too thick or too thin;
scaling effects upon inclusion can create a mismatch between sizes in
the plot and sizes in the text; the automatic grid generated by most
programs is mostly distracting; the automatic ticks generated by most
programs are cryptic numerics (try adding a tick reading ``$\pi$'' at
the right point); most programs make it very easy to create ``chart
junk'' in a most convenient fashion; arrows and plot marks will almost
never match the arrows used in the rest of the document. This list is
not exhaustive, unfortunately.

There are basically three ways of creating plots using \tikzname:

\begin{enumerate}
\item Use the |plot| path operation. How this works is explained in
  the present section. This is the most ``basic'' of the three options
  and forces you to do a lot of things ``by hand'' like adding axes or
  ticks.
\item Use the |datavisualization| path command, which is documented in
  Part~\ref{part-dv}. This command is much more powerful than the
  |plot| path operation and produces complete plots including axes and
  ticks. The downside is that you cannot use it to ``just'' quickly
  plot a simple curve (or, more precisely, it is hard to use it in
  this way).
\item Use the |pgfplots| package, which is basically an alternative to
  the |datavisualization| command. While the underlying philosophy of
  this package is not as ``ambitious'' as that of the command
  |datavisualization|, it is somewhat more mature, has a
  simpler design, and wider support base. 
\end{enumerate}


\subsection{The Plot Path Operation}

The |plot| path operation can be used to append a line or curve to the path
that goes through a large number of coordinates. These coordinates are
either given in a simple list of coordinates, read from some file, or
they are computed on the fly.

The syntax of the |plot| comes in different versions.

\begin{pathoperation}{--plot}{\meta{further arguments}}
  This operation plots the curve through the coordinates specified in
  the \meta{further arguments}. The current (sub)path is simply
  continued, that is, a line-to operation to the first point of the
  curve is implicitly added. The details of the \meta{further
    arguments}  will be explained in a moment.
\end{pathoperation}

\begin{pathoperation}{plot}{\meta{further arguments}}
  This operation plots the curve through the coordinates specified in
  the \meta{further arguments} by first ``moving'' to the first
  coordinate of the curve.
\end{pathoperation}

The \meta{further arguments} are used in different ways to
specifying the coordinates of the points to be plotted:

\begin{enumerate}
\item
  \opt{|--|}|plot|\oarg{local options}\declare{|coordinates{|\meta{coordinate
    1}\meta{coordinate 2}\dots\meta{coordinate $n$}|}|}
\item
  \opt{|--|}|plot|\oarg{local options}\declare{|file{|\meta{filename}|}|}
\item
  \opt{|--|}|plot|\oarg{local options}\declare{\meta{coordinate expression}}
\item
  \opt{|--|}|plot|\oarg{local options}\declare{|function{|\meta{gnuplot formula}|}|}
\end{enumerate}

These different ways are explained in the following.


\subsection{Plotting Points Given Inline}

Points can be given directly in the \TeX-file
as in the following example:

\begin{codeexample}[]
\tikz \draw plot coordinates {(0,0) (1,1) (2,0) (3,1) (2,1) (10:2cm)};
\end{codeexample}

Here is an example showing the difference between |plot| and |--plot|:

\begin{codeexample}[]
\begin{tikzpicture}
  \draw (0,0) -- (1,1) plot coordinates {(2,0)  (4,0)};
  \draw[color=red,xshift=5cm]
        (0,0) -- (1,1) -- plot coordinates {(2,0)  (4,0)};
\end{tikzpicture}
\end{codeexample}


\subsection{Plotting Points Read From an External File}

The second way of specifying points is to put them in an external
file named \meta{filename}. Currently, the only file format that
\tikzname\ allows is the following: Each line of the \meta{filename}
should contain one line starting with two numbers, separated by a
space. A line may also be empty or, if it starts with |#| or |%| it is
considered empty. For such lines, a ``new data set'' is started,
typically resulting in a new subpath being started in the plot (see
Section~\ref{section-plot-jumps} on how to change this behaviour, if
necessary). For lines containing two numbers, they must be separated
by a space. They may be following by arbitrary text, which is ignored,
\emph{except} if it is |o| or |u|. In the first case, the point is
considered to be an \emph{outlier} and normally also results in a new
subpath being started. In the second case, the point is considered to
be \emph{undefined}, which also results in a new subpath being
started. Again, see Section~\ref{section-plot-jumps} on how to change
this, if necessary. (This is exactly the format that \textsc{gnuplot}
produces when you say |set terminal table|.) 

\begin{codeexample}[]
\tikz \draw plot[mark=x,smooth] file {./text-zh/plots/pgfmanual-sine.table};
\end{codeexample}

The file |./text-zh/plots/pgfmanual-sine.table| reads:
\begin{codeexample}[code only]
#Curve 0, 20 points
#x y type
0.00000 0.00000  i
0.52632 0.50235  i
1.05263 0.86873  i
1.57895 0.99997  i
...
9.47368 -0.04889  i
10.00000 -0.54402  i
\end{codeexample}
It was produced from the following source, using |gnuplot|:
\begin{codeexample}[code only]
set table  "./text-zh/plots/pgfmanual-sine.table"
set format "%.5f"
set samples 20
plot [x=0:10] sin(x)
\end{codeexample}

The \meta{local options} of the |plot| operation are local to each
plot and do not affect other plots ``on the same path.'' For example,
|plot[yshift=1cm]| will locally shift the plot 1cm upward. Remember,
however, that most options can only be applied to paths as a
whole. For example, |plot[red]| does not have the effect of making the
plot red. After all, you are trying to ``locally'' make part of the
path red, which is not possible.

\subsection{Plotting a Function}
\label{section-tikz-plot}

When you plot a function, the coordinates of the plot data can be
computed by evaluating a mathematical expression. Since \pgfname\
comes with a mathematical engine, you can specify this expression and
then have \tikzname\ produce the desired coordinates for you,
automatically.

Since this case is quite common when plotting a function, the syntax
is easy: Following the |plot| command and its local options, you
directly provide a \meta{coordinate expression}. It looks like a
normal coordinate, but inside you may use a special macro, which is
|\x| by default, but this can be changed using the |variable|
option. The \meta{coordinate expression} is then evaluated for
different values for |\x| and the resulting coordinates are plotted.

Note that you will often have to put the $x$- or $y$-coordinate inside
braces, namely whenever you use an expression involving a parenthesis.

The following options influence how the \meta{coordinate expression}
is evaluated:
\begin{key}{/tikz/variable=\meta{macro} (initially x)}
  Sets the macro whose value is set to the different values when
  \meta{coordinate expression} is evaluated.
\end{key}

\begin{key}{/tikz/samples=\meta{number} (initially 25)}
  Sets the number of samples used in the plot.
\end{key}

\begin{key}{/tikz/domain=\meta{start}|:|\meta{end} (initially -5:5)}
  Sets the domain from which the samples are taken.
\end{key}

\begin{key}{/tikz/samples at=\meta{sample list}}
  This option specifies a list of positions for which the
  variable should be evaluated. For instance, you can say
  |samples at={1,2,8,9,10}| to have the variable evaluated exactly for
  values $1$, $2$, $8$, $9$, and $10$. You can use the |\foreach|
  syntax, so you can use |...| inside the \meta{sample list}.

  When this option is used, the |samples| and |domain| option are
  overruled. The other way round, setting either |samples| or
  |domain| will overrule this option.
\end{key}

\begin{codeexample}[]
\begin{tikzpicture}[domain=0:4]
  \draw[very thin,color=gray] (-0.1,-1.1) grid (3.9,3.9);

  \draw[->] (-0.2,0) -- (4.2,0) node[right] {$x$};
  \draw[->] (0,-1.2) -- (0,4.2) node[above] {$f(x)$};

  \draw[color=red]    plot (\x,\x)             node[right] {$f(x) =x$};
  % \x r means to convert '\x' from degrees to _r_adians:
  \draw[color=blue]   plot (\x,{sin(\x r)})    node[right] {$f(x) = \sin x$};
  \draw[color=orange] plot (\x,{0.05*exp(\x)}) node[right] {$f(x) = \frac{1}{20} \mathrm e^x$};
\end{tikzpicture}
\end{codeexample}

\begin{codeexample}[]
\tikz \draw[scale=0.5,domain=-3.141:3.141,smooth,variable=\t]
  plot ({\t*sin(\t r)},{\t*cos(\t r)});
\end{codeexample}

\begin{codeexample}[]
\tikz \draw[domain=0:360,smooth,variable=\t]
  plot ({sin(\t)},\t/360,{cos(\t)});
\end{codeexample}


\subsection{Plotting a Function Using Gnuplot}
\label{section-tikz-gnuplot}

Often, you will want to plot points that are given via a function like
$f(x) = x \sin x$. Unfortunately, \TeX\ does not really have enough
computational power to generate the points of such a function
efficiently (it is a text processing program, after all). However,
if you allow it, \TeX\ can try to call external programs that can
easily produce the necessary points. Currently, \tikzname\ knows how to
call \textsc{gnuplot}.

When \tikzname\ encounters your operation
|plot[id=|\meta{id}|] function{x*sin(x)}| for
the first time, it will create a file called
\meta{prefix}\meta{id}|.gnuplot|, where \meta{prefix} is |\jobname.| by
default, that is, the name of your main |.tex| file. If no \meta{id} is
given, it will be empty, which is alright, but it is better when each
plot has a unique \meta{id} for reasons explained in a moment. Next,
\tikzname\ writes some initialization code into this file followed by
|plot x*sin(x)|. The initialization code sets up things
such that the |plot| operation will write the coordinates into another
file called \meta{prefix}\meta{id}|.table|. Finally, this table file
is read as if you had said |plot file{|\meta{prefix}\meta{id}|.table}|.

For the plotting mechanism to work, two conditions must be met:
\begin{enumerate}
\item
  You must have allowed \TeX\ to call external programs. This is often
  switched off by default since this is a security risk (you might,
  without knowing, run a \TeX\ file that calls all sorts of ``bad''
  commands). To enable this ``calling external programs'' a command
  line option must be given to the \TeX\ program. Usually, it is
  called something like |shell-escape| or |enable-write18|. For
  example, for my |pdflatex| the option |--shell-escape| can be
  given.
\item
  You must have installed the |gnuplot| program and \TeX\ must find it
  when compiling your file.
\end{enumerate}

Unfortunately, these conditions will not always be met. Especially if
you pass some source to a coauthor and the coauthor does not have
\textsc{gnuplot} installed, he or she will have trouble compiling your
files.

For this reason, \tikzname\ behaves differently when you compile your
graphic for the second time: If upon reaching
|plot[id=|\meta{id}|] function{...}| the file \meta{prefix}\meta{id}|.table|
already exists \emph{and} if the \meta{prefix}\meta{id}|.gnuplot| file
contains what \tikzname\ thinks that it ``should'' contain, the |.table|
file is immediately read without trying to call a |gnuplot|
program. This approach has the following advantages:
\begin{enumerate}
\item
  If you pass a bundle of your |.tex| file and all |.gnuplot| and
  |.table| files to someone else, that person can \TeX\ the |.tex|
  file without having to have |gnuplot| installed.
\item
  If the |\write18| feature is switched off for security reasons (a
  good idea), then, upon the first compilation of the |.tex| file, the
  |.gnuplot| will still be generated, but not the |.table|
  file. You can then simply call |gnuplot| ``by hand'' for each
  |.gnuplot| file, which will produce all necessary |.table| files.
\item
  If you change the function that you wish to plot or its
  domain, \tikzname\ will automatically try to regenerate the |.table|
  file.
\item
  If, out of laziness, you do not provide an |id|, the same |.gnuplot|
  will be used for different plots, but this is not a problem since
  the |.table| will automatically be regenerated for each plot
  on-the-fly. \emph{Note: If you intend to share your files with
  someone else, always use an id, so that the file can by typeset
  without having \textsc{gnuplot} installed.} Also, having unique ids
  for each plot will improve compilation speed since no external
  programs need to be called, unless it is really necessary.
\end{enumerate}

When you use |plot function{|\meta{gnuplot formula}|}|, the \meta{gnuplot
  formula} must be given in the |gnuplot| syntax, whose details are
beyond the scope of this manual. Here is the ultra-condensed
essence: Use |x| as the variable and use the C-syntax for normal
plots, use |t| as the variable for parametric plots. Here are some examples:

\begin{codeexample}[]
\begin{tikzpicture}[domain=0:4]
  \draw[very thin,color=gray] (-0.1,-1.1) grid (3.9,3.9);

  \draw[->] (-0.2,0) -- (4.2,0) node[right] {$x$};
  \draw[->] (0,-1.2) -- (0,4.2) node[above] {$f(x)$};

  \draw[color=red]    plot[id=x]   function{x}           node[right] {$f(x) =x$};
  \draw[color=blue]   plot[id=sin] function{sin(x)}      node[right] {$f(x) = \sin x$};
  \draw[color=orange] plot[id=exp] function{0.05*exp(x)} node[right] {$f(x) = \frac{1}{20} \mathrm e^x$};
\end{tikzpicture}
\end{codeexample}


The plot is influenced by the following options: First, the options
|samples| and |domain| explained earlier. Second, there are some more
specialized options.

\begin{key}{/tikz/parametric=\meta{boolean} (default true)}
  Sets whether the plot is a parametric plot. If true, then |t| must
  be used instead of |x| as the parameter and two comma-separated
  functions must be given in the \meta{gnuplot formula}. An example is
  the following:
\begin{codeexample}[]
\tikz \draw[scale=0.5,domain=-3.141:3.141,smooth]
  plot[parametric,id=parametric-example] function{t*sin(t),t*cos(t)};
\end{codeexample}
\end{key}

\begin{key}{/tikz/range=\meta{start}|:|\meta{end}}
  This key sets the range of the plot. If set, all points whose
  $y$-coordinates lie outside this range will be considered to be
  outliers and will cause jumps in the plot, by default:
\begin{codeexample}[]
\tikz \draw[scale=0.5,domain=-3.141:3.141, samples=100, smooth, range=-3:3]
  plot[id=tan-example] function{tan(x)};
\end{codeexample}
\end{key}

\begin{key}{/tikz/yrange=\meta{start}|:|\meta{end}}
  Same as |range|.
\end{key}

\begin{key}{/tikz/xrange=\meta{start}|:|\meta{end}}
  Set the $x$-range. This makes sense only for parametric plots.
\begin{codeexample}[]
\tikz \draw[scale=0.5,domain=-3.141:3.141,smooth,xrange=0:1]
  plot[parametric,id=parametric-example-cut] function{t*sin(t),t*cos(t)};
\end{codeexample}
\end{key}

\begin{key}{/tikz/id=\meta{id}}
  Sets the identifier of the current plot. This should be a unique
  identifier for each plot (though things will also work if it is not,
  but not as well, see the explanations above). The \meta{id} will be
  part of a filename, so it should not contain anything fancy like |*|
  or |$|.%$
\end{key}

\begin{key}{/tikz/prefix=\meta{prefix}}
  The \meta{prefix} is put before each plot file name. The default is
  |\jobname.|, but
  if you have many plots, it might be better to use, say |plots/| and
  have all plots placed in a directory. You have to create the
  directory yourself.
\end{key}

\begin{key}{/tikz/raw gnuplot}
  This key causes the \meta{gnuplot formula} to be passed on to
  \textsc{gnuplot} without setting up the samples or the |plot|
  operation. Thus, you could write
\begin{codeexample}[code only]
plot[raw gnuplot,id=raw-example] function{set samples 25; plot sin(x)}
\end{codeexample}
  This can be
  useful for complicated things that need to be passed to
  \textsc{gnuplot}. However, for really complicated situations you
  should create a special external generating \textsc{gnuplot} file
  and use the |file|-syntax to include the table ``by hand.''
\end{key}

The following styles influence the plot:
\begin{stylekey}{/tikz/every plot (initially \normalfont empty)}
  This style is installed in each plot, that is, as if you always said
\begin{codeexample}[code only]
  plot[every plot,...]
\end{codeexample}
 This is most useful for globally setting a prefix for all plots by saying:
\begin{codeexample}[code only]
\tikzset{every plot/.style={prefix=./text-zh/plots/}}
\end{codeexample}
\end{stylekey}



\subsection{Placing Marks on the Plot}

As we saw already, it is possible to add \emph{marks} to a plot using
the |mark| option. When this option is used, a copy of the plot
mark is placed on each point of the plot. Note that the marks are
placed \emph{after} the whole path has been drawn/filled/shaded. In
this respect, they are handled like text nodes.

In detail, the following options govern how marks are drawn:
\begin{key}{/tikz/mark=\meta{mark mnemonic}}
  Sets the mark to a mnemonic that has previously been defined using
  the |\pgfdeclareplotmark|. By default, |*|, |+|, and |x| are available,
  which draw a filled circle, a plus, and a cross as marks. Many more
  marks become available when the library |plotmarks| is
  loaded. Section~\ref{section-plot-marks} lists the available plot
  marks.

  One plot mark is special: the |ball| plot mark is available only
  in \tikzname. The |ball color| option determines the balls's color. Do not use
  this option with a large number of marks since it will take very long
  to render in PostScript.

  \begin{tabular}{lc}
    Option & Effect \\\hline \vrule height14pt width0pt
    \plotmarkentrytikz{ball}
  \end{tabular}
\end{key}

\begin{key}{/tikz/mark repeat=\meta{r}}
  This option tells \tikzname\ that only every $r$th mark should be
  drawn.

\begin{codeexample}[]
\tikz \draw plot[mark=x,mark repeat=3,smooth] file {./text-zh/plots/pgfmanual-sine.table};
\end{codeexample}
\end{key}

\begin{key}{/tikz/mark phase=\meta{p}}
  This option tells \tikzname\ that the first mark to be draw should
  be the $p$th, followed by the $(p+r)$th, then the $(p+2r)$th, and so
  on.

\begin{codeexample}[]
\tikz \draw plot[mark=x,mark repeat=3,mark phase=6,smooth] file {./text-zh/plots/pgfmanual-sine.table};
\end{codeexample}
\end{key}

\begin{key}{/tikz/mark indices=\meta{list}}
  This option allows you to specify explicitly the indices at which a
  mark should be placed. Counting starts with 1. You can use the
  |\foreach| syntax, that is, |...| can be used.

\begin{codeexample}[]
\tikz \draw plot[mark=x,mark indices={1,4,...,10,11,12,...,16,20},smooth]
  file {./text-zh/plots/pgfmanual-sine.table};
\end{codeexample}
\end{key}

\begin{key}{/tikz/mark size=\meta{dimension}}
  Sets the size of the plot marks. For circular plot marks,
  \meta{dimension} is the radius, for other plot marks
  \meta{dimension} should be about half the width and height.

  This option is not really necessary, since you achieve the same
  effect by specifying |scale=|\meta{factor} as a local option, where
  \meta{factor} is the quotient of the desired size and the default
  size. However, using |mark size| is a bit faster and more natural.
\end{key}

\begin{stylekey}{/tikz/every mark}
  This style is installed before drawing plot marks. For example,
  you can scale (or otherwise transform) the plot mark or set its
  color.
\end{stylekey}

\begin{key}{/tikz/mark options=\meta{options}}
	Redefines |every mark| such that it sets \marg{options}.
\begin{codeexample}[]
\tikz \fill[fill=blue!20]
  plot[mark=triangle*,mark options={color=blue,rotate=180}]
    file{./text-zh/plots/pgfmanual-sine.table} |- (0,0);
\end{codeexample}
\end{key}
	
\begin{stylekey}{/tikz/no marks}
	Disables markers (the same as |mark=none|).
\end{stylekey}
\begin{stylekey}{/tikz/no markers}
	Disables markers (the same as |mark=none|).
\end{stylekey}



\subsection{Smooth Plots, Sharp Plots, Jump Plots, Comb Plots and Bar Plots}

There are different things the |plot| operation can do with the points
it reads from a file or from the inlined list of points. By default,
it will connect these points by straight lines. However, you can also
use options to change the behavior of |plot|.

\begin{key}{/tikz/sharp plot}
  This is the default and causes the points to be connected by
  straight lines. This option is included only so that you can
  ``switch back'' if you ``globally'' install, say, |smooth|.
\end{key}

\begin{key}{/tikz/smooth}
  This option causes the points on the path to be connected using a
  smooth curve:

\begin{codeexample}[]
\tikz\draw plot[smooth] file{./text-zh/plots/pgfmanual-sine.table};
\end{codeexample}

  Note that the smoothing algorithm is not very intelligent. You will
  get the best results if the bending angles are small, that is, less
  than about $30^\circ$ and, even more importantly, if the distances
  between points are about the same all over the plotting path.
\end{key}

\begin{key}{/tikz/tension=\meta{value}}
  This option influences how ``tight'' the smoothing is. A lower value
  will result in sharper corners, a higher value in more ``round''
  curves. A value of $1$ results in a circle if four points at
  quarter-positions on a circle are given. The default is $0.55$. The
  ``correct'' value depends on the details of plot.

\begin{codeexample}[]
\begin{tikzpicture}[smooth cycle]
  \draw                 plot[tension=0.2]
    coordinates{(0,0) (1,1) (2,0) (1,-1)};
  \draw[yshift=-2.25cm] plot[tension=0.5]
    coordinates{(0,0) (1,1) (2,0) (1,-1)};
  \draw[yshift=-4.5cm]  plot[tension=1]
    coordinates{(0,0) (1,1) (2,0) (1,-1)};
\end{tikzpicture}
\end{codeexample}
\end{key}

\begin{key}{/tikz/smooth cycle}
  This option causes the points on the path to be connected using a
  closed smooth curve.

\begin{codeexample}[]
\tikz[scale=0.5]
  \draw plot[smooth cycle] coordinates{(0,0) (1,0) (2,1) (1,2)}
        plot               coordinates{(0,0) (1,0) (2,1) (1,2)} -- cycle;
\end{codeexample}
\end{key}

\begin{key}{/tikz/const plot}
  This option causes the points on the path to be connected using
  piecewise constant series of lines: 

\begin{codeexample}[]
\tikz\draw plot[const plot] file{./text-zh/plots/pgfmanual-sine.table};
\end{codeexample}
\end{key}

\begin{key}{/tikz/const plot mark left}
  Just an alias for |/tikz/const plot|.
\begin{codeexample}[]
\tikz\draw plot[const plot mark left,mark=*] file{./text-zh/plots/pgfmanual-sine.table};
\end{codeexample}
\end{key}

\begin{key}{/tikz/const plot mark right}
  A variant of |/tikz/const plot| which places its mark on the right ends:
\begin{codeexample}[]
\tikz\draw plot[const plot mark right,mark=*] file{./text-zh/plots/pgfmanual-sine.table};
\end{codeexample}
\end{key}

\begin{key}{/tikz/const plot mark mid}
  A variant of |/tikz/const plot| which places its mark in the middle
  of the horizontal lines: 
\begin{codeexample}[]
\tikz\draw plot[const plot mark mid,mark=*] file{./text-zh/plots/pgfmanual-sine.table};
\end{codeexample}
  More precisely, it generates vertical lines in the middle between
  each pair of consecutive points. If the mesh width is constant, this
  leads to symmetrically placed marks (``middle''). 
\end{key}


\begin{key}{/tikz/jump mark left}
  This option causes the points on the path to be drawn using
  piecewise constant, non-connected series of lines. If there are any
  marks, they will be placed on left open ends: 

\begin{codeexample}[]
\tikz\draw plot[jump mark left, mark=*] file{./text-zh/plots/pgfmanual-sine.table};
\end{codeexample}
\end{key}

\begin{key}{/tikz/jump mark right}
  This option causes the points on the path to be drawn using
  piecewise constant, non-connected series of lines. If there are any
  marks, they will be placed on right open ends: 

\begin{codeexample}[]
\tikz\draw plot[jump mark right, mark=*] file{./text-zh/plots/pgfmanual-sine.table};
\end{codeexample}
\end{key}

\begin{key}{/tikz/jump mark mid}
  This option causes the points on the path to be drawn using
  piecewise constant, non-connected series of lines. If there are any
  marks, they will be placed in the middle of the horizontal line
  segments: 

\begin{codeexample}[]
\tikz\draw plot[jump mark right, mark=*] file{./text-zh/plots/pgfmanual-sine.table};
\end{codeexample}

  In case of non--constant mesh widths, the same remarks as for |const plot mark mid| apply.
\end{key}

\begin{key}{/tikz/ycomb}
  This option causes the |plot| operation to interpret the plotting
  points differently. Instead of connecting them, for each point of
  the plot a straight line is added to the path from the $x$-axis to the point,
  resulting in a sort of ``comb'' or ``bar diagram.''

\begin{codeexample}[]
\tikz\draw[ultra thick] plot[ycomb,thin,mark=*] file{./text-zh/plots/pgfmanual-sine.table};
\end{codeexample}

\begin{codeexample}[]
\begin{tikzpicture}[ycomb]
  \draw[color=red,line width=6pt]
    plot coordinates{(0,1) (.5,1.2) (1,.6) (1.5,.7) (2,.9)};
  \draw[color=red!50,line width=4pt,xshift=3pt]
    plot coordinates{(0,1.2) (.5,1.3) (1,.5) (1.5,.2) (2,.5)};
\end{tikzpicture}
\end{codeexample}
\end{key}


\begin{key}{/tikz/xcomb}
  This option works like |ycomb| except that the bars are horizontal.

\begin{codeexample}[]
\tikz \draw plot[xcomb,mark=x] coordinates{(1,0) (0.8,0.2) (0.6,0.4) (0.2,1)};
\end{codeexample}
\end{key}


\begin{key}{/tikz/polar comb}
  This option causes a line from the origin to the point to be added
  to the path for each plot point.

\begin{codeexample}[]
\tikz \draw plot[polar comb,
     mark=pentagon*,mark options={fill=white,draw=red},mark size=4pt]
   coordinates {(0:1cm) (30:1.5cm) (160:.5cm) (250:2cm) (-60:.8cm)};
\end{codeexample}
\end{key}

\begin{key}{/tikz/ybar}
  This option produces fillable bar plots. It is thus very similar to
  |ycomb|, but it employs rectangular shapes instead of line-to
  operations. It thus allows to use any fill- or pattern style. 

\begin{codeexample}[]
\tikz\draw[draw=blue,fill=blue!60!black] plot[ybar] file{./text-zh/plots/pgfmanual-sine.table};
\end{codeexample}

\begin{codeexample}[]
\begin{tikzpicture}[ybar]
  \draw[color=red,fill=red!80,bar width=6pt]
    plot coordinates{(0,1) (.5,1.2) (1,.6) (1.5,.7) (2,.9)};
  \draw[color=red!50,fill=red!20,bar width=4pt,bar shift=3pt]
    plot coordinates{(0,1.2) (.5,1.3) (1,.5) (1.5,.2) (2,.5)};
\end{tikzpicture}
\end{codeexample}
  The use of |bar width| and |bar shift| is explained in the plot
  handler library documentation,
  section~\ref{section-plotlib-bar-handlers}. Please refer to
  page~\pageref{key-bar-width}.  
\end{key}

\begin{key}{/tikz/xbar}
  This option works like |ybar| except that the bars are horizontal.

\begin{codeexample}[]
\tikz \draw[pattern=north west lines] plot[xbar]
   coordinates{(1,0) (0.4,1) (1.7,2) (1.6,3)};
\end{codeexample}
\end{key}

\begin{key}{/tikz/ybar interval}
  As |/tikz/ybar|, this options produces vertical bars. However, bars
  are centered at coordinate \emph{intervals} instead of interval
  edges, and the bar's width is also determined relatively to the
  interval's length: 

\begin{codeexample}[]
\begin{tikzpicture}[ybar interval,x=10pt]
  \draw[color=red,fill=red!80]
    plot coordinates{(0,2) (2,1.2) (3,.3) (5,1.7) (8,.9) (9,.9)};
\end{tikzpicture}
\end{codeexample}
  Since there are $N$ intervals $[x_i,x_{i+1}]$ for given $N+1$
  coordinates, you will always have one coordinate more than bars. The
  last $y$ value will be ignored. 

  You can configure relative shifts and relative bar widths, which is
  explained in the plot handler library documentation,
  section~\ref{section-plotlib-bar-handlers}. Please refer to
  page~\pageref{key-bar-interval-width}. 
\end{key}

\begin{key}{/tikz/xbar interval}
  Works like |ybar interval|, but for horizontal bar plots.

\begin{codeexample}[]
\begin{tikzpicture}[xbar interval,x=0.5cm,y=0.5cm]
  \draw[color=red,fill=red!80]
    plot coordinates {(3,0) (2,1) (4,1.5) (1,4) (2,6) (2,7)};
\end{tikzpicture}
\end{codeexample}
\end{key}

\begin{key}{/tikz/only marks}
  This option causes only marks to be shown; no path segments are
  added to the actual path. This can be useful for quickly adding some
  marks to a path.

\begin{codeexample}[]
\tikz \draw (0,0) sin (1,1) cos (2,0)
  plot[only marks,mark=x] coordinates{(0,0) (1,1) (2,0) (3,-1)};
\end{codeexample}
\end{key}

% % Copyright 2006 by Till Tantau
%
% This file may be distributed and/or modified
%
% 1. under the LaTeX Project Public License and/or
% 2. under the GNU Free Documentation License.
%
% See the file doc/generic/pgf/licenses/LICENSE for more details.


\section{Transparency}

\label{section-tikz-transparency}


\subsection{Overview}

Normally, when you paint something using any of \tikzname's commands
(this includes stroking, filling, shading, patterns, and images), the
newly painted objects totally obscure whatever was painted earlier in
the same area.

You can change this behaviour by using something that can be thought
of as ``(semi)transparent colors.'' Such colors do not completely
obscure the background, rather they blend the background with the new
color. At first sight, using such semitransparent colors might seem quite
straightforward, but the math going on in the background is quite
involved and the correct handling of transparency fills some 64 pages
in the PDF specification.

In the present section, we start with the different ways of specifying
``how transparent'' newly drawn objects should be. The simplest way is
to just specify a percentage like ``60\% transparent.'' A much more
general way is to use something that I call a \emph{fading,} also
known as a soft mask or a mask.

At the end of the section we address the problem of creating so-called
\emph{transparency groups}. This problem arises when you paint over a
position several times with a semitransparent color. Sometimes you
want the effect to accumulate, sometimes you do not.

\emph{Note:} Transparency is best supported by the pdf\TeX\
driver. The \textsc{svg} driver also has some support. For PostScript
output, opacity is rendered correctly only with the most recent
versions of Ghostscript. Printers and other programs will typically
ignore the opacity setting.



\subsection{Specifying a Uniform Opacity}

Specifying a stroke and/or fill opacity is quite easy using the
following options.


\begin{key}{/tikz/draw opacity=\meta{value}}
  This option sets ``how transparent'' lines should be. A value of |1|
  means ``fully opaque'' or ``not transparent at all,'' a value of |0|
  means ``fully transparent'' or ``invisible.'' A value of |0.5|
  yields lines that are semitransparent.

  Note that when you use PostScript as your output format,
  this option works only with recent versions of Ghostscript.

\begin{codeexample}[]
\begin{tikzpicture}[line width=1ex]
  \draw (0,0) -- (3,1);
  \filldraw [fill=yellow!80!black,draw opacity=0.5] (1,0) rectangle (2,1);
\end{tikzpicture}
\end{codeexample}
\end{key}

Note that the |draw opacity| options only sets the opacity of drawn
lines. The opacity of fillings is set using the option
|fill opacity| (documented in Section~\ref{section-fill-opacity}. The
option |opacity| sets both at the same time.

\begin{key}{/tikz/opacity=\meta{value}}
  Sets both the drawing and filling opacity to \meta{value}.

  The following predefined styles make it easier to use this option:
  \begin{stylekey}{/tikz/transparent}
    Makes everything totally transparent and, hence, invisible.

\begin{codeexample}[]
\tikz{\fill[red]             (0,0)   rectangle (1,0.5);
      \fill[transparent,red] (0.5,0) rectangle (1.5,0.25); }
\end{codeexample}
  \end{stylekey}

  \begin{stylekey}{/tikz/ultra nearly transparent}
    Makes everything, well, ultra nearly transparent.

\begin{codeexample}[]
\tikz{\fill[red]                      (0,0)   rectangle (1,0.5);
      \fill[ultra nearly transparent] (0.5,0) rectangle (1.5,0.25); }
\end{codeexample}
  \end{stylekey}

  \begin{stylekey}{/tikz/very nearly transparent}
\begin{codeexample}[]
\tikz{\fill[red]                     (0,0)   rectangle (1,0.5);
      \fill[very nearly transparent] (0.5,0) rectangle (1.5,0.25); }
\end{codeexample}
  \end{stylekey}

  \begin{stylekey}{/tikz/nearly transparent}
\begin{codeexample}[]
\tikz{\fill[red]                (0,0)   rectangle (1,0.5);
      \fill[nearly transparent] (0.5,0) rectangle (1.5,0.25); }
\end{codeexample}
  \end{stylekey}

  \begin{stylekey}{/tikz/semitransparent}
\begin{codeexample}[]
\tikz{\fill[red]             (0,0)   rectangle (1,0.5);
      \fill[semitransparent] (0.5,0) rectangle (1.5,0.25); }
\end{codeexample}
  \end{stylekey}

  \begin{stylekey}{/tikz/nearly opaque}
\begin{codeexample}[]
\tikz{\fill[red]           (0,0)   rectangle (1,0.5);
      \fill[nearly opaque] (0.5,0) rectangle (1.5,0.25); }
\end{codeexample}
  \end{stylekey}

  \begin{stylekey}{/tikz/very nearly opaque}
\begin{codeexample}[]
\tikz{\fill[red]                (0,0)   rectangle (1,0.5);
      \fill[very nearly opaque] (0.5,0) rectangle (1.5,0.25); }
\end{codeexample}
  \end{stylekey}

  \begin{stylekey}{/tikz/ultra nearly opaque}
\begin{codeexample}[]
\tikz{\fill[red]                 (0,0)   rectangle (1,0.5);
      \fill[ultra nearly opaque] (0.5,0) rectangle (1.5,0.25); }
\end{codeexample}
  \end{stylekey}

  \begin{stylekey}{/tikz/opaque}
    This yields completely opaque drawings, which is the default.
\begin{codeexample}[]
\tikz{\fill[red]    (0,0)   rectangle (1,0.5);
      \fill[opaque] (0.5,0) rectangle (1.5,0.25); }
\end{codeexample}
  \end{stylekey}
\end{key}


\begin{key}{/tikz/fill opacity=\meta{value}}
  This option sets the opacity of fillings. In addition to filling
  operations, this opacity also applies to text and images.

  Note, again, that when you use PostScript as your output format,
  this option works only with recent versions of Ghostscript.

\begin{codeexample}[]
\begin{tikzpicture}[thick,fill opacity=0.5]
  \filldraw[fill=red]   (0:1cm)    circle (12mm);
  \filldraw[fill=green] (120:1cm)  circle (12mm);
  \filldraw[fill=blue]  (-120:1cm) circle (12mm);
\end{tikzpicture}
\end{codeexample}

\begin{codeexample}[]
\begin{tikzpicture}
  \fill[red] (0,0) rectangle (3,2);

  \node                   at (0,0) {\huge A};
  \node[fill opacity=0.5] at (3,2) {\huge B};
\end{tikzpicture}
\end{codeexample}
\end{key}

\begin{key}{/tikz/text opacity=\meta{value}}
  Sets the opacity of text labels, overriding the |fill opacity| setting.
\begin{codeexample}[]
\begin{tikzpicture}[every node/.style={fill,draw}]
  \draw[line width=2mm,blue!50,line cap=round] (0,0) grid (3,2);

  \node[opacity=0.5] at (1.5,2) {Upper node};
  \node[draw opacity=0.8,fill opacity=0.2,text opacity=1]
    at (1.5,0) {Lower node};
\end{tikzpicture}
\end{codeexample}
\end{key}


Note the following effect: If you set up a certain opacity for stroking
or filling and you stroke or fill the same area twice, the effect
accumulates:

\begin{codeexample}[]
\begin{tikzpicture}[fill opacity=0.5]
  \fill[red] (0,0) circle (1);
  \fill[red] (1,0) circle (1);
\end{tikzpicture}
\end{codeexample}

Often, this is exactly what you intend, but not always. You can use
transparency groups, see the end of this section, to change this.



\subsection{Blend Modes}
\label{section-blend-modes}

A \emph{blend mode} specifies how colors mix when you paint on a
canvas. Normally, if you paint a red box on a green circle, the red
color will completely replace the green circle. However, in some
situations you might also wish the red color to somehow ``mix'' or
``blend'' with the green circle. We already saw that, using transparency,
we can draw something without completely obscuring the
background. \emph{Blending} is a similar operation, only here we mix
colors in more complicated ways.

\emph{Note:} Blending is a rather ``advanced'' feature of
\textsc{pdf}. Most renderers, let alone printers, will have trouble
rendering blending correctly.

\begin{key}{/tikz/blend mode=\meta{mode}}
  Sets the current blend mode to \meta{mode}. Here \meta{mode} must be
  one of the modes listed below. More details on these modes can also
  be found in  Section 7.2.4 of the \textsc{pdf} Specification, version 1.7.

  In the following example, the blend mode is only used and set inside
  a transparency group (see also
  Section~\ref{section-transparency-groups}). This is because most
  renderers (viewing 
  programs) have trouble rendering blending correctly otherwise. For
  instance, at the time of writing, the versions of Adobe's Reader and
  Apple's Preview render the following drawing very differently, if
  the transparency group is not used in the following example.

\begin{codeexample}[]
\tikz {
  \begin{scope}[transparency group]
    \begin{scope}[blend mode=screen] 
      \fill[red!90!black]   ( 90:.6) circle (1);
      \fill[green!80!black] (210:.6) circle (1);
      \fill[blue!90!black]  (330:.6) circle (1);
    \end{scope}
  \end{scope}
}
\end{codeexample}

  Because of the trouble with rendering blending correctly outside
  transparency groups, there is a special key that establishes a
  transparency group and sets a blend mode simultaneously:
  
  \begin{key}{/tikz/blend group=\meta{mode}}
    This key can only be used with a scope (like
    |transparency group|). It will cause the current scope to become a
    transparency group and, inside this group, the blend mode will be
    set to \meta{mode}.

\begin{codeexample}[]
\tikz [blend group=screen] {
  \fill[red!90!black]   ( 90:.6) circle (1);
  \fill[green!80!black] (210:.6) circle (1);
  \fill[blue!90!black]  (330:.6) circle (1);
}
\end{codeexample}
  \end{key}

  Here is an overview of the effects of the different available blend
  modes. In the examples, we always have three circles drawn on
  top of each other (as in the example code earlier): We start with a
  triple of pure red, green, and blue. Below it, we have a triple of
  light versions of these three colors (|red!50|, |green!50|, and
  |blue!50|). Next comes the triple  yellow, cyan, and magenta; again
  with a triple of light versions below it. The large example consists
  of three balls (produced using |ball color|) having the colors red,
  green, and blue, are drawn on top of each other just like the
  circles.  
  
  \definecolor{rg}{rgb}{1,1,0}
  \definecolor{gb}{rgb}{0,1,1}
  \definecolor{br}{rgb}{1,0,1}
  
  \def\makeline#1#2#3{\leavevmode
    \hbox to 4cm{#1\hss}\ \hbox to
    2cm{#2\hss}\ \begin{minipage}[t]{9cm}\raggedright#3\end{minipage}\par
    \textcolor{black!25}{\hrule height1pt}
  }

  \def\showmode#1#2{
    \makeline{
    \tikz [blend mode=#1,baseline=-.5ex] {
      \fill[red]      ( 90:.5em) circle (.75em);
      \fill[green]    (210:.5em) circle (.75em);
      \fill[blue]     (330:.5em) circle (.75em);
      \scoped[yshift=-2.5em]{
        \fill[red!50]   ( 90:.5em) circle (.75em);
        \fill[green!50] (210:.5em) circle (.75em);
        \fill[blue!50]  (330:.5em) circle (.75em);
      }
    }
    \tikz [blend mode=#1,baseline=-.5ex] {
      \fill[rg]   ( 90:.5em) circle (.75em);
      \fill[gb]  (210:.5em) circle (.75em);
      \fill[br]    (330:.5em) circle (.75em);
      \scoped[yshift=-2.5em]{
        \fill[rg!50]  ( 90:.5em) circle (.75em);
        \fill[gb!50]  (210:.5em) circle (.75em);
        \fill[br!50]  (330:.5em) circle (.75em);
      }
    }
    \tikz [blend mode=#1,baseline=-.5ex+1.25em] {
      \shade[ball color=red]      ( 90:1em) circle (1.5em);
      \shade[ball color=green]    (210:1em) circle (1.5em);
      \shade[ball color=blue]     (330:1em) circle (1.5em);
    }}{|#1|}{#2}}

  \medskip
  \makeline{\emph{Example}}{\emph{Mode}}{\emph{Explanations quoted from Table 7.2 of the
      \textsc{pdf} Specification, Version 1.7}}
  \showmode{normal}{When painting a pixel with a some color (called
    the ``source color''), the background color
      (called the ``backdrop'') is completely ignored.}
    \showmode{multiply}{Multiplies the backdrop and source color
      values. The result color is always at least as dark as
      either of the two constituent colors. Multiplying any color with
      black produces black; multiplying with white leaves the original
      color unchanged. Painting successive overlapping objects with a
      color other than black or white produces progressively darker
      colors.}
    \showmode{screen}{Multiplies the complements of the backdrop and
      source color       values, then complements the result. The
      result color is always 
      at least as light as either of the two constituent
      colors. Screening any color with white produces white; screening
      with black leaves the original color unchanged. The effect is
      similar to projecting multiple photographic slides
      simultaneously onto a single screen.}
    \showmode{overlay}{Multiplies or screens the colors, depending on
      the backdrop color value. Source colors overlay the backdrop
      while preserving its highlights and shadows. The backdrop color
      is not replaced but is mixed with the source color to reflect
      the lightness or darkness of the backdrop.}
    \showmode{darken}{Selects the darker of the backdrop and source
      colors. The backdrop is replaced with the source where the
      source is darker; otherwise, it is left unchanged.}
    \showmode{lighten}{Selects the lighter of the backdrop and source
      colors. The backdrop is replaced with the source where the
      source is lighter; otherwise, it is left unchanged.}
    \showmode{color dodge}{Brightens the backdrop color to reflect the
      source color. Painting with black produces no changes.}
    \showmode{color burn}{Darkens the backdrop color to reflect the
      source color. Painting with white produces no change.}
    \showmode{hard light}{Multiplies or screens the colors, depending
      on the source color value. The effect is similar to shining a
      harsh spotlight on the backdrop.}
    \showmode{soft light}{Darkens or lightens the colors, depending on
      the source color value. The effect is similar to shining a
      diffused spotlight on the backdrop.}
    \showmode{difference}{Subtracts the darker of the two constituent
      colors from the lighter color. Painting with white inverts the
      backdrop color; painting with black produces no change.}
    \showmode{exclusion}{Produces an effect similar to that of the
      Difference mode but lower in contrast. Painting with white
      inverts the backdrop color; painting with black produces no
      change.}
    \showmode{hue}{Creates a color with the hue of the source color
      and the saturation and luminosity of the backdrop color.} 
    \showmode{saturation}{Creates a color with the saturation of the
      source color and the hue and luminosity of the backdrop
      color. Painting with this mode in an area of the backdrop that
      is a pure gray (no saturation) produces no change.}
    \showmode{color}{Creates a color with the hue and saturation of
      the source color and the luminosity of the backdrop color. This
      preserves the gray levels of the backdrop and is useful for
      coloring monochrome images or tinting color images.}
    \showmode{luminosity}{Creates a color with the luminosity of the
      source color and the hue and saturation of the backdrop
      color. This produces an inverse effect to that of the Color
      mode.}
  
\end{key}



\subsection{Fadings}

For complicated graphics, uniform transparency settings are not always
sufficient. Suppose, for instance, that while you paint a picture, you
want the transparency to vary smoothly from completely opaque to
completely transparent. This is a ``shading-like'' transparency. For
such a form of transparency I will use the term \emph{fading} (as a
noun). They are also known as \emph{soft masks}, \emph{opacity masks},
\emph{masks}, or \emph{soft clips}.


\subsubsection{Creating Fadings}

How do we specify a fading? This is a bit of an art since the
underlying mechanism is quite powerful, but a bit difficult to use.

Let us start with a bit of terminology. A \emph{fading} specifies for
each point of an area the transparency of that point. This transparency
can by any number between 0 and 1. A \emph{fading picture} is a normal
graphic that, in a way to be described in a moment, determines the
transparency of points inside the fading. Each fading has an
underlying fading picture.

The fading picture is a normal graphic drawn using any of the normal
graphic drawing commands. A fading and its fading picture are related
as follows: Given any point of the fading, the transparency of this
point is determined by the luminosity of the fading picture at the
same position. The luminosity of a point determines ``how bright'' the
point is. The brighter the point in the fading picture, the more
opaque is the point in the fading. In particular, a white point of the
fading picture is completely opaque in the fading and a black point of
the fading picture is completely transparent in the fading. (The
background of the fading picture is always transparent in the fading
as if the background were black.)

It is rather counter-intuitive that a \emph{white} pixel of the fading
picture will be \emph{opaque} in the fading and a \emph{black} pixel
will be \emph{transparent}. For this reason, \tikzname\ defines a
color called |transparent| that is the same as |black|. The nice thing
about this definition is that the color
|transparent!|\meta{percentage} in the fading picture yields a
pixel that is \meta{percentage} percent transparent in the fading.

Turning a fading picture into a normal picture is achieved using the
following commands, which are \emph{only defined in the library},
namely the library |fadings|. So, to use them, you have to say
|\usetikzlibrary{fadings}| first.

\begin{environment}{{tikzfadingfrompicture}\oarg{options}}
  This command works like a |{tikzpicture}|, only the picture is not
  shown, but instead a fading is defined based on this picture. To set
  the name of the picture, use the |name| option (which is normally
  used to set the name of a node).
  \begin{key}{/tikz/name=\marg{name}}
    Use this option with the |{tikzfadingfrompicture}| environment to
    set the name of the fading. You \emph{must} provide this option.
  \end{key}

  The following shading is 2cm by 2cm and gets more and more
  transparent from left to right, but is 50\% transparent for a large
  circle in the middle.
{\tikzexternaldisable
\begin{codeexample}[]
\begin{tikzfadingfrompicture}[name=fade right]
  \shade[left color=transparent!0,
         right color=transparent!100] (0,0) rectangle (2,2);
  \fill[transparent!50] (1,1) circle (0.7);
\end{tikzfadingfrompicture}

% Now we use the fading in another picture:
\begin{tikzpicture}
  % Background
  \fill [black!20] (-1.2,-1.2) rectangle (1.2,1.2);
  \pattern [pattern=checkerboard,pattern color=black!30]
                   (-1.2,-1.2) rectangle (1.2,1.2);

  \fill [path fading=fade right,red] (-1,-1) rectangle (1,1);
\end{tikzpicture}
\end{codeexample}
  In the next example we create a fading picture that contains some
  text. When the fading is used, we only see the shading ``through
  it.''
\begin{codeexample}[]
\begin{tikzfadingfrompicture}[name=tikz]
  \node [text=transparent!20]
    {\fontfamily{ptm}\fontsize{45}{45}\bfseries\selectfont Ti\emph{k}Z};
\end{tikzfadingfrompicture}

% Now we use the fading in another picture:
\begin{tikzpicture}
  \fill [black!20] (-2,-1) rectangle (2,1);
  \pattern [pattern=checkerboard,pattern color=black!30]
                   (-2,-1) rectangle (2,1);

  \shade[path fading=tikz,fit fading=false,
         left color=blue,right color=black]
    (-2,-1) rectangle (2,1);
\end{tikzpicture}
\end{codeexample}
}%

  The same effect can also be achieved using knockout groups, see
  Section~\ref{section-transparency-groups}.
\end{environment}

\begin{plainenvironment}{{tikzfadingfrompicture}\oarg{options}}
  The plain\TeX\ version of the environment.
\end{plainenvironment}

\begin{contextenvironment}{{tikzfadingfrompicture}\oarg{options}}
  The Con\TeX t version of the environment.
\end{contextenvironment}

\begin{command}{\tikzfading\oarg{options}}
  This command is used to define a fading similarly to the way a
  shading is defined. In the \meta{options} you should
  \begin{enumerate}
  \item use the |name=|\meta{name} option to set a name for the fading,
  \item use the |shading| option to set the name of the shading that
    you wish to use,
  \item extra options for setting the colors of the shading (typically
    you will set them to the color |transparent!|\meta{percentage}).
  \end{enumerate}
  Then, a new fading named \meta{name} will be created based on the
  shading.

\begin{codeexample}[]
\tikzfading[name=fade right,
            left color=transparent!0,
            right color=transparent!100]

% Now we use the fading in another picture:
\begin{tikzpicture}
  % Background
  \fill [black!20] (-1.2,-1.2) rectangle (1.2,1.2);
  \path [pattern=checkerboard,pattern color=black!30]
                   (-1.2,-1.2) rectangle (1.2,1.2);

  \fill [red,path fading=fade right] (-1,-1) rectangle (1,1);
\end{tikzpicture}
\end{codeexample}

\begin{codeexample}[]
\tikzfading[name=fade out,
            inner color=transparent!0,
            outer color=transparent!100]

% Now we use the fading in another picture:
\begin{tikzpicture}
  % Background
  \fill [black!20] (-1.2,-1.2) rectangle (1.2,1.2);
  \path [pattern=checkerboard,pattern color=black!30]
                   (-1.2,-1.2) rectangle (1.2,1.2);

  \fill [blue,path fading=fade out] (-1,-1) rectangle (1,1);
\end{tikzpicture}
\end{codeexample}
\end{command}



\subsubsection{Fading a Path}

A fading specifies for each pixel of a certain area how transparent
this pixel will be. The following options are used to install such a
fading for the current scope or path.

\pgfdeclarefading{fade down}{%
  \tikzset{top color=pgftransparent!0,bottom color=pgftransparent!100}
  \pgfuseshading{axis}
}
\pgfdeclarefading{fade inside}{%
  \tikzset{inner color=pgftransparent!90,outer color=pgftransparent!30}
  \pgfuseshading{radial}
}

\begin{key}{/tikz/path fading=\meta{name} (default \normalfont scope's setting)}
  This option tells \tikzname\ that the current path should be faded
  with the fading \meta{name}. If no \meta{name} is given, the
  \meta{name} set for the whole scope is used. Similarly to options
  like |draw| or |fill|, this option is reset for each path, so you
  have to add it to each path that should be faded. You can also
  specify |none| as \meta{name}, in which case fading for the path
  will be switched off in case it has been switched on by previous
  options or styles.
\begin{codeexample}[]
\begin{tikzpicture}[path fading=south]
  % Checker board
  \fill [black!20] (0,0) rectangle (4,3);
  \pattern [pattern=checkerboard,pattern color=black!30]
                   (0,0) rectangle (4,3);

  \fill [color=blue]                   (0.5,1.5) rectangle +(1,1);
  \fill [color=blue,path fading=north] (2.5,1.5) rectangle +(1,1);

  \fill [color=red,path fading]        (1,0.75) ellipse (.75 and .5);
  \fill [color=red]                    (3,0.75) ellipse (.75 and .5);
\end{tikzpicture}
\end{codeexample}

  \begin{key}{/tikz/fit fading=\meta{boolean} (default true, initially true)}
    When set to |true|, the fading is shifted and resized (in exactly
    the same way as a shading) so that it covers the current
    path. When set to |false|, the fading is only shifted so that it
    is centered on the path's center, but it is not resized. This can
    be useful for special-purpose fadings, for instance when you use a
    fading to ``punch out'' something.                                     
  \end{key}

  \begin{key}{/tikz/fading transform=\meta{transformation options}}
    The \meta{transformation options} are applied to the fading before
    it is used. For instance, if \meta{transformation options} is set
    to |rotate=90|, the fading is rotated by 90 degrees.
\begin{codeexample}[]
\begin{tikzpicture}[path fading=fade down]
  % Checker board
  \fill [black!20] (0,0) rectangle (4,1.5);
  \path [pattern=checkerboard,pattern color=black!30] (0,0) rectangle (4,1.5);

  \fill [red,path fading,fading transform={rotate=90}]
    (1,0.75) ellipse (.75 and .5);
  \fill [red,path fading,fading transform={rotate=30}]
    (3,0.75) ellipse (.75 and .5);
\end{tikzpicture}
\end{codeexample}
  \end{key}

  \begin{key}{/tikz/fading angle=\meta{degree}}
    A shortcut for |fading transform={rotate=|\meta{degree}|}|.
  \end{key}

  Note that you can ``fade just about anything.'' In particular, you
  can fade a shading.

\begin{codeexample}[]
\begin{tikzpicture}
  % Checker board
  \fill [black!20] (0,0) rectangle (4,4);
  \path [pattern=checkerboard,pattern color=black!30] (0,0) rectangle (4,4);

  \shade [ball color=blue,path fading=south] (2,2) circle (1.8);
\end{tikzpicture}
\end{codeexample}

  The |fade inside| of the following example is more transparent in the middle than on the
  outside.

\begin{codeexample}[]
\tikzfading[name=fade inside,
            inner color=transparent!80,
            outer color=transparent!30]
\begin{tikzpicture}
  % Checker board
  \fill [black!20] (0,0) rectangle (4,4);
  \path [pattern=checkerboard,pattern color=black!30] (0,0) rectangle (4,4);

  \shade [ball color=red] (3,3) circle (0.8);
  \shade [ball color=white,path fading=fade inside] (2,2) circle (1.8);
\end{tikzpicture}
\end{codeexample}

  Note that adding the |path fading| option to a node fades the
  (background) path, not the text itself. To fade the text, you need
  to use a scope fading (see below).
\end{key}

Note that using fadings in conjunction with patterns can create
visually rather pleasing effects:
\begin{codeexample}[]
\tikzfading[name=middle,
            top color=transparent!50,
            bottom color=transparent!50,
            middle color=transparent!20]
\begin{tikzpicture}
  \node      [circle,circular drop shadow,
              pattern=horizontal lines dark blue,
              path fading=south,
              minimum size=3.6cm] {};
  \pattern   [path fading=north,
              pattern=horizontal lines dark gray]
    (0,0) circle (1.8cm);
  \pattern   [path fading=middle,
              pattern=crosshatch dots light steel blue]
    (0,0) circle (1.8cm);
\end{tikzpicture}
\end{codeexample}


\subsubsection{Fading a Scope}

In addition to fading individual paths, you may also wish to ``fade a
scope,'' that is, you may wish to install a fading that is used
globally to specify the transparency for all objects drawn inside a
scope. This effect can also be thought of as a ``soft clip'' and it
works in a similar way: You add the |scope fading| option to a path in
a scope -- typically the first one -- and then all subsequent drawings
in the scope are faded. You will use a |transparency group| in
conjunction, see the end of this section.

\begin{key}{/tikz/scope fading=\meta{fading}}
  In principle, this key works in exactly the same way as the
  |path fading| key. The only difference is, that the effect of the
  fading will persist after the current path till the end of the
  scope. Thus, the \meta{fading} is applied to all subsequent drawings
  in the current scope, not just to the current path. In this regard,
  the option works very much like the |clip| option. (Note, however,
  that, unlike the |clip| option, fadings to not accumulate unless a
  transparency group is used.)

  The keys |fit fading| and |fading transform| have the same effect as
  for |path fading|. Also that, just as for |path fading|, providing
  the |scope fading| option with a |{scope}| only sets the name of the
  fading to be used. You have to explicitly provide the |scope fading|
  with a path to actually install a fading.

\begin{codeexample}[]
\begin{tikzpicture}
  \fill [black!20] (-2,-2) rectangle (2,2);
  \pattern [pattern=checkerboard,pattern color=black!30]
                   (-2,-2) rectangle (2,2);

  % The bounding box of the shading:
  \draw [red] (-50bp,-50bp) rectangle (50bp,50bp);

  \path [scope fading=south,fit fading=false] (0,0);
  % fading is centered at its natural size

  \fill[red]   ( 90:1) circle (1);
  \fill[green] (210:1) circle (1);
  \fill[blue]  (330:1) circle (1);
\end{tikzpicture}
\end{codeexample}

  In the following example we resize the fading to the size of the
  whole picture:
\begin{codeexample}[]
\begin{tikzpicture}
  \fill [black!20] (-2,-2) rectangle (2,2);
  \pattern [pattern=checkerboard,pattern color=black!30]
                   (-2,-2) rectangle (2,2);

  \path [scope fading=south] (-2,-2) rectangle (2,2);

  \fill[red]   ( 90:1) circle (1);
  \fill[green] (210:1) circle (1);
  \fill[blue]  (330:1) circle (1);
\end{tikzpicture}
\end{codeexample}

  Scope fadings are also needed if you wish to fade a node.
\begin{codeexample}[]
\tikz \node [scope fading=south,fading angle=45,text width=3.5cm]
{
  This is some text that will fade out as we go right
  and down. It is pretty hard to achieve this effect in
  other ways.
};
\end{codeexample}

\end{key}


\subsection{Transparency Groups}
\label{section-transparency-groups}

Consider the following cross and sign. They ``look wrong'' because we
can see how they were constructed, while this is not really part of
the desired effect.

\begin{codeexample}[]
\begin{tikzpicture}[opacity=.5]
  \draw [line width=5mm] (0,0) -- (2,2);
  \draw [line width=5mm] (2,0) -- (0,2);
\end{tikzpicture}
\end{codeexample}

\begin{codeexample}[]
\begin{tikzpicture}
  \node at (0,0) [forbidden sign,line width=2ex,draw=red,fill=white] {Smoking};

  \node [opacity=.5]
        at (2,0) [forbidden sign,line width=2ex,draw=red,fill=white] {Smoking};
\end{tikzpicture}
\end{codeexample}

Transparency groups are used to render them correctly:

\begin{codeexample}[]
\begin{tikzpicture}[opacity=.5]
  \begin{scope}[transparency group]
    \draw [line width=5mm] (0,0) -- (2,2);
    \draw [line width=5mm] (2,0) -- (0,2);
  \end{scope}
\end{tikzpicture}
\end{codeexample}

\begin{codeexample}[]
\begin{tikzpicture}
  \node at (0,0) [forbidden sign,line width=2ex,draw=red,fill=white] {Smoking};

  \begin{scope}[opacity=.5,transparency group]
    \node at (2,0) [forbidden sign,line width=2ex,draw=red,fill=white]
      {Smoking};
  \end{scope}
\end{tikzpicture}
\end{codeexample}

\begin{key}{/tikz/transparency group=\oarg{options}}
  This option can be given to a |scope|. It will have the following
  effect: The scope's contents is stroked\,/\penalty0\,filled
  ``ignoring any outside transparency.'' This means, all previous
  transparency settings are ignored (you can still set transparency
  inside the group, but never mind). For instance, in the forbidden
  sign example, the whole sign is first painted (conceptually) like
  the image on the left hand side. Note that some pixels of the sign
  are painted multiple times (up to three times), but only the last
  color ``wins.''

  Then, when the scope is finished, it is painted as a whole. The
  \emph{fill} transparency settings are now applied to the resulting
  picture. For instance, the pixel that has been painted three times
  is just red at the end, so this red color will be blended with
  whatever is ``behind'' the group on the page.

\begin{codeexample}[]
\begin{tikzpicture}
  \pattern[pattern=checkerboard,pattern color=black!15](-1,-1) rectangle (3,1);
  \node at (0,0) [forbidden sign,line width=2ex,draw=red,fill=white] {Smoking};

  \begin{scope}[transparency group,opacity=.5]
    \node at (2,0) [forbidden sign,line width=2ex,draw=red,fill=white]
      {Smoking};
  \end{scope}
\end{tikzpicture}
\end{codeexample}

  Note that in the example, the |opacity=.5| is not active inside the
  transparency group: The group is only established at beginning of
  the scope and all options given to the |{scope}| environment are set
  before the group is established. To change the opacity \emph{inside}
  the group, you need to open another scope inside it or use the
  |opacity| key with a command inside the group:

\begin{codeexample}[]
\begin{tikzpicture}
  \pattern[pattern=checkerboard,pattern color=black!15](-1,-1) rectangle (3,1);
  \node at (0,0) [forbidden sign,line width=2ex,draw=red,fill=white] {Smoking};

  \begin{scope}[transparency group,opacity=.5]
    \node (s) at (2,0) [forbidden sign,line width=2ex,draw=red,fill=white]
    {Smoking};

    \draw [opacity=.5, line width=2ex, blue] (1.2,0) -- (2.8,0);
  \end{scope}
\end{tikzpicture}
\end{codeexample}

  The \meta{options} are a list of comma-separated options:
  \begin{itemize}
  \item \declare{|knockout|} When this option is given inside the
    \meta{options}, the group becomes a so-called \emph{knockout}
    group. This means, essentially, that inside the group everything
    is painted as if the ``opacity'' of a line or area were just
    another color channel. In particular, if you paint a pixel with
    opacity $0$ inside a knockout group, this pixel becomes perfectly
    transparent immediately. In contrast, painting a pixel with
    something of opacity 0 normally has no effect.

    Not all renderers, let alone printers, will support
    this. At the time of writing, Apple's Preview will not show the
    following correctly (you should see the text \tikzname\ in the
    middle): 
\begin{codeexample}[]
\begin{tikzpicture}
  \shade [left color=red,right color=blue] (-2,-1) rectangle (2,1);
  \begin{scope}[transparency group=knockout]
    \fill [white] (-1.9,-.9) rectangle (1.9,.9);
    \node [opacity=0,font=\fontfamily{ptm}\fontsize{45}{45}\bfseries]
          {Ti\emph{k}Z};
  \end{scope}
\end{tikzpicture}
\end{codeexample}
   In the example, we first draw a large shading and then, inside the
   transparency group ``overwrite'' most of this shading by a big
   white rectangle. The interesting part is the text of the node,
   which has opacity |0|. Normally, this would mean that nothing is
   shown. However, in a knockout group, we ``paint'' the text with an
   ``opacity zero'' color. The effect is that part of the totally
   opaque white rectangle gets overwritten by a perfectly transparent
   area (namely exactly the area taken up by the pixels of the
   text). When this whole knockout group is then placed on top of the
   shading, the shading will ``shine through'' at the knocked-out
   pixels.

  \item \declare{|isolated|}|=false| A group can be isolated or
    not. By default, they are isolated, since this is typically what you
    want. For details on what isolated groups are, exactly, see
    Section~7.3.4 of the \textsc{pdf} Specification, version 1.7.
  \end{itemize}

  Note that when a transparency group is created, \tikzname\ must
  correctly determine the size of the material inside the
  group. Usually, this is no problem, but when you use things like
  |overlay| or |transform canvas|, trouble may result. In this case,
  please consult Section~\ref{section-transparency} on how to sidestep
  this problem in such cases.
\end{key}




%%% Local Variables:
%%% mode: latex
%%% TeX-master: "pgfmanual"
%%% End:

% % Copyright 2008 by Mark Wibrow
%
% This file may be distributed and/or modified
%
% 1. under the LaTeX Project Public License and/or
% 2. under the GNU Free Documentation License.
%
% See the file doc/generic/pgf/licenses/LICENSE for more details.

\section{Decorated Paths}

\label{section-tikz-decorations}


\subsection{Overview}

Decorations are a general concept to make (sub)paths ``more
interesting.'' Before we have a look at the details, let us have a
look at some examples:

\begin{codeexample}[]
\begin{tikzpicture}[thick]
  \draw                                                (0,3)   -- (3,3);
  \draw[decorate,decoration=zigzag]                    (0,2.5) -- (3,2.5);
  \draw[decorate,decoration=brace]                     (0,2)   -- (3,2);
  \draw[decorate,decoration=triangles]                 (0,1.5) -- (3,1.5);
  \draw[decorate,decoration={coil,segment length=4pt}] (0,1)   -- (3,1);
  \draw[decorate,decoration={coil,aspect=0}]           (0,.5)  -- (3,.5);
  \draw[decorate,decoration={expanding waves,angle=7}] (0,0)   -- (3,0);
\end{tikzpicture}
\end{codeexample}

\begin{codeexample}[]
\begin{tikzpicture}
  \node [fill=red!20,draw,decorate,decoration={bumps,mirror},
         minimum height=1cm]
  {Bumpy};
\end{tikzpicture}
\end{codeexample}

\begin{codeexample}[]
\begin{tikzpicture}
  \filldraw[fill=blue!20]                    (0,3)
  decorate [decoration=saw]             { -- (3,3) }
  decorate [decoration={coil,aspect=0}] { -- (2,1) }
  decorate [decoration=bumps]           { -| (0,3) };
\end{tikzpicture}
\end{codeexample}

\begin{codeexample}[]
\begin{tikzpicture}
  \node [fill=yellow!50,draw,thick, minimum height=2cm, minimum width=3cm,
         decorate, decoration={random steps,segment length=3pt,amplitude=1pt}]
    {Saved from trash};
\end{tikzpicture}
\end{codeexample}

The general idea of decorations is the following: First, you construct
a path using the usual path construction commands. The resulting path
is, in essence, a series of straight and curved lines. Instead of
directly using this path for filling or drawing, you can then specify
that it should form the basis for a decoration. In this case,
depending on which decoration you use, a new path is constructed
``along'' the path you specified. For instance, with the |zigzag|
decoration, the new path is a zigzagging line that goes along the old
path.

Let us have a look at an example: In the first picture, we see a path
that consists of a line, an arc, and a line. In the second picture,
this path has been used as the basis of a decoration.

\begin{codeexample}[]
\tikz \fill
  [fill=blue!20,draw=blue,thick] (0,0) -- (2,1) arc (90:-90:.5) -- cycle;
\end{codeexample}
\begin{codeexample}[]
\tikz \fill [decorate,decoration={zigzag}]
  [fill=blue!20,draw=blue,thick] (0,0) -- (2,1) arc (90:-90:.5) -- cycle;
\end{codeexample}

It is also possible to decorate only a subpath (the exact syntax will
be explained later in this section).
\begin{codeexample}[]
\tikz \fill [decoration={zigzag}]
  [fill=blue!20,draw=blue,thick] (0,0) -- (2,1)
    decorate { arc (90:-90:.5) } -- cycle;
\end{codeexample}

The |zigzag| decoration will be called a \emph{path
  morphing} decoration because it morphs a path into a different, but
topologically equivalent path. Not all decorations are path
morphing; rather there are three kinds of decorations.


\begin{enumerate}
\item The just-mentioned \emph{path morphing} decorations morph the
  path in the sense that what used to be a straight  line might
  afterwards be a squiggly line or might have bumps. However, a line
  is still and a line and path deforming decorations do not change
  the number of subpaths.

  Examples of such decorations are the |snake| or the |zigzag|
  decoration. Many such decorations are defined in the library
  |decorations.pathmorphing|.

\item \emph{Path replacing} decorations completely replace the
  path by a different path that is only ``loosely based'' on the
  original path. For instance, the |crosses| decoration replaces a path
  by a path consisting of a sequence of crosses. Note how in the
  following example filling the path has no effect since the path
  consist only of (numerous) unconnected straight line subpaths:
\begin{codeexample}[]
\tikz \fill [decorate,decoration={crosses}]
  [fill=blue!20,draw=blue,thick] (0,0) -- (2,1) arc (90:-90:.5) -- cycle;
\end{codeexample}

  Examples of path replacing decorations are |crosses| or |ticks| or
  |shape backgrounds|. Such decorations are defined in the library
  |decorations.pathreplacing|, but also in |decorations.shapes|.

\item \emph{Path removing} decorations completely remove the
  to-be-decorated path. Thus, they have no effect on the main path
  that is being constructed. Instead, they typically have numerous
  \emph{side  effects}. For instance, they might ``write some text''
  along the (removed) path or they might place nodes along this
  path. Note that for such decorations the path usage command for the
  main path have no influence on how the decoration looks like.

\begin{codeexample}[]
\tikz \fill [decorate,decoration={text along path,
               text=This is a text along a path. Note how the path is lost.}]
  [fill=blue!20,draw=blue,thick] (0,0) -- (2,1) arc (90:-90:.5) -- cycle;
\end{codeexample}
\end{enumerate}

Decorations are defined in different decoration libraries, see
Section~\ref{section-library-decorations} for details. It is also
possible to define your own decorations, see
Section~\ref{section-base-decorations}, but you need to use the
\pgfname\ basic layer and a bit of theory is involved.

Decorations can be used to decorate already decorated paths. In the
following three graphics, we start with a simple path, then decorate
it once, and then decorate the decorated path once more.

\begin{codeexample}[]
\tikz \fill [fill=blue!20,draw=blue,thick]
  (0,0) rectangle (3,2);
\end{codeexample}
\begin{codeexample}[]
\tikz \fill [fill=blue!20,draw=blue,thick]
  decorate[decoration={zigzag,segment length=10mm,amplitude=2.5mm}]
    { (0,0) rectangle (3,2) };
\end{codeexample}
\begin{codeexample}[]
\tikz \fill [fill=blue!20,draw=blue,thick]
  decorate[decoration={crosses,segment length=2mm}] {
    decorate[decoration={zigzag,segment length=10mm,amplitude=2.5mm}] {
      (0,0) rectangle (3,2)
    }
  };
\end{codeexample}

One final word of warning: Decorations can be pretty slow to
typeset and they can be inaccurate. The reason is that \pgfname\ has
to a \emph{lot} of rather difficult computations in the background and
\TeX\ is not very good at doing math. Decorations are fastest when
applied to straight line segments, but even then they are much slower
than other alternatives. For instance, the |ticks| decoration can be
simulated by clever use of a dashing pattern and the dashing pattern
will literally be thousands of times faster to typeset. However, for
most decorations there are no real alternatives.

\begin{tikzlibrary}{decorations}
  In order to use decorations, you first have to load a decoration
  library. This |decoration| library defines the basic options
  described in the following, but it does not define any new
  decorations. This is done by libraries like
  |decorations.text|. Since these more specialized libraries include
  the |decoration| library automatically, you usually do not have to
  bother about it.
\end{tikzlibrary}



\subsection{Decorating a Subpath Using the Decorate Path Command}

The most general way to decorate a (sub)path is the following path
command.

\begin{pathoperation}{decorate}{\opt{\oarg{options}}\marg{subpath}}
  This path operation causes the \meta{subpath} to be
  decorated using the current decoration. Depending on the decoration,
  this may or may not extend the current path.
\begin{codeexample}[]
\begin{tikzpicture}
  \draw [help lines] grid (3,2);
  \draw decorate [decoration={name=zigzag}]
         { (0,0) .. controls (0,2) and (3,0) .. (3,2) |- (0,0) };
\end{tikzpicture}
\end{codeexample}
  The path can include straight lines, curves,
  rectangles, arcs, circles, ellipses, and even already decorated
  paths (that is, you can nest applications of the |decorate| path
  command, see below).

  Due to the limits on the precision in  \TeX, some inaccuracies in
  positioning when crossing input segment boundaries may occasionally be
  found.

  You can use nodes normally inside the \meta{subpath}.
\begin{codeexample}[]
\begin{tikzpicture}
  \draw [help lines] grid (3,2);
  \draw decorate [decoration={name=zigzag}]
    { (0,0) -- (2,2) node (hi) [left,draw=red] {Hi!} arc(90:0:1)};

  \draw [blue] decorate [decoration={crosses}] {(3,0) -- (hi)};
\end{tikzpicture}
\end{codeexample}

  The following key is used to select the decoration and also to
  select further ``rendering options'' for the decoration.

  \begin{key}{/pgf/decoration=\meta{decoration options}}
    \keyalias{tikz}
    This option is used to specify which decoration is used and how it
    will look like. Note that this key will \emph{not} cause any
    decorations to be applied, immediately. It takes the |decorate| path
    command or the |decorate| option to actually decorate a path. The
    |decoration| option is only used to specify which decoration should
    be used, in principle. You can also use this option at the
    beginning of a picture or a scope to specify the decoration to be
    used with each invocation of the |decorate| path
    command. Naturally, any local options of the |decorate| path
    command override these ``global'' options.
\begin{codeexample}[]
\begin{tikzpicture}[decoration=zigzag]
  \draw       decorate                      {(0,0) -- (3,2)};
  \draw [red] decorate [decoration=crosses] {(0,2) -- (3,0)};
\end{tikzpicture}
\end{codeexample}

    The \meta{decoration options} are special options
    (which have the path prefix |/pgf/decoration/|) that determine the
    properties of the decoration. Which options are appropriate for a
    decoration strongly depend on the decoration, you will have to look
    up the appropriate options in the documentation of the decoration,
    see Section~\ref{section-library-decorations}.

    There is one option (available only in \tikzname) that is special:
    \begin{key}{/pgf/decoration/name=\meta{name} (initially none)}
      Use this key to set which decoration is to be used. The
      \meta{name} can both be a decoration or a meta-decoration (you
      need to worry about the difference only if you wish to define
      your own decorations).

      If you set \meta{name} to |none|, no decorations are added.
\begin{codeexample}[]
\begin{tikzpicture}
  \draw [help lines] grid (3,2);
  \draw decorate [decoration={name=zigzag}]
         { (0,0) .. controls (0,2) and (3,0) .. (3,2) };
\end{tikzpicture}
\end{codeexample}
      Since this option is used so often, you can also leave out the
      |name=| part. Thus, the above example can be rewritten more
      succinctly:
\begin{codeexample}[]
\begin{tikzpicture}
  \draw [help lines] grid (3,2);
  \draw decorate [decoration=zigzag]
         { (0,0) .. controls (0,2) and (3,0) .. (3,2) };
\end{tikzpicture}
\end{codeexample}
      In general, when \meta{decoration options} are parsed, for each
      unknown key it is checked whether that key happens to be a
      (meta-)decoration and, if so, the |name| option is executed for
      this key.
    \end{key}

    Further options allow you to adjust the position of decorations
    relative to the to-be-decorated path. See
    Section~\ref{section-decorations-adjust} below for details.
  \end{key}

  Recall that some decorations actually completely remove the
  to-be-decorated path. In such cases, the construction of the main
  path is resumed after the |decorate| path command ends.

\begin{codeexample}[]
\begin{tikzpicture}[decoration={text along path,text=
      around and around and around and around we go}]

  \draw (0,0) -- (1,1) decorate { -- (2,1) } -- (3,0);
\end{tikzpicture}
\end{codeexample}

  It is permissible to nest |decorate| commands. In this case, the
  path resulting from the first decoration process is used as the
  to-be-decorated path for the second decoration process. This is
  especially useful for drawing fractals. The |Koch snowflake|
  decoration replaces a straight line like \tikz\draw (0,0) -- (1,0);
  by \tikz[decoration=Koch snowflake] \draw decorate{(0,0) --
    (1,0)};. Repeatedly applying this transformation to a triangle
  yields a fractal that looks a bit like a snowflake, hence the name.
\begin{codeexample}[]
\begin{tikzpicture}[decoration=Koch snowflake,draw=blue,fill=blue!20,thick]
  \filldraw (0,0) -- ++(60:1) -- ++(-60:1) -- cycle ;
  \filldraw decorate{ (0,-1) -- ++(60:1) -- ++(-60:1) -- cycle };
  \filldraw decorate{ decorate{ (0,-2.5) -- ++(60:1) -- ++(-60:1) -- cycle }};
\end{tikzpicture}
\end{codeexample}
\end{pathoperation}



\subsection{Decorating a Complete Path}

You may sometimes wish to decorate a path over whose construction you
have no control. For instance, the path of the background of a node is
created without having a chance to issue a |decorate| path
command. In such cases you can use the following option, which allows
you to decorate a path ``after the fact.''

\begin{key}{/tikz/decorate=\opt{\meta{boolean}} (default true)}
  When this key is set, the whole path is decorated after it has been
  finished. The decoration used for decorating the path is set via the
  |decoration| way, in exactly the same way as for the |decorate| path
  command. Indeed, the following two commands have the same effect:
  \begin{enumerate}
  \item |\path decorate[|\meta{options}|] {|\meta{path}|};|
  \item |\path [decorate,|\meta{options}|] |\meta{path}|;|
  \end{enumerate}
  The main use or the |decorate| option is the you can also use it
  with the nodes. It then causes the background path of the node to be
  decorated. Note that you can decorate a background path only once in
  this manner. That is, in contrast to the |decorate| path command you
  cannot apply this option twice (this would just set it to |true|,
  once more).

\begin{codeexample}[]
\begin{tikzpicture}[decoration=zigzag]
  \draw [help lines] (0,0) grid (3,5);

  \draw [fill=blue!20,decorate] (1.5,4) circle (1cm);

  \node at (1.5,2.5) [fill=red!20,decorate,ellipse] {Ellipse};

  \node at (1.5,1) [inner sep=6mm,fill=red!20,decorate,ellipse,decoration=
    {text along path,text={This is getting silly}}] {Ellipse};
\end{tikzpicture}
\end{codeexample}

  In the last example, the |text along path| decoration removes the
  path. In such cases it is useful to use a pre- or postaction to
  cause the decoration to be applied only before or after the main
  path has been used. Incidentally, this is another application of the
  |decorate| option that you cannot achieve with the decorate path
  command.
\begin{codeexample}[]
\begin{tikzpicture}[decoration=zigzag]
  \node at (1.5,1) [inner sep=6mm,fill=red!20,ellipse,
    postaction={decorate,decoration=
    {text along path,text={This is getting silly}}}] {Ellipse};
\end{tikzpicture}
\end{codeexample}
  Here is more useful example, where a postaction is used to add the
  path after the main path has been drawn.
% \catcode`\|12 % !?
\begin{codeexample}[]
\begin{tikzpicture}
\draw [help lines] grid (3,2);
\fill [draw=red,fill=red!20,
         postaction={decorate,decoration={raise=2pt,text along path,
           text=around and around and around and around we go}}]
  (0,1) arc (180:-180:1.5cm and 1cm);
\end{tikzpicture}
\end{codeexample}
\end{key}


\subsection{Adjusting Decorations}

\label{section-decorations-adjust}

\subsubsection{Positioning Decorations Relative to the To-Be-Decorate Path}

The following option, which are only available with \tikzname, allow
you to modify the positioning of decorations relative to the
to-be-decorated path.

\begin{key}{/pgf/decoration/raise=\meta{dimension} (initially 0pt)}
  The segments of the decoration are raised by \meta{dimension}
  relative to the to-be-decorated path. More precisely, the segments
  of the path are offset by this much ``to the left'' of the path as
  we travel along the path. This raising is done after and in addition
  to any transformations set using the |transform| option (see below).

  A negative \meta{dimension} will offset the decoration ``to the
  right'' of the to-be-decorated path.
\begin{codeexample}[]
\begin{tikzpicture}
  \draw [help lines] (0,0) grid (3,2);

  \draw (0,0) -- (1,1) arc (90:0:2 and 1);
  \draw      decorate [decoration=crosses]
        { (0,0) -- (1,1) arc (90:0:2 and 1) };
  \draw[red] decorate [decoration={crosses,raise=5pt}]
        { (0,0) -- (1,1) arc (90:0:2 and 1) };
\end{tikzpicture}
\end{codeexample}
\end{key}

\begin{key}{/pgf/decoration/mirror=\opt{\meta{boolean}}}
  Causes the segments of the decoration to be mirrored along the
  to-be-decorated path. This is done after and in addition to any
  transformations set using the |transform| and/or |raise| options.
\begin{codeexample}[]
\begin{tikzpicture}
  \node (a)          {A};
  \node (b) at (2,1) {B};
  \draw                                                    (a) -- (b);
  \draw[decorate,decoration=brace]                         (a) -- (b);
  \draw[decorate,decoration={brace,mirror},red]            (a) -- (b);
  \draw[decorate,decoration={brace,mirror,raise=5pt},blue] (a) -- (b);
\end{tikzpicture}
\end{codeexample}
\end{key}


\begin{key}{/pgf/decoration/transform=\meta{transformations}}
  This key allows you to specify general \meta{transformations} to be
  applied to the segments of a decoration. These transformations are
  applied before and independently of |raise| and |mirror|
  transformations. The \meta{transformations} should be normal
  \tikzname\ transformations like |shift| or |rotate|.

  In the following example the |shift only| transformation is used to
  make sure that the crosses are \emph{not} sloped along the path.
\begin{codeexample}[]
\begin{tikzpicture}
  \draw [help lines] (0,0) grid (3,2);

  \draw (0,0) -- (1,1) arc (90:0:2 and 1);
  \draw[red,very thick] decorate [decoration={
               crosses,transform={shift only},shape size=1.5mm}]
        { (0,0) -- (1,1) arc (90:0:2 and 1) };
\end{tikzpicture}
\end{codeexample}
\end{key}


\subsubsection{Starting and Ending Decorations Early or Late}

You sometimes may wish to ``end'' a decoration a bit early on the
path. For instance, you might wish a |snake| decoration to stop 5mm
before the end of the path and to continue in a straight line. There
are different ways of achieving this effect, but the easiest may be
the |pre| and |post| options, which only have an effect in
\tikzname. Note, however, that they can only be used with decorations,
not with meta-decorations.

\begin{key}{/pgf/decoration/pre=\meta{decoration} (initially lineto)}
  This key sets a decoration that should be used before the main
  decoration starts. The \meta{decoration} will be used for a length
  of |pre length|, which |0pt| by default. Thus, for the |pre| option
  to have any effect, you also need to set the |pre length| option.
\begin{codeexample}[]
\begin{tikzpicture}
\tikz [decoration={zigzag,pre=lineto,pre length=1cm}]
  \draw [decorate] (0,0) -- (2,1) arc (90:0:1);
\end{tikzpicture}
\end{codeexample}
\begin{codeexample}[]
\begin{tikzpicture}
\tikz [decoration={zigzag,pre=moveto,pre length=1cm}]
  \draw [decorate] (0,0) -- (2,1) arc (90:0:1);
\end{tikzpicture}
\end{codeexample}
\begin{codeexample}[]
\begin{tikzpicture}
\tikz [decoration={zigzag,pre=crosses,pre length=1cm}]
  \draw [decorate] (0,0) -- (2,1) arc (90:0:1);
\end{tikzpicture}
\end{codeexample}

  Note that the default |pre| option is |lineto|, not |curveto|. This
  means that the default |pre| decoration will not follow curves (for
  efficiency reasons). Change the |pre| key to |curveto| if you have a
  curved path.
\begin{codeexample}[]
\begin{tikzpicture}
\tikz [decoration={zigzag,pre length=3cm}]
  \draw [decorate] (0,0) -- (2,1) arc (90:0:1);
\end{tikzpicture}
\end{codeexample}
\begin{codeexample}[]
\begin{tikzpicture}
\tikz [decoration={zigzag,pre=curveto,pre length=3cm}]
  \draw [decorate] (0,0) -- (2,1) arc (90:0:1);
\end{tikzpicture}
\end{codeexample}
\end{key}

\begin{key}{/pgf/decoration/pre length=\meta{dimension} (initially 0pt)}
  This key sets the distance along which the pre-decoration should be
  used. If you do not need/wish a pre-decoration, set this key to
  |0pt| (exactly this string, not just to something that evaluates to
  the same things such as |0cm|).
\end{key}

\begin{key}{/pgf/decorations/post=\meta{decoration} (initially
    lineto)}
  Works like |pre|, only for the end of the decoration.
\end{key}

\begin{key}{/pgf/decorations/post length=\meta{dimension} (initially
    0pt)}
  Works like |pre length|, only for the end of the decoration.
\end{key}

Here is a typical example that shows how these keys can be used:

\begin{codeexample}[]
\begin{tikzpicture}
  [decoration=snake,
   line around/.style={decoration={pre length=#1,post length=#1}}]

  \draw[->,decorate]                  (0,0)    -- ++(3,0);
  \draw[->,decorate,line around=5pt]  (0,-5mm) -- ++(3,0);
  \draw[->,decorate,line around=1cm]  (0,-1cm) -- ++(3,0);
\end{tikzpicture}
\end{codeexample}



\endinput

% % Copyright 2006 by Till Tantau
%
% This file may be distributed and/or modified
%
% 1. under the LaTeX Project Public License and/or
% 2. under the GNU Free Documentation License.
%
% See the file doc/generic/pgf/licenses/LICENSE for more details.


\section{Transformations}

\pgfname\ has a powerful transformation mechanism that is similar to the
transformation capabilities of \textsc{metafont}. The present section explains
how you can access it in \tikzname.


\subsection{The Different Coordinate Systems}

It is a long process from  a coordinate like, say, $(1,2)$ or
$(1\mathrm{cm},5\mathrm{pt})$, to the position a point is finally placed on the
display or paper. In order to find out where the point should go, it is
constantly ``transformed'', which means that it is mostly shifted around and
possibly rotated, slanted, scaled, and otherwise mutilated.

In detail, (at least) the following transformations are applied to a coordinate
like $(1,2)$ before a point on the screen is chosen:
%
\begin{enumerate}
    \item \pgfname\ interprets a coordinate like $(1,2)$  in its
        $xy$-coordinate system as ``add the current $x$-vector once and the
        current $y$-vector twice to obtain the new point''.
    \item \pgfname\ applies its coordinate transformation matrix to the
        resulting coordinate. This yields the final position of the point
        inside the picture.
    \item The backend driver (like |dvips| or |pdftex|) adds transformation
        commands such that the coordinate is shifted to the correct position in
        \TeX's page coordinate system.
    \item \textsc{pdf} (or PostScript) apply the canvas transformation matrix
        to the point, which can once more change the position on the page.
    \item The viewer application or the printer applies the device
        transformation matrix to transform the coordinate to its final pixel
        coordinate on the screen or paper.
\end{enumerate}

In reality, the process is even more involved, but the above should give the
idea: A point is constantly transformed by changes of the coordinate system.

In \tikzname, you only have access to the first two coordinate systems: The
$xy$-coordinate system and the coordinate transformation matrix (these will be
explained later). \pgfname\ also allows you to change the canvas transformation
matrix, but you have to use commands of the core layer directly to do so and
you ``better know what you are doing'' when you do this. The moment you start
modifying the canvas matrix, \pgfname\ immediately loses track of all
coordinates and shapes, anchors, and bounding box computations will no longer
work.


\subsection{The XY- and XYZ-Coordinate Systems}
\label{section-xyz}

The first and easiest coordinate systems are \pgfname's $xy$- and
$xyz$-coordinate systems. The idea is very simple: Whenever you specify a
coordinate like |(2,3)| this means $2v_x + 3v_y$, where $v_x$ is the current
\emph{$x$-vector} and $v_y$ is the current \emph{$y$-vector}. Similarly, the
coordinate |(1,2,3)| means $v_x + 2v_y + 3v_z$.

Unlike other packages, \pgfname\ does not insist that $v_x$ actually has a
$y$-component of $0$, that is, that it is a horizontal vector. Instead, the
$x$-vector can point anywhere you want. Naturally, \emph{normally} you will
want the $x$-vector to point horizontally.

One undesirable effect of this flexibility is that it is not possible to
provide mixed coordinates as in $(1,2\mathrm{pt})$. Life is hard.

To change the $x$-, $y$-, and $z$-vectors, you can use the following options:

\begin{key}{/tikz/x=\meta{value} (initially 1cm)}
    If \meta{value} is a dimension, the $x$-vector of \pgfname's
    $xyz$-coordinate system is set up to point \meta{value} to the right, that
    is, to $(\meta{value},0pt)$.
    %
\begin{codeexample}[]
\begin{tikzpicture}
  \draw                  (0,0)   -- +(1,0);
  \draw[x=2cm,color=red] (0,0.1) -- +(1,0);
\end{tikzpicture}
\end{codeexample}

\begin{codeexample}[]
\tikz \draw[x=1.5cm] (0,0) grid (2,2);
\end{codeexample}

    The last example shows that the size of steppings in grids, just like all
    other dimensions, are not affected by the $x$-vector. After all, the
    $x$-vector is only used to determine the coordinate of the upper right
    corner of the grid.

    If \meta{value} is a coordinate, the $x$-vector of \pgfname's
    $xyz$-coordinate system to the specified coordinate. If \meta{value}
    contains a comma, it must be put in braces.
    %
\begin{codeexample}[]
\begin{tikzpicture}
  \draw                            (0,0) -- (1,0);
  \draw[x={(2cm,0.5cm)},color=red] (0,0) -- (1,0);
\end{tikzpicture}
\end{codeexample}

    You can use this, for example, to exchange the meaning of the $x$- and
    $y$-coordinate.
    %
\begin{codeexample}[]
\begin{tikzpicture}[smooth]
  \draw plot coordinates{(1,0) (2,0.5) (3,0) (3,1)};
  \draw[x={(0cm,1cm)},y={(1cm,0cm)},color=red]
        plot coordinates{(1,0) (2,0.5) (3,0) (3,1)};
\end{tikzpicture}
\end{codeexample}
    %
\end{key}

\begin{key}{/tikz/y=\meta{value} (initially 1cm)}
    Works like the |x=| option, only if \meta{value} is a dimension, the
    resulting vector points to $(0,\meta{value})$.
\end{key}

\begin{key}{/tikz/z=\meta{value} (initially \normalfont$-3.85$mm)}
    Works like the |y=| option, but now a dimension is the point
    $(\meta{value},\meta{value})$.
    %
\begin{codeexample}[]
\begin{tikzpicture}[z=-1cm,->,thick]
  \draw[color=red] (0,0,0) -- (1,0,0);
  \draw[color=blue] (0,0,0) -- (0,1,0);
  \draw[color=orange] (0,0,0) -- (0,0,1);
\end{tikzpicture}
\end{codeexample}
    %
\end{key}


\subsection{Coordinate Transformations}

\pgfname\ and \tikzname\ allow you to specify \emph{coordinate
transformations}. Whenever you specify a coordinate as in |(1,0)| or
|(1cm,1pt)| or |(30:2cm)|, this coordinate is first ``reduced'' to a position
of the form ``$x$ points to the right and $y$ points upwards''. For example,
|(1in,5pt)| is reduced to ``$72\frac{72}{100}$ points to the right and 5 points
upwards'' and |(90:100pt)| means ``0pt to the right and 100 points upwards''.

The next step is to apply the current \emph{coordinate transformation matrix}
to the coordinate. For example, the coordinate transformation matrix might
currently be set such that it adds a certain constant to the $x$ value. Also,
it might be set up such that it, say, exchanges the $x$ and $y$ value. In
general, any ``standard'' transformation like translation, rotation, slanting,
or scaling or any combination thereof is possible. (Internally, \pgfname\ keeps
track of a coordinate transformation matrix very much like the concatenation
matrix used by \textsc{pdf} or PostScript.)
%
\begin{codeexample}[]
\begin{tikzpicture}
  \draw[help lines] (0,0) grid (3,2);
  \draw (0,0) rectangle (1,0.5);
  \begin{scope}[xshift=1cm]
    \draw             [red]    (0,0) rectangle (1,0.5);
    \draw[yshift=1cm] [blue]   (0,0) rectangle (1,0.5);
    \draw[rotate=30]  [orange] (0,0) rectangle (1,0.5);
  \end{scope}
\end{tikzpicture}
\end{codeexample}

The most important aspect of the coordinate transformation matrix is \emph{that
it applies to coordinates only!} In particular, the coordinate transformation
has no effect on things like the line width or the dash pattern or the shading
angle. In certain cases, it is not immediately clear whether the coordinate
transformation matrix \emph{should} apply to a certain dimension. For example,
should the coordinate transformation matrix apply to grids? (It does.) And what
about the size of arced corners? (It does not.) The general rule is: ``If there
is no `coordinate' involved, even `indirectly', the matrix is not applied.''.
However, sometimes, you simply have to try or look it up in the documentation
whether the matrix will be applied.

Setting the matrix cannot be done directly. Rather, all you can do is to
``add'' another transformation to the current matrix. However, all
transformations are local to the current \TeX-group. All transformations are
added using graphic options, which are described below.

Transformations apply immediately when they are encountered ``in the middle of
a path'' and they apply only to the coordinates on the path following the
transformation option.
%
\begin{codeexample}[]
\tikz \draw (0,0) rectangle (1,0.5) [xshift=2cm] (0,0) rectangle (1,0.5);
\end{codeexample}

A final word of warning: You should refrain from using ``aggressive''
transformations like a scaling of a factor of 10\,000. The reason is that all
transformations are done using \TeX, which has a fairly low accuracy.
Furthermore, in certain situations it is necessary that \tikzname\
\emph{inverts} the current transformation matrix and this will fail if the
transformation matrix is badly conditioned or even singular (if you do not know
what singular matrices are, you are blessed).

\begin{key}{/tikz/shift={\ttfamily\char`\{}\meta{coordinate}{\ttfamily\char`\}}}
    Adds the \meta{coordinate} to all coordinates.
    %
\begin{codeexample}[]
\begin{tikzpicture}
  \draw[help lines] (0,0) grid (3,2);
  \draw                       (0,0) -- (1,1) -- (1,0);
  \draw[shift={(1,1)},blue]   (0,0) -- (1,1) -- (1,0);
  \draw[shift={(30:1cm)},red] (0,0) -- (1,1) -- (1,0);
\end{tikzpicture}
\end{codeexample}
    %
\end{key}

\begin{key}{/tikz/shift only}
    This option does not take any parameter. Its effect is to cancel all
    current transformations except for the shifting. This means that the origin
    will remain where it is, but any rotation around the origin or scaling
    relative to the origin or skewing will no longer have an effect.

    This option is useful in situations where a complicated transformation is
    used to ``get to a position'', but you then wish to draw something
    ``normal'' at this position.
    %
\begin{codeexample}[]
\begin{tikzpicture}
  \draw[help lines] (0,0) grid (3,2);
  \draw                                      (0,0) -- (1,1) -- (1,0);
  \draw[rotate=30,xshift=2cm,blue]           (0,0) -- (1,1) -- (1,0);
  \draw[rotate=30,xshift=2cm,shift only,red] (0,0) -- (1,1) -- (1,0);
\end{tikzpicture}
\end{codeexample}
    %
\end{key}

\begin{key}{/tikz/xshift=\meta{dimension}}
    Adds \meta{dimension} to the $x$ value of all coordinates.
    %
\begin{codeexample}[]
\begin{tikzpicture}
  \draw[help lines] (0,0) grid (3,2);
  \draw                   (0,0) -- (1,1) -- (1,0);
  \draw[xshift=2cm,blue]  (0,0) -- (1,1) -- (1,0);
  \draw[xshift=-10pt,red] (0,0) -- (1,1) -- (1,0);
\end{tikzpicture}
\end{codeexample}
    %
\end{key}

\begin{key}{/tikz/yshift=\meta{dimension}}
    Adds \meta{dimension} to the $y$ value of all coordinates.
\end{key}

\begin{key}{/tikz/scale=\meta{factor}}
    Multiplies all coordinates by the given \meta{factor}. The \meta{factor}
    should not be excessively large in absolute terms or very close to zero.
    %
\begin{codeexample}[]
\begin{tikzpicture}
  \draw[help lines] (0,0) grid (3,2);
  \draw               (0,0) -- (1,1) -- (1,0);
  \draw[scale=2,blue] (0,0) -- (1,1) -- (1,0);
  \draw[scale=-1,red] (0,0) -- (1,1) -- (1,0);
\end{tikzpicture}
\end{codeexample}
    %
\end{key}

\begin{key}{/tikz/scale around={\ttfamily\char`\{}\meta{factor}|:|\meta{coordinate}{\ttfamily\char`\}}}
    Scales the coordinate system by \meta{factor}, with the ``origin of
    scaling'' centered on \meta{coordinate} rather than the origin.
    %
\begin{codeexample}[]
\begin{tikzpicture}
  \draw[help lines] (0,0) grid (3,2);
  \draw                             (0,0) -- (1,1) -- (1,0);
  \draw[scale=2,blue]               (0,0) -- (1,1) -- (1,0);
  \draw[scale around={2:(1,1)},red] (0,0) -- (1,1) -- (1,0);
\end{tikzpicture}
\end{codeexample}
    %
\end{key}

\begin{key}{/tikz/xscale=\meta{factor}}
    Multiplies only the $x$-value of all coordinates by the given
    \meta{factor}.
\begin{codeexample}[]
\begin{tikzpicture}
  \draw[help lines] (0,0) grid (3,2);
  \draw                (0,0) -- (1,1) -- (1,0);
  \draw[xscale=2,blue] (0,0) -- (1,1) -- (1,0);
  \draw[xscale=-1,red] (0,0) -- (1,1) -- (1,0);
\end{tikzpicture}
\end{codeexample}
    %
\end{key}

\begin{key}{/tikz/yscale=\meta{factor}}
    Multiplies only the $y$-value of all coordinates by \meta{factor}.
\end{key}

\begin{key}{/tikz/xslant=\meta{factor}}
    Slants the coordinate horizontally by the given \meta{factor}:
    %
\begin{codeexample}[]
\begin{tikzpicture}
  \draw[help lines] (0,0) grid (3,2);
  \draw                (0,0) -- (1,1) -- (1,0);
  \draw[xslant=2,blue] (0,0) -- (1,1) -- (1,0);
  \draw[xslant=-1,red] (0,0) -- (1,1) -- (1,0);
\end{tikzpicture}
\end{codeexample}
    %
\end{key}

\begin{key}{/tikz/yslant=\meta{factor}}
    Slants the coordinate vertically by the given \meta{factor}:
    %
\begin{codeexample}[]
\begin{tikzpicture}
  \draw[help lines] (0,0) grid (3,2);
  \draw                (0,0) -- (1,1) -- (1,0);
  \draw[yslant=2,blue] (0,0) -- (1,1) -- (1,0);
  \draw[yslant=-1,red] (0,0) -- (1,1) -- (1,0);
\end{tikzpicture}
\end{codeexample}
    %
\end{key}

\begin{key}{/tikz/rotate=\meta{degree}}
    Rotates the coordinate system by \meta{degree}:
    %
\begin{codeexample}[]
\begin{tikzpicture}
  \draw[help lines] (0,0) grid (3,2);
  \draw                 (0,0) -- (1,1) -- (1,0);
  \draw[rotate=40,blue] (0,0) -- (1,1) -- (1,0);
  \draw[rotate=-20,red] (0,0) -- (1,1) -- (1,0);
\end{tikzpicture}
\end{codeexample}
    %
\end{key}

\begin{key}{/tikz/rotate around={\ttfamily\char`\{}\meta{degree}|:|\meta{coordinate}{\ttfamily\char`\}}}
    Rotates the coordinate system by \meta{degree} around the point
    \meta{coordinate}.
    %
\begin{codeexample}[]
\begin{tikzpicture}
  \draw[help lines] (0,0) grid (3,2);
  \draw                                (0,0) -- (1,1) -- (1,0);
  \draw[rotate around={40:(1,1)},blue] (0,0) -- (1,1) -- (1,0);
  \draw[rotate around={-20:(1,1)},red] (0,0) -- (1,1) -- (1,0);
\end{tikzpicture}
\end{codeexample}
    %
\end{key}

\begin{key}{/tikz/rotate around x=\meta{angle}}
    This key sets the $x$, $y$ and $z$ vectors of the \pgfname\
    $xyz$-coordinate system so that they are rotated by \meta{angle} around the
    axis corresponding to the $x$-vector. The rotation is applied so that when
    looking towards the origin along this axis, positive angles result in an
    anticlockwise rotation.
    %
\begin{codeexample}[]
\begin{tikzpicture}[>=stealth]
  \draw [->] (0,0,0) -- (2,0,0) node [at end, right] {$x$};
  \draw [->] (0,0,0) -- (0,2,0) node [at end, left]  {$y$};
  \draw [->] (0,0,0) -- (0,0,2) node [at end, left]  {$z$};

  \draw [red,   rotate around x=0]  (0,0,0) -- (1,1,0) -- (1,0,0);
  \draw [green, rotate around x=45] (0,0,0) -- (1,1,0) -- (1,0,0);
  \draw [blue,  rotate around x=90] (0,0,0) -- (1,1,0) -- (1,0,0);
\end{tikzpicture}
\end{codeexample}
    %
\end{key}

\begin{key}{/tikz/rotate around y=\meta{angle}}
    This key sets the $x$, $y$ and $z$ vectors of the \pgfname\
    $xyz$-coordinate system so that they are rotated by \meta{angle} around
    the axis corresponding to the $y$-vector. The rotation is applied so that
    when looking towards the origin along this axis, positive angles result in
    an anticlockwise rotation.
    %
\begin{codeexample}[]
\begin{tikzpicture}[>=stealth]
  \draw [->] (0,0,0) -- (2,0,0) node [at end, right] {$x$};
  \draw [->] (0,0,0) -- (0,2,0) node [at end, left]  {$y$};
  \draw [->] (0,0,0) -- (0,0,2) node [at end, left]  {$z$};

  \draw [red,   rotate around y=0]   (0,0,0) -- (1,1,0) -- (1,0,0);
  \draw [green, rotate around y=-45] (0,0,0) -- (1,1,0) -- (1,0,0);
  \draw [blue,  rotate around y=-90] (0,0,0) -- (1,1,0) -- (1,0,0);
\end{tikzpicture}
\end{codeexample}
    %
\end{key}

\begin{key}{/tikz/rotate around z=\meta{angle}}
    This key sets the $x$, $y$ and $z$ vectors of the \pgfname\
    $xyz$-coordinate system so that they are rotated by \meta{angle} around the
    axis corresponding to the $z$-vector. The rotation is applied so that when
    looking towards the origin along this axis, positive angles result in an
    anticlockwise rotation.
    %
\begin{codeexample}[]
\begin{tikzpicture}[>=stealth]
  \draw [->] (0,0,0) -- (2,0,0) node [at end, right] {$x$};
  \draw [->] (0,0,0) -- (0,2,0) node [at end, left]  {$y$};
  \draw [->] (0,0,0) -- (0,0,2) node [at end, left]  {$z$};

  \draw [red,   rotate around z=0]  (0,0) -- (1,1) -- (1,0);
  \draw [green, rotate around z=45] (0,0) -- (1,1) -- (1,0);
  \draw [blue,  rotate around z=90] (0,0) -- (1,1) -- (1,0);
\end{tikzpicture}
\end{codeexample}
    %
\end{key}

\begin{key}{/tikz/cm={\ttfamily\char`\{}\meta{$a$}|,|\meta{$b$}|,|\meta{$c$}|,|\meta{$d$}|,|\meta{coordinate}{\ttfamily\char`\}}}
    applies the following transformation to all coordinates: Let $(x,y)$ be the
    coordinate to be transformed and let \meta{coordinate} specify the point
    $(t_x,t_y)$. Then the new coordinate is given by
    $\left(\begin{smallmatrix} a & c \\ b & d\end{smallmatrix}\right)
    \left(\begin{smallmatrix} x \\ y \end{smallmatrix}\right) +
    \left(\begin{smallmatrix} t_x \\ t_y \end{smallmatrix}\right)$.
    Usually, you do not use this option directly.
    %
\begin{codeexample}[]
\begin{tikzpicture}
  \draw[help lines] (0,0) grid (3,2);
  \draw                             (0,0) -- (1,1) -- (1,0);
  \draw[cm={1,1,0,1,(0,0)},blue]    (0,0) -- (1,1) -- (1,0);
  \draw[cm={0,1,1,0,(1cm,1cm)},red] (0,0) -- (1,1) -- (1,0);
\end{tikzpicture}
\end{codeexample}
    %
\end{key}

\begin{key}{/tikz/reset cm}
    Completely resets the coordinate transformation matrix to the identity
    matrix. This will destroy not only the transformations applied in the
    current scope, but also all transformations inherited from surrounding
    scopes. Do not use this option, unless you really, really know what you are
    doing.
\end{key}


\subsection{Canvas Transformations}

A \emph{canvas transformation}, see
Section~\ref{section-design-transformations} for details, is best thought of as
a transformation in which the drawing canvas is stretched or rotated. Imaging
writing something on a balloon (the canvas) and then blowing air into the
balloon: Not only does the text become larger, the thin lines also become
larger. In particular, if you scale the canvas by a factor of two, all lines
are twice as thick.

Canvas transformations should be used with great care. In most circumstances
you do \emph{not} want line widths to change in a picture as this creates
visual inconsistency.

Just as important, when you use canvas transformations \emph{\pgfname\ loses
track of positions of nodes and of picture sizes} since it does not take the
effect of canvas transformations into account when it computes coordinates of
nodes (do not, however, rely on this; it may change in the future).

Finally, note that a canvas transformation always applies to a path as a whole,
it is not possible (as for coordinate transformations) to use different
transformations in different parts of a path.

In short, you should not use canvas transformations unless you really know what
you are doing.

\begin{key}{/tikz/transform canvas=\meta{options}}
    The \meta{options} should contain coordinate transformations options like
    |scale| or |xshift|. Multiple options can be given, their effects
    accumulate in the usual manner. The effect of these \meta{options}
    (immediately) changes the current canvas transformation matrix. The
    coordinate transformation matrix is not changed. Tracking of the picture
    size is (locally) switched off and the node coordinate will no longer be
    correct.
    %
\begin{codeexample}[]
\begin{tikzpicture}
  \draw[help lines] (0,0) grid (3,2);
  \draw                                    (0,0) -- (1,1) -- (1,0);
  \draw[transform canvas={scale=2},blue]   (0,0) -- (1,1) -- (1,0);
  \draw[transform canvas={rotate=180},red] (0,0) -- (1,1) -- (1,0);
\end{tikzpicture}
\end{codeexample}
    %
\end{key}
