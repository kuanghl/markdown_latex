\chapter{原版跋}

首先向一路披荆斩棘看到这里的读者表示祝贺,至少在精神上你已经成为一名合格的 \LaTeX{}er。从此你生是 \LaTeX 的人,死是 \LaTeX 的鬼。Once black, never back。没有坚持到这里的同学自然已经重新投向“邪恶”的MS Word,毕竟那里点个按钮就可以插入图形,点个下拉框就可以选择字体。

当然 \LaTeX{}er也有简单的出路,就是只使用缺省设置,尽量少用插图;不必理会点阵、矢量,也不必理会 Type 1, Type 3, TrueType, OpenType。因为内容高于形式,你把文章的版面、字体搞得再漂亮,它也不会因此成为《红楼梦》;而《红楼梦》即使是手抄本,也依然是不朽的名著。

包老师曾经以为 \LaTeX 和 Word 的关系就好像是《笑傲江湖》中华山的气宗和剑宗,头十年剑宗进步快,中间十年打个平手,再往后气宗就遥遥领先。至于令狐冲的无招胜有招,风清扬的神龙见首不见尾又是另一重境界,普通人恐怕只能望其颈背。

费尽九牛二虎之力熬到本文杀青的时候,才发现从前的想法很傻很天真。让我们挥一挥衣袖,不带走一片云,卧薪尝胆忍辱负重,耐心等待 \XeTeX 和 \LuaTeX。

\begin{flushright}
  2008年7月于法戈
\end{flushright}
